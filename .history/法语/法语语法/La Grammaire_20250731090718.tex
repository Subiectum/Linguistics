\documentclass[UTF8]{report}
\usepackage{ctex}
\usepackage{amsmath}
\usepackage{amssymb}
\usepackage{graphicx}
\usepackage{float}
\usepackage{tabularx}
\usepackage{ragged2e}
\usepackage{multirow}

\usepackage[table]{xcolor}
\usepackage{booktabs}
\usepackage{array}
\usepackage{adjustbox}
\usepackage[table]{xcolor}
\usepackage{longtable}
\usepackage{subcaption}
\usepackage[breaklinks,colorlinks,linkcolor=black,citecolor=black,urlcolor=black]{hyperref}
\usepackage{geometry}
\usepackage{tikz}

\usepackage{booktabs}

\newcolumntype{Y}{>{\RaggedRight\arraybackslash\leavevmode}X} % <-- Changed name to Y
% Define a new column type for justified text (left-aligned)
\newcolumntype{L}[1]{>{\RaggedRight\arraybackslash}p{#1}}

\setcounter{secnumdepth}{3}

\special{dvipdfmx:config z 0} %取消PDF压缩,加快速度,最终版本生成的时候最好把这句话注释掉
\geometry{a4paper,left=2cm,right=2cm,top=2cm,bottom=2cm}

\usepackage{newpxtext,newpxmath}

\usepackage{titlesec}
\titleformat{\paragraph}
{\normalfont\normalsize\bfseries\itshape}{\theparagraph}{1em}{}
\titlespacing*{\paragraph}
{0pt}{3.25ex plus 1ex minus .2ex}{1.5ex plus .2ex}





\usepackage{enumitem}
\renewlist{enumerate}{enumerate}{10}
\setlist[enumerate]{
    label*=\arabic*., % 注意结尾多一个点
    format=\bfseries,
    leftmargin=2em
}
\setlist[enumerate,1]{label=\arabic*.} % 一级编号加点

\usepackage{hyperref}
\usepackage{bookmark}


\usepackage{caption}
\captionsetup[table]{
    position=below,    
    justification=raggedright, 
    singlelinecheck=false, 
    font=small,        
    labelsep=space     
}


\begin{document}


\chapter{Adjectif}
\section{La formation des adjectifs}
\subsection{Les adjectifs dérivés d’un nom}
\textit{-ique, -eux, -aire, -ain, -ais, -an, -ois, -el, -esque, -ier, -iste, -u, -ard, -in.}
\begin{enumerate}
    \item -u se combine avec des noms concrets désignant des parties du corps ou des réalités naturelles (barbu, feuillu).
    \item -eux se combine à des noms de qualité, de matière ou d’évènements météoro- logiques (brumeux, laineux, vertueux).
\end{enumerate}

\subsection{Les adjectifs dérivés d’un verbe}
\textit{-able, -ible,  -eur (féminin -euse) et -eur (féminin -rice), -if, -oire}

\subsection{Les adjectifs dérivés d’un adjectif}
\textit{-asse, -âtre, -aud, -if, -ingue, -issime, -ot, -(l)et}

\begin{enumerate}
    \item -ot, -(l)et : diminutifs
    \item -ard : s’accompagne d’un effet dépréciatif
    \item -issime : un haut degré
\end{enumerate}



\section{ Le genre et le nombre des adjectifs}

\subsection{Singulier}
\begin{table}[H]

\begin{adjustbox}{max width = \textwidth}
        \centering
        \begin{tabular}{|l|l|l|l|}
        \hline
        \rowcolor{cyan!20}
        \textbf{ÉCRIT} & \textbf{ORAL} & \textbf{EXEMPLES} & \textbf{NOMBRE D'ADJECTIFS} \\
        \hline
        féminin = masculin & féminin = masculin & \textit{facile, énorme, rouge} & 3 503 \\
        \hline
        féminin = masculin + -e & féminin = masculin & \textit{joli / jolie, net / nette, public / publique} & 2 650 \\
        avec adaptation éventuelle & & & \\
        de la consonne & & & \\
        \hline
        féminin = masculin + -e & féminin = masculin & \textit{charmant / charmante, grand / grande,} & 1 962 \\
        avec adaptation éventuelle & + consonnes & \textit{jaloux / jalouse, petit / petite, roux / rousse} & \\
        de la consonne & & & \\
        \hline
        masculin en -f & masculin en /f/ & \textit{fautif / fautive, vif / vive} & 271 \\
        féminin en -ve & féminin en /v/ & & \\
        \hline
        féminin = masculin + -e & masculin en voyelle nasale & \textit{bon / bonne, plein / pleine, plan / plane} & 508 \\
        avec adaptation éventuelle & féminin en consonne nasale & & \\
        de la consonne & & & \\
        \hline
        \rowcolor{cyan!20}
        \textbf{ÉCRIT} & \textbf{ORAL} & \textbf{EXEMPLES} & \textbf{NOMBRE D'ADJECTIFS} \\
        \hline
        masculin en -\textit{eur} & masculin en /œr/ & \textit{trompeur / trompeuse, révélateur / révélatrice} & 327 \\
        féminin en -\textit{euse} ou -\textit{rice} & féminin en /øz/ ou /ris/ & & \\
        \hline
        cas restants & cas restants & \textit{beau / belle, mou / molle, vieux / vieille} & 13 \\
        \hline
        \end{tabular}
\end{adjustbox}
\end{table}

\subsection{Pluriel}
\begin{enumerate}
    \item + -s : masculin ou féminin
    \begin{itemize}
        \item fou / fous, grand / grands, petit / petits
        \item folle / folles, grande / grandes, petite / petites, normale / normales
    \end{itemize}
    \item + -x : masculin
    \begin{itemize}
        \item beau / beaux, nouveau / nouveaux
    \end{itemize}
\end{enumerate}



\section{Fonction d’ajout}

\begin{table}[H]
    \centering 
    \begin{tabular}{|l|l|}
    \hline
    \rowcolor{cyan!20}
    \textbf{FONCTION} & \textbf{EXEMPLES} \\
    \hline
    \rowcolor{cyan!20}
    \multicolumn{2}{|c|}{{accord avec le sujet}} \\
    \hline
    \textit{ajout} à la phrase, & \textit{Silencieuse, Marie aurait mieux convaincu.} \\
    avec prosodie incidente & \textit{Elle baisse la tête, silencieuse.} \\
    \hline
    \rowcolor{cyan!20}
    \multicolumn{2}{|c|}{{accord avec le nom modifié}} \\
    \hline
    \textit{ajout} au verbe & \textit{Marie est partie contente.} \\
    & \textit{Elle loue son appartement vide.} \\
    \hline
    \textit{apposé} (\textit{ajout} au SN), & \textit{avec la robe, trop petite, qu'elle avait achetée} \\
    avec prosodie incidente & \\
    \hline
    \textit{épithète} & \textit{une robe rouge, une petite jupe} \\
    (\textit{ajout} au nom ou au SN) & \\
    \hline
    \rowcolor{cyan!20}
    \multicolumn{2}{|c|}{{adjectif invariable}} \\
    \hline
    \textit{ajout} au verbe & \textit{Marie a refusé net votre proposition.} \\
    \hline
    \textit{ajout} à la phrase, & \textit{Plus grave encore, Marie ne nous a pas reconnus.} \\
    avec prosodie incidente & \\
    \hline
    \end{tabular}
\end{table}




\subsection{épithète}

\begin{enumerate}
    \item 形容词quitte两清的, d’accord, sauf安全的,只能作为attribut,不能作为épithète
    \begin{itemize}
        \item On est quitte.
        \item * J’ai parlé à des amis quittes.
        \item Alex est sauf, dieu merci !
    \end{itemize}
    \begin{enumerate}
        \item 固定搭配sain et sauf安然无恙可作为épithète
        \begin{itemize}
            \item un homme sain et sauf
        \end{itemize}
    \end{enumerate}
    \item 形容词feu已故的, seul et tout可用在冠词前
    \begin{itemize}
        \item J’ai pensé à feue la reine
        \item Seul un général assistait à la cérémonie.
        \item Dans toute sa jeunesse, il était fort timide.
    \end{itemize}
    \item pronom + de + adjectif épithète
    \begin{enumerate}
        \item pronoms personnels只有moi seul, nous autres, vous tous, etc
        \item indéfinis (quelque chose, quelqu’un)
        \begin{itemize}
            \item Lise s’enthousiasme devant [quelque chose de nouveau] 
        \end{itemize}
        \item quantifieurs (personne, rien)
        \begin{itemize}
            \item Rien de bon ne peut venir de lui.
        \end{itemize}
        \item choix libre (quiconque, quoi que ce soit, n’importe quoi, je ne sais quoi)
        \begin{itemize}
            \item Nous n’avons pas lu [quoi que ce soit d’intéressant]
        \end{itemize}
        \item pronoms interrogatifs, de + adjectif可与代词分开而出现在动词后
        \begin{itemize}
            \item Qui d’autre as-tu rencontré ?
            \item Qui as-tu rencontré d’autre ?
            \item À quoi de déplaisant t’attends-tu ?
            \item À quoi t’attends-tu de déplaisant ?
        \end{itemize}
        \begin{enumerate}
            \item que的épithète只能出现在动词后
            \begin{itemize}
                \item * Que d’intéressant fais-tu ?
                \item Que fais-tu d’intéressant ?
            \end{itemize}
        \end{enumerate}
        \item pronom démonstratif, de + adjectif可与其分开
        \begin{itemize}
            \item non seulement dans ceci [de superficiel] qu’est son intrigue son timbre, sa tonalité.
            \item ce qui est arrivé [de pire] à l’humanité depuis la grippe espagnole
        \end{itemize}
        \item indéfinis autrui, d’aucuns, l’un, les uns;quantifieurs chacun, tous, tout不能跟adjectif épithète,只能用adjectif en apposition
        \begin{itemize}
            \item Les uns, malades, sont partis. | * Les uns (de) malades sont partis.
            \item Chacun, malade, est parti. | * Chacun (de) malade est parti.
        \end{itemize}
    \end{enumerate}
\end{enumerate}

\subsection{apposé}
\begin{itemize}
    \item Les chatons, déjà lestes et indisciplinés, vont gaillardement vers leur quatrième mois.
    \item Ce sont les seules fois où les propriétaires l’ont vue dans une autre robe que celle, grise et noire, qu’elle portait d’habitude.
\end{itemize}


\subsection{Modifieur de nom et ajout au verbe}

\subsubsection{L’adjectif ajout au verbe modifieur de nom sujet}
il s’accorde avec sujet
\begin{itemize}
    \item Jules s’est réveillé joyeux.
    \item Lise est arrivée contente, mais épuisée.
    \item Elles vinrent hier nombreuses à notre petite fête.
\end{itemize}


\subsubsection{L’adjectif ajout au verbe, modifieur de nom complément}
il s’accorde avec complément 
\begin{itemize}
    \item Lise a retrouvé Rose épanouie
    \item Je bois le monbazillac très frais.
    \item Paul a rencontré Lucie jeune.
\end{itemize}

\subsection{Invariable ajout au verbe ou à la phrase}
\subsubsection{L’adjectif invariable modifieur de verbe}
被称作adjectif adverbial
\begin{itemize}
    \item Jules a refusé net ma proposition.
    \item Parlons franc, travaillons dur.
    \item Il faut téléphoner rusé, jouer serré, se maquiller pâle cet hiver.
\end{itemize}

\begin{enumerate}
    \item 与真正副词的区别
    \begin{enumerate}
        \item 它不能置于l’auxiliaire et le participe passé之间
        \begin{itemize}
            \item * Jules a net refusé ma proposition.
            \item Jules a nettement refusé ma proposition.
            \item * Lise a dur travaillé.
            \item Lise a durement travaillé.
        \end{itemize}
        \item 不能在infinitif前
        \begin{itemize}
            \item * Il faudrait dur travailler.
            \item Il faudrait durement travailler.
        \end{itemize}
    \end{enumerate}
\end{enumerate}

\subsubsection{L’adjectif invariable connecteur}
comparatifs en début de phrase,作为connecteur de discours
\begin{itemize}
    \item Pire encore, tout est à refaire en partant de plus bas qu’on était parti, c’est décourageant.
    \item Plus curieux, Lise a choisi le poulet au gingembre.
    \item Lise a, plus ennuyeux, démissionné.
    \item Lise, plus ennuyeux, a démissionné.
    \item * Lise a démissionné, plus ennuyeux. (不能用于句尾)
\end{itemize}


\section{Fonction de complément}


\subsection{L’adjectif attribut du sujet}
il s’accorde en genre et en nombre avec ce sujet
\begin{enumerate}
    \item 可被pour, comme引导
    \begin{itemize}
        \item Jules passe pour gentil.
        \item Rose apparait comme timide, alors qu’elle ne l’est guère.
    \end{itemize}
    \item 可被le指代,invariable
    \begin{itemize}
        \item Lise devient jolie.|Elle le devient.
        \item Les arbres sont beaux.|Ils le sont.
    \end{itemize}
    \begin{enumerate}
        \item 这些动词后的attribut不能被le指代:avoir l’air, faire, tomber, tourner;verbes réfléchis ;与attribut通过介词引导的动词
        \begin{itemize}
            \item Lise est tombée malade/a l’air triste.
            \item * Lise l’est tombée/l’en a l’air.
            \item Jules fait jeune avec cette veste de toile.|*Jules le fait.
            \item Antoine se fait tendre/se montre plus loquace.|*Antoine se le fait/se le montre.
            \item Lise passe pour espiègle.|*Lise le passe.
        \end{itemize}
    \end{enumerate}
    \item 用comment或quoi对其提问
    \begin{itemize}
        \item Comment est Jules ? Il est vulnérable/ironique/timide.
        \item Jules est quoi ? Il est catholique/alsacien/socialiste.
    \end{itemize}
\end{enumerate}

\subsection{L’adjectif attribut du complément}
Il s’accorde en genre et en nombre avec ce complément.
\begin{enumerate}
    \item 可用comment 和quoi对其提问
    \begin{itemize}
        \item Vous avez trouvé le spectacle comment ?
        \item On l’a déclaré quoi, innocent ou coupable ?
    \end{itemize}
    \item 其不能被le指代,只有complément direct才能被指代
    \begin{itemize}
        \item Max le croit magnifique, ce spectacle.
        \item * Magnifique, Max le croit ce spectacle.
        \item Nous l’avons trouvé admirable.
        \item * Nous l’avons trouvé le spectacle.
    \end{itemize}
    \begin{enumerate}
        \item rester, il y a中只能用en指代complément nominal
        \begin{itemize}
            \item Il y en a un malade.
            \item Il en reste un vacant.
        \end{itemize}
    \end{enumerate}
    \item 可被comme引导;comme与comme adverb不同,其在这里表示atténuation ou approximation
    \begin{itemize}
        \item Jules était comme ému.
        \item Lise l’a trouvé comme rajeuni.
    \end{itemize}
    \item 可被de引导
    \begin{enumerate}
        \item de用于当complément nominal est indéfini时
        \begin{itemize}
            \item Je vois deux étudiants de disponibles.
            \item Voilà mon humeur (de) gâchée.
            \item Il y a une tarte de prête.
        \end{itemize}
        \item de用于表示限制(ne)……que, seulement的句子中,可与complément nominal défini连用
        \begin{itemize}
            \item Je ne vois que Lise de compétente pour ce travail.
            \item Lise ne rend que Paul (de) fou.
            \item Le spectacle n’a laissé que Lise (de) ravie.
        \end{itemize}
        \item complément是infinitif或subordonnée complétive时,形容词前不能用de
        \begin{itemize}
            \item * Je trouve d’agréable de me reposer.
            \item * Jules estime de nécessaire seulement que tu te fasses représenter.
        \end{itemize}
        \item 只有de引导的形容词可以作为antéposition
        \begin{itemize}
            \item De prête, il reste une tarte.
            \item * Prête, il reste une tarte.
        \end{itemize}
    \end{enumerate}
\end{enumerate}

\subsubsection{distinguer adjectif attribut du complément et épithète}
\begin{enumerate}
    \item 只有adjectif attribut du complément可用于antéposition de l’adjectif或clivée avec le complément
    \begin{itemize}
        \item De libres, il reste trois places au balcon.
        \item C’est [trois places] qu’il reste de libres au balcon.
        \item * D’intéressant, Lise m’a raconté quelque chose.
        \item * C’est quelque chose que Lise m’a raconté d’intéressant.
    \end{itemize}
    \item 只有épithète的整个syntagme可以被le取代
        \begin{itemize}
            \item Max l’a regardé avec émerveillement, ce spectacle magnifique.
            \item Max le croit magnifique, ce spectacle.
            \item * Magnifique, Max le croit ce spectacle.
        \end{itemize}
\end{enumerate}

\subsubsection{complément invariable d’un verbe}
\begin{enumerate}
    \item Je veux manger sain.
    \item On parlait français.
    \item Paul dit vrai, voit double, risque gros.
    \item  je n’ai plus les yeux pour écrire si long 
\end{enumerate}





\subsection{L’adjectif invariable complément de préposition}
\subsubsection{En + adjectif}

\begin{enumerate}
    \item attribut
    \begin{itemize}
        \item La Mariée était en noir
    \end{itemize}
    \item ajout au verbe
    \begin{itemize}
        \item Liseveutpeindresachambreenrose,s’habillerenvert,semarierenblanc,arriver dimanche en turquoise.
    \end{itemize}
    \item épithète
    \begin{itemize}
        \item On a vu entrer une jolie fille en jaune.
    \end{itemize}
\end{enumerate}


\subsubsection{De + adjectif}
描述un changement d’état,l’état initial du procès;根据主语而变格

\begin{enumerate}
    \item 与devnir连用时,de + adjectif置于句首并用逗号隔开
    \begin{itemize}
        \item De pâle, Jules est devenu blême.
        \item D’hésitants, ils deviendront résolus.
        \item * Jules est devenu blême, de pâle.
    \end{itemize}
    \item de + adjectif可与subordonnée relative en que连用,此时其可置于句尾
    \begin{itemize}
        \item De coquette qu’elle était, Lise est devenue sage.
        \item Jules est devenu blême, de pâle qu’il était.
    \end{itemize}
\end{enumerate}

\subsubsection{Pour + adjectif/infinitif}
表示“至于/鉴于……,若论……”,置于句首并用逗号隔开
\begin{enumerate}
    \item Pour pleurer, il pleure.
    \item Pour de la douceur, c’était de la douceur.
    \item Pour de la douceur, ça en était bel et bien.
\end{enumerate}

\subsubsection{En fait de + adjectif/infinitif}
与pour意思相同
\begin{enumerate}
    \item en fait de génial c’était plutôt éprouvant. 
    \item En fait de courgettes, il a rapporté des concombres.
    \item En fait de pleurer, il pleurniche plutôt.
\end{enumerate}


\subsection{L’adjectif comparatif complément de verbe ou de préposition}
作为il y a, arriver, protéger的compléments directs;形容词保持invariable;形容词不是比较级;该结构中形容词只能为比较级而不是原形
\begin{enumerate}
    \item Il y a plus drôle. | * Il y a drôle.
    \item Il va arriver pire. | * Il va arriver mauvais.
    \item pour demander à protéger plus pitoyable que lui. | * Il a cherché à protéger pitoyable.
\end{enumerate}

\subsubsection{À / pour /avec + adjectif comparatif}
该结构中形容词只能为比较级而不是原形
\begin{itemize}
    \item Quand tu t’attaqueras à plus vieux, tu verras ! | * Quand tu t’attaqueras à vieux.
    \item Je cède la parole à plus informé que moi. | * Je cède la parole à informé.
    \item Ont-ils changé d’appartement pour plus grand ? | * Ont-ils changé d’appartement pour grand ?
    \item on bouchait maintenant les trous avec plus vieux que lui. | * On bouchait les trous avec vieux.
\end{itemize}

\section{Fonction de extrait et périphérique}
\subsection{extraits}
\subsubsection{antéposition avec inversion du sujet }
\begin{enumerate}
    \item 主语为sujet nominal,而不用proforme, infinitif ou subordonné
    \begin{itemize}
        \item Bouleversante était l’entrevue.
        \item * Bouleversante était-elle.
        \item ? Plus bouleversant était de partir/qu’il parte.
    \end{itemize}
    \item 作为attribut du sujet,,动词为être, rester, sembler
    \item adjectif常用comparatif
    \begin{itemize}
        \item Jules est d’un tempérament très vif. Plus douce est sa sœur.
        \item Si grandes étaient notre obsession, et notre confiance, à l’époque, que nous en aurions juré
    \end{itemize}
    \item tel作为attribut du sujet,或作为attribut du complément且complément已经被pronominalisé时,可用于该结构
    \begin{itemize}
        \item Je viendrai. Telle est ma décision.
        \item Jules était timide, ou du moins, tel le jugeait Rosine.
    \end{itemize}
    \item sujet可inversé或pronominal
    \begin{itemize}
        \item Tel était Paul, tel sera Jules.
        \item Tels nous sommes, tels nous demeurerons.
    \end{itemize}
\end{enumerate}



\subsubsection{L’adjectif antéposé dans  comparative corrélative}
\begin{enumerate}
    \item sujet可inversé或pronominal
    \begin{itemize}
        \item Plus brillante est l’interprétation, plus profond est le ravissement de l’auditeur.
        \item Plus vieux vous serez, moins riche vous serez.
        \item 
    \end{itemize}
    \item attribut du sujet
    \begin{itemize}
        \item Plus vieux vous serez, moins riche vous serez.
    \end{itemize}
    \item attribut du complément 
    \begin{itemize}
        \item Plus vieux vous boirez ce vin, meilleur vous le trouverez.
    \end{itemize}
    \item complément invariable de verbe
    \begin{itemize}
        \item Plus jeune vous choisirez votre assurance décès, moins cher vous la paierez.
    \end{itemize}
\end{enumerate}



\subsubsection{subordonnée exclamative}





\subsubsection{subordonnée concessive}
\begin{enumerate}
    \item si / pour / tout / aussi + adjectif + que / sujet suffixé 
    \item 从句大部分用subjonctif,但与tout连用时用indicatif
\end{enumerate}
\begin{itemize}
    \item si futés que nous soyons, nous nous laissons encore prendre à ça !
    \item Pour isolée et presque sauvage que fût leur existence, la grand-mère et sa petite fille ne pouvaient éviter un certain nombre de contacts et même de relations.
    \item Si beau soit-il, je préfère tout de même son frère.
    \item Tout habile qu’il est, il ne me convainc pas.
    \item Aussi sympathique que tu l’aies jugé, méfie-toi quand même de lui.
\end{itemize}

\subsection{périphérique}
该结构中的adjectif只作为ajout de sujet,repris par la proforme le.
\subsubsection{en début de phrase }
\begin{itemize}
    \item Content, il l’était en effet, au point d’être redevenu beau malgré son visage brûlé
    \item Inquiet de tout, Jules l’est.
    \item Inquiet, Jules l’est de tout.
\end{itemize}
\subsubsection{en fin de phrase }
\begin{itemize}
    \item Jules l’est, cordial.
\end{itemize}

\chapter{Proforme}
\section{Pronom et Proforme}
\subsection{La forme des pronoms}
\begin{longtable}{|>{\raggedright\arraybackslash}p{4cm}|>{\raggedright\arraybackslash}p{5cm}|>{\raggedright\arraybackslash}p{6cm}|}
\hline
\rowcolor{cyan!20}
\textbf{PRONOMS} & \textbf{SIMPLES} & \textbf{COMPLEXES} \\
\hline
\textbf{de choix libre et concessifs} & \textit{quiconque, quoi} & \textit{n’importe lequel, n’importe qui, n’importe quoi, qui que ce soit, quoi que ce soit} \\
\hline
\textbf{démonstratifs} & \textit{ça, ce, ceci, cela, celui} & \textit{celui-ci, celui-là, ce dernier} \\
\hline
\textbf{indéfinis} & \textit{autrui, on, qui, soi, tel, untel} & 
\textit{autre chose, autre part, d’aucuns, Dieu sait qui, Dieu sait quoi, Dieu sait lequel, grand-chose, grand monde, je ne sais lequel, je ne sais qui, je ne sais quoi, l’on, l’un, on ne sait lequel, on ne sait qui, on ne sait quoi, quelqu’un, quelque chose, quelque part, quelques-uns} \\
\hline
\textbf{interrogatifs} & \textit{lequel, que, qui, quid, quoi} & \textit{qui est-ce que, qui est-ce qui, qu’est-ce que, qu’est-ce qui} \\
\hline
\textbf{personnels} & \textit{je, tu, il, ils, elle, elles, nous, vous, moi, toi, lui, eux} & 
\textit{moi-même, toi-même, elle-même, lui-même, nous-mêmes, vous-mêmes, eux-mêmes, elles-mêmes, \% nous autres, \% vous autres, \% eux-autres} \\
\hline
\textbf{quantifieurs} & \textit{chacun, personne, rien, tous, tout} & \textit{nulle part, tout le monde} \\
\hline
\textbf{relatifs} & \textit{lequel, qui, quoi} & — \\
\hline
\textbf{relatifs sans antécédent} & \textit{qui, quiconque, quoi} & — \\
\hline
\textbf{temporels} & \textit{aujourd’hui, demain, hier} & \textit{avant-hier, après-demain} \\
\hline
\caption{ Les pronoms simples et complexes}
\end{longtable}

\subsubsection{La distinction entre pronoms et déterminants}

\begin{table}[H]
\centering
\begin{tabular}{|l|l|l|}
\hline
\rowcolor{cyan!20}
\textbf{} & \textbf{DÉTERMINANT} & \textbf{PRONOM} \\ \hline
démonstratif & \textit{ce} & \textit{ceci, cela, celui-ci, celui-là} \\ \hline
indéfini & \textit{quelque} & \textit{quelqu’un, quelque chose} \\ \hline
interrogatif & \textit{quel} & \textit{lequel} \\ \hline
quantifieur & \textit{chaque} & \textit{chacun} \\ \hline
\end{tabular}

\end{table}

\subsection{La variation des pronoms}
\begin{table}[H]
\centering
\small
\renewcommand{\arraystretch}{1.4}
\begin{tabular}{|>{\centering\arraybackslash}m{3.5cm}|>{\centering\arraybackslash}m{2.5cm}|>{\centering\arraybackslash}m{2.5cm}|>{\centering\arraybackslash}m{2.5cm}|>{\centering\arraybackslash}m{2.5cm}|}
\hline
\rowcolor{cyan!20}
\textbf{PRONOMS} & \multicolumn{2}{c|}{\textbf{SINGULIER}} & \multicolumn{2}{c|}{\textbf{PLURIEL}} \\
\hline
& \textbf{féminin} & \textbf{masculin} & \textbf{féminin} & \textbf{masculin} \\
\hline
\textbf{de choix libre} 
& \textit{n’importe laquelle} 
& \textit{n’importe lequel} 
& \textit{n’importe lesquelles} 
& \textit{n’importe lesquels} \\
\hline
\textbf{démonstratifs} 
& \textit{celle} 
& \textit{celui} 
& \textit{celles} 
& \textit{ceux} \\
\cline{2-5}
& \textit{cette dernière} 
& \textit{ce dernier} 
& \textit{ces dernières} 
& \textit{ces derniers} \\
\hline
\textbf{indéfinis} 
& \textit{telle} 
& \textit{tel} 
& \textit{telles} 
& \textit{tels} \\
\cline{2-5}
& \textit{l’une} 
& \textit{l’un} 
& \textit{les unes} 
& \textit{les uns} \\
\cline{2-5}
& \textit{je ne sais laquelle} 
& \textit{je ne sais lequel} 
& \textit{je ne sais lesquelles} 
& \textit{je ne sais lesquels} \\
\cline{2-5}
& \textit{on ne sait laquelle} 
& \textit{on ne sait lequel} 
& \textit{on ne sait lesquelles} 
& \textit{on ne sait lesquels} \\
\cline{2-5}
& — & — 
& \textit{quelques-unes} 
& \textit{quelques-uns} \\
\cline{2-5}
& \textit{Unetelle, une telle} 
& \textit{Untel, un tel} 
& — & — \\
\hline
\textbf{interrogatif, relatif} 
& \textit{laquelle} 
& \textit{lequel} 
& \textit{lesquelles} 
& \textit{lesquels} \\
\hline
\textbf{personnels} 
& \textit{elle} 
& \textit{il} 
& \textit{elles} 
& \textit{ils} \\
\cline{2-5}
& \textit{elle} 
& \textit{lui} 
& \textit{elles} 
& \textit{eux} \\
\hline
\textbf{quantifieurs} 
& \textit{chacune} 
& \textit{chacun} 
& — & — \\
\cline{2-5}
& — & — 
& \textit{toutes} 
& \textit{tous} \\
\hline
\end{tabular}
\caption{La variation des pronoms en genre et nombre}
\end{table}

\newpage
\subsection{Les pronoms animés ou inanimés}
\begin{longtable}{|>{\raggedright\arraybackslash}p{3cm}|
                    >{\raggedright\arraybackslash}p{3.5cm}|
                    >{\raggedright\arraybackslash}p{3.5cm}|
                    >{\raggedright\arraybackslash}p{3.5cm}|}
\hline
\rowcolor{cyan!20}
\textbf{PRONOMS} & \textbf{ANIMÉ} & \textbf{INANIMÉ} & \textbf{NON RESTREINT} \\
\hline
\textbf{démonstratifs} & --- & \textit{ça, ce, ceci, cela} & \textit{celui, celui-ci, celui-là, ce dernier} \\
\hline
\rowcolor{white}
\textbf{indéfinis} & 
\textit{autrui, d’aucuns, Dieu sait qui,} \newline 
\textit{grand monde, je ne sais qui, l’on, on,} \newline
\textit{on ne sait qui, quelqu’un, qui, soi,} \newline
\textit{soi-même, untel} &
\textit{autre chose, autre part, Dieu sait quoi,} \newline 
\textit{grand-chose, je ne sais quoi,} \newline 
\textit{on ne sait quoi, quelque chose,} \newline 
\textit{quelque part} &
\textit{Dieu sait lequel, je ne sais lequel,} \newline 
\textit{les uns, l’un, on ne sait lequel,} \newline 
\textit{quelques-uns, tel} \\
\hline
\textbf{personnels} & 
\textit{je, moi, tu, toi, nous, vous} \newline 
\textit{moi-même, toi-même, nous-mêmes,} \newline 
\textit{vous-mêmes} & --- &
\textit{il, elle, lui, ils, elles, eux} \\
\hline
\rowcolor{white}
\textbf{quantifieurs, concessifs} \newline 
\textbf{et de choix libre} &
\textit{personne, tout le monde ;} \newline 
\textit{n’importe qui, quiconque,} \newline 
\textit{qui que ce soit} &
\textit{nulle part, rien, tout ;} \newline 
\textit{n’importe quoi, quoi, quoi que ce soit} &
\textit{chacun, tous ;} \newline 
\textit{n’importe lequel} \\
\hline
\textbf{interrogatifs, relatifs,} \newline 
\textbf{relatifs sans antécédent} &
\textit{qui, qui est-ce qui, qui est-ce que,} \newline 
\textit{quiconque} &
\textit{que, quid, qu’est-ce qui,} \newline 
\textit{qu’est-ce que, quoi} &
\textit{lequel} \\
\hline
\caption{ Les pronoms animés, inanimés ou non restreints}
\end{longtable}
\begin{itemize}
    \item Les pronoms qui réfèrent seulement à un animé, ou seulement à un inanimé, sont généralement \textbf{invariables}
\end{itemize}

\subsection{forme forte et forme faible}

\begin{table}[H]
    \small
\centering
\renewcommand{\arraystretch}{1.5}
\begin{tabular}{|>{\bfseries}l|l|l|l|}
\hline
\rowcolor{cyan!20}
\textbf{PRONOMS} & \textbf{FORTS} & \textbf{FAIBLES} & \textbf{FORMES INDIFFÉRENCIÉES} \\
\hline
indéfinis & \textit{soi} & \textit{on} & \textit{---} \\
\hline
démonstratifs & \textit{ceci, cela, celui, ça} & \textit{ce} & \textit{---} \\
\hline
interrogatifs & \textit{lequel, qui, quoi} & \textit{que} & \textit{---} \\
\hline
personnels & 
\begin{tabular}{@{}l@{}}
\textit{moi, toi, lui, eux,} \\
\textit{lui-même, moi-même,} \\
\textit{toi-même, elle-même,} \\
\textit{nous-mêmes, vous-mêmes,} \\
\textit{eux-mêmes, elles-mêmes}
\end{tabular}
& 
\begin{tabular}{@{}l@{}}
\textit{je, tu, il, ils}
\end{tabular} 
& 
\begin{tabular}{@{}l@{}}
\textit{nous, vous,} \\
\textit{elle, elles}
\end{tabular} \\
\hline
\end{tabular}
\caption{Les pronoms forts et les pronoms faibles}
\end{table}


\subsection{Proforme}
\begin{table}[H]
\small
\centering
\begin{tabular}{|p{3cm}|p{5cm}|p{5cm}|}
\hline
\rowcolor{cyan!20}
\textbf{CATÉGORIE} & \textbf{EXPRESSION} & \textbf{EXEMPLES} \\ \hline
proforme préfixée ou suffixée & \textit{me, te, nous, vous, le, la, les, lui, leur, se, y, en} & Paul \textit{nous en} parlera. Regarde-\textit{les} ! \\ \hline
\textbf{DÉTERMINANT} & \textit{Dieu sait quel, je ne sais quel, lequel (relatif), n'importe quel, on ne sait quel, quel, quelque, mon, ton, son, notre, votre, leur} & \textit{mon} chat, J'ai trouvé un notaire, \textit{lequel} notaire habite loin. \textit{Quel} chat ? \textit{Quel} chat ! \\ \hline
\textbf{ADJECTIF} & \textit{quel, tel, tel quel} & \textit{Quel} est ton nom ? \\ \hline
\textbf{ADVERBE} & \textit{ainsi, alors, combien, comme, comment, Dieu sait comment, Dieu sait quand, je ne sais comment, je ne sais quand, n'importe comment, n'importe quand, non, on ne sait quand, oui, pourquoi, quand, que (exclamatif, comparatif), si} & \textit{Comme} tu es belle ! \textit{Comment} t'appelles-tu ? \\ \hline
\textbf{PRÉPOSITION} & \textit{auquel, duquel, ici, là, là-bas, où, -ci, -là, Dieu sait où, je ne sais où, n'importe où, on ne sait où, où que ce soit, partout} & \textit{Duquel} parles-tu ? Viens \textit{ici} ! \textit{Où} vas-tu ? \\ \hline
proforme verbale & le faire, faire de même, en faire autant & Paul va dormir et je vais \textit{en faire autant}.  \\ \hline
\textbf{SYNTAGME NOMINAL} sans nom & & Je sien, les autres, un autre, certains \\ \hline
\end{tabular}
\label{tab:grammatical}
\caption{Les principales proformes (hors pronoms)}
\end{table}

\subsection{Se et les verbes réfléchis}





\section{Les proformes personnelles faibles}
\begin{table}[H]
\centering
\renewcommand{\arraystretch}{1.5}
\begin{tabular}{|>{\bfseries}l|>{\itshape}l|>{\itshape}l|}
\hline
\rowcolor{cyan!20}
\textbf{PERSONNE} & \textbf{SINGULIER} & \textbf{PLURIEL} \\
\hline
1° & je, me & nous \\
\hline
2° & tu, te & vous \\
\hline
3° & 
\begin{tabular}{@{}>{\itshape}l@{}}
fém. elle, la \\
masc. il, le \\
se, lui, en, y
\end{tabular} 
& 
\begin{tabular}{@{}>{\itshape}l@{}}
fém. elles \\
masc. ils \\
les, leur
\end{tabular} \\
\hline
\end{tabular}

\end{table}

\subsection{General}
\subsubsection{La distinction entre forme faible et forme forte}
\begin{enumerate}
    \item 弱形式不能单独使用
    \begin{itemize}
        \item Qui a coupé l’eau ? — * Il!
        \item Qui a coupé l’eau ? — Lui !
    \end{itemize}
    \item 弱形式不能用于连词间
    \begin{itemize}
        \item Lui et moi dinerons ensemble
        \item * Pierre et je dinerons ensemble
    \end{itemize}
    \begin{enumerate}
        \item 但可以使用il ou elle, le ou la
    \end{enumerate}
    \item 弱形式不能被修饰
    \begin{itemize}
        \item * Tu qui as une belle voix, lis-nous ce texte
        \item Toi qui as une belle voix, lis-nous ce texte
        \item *Il seul peut réussir.
        \item Lui seul peut réussir
    \end{itemize}
\end{enumerate}

\subsubsection{Proformes personnelles faibles et formes verbales}
\begin{enumerate}
    \item indicatif ou subjonctif中,弱形式总置于动词前
    \begin{enumerate}
        \item 弱形式与动词间不能插入其他单词
        \begin{itemize}
            \item * Il, évidemment, viendra
            \item * Il ne, je crois, viendra pas
            \item Paul, je crois, viendra
        \end{itemize}
        \item 弱形式主语也可置于动词后,通过连字符(动词以元音结尾时加t)
        \begin{itemize}
            \item Paul avait besoin d’un espoir, fût-il ténu.
            \item Viendras-tu
            \item Viendra-t-il
            \item Ainsi soit-il
            \item Puisse-t-il vous aider
        \end{itemize}
        \item 弱形式需置于participe présent前
        \begin{itemize}
            \item Paul a sursauté en me voyant
            \item * Paul a sursauté en me, je crois, voyant
        \end{itemize}
        \item 弱形式不能置于participes passés ou passifs前
        \begin{itemize}
            \item Paul les a vus/*a les vus
            \item Sorti de prison / * En sorti, Jim avait refait sa vie
            \item Le livre lui est dédié/*est lui dédié
        \end{itemize}
    \end{enumerate}
    \item 弱形式需置于infinitif前,但可被特殊副词bien, mieux, pas, rien分开
    \begin{itemize}
        \item Cette simultanéité, à y bien réfléchir, n’est qu’un autre nom pour la mutualité amoureuse
        \item La finaude entrait dans tous tes soupçons, épousait tes mauvaises querelles pour te mieux tenir
    \end{itemize}
    \item 弱形式在impératif中
    \begin{enumerate}
        \item 肯定式需通过连字符置于动词后
        \begin{itemize}
            \item Prends-le en main
            \item Donne-le-lui
            \item Donne-m’en aussi un brin, Muller
        \end{itemize}
        \item 否定式需与ne一起置于动词前
        \begin{itemize}
            \item Ne le prends pas !
            \item * Ne prends-le pas !
            \item Ne lui donne pas !
            \item * Ne donne-lui pas !
        \end{itemize}
        \begin{enumerate}
            \item 口语中,ne会省略,弱形式可置于动词前后
            \begin{itemize}
                \item Le prends pas 
                \item Lui donne pas
                \item Dis-moi-le pas
            \end{itemize}
        \end{enumerate}
    \end{enumerate}
\end{enumerate}

\subsection{Le pronom personnel faible sujet}

\subsubsection{Le pronom faible sujet préverbal}
\begin{enumerate}
    \item 第一二人称代词在coordination de verbe中通常要重复出现
    \begin{itemize}
        \item Tu rentres ou tu sors ?
        \item ? Tu rentres ou sors ?
    \end{itemize}
    \item 第三人称代词avec le passé simple在coordination de verbe中可省略
    \begin{itemize}
        \item Il pâlit et serra les dents, essaya de se lever, eut un malaise, un frisson.
        \item Il regarda la carte et changea une nouvelle fois de ton.
    \end{itemize}
\end{enumerate}

\subsubsection{Les proformes sujets suffixées au verbe}
加连字符,动词以元音结尾时加-t-
\begin{enumerate}
    \item proformes sujets suffixées出现在其他sujet (nominal ou pronominal) inversé不能出现时
    \begin{itemize}
        \item * Viendra Paul ?
        \item * Viendra lui?
        \item Paul viendra-t-il ?
    \end{itemize}
    \item 在coordination de verbe中必须要重复出现
    \begin{itemize}
        \item Appelleront-ils ou écriront-ils ?
        \item * Appelleront ou écriront-ils ?
    \end{itemize}
\end{enumerate}




\subsection{L’ordre}




\subsubsection{L’ordre des proformes faibles préverbales}
\begin{table}[H]
\centering
\renewcommand{\arraystretch}{1.5}
\begin{tabular}{|>{\itshape}c|>{\itshape}l|>{\itshape}l|>{\itshape}l|>{\itshape}l|l|}
\hline
\textbf{I} & \textbf{II} & \textbf{III} & \textbf{IV} & \textbf{V} &  \\
\hline
me, te & le, la & lui & y & en & verb \\
nous, vous & les & leur & & & \\
se & & & & & \\
\cline{1-5}
\hline
\end{tabular}
\end{table}

\begin{enumerate}
    \item 每一列在动词前只能出现一个proforme
    \begin{itemize}
        \item *Paul me vous présenter
        \item Paul me présentera à vous
        \item *Paul se vous présentera
        \item Paul se présentera à vous
    \end{itemize}
    \item le指代complément direct时是variable,指代attribut du sujet是invariable
    \begin{itemize}
        \item Paul nous donne son sac.|Paul nous le donne.
        \item Paul nous semble astucieux.|Paul nous le semble
    \end{itemize}
    \begin{enumerate}
        \item le指代complément direct时可与y或en连用,但指代attribut du sujet时不行
        \begin{itemize}
            \item Nous amènerons Paul à la gare.|Nous l’y amènerons
            \item Paul est très actif à Paris.|*Paul l’y est.
        \end{itemize}
    \end{enumerate}
    \item 第一列与第三列不能同时出现
    \begin{itemize}
        \item Paul se présentera à eux.|*Paul se leur présentera
        \item Marie me présentera à lui.|*Marie me lui présentera
    \end{itemize}
\end{enumerate}

\subsubsection{L’ordre des proformes faibles à l’impératif}
\begin{table}[H]
\centering
\begin{tabular}{|l|l|l|l|l|}
\hline
\textbf{} & \textbf{I} & \textbf{II} & \textbf{III} & \textbf{IV} \\
\hline
verbe impératif & \textit{-le, -la, -l'} & \textit{-moi, -me} & \textit{-y} & \textit{-en} \\
& \textit{-les} & \textit{-toi, -te} & & \\
& & \textit{-lui, -leur} & & \\
& & \textit{-nous, -vous} & & \\
\hline
\end{tabular}
\end{table}

\begin{enumerate}
    \item 每一列只能出现一个proforme
    \begin{itemize}
        \item * Présentez-moi-vous
        \item Présentez-vous à moi
        \item * Présente-moi-lui 
        \item Présente-moi à lui
    \end{itemize}
\end{enumerate}

\subsection{en}
\subsubsection{connecteur}
en表示 la situation résultante =de ce fait, du coup

\begin{enumerate}
    \item par une coordonnée
    \begin{itemize}
        \item Paul était sur les lieux de l’explosion, et il en est resté sourd
    \end{itemize}
    \item par un ajout au participe présent
    \begin{itemize}
        \item Ayant vécu cette triste expérience, Léa en est devenue hostile à tout changement
    \end{itemize}
    \item par la phrase précédente
    \begin{itemize}
        \item C’est en traversant le pont que je lui ai raconté l’histoire. J’en ai oublié de lui recommander de regarder l’eau grise
    \end{itemize}
    \item par une séquence de plusieurs phrases
    \begin{itemize}
        \item Tout le monde la crut morte. Mais non, elle ronflait. Elle avait dit ce qu’elle avait à dire et s’était endormie. Étrange grand-mère. Clara en resta muette
    \end{itemize}
\end{enumerate}

\subsubsection{Les expressions figées}
\begin{table}[H]
\centering
\begin{tabular}{|>{\RaggedRight\arraybackslash}m{4cm}|>{\RaggedRight\arraybackslash}m{9cm}|} % Adjust column widths as needed
\hline
\multicolumn{2}{|c|}{\textbf{EXPRESSION VERBALE}} \\
\hline
avec un attribut & \textit{n'en est pas moins + adj., n'en être que plus + adj., en rester + adj.} \\
\hline
avec un complément en \textit{à} & \textit{en coûter (à), en faire accroire (à), en imposer (à), s'en prendre (à), en référer (à), s'en référer (à), s'en tenir (à), en vouloir (à)} \\
\hline
avec un complément en \textit{de} & \textit{c'en est fait (de), en avoir assez (de), en avoir marre (de)} \\
\hline
avec un autre complément & \textit{il s'en faut de beaucoup / de peu (que), s'en aller, en dire de belles (sur), en apprendre de belles / de drôles (sur), en finir (avec), s'en retourner, s'en faire (pour)} \\
\hline
avec un complément figé & \textit{en faire de belles, en faire de drôles, en prendre à son aise, n'en rien faire, ne pas en mener large, en rester là, en voir des vertes et des pas mûres} \\
\hline
entièrement figée & \textit{En veux-tu, en voilà.} \\
& \textit{C'en est trop.} \\
& \textit{Il en est / va ainsi.} \\
& \textit{Il en est / va de même.} \\
\hline
\end{tabular}
\caption{Quelques expressions verbales figées comportant en
}
\end{table}


\subsubsection{Le complément nominal}
\begin{enumerate}
    \item en指代complément nominal indéfini,由de, des, du de la引导
    \begin{enumerate}
        \item un complément direct
        \begin{itemize}
            \item Paul achètera du pain.|Paul en achètera
            \item Paul prend de la farine.|Paul en prend
            \item Paul achète de belles pommes.|Paul en achète
            \item La fenêtre a couté 300 euros et le volet en a couté 200
        \end{itemize}
        \item un attribut du sujet
        \begin{itemize}
            \item On cherchait des médecins et ils en étaient bien
        \end{itemize}
        \item impersonnel,可看作complément direct
        \begin{itemize}
            \item Il arrive des accidents. Il en arrive
            \item Il arrive souvent des accidents à ce carrefour. Il en est arrivé hier.
        \end{itemize}
        \item 由定冠词le, la, les引导时,不用en
        \begin{itemize}
            \item Paul achètera ce livre.|Paul l’achètera
            \item Paul prend la farine.|Paul la prend.
            \item Paul a lu les romans de Balzac.|Paul les a lus
        \end{itemize}
    \end{enumerate}
    \item en动词后可跟随quantité
    \begin{enumerate}
        \item un déterminant
        \begin{itemize}
            \item Paul en a étudié plusieurs
        \end{itemize}
        \item un adjectif cardinal
        \begin{itemize}
            \item Paul en a lu deux
        \end{itemize}
        \item un adverbe
        \begin{itemize}
            \item Paul en a acheté beaucoup
        \end{itemize}
        \item un syntagme nominal
        \begin{itemize}
            \item Paul en a une dizaine
        \end{itemize}
    \end{enumerate}
    \item en动词后可跟随modifieur
    \begin{enumerate}
        \item un syntagme prépositionnel
        \begin{itemize}
            \item Des chaussettes J’en ai acheté en soie et en coton
        \end{itemize}
        \item une subordonnée relative
        \begin{itemize}
            \item Peu de gens étaient disponibles mais on en a rencontré qui étaient prêts à nous aider
            \item 
        \end{itemize}
        \item un adjectif épithète  ou apposé
        \begin{itemize}
            \item À propos de sole J’en ai acheté une belle
            \item J’avais déjà une place et j’en ai trouvé deux autres
        \end{itemize}
    \end{enumerate}
\end{enumerate}

\subsubsection{Le complément oblique}
\begin{enumerate}
    \item \textbf{complément oblique d’un verbe}
    \begin{enumerate}
        \item un complément prépositionnel introduit par de
        \begin{itemize}
            \item Il en parle à tout le monde, de ses voyages
            \item Paul en est, de mes amis/de la partie
            \item On en a reçu ce livre, de la bibliothèque
        \end{itemize}
        \begin{enumerate}
            \item de引导日期、尺度时,不能被en替代
            \begin{itemize}
                \item Ces chiffres datent du mois de juillet.|*Ces chiffres en datent
                \item On a déplacé la table de 3 centimètres.|*On en a déplacé la table.
            \end{itemize}
        \end{enumerate}
        \item un complément infinitif introduit par de(infinitf必须作为complément oblique)
        \begin{itemize}
            \item On ne se souvenait pas d’avoir étudié ce texte.|On ne s’en souvenait pas
            \item Paul rêve de partir.|Paul en rêve
            \item Paul nous a promis de venir demain.|*Paul nous en a promis
            \item Il craint d’être licencié.|*Il en craint.
            \item Paul nous l’a promis, de venir demain
        \end{itemize}
        \item une subordonnée introduite par que ou de ce que((从句必须作为complément oblique)
        \begin{itemize}
            \item Paul rêve que son frère le rejoigne.|Paul en rêve
            \item Nous craignons que Paul parte.|Nous le craignons.|*Nous en craignons
        \end{itemize}
        \item 宾语从句中en有时可指代主句的无生命主语,主句主语有生命时用lui
        \begin{itemize}
            \item Cette maison a besoin qu’on s’en occupe.
            \item Cette question mérite qu’on en discute
            \item Émile croit que Sophie veut s’en débarrasser. en ≠ Émile
            \item Émile craint qu’on parle de lui
        \end{itemize}
    \end{enumerate}
    \item \textbf{complément oblique d’un adjectif}:de
    \begin{enumerate}
        \item le complément d’un adjectif attribut
        \begin{itemize}
            \item Elle en est fière, de ce tableau.
            \item Elle en est satisfaite, de partir en Chine.
            \item Elle en sera heureuse, que vous l’invitiez
        \end{itemize}
        \item le complément d’un attribut non adjectival
        \begin{itemize}
            \item Paul est maire de ce village depuis vingt ans.|Paul en est maire depuis vingt ans
            \item Joseph est le gardien de notre immeuble.|Joseph en est le gardien
        \end{itemize}
    \end{enumerate}
    \item \textbf{un complément de nom}:de
    \begin{itemize}
        \item Le bateau Cette statue en ornait sans doute la proue
        \item Ces livres Il en a lu à peu près la moitié
        \item Il est question de légaliser certaines drogues mais on continuera d’en surveiller l’usage.
    \end{itemize}
    \begin{enumerate}
        \item 表示origin和approximation (genre de, sorte de, type de)的短语不能被en替代
        \begin{itemize}
            \item La plage * Il en organisera le retour dès 18 heures
            \item J’ai pris une espèce de sac.|*J’en ai pris une espèce
            \item J’ai acheté une proue de bateau.|*J’en ai acheté une proue
        \end{itemize}
        \item 可被en替代的名词也可被dont替代
        \begin{itemize}
            \item Voici le bateau dont cette statue ornait la proue
            \item Voici les livres dont j’ai lu la moitié.
        \end{itemize}
        \item \textbf{le complément du sujet nominal}
        \begin{enumerate}
            \item le sujet est un nom de partie
            \begin{itemize}
                \item (La table) Le pied en est cassé
                \item (Ce village) Son maire aide tout le monde
            \end{itemize}
            \item le sujet n’est pas agentif
            \begin{itemize}
                \item J’ai malheureusement oublié le détail de cette épopée paternelle. Le sujet n’en changera jamais
            \end{itemize}
            \item la phrase ne décrit pas un évènement, mais une propriété
            \begin{itemize}
                \item (Une bière) Le gout en a surpris mes amis
                \item (Le putois) L’odeur en est insupportable
            \end{itemize}
            \item en指代devoir, pouvoir ou sembler这些直接接不定式的动词的主语时,en放在不定式动词前
            \begin{itemize}
                \item Il n’est pas possible d’opérer n’importe quel transfert de techniques d’un champ de recherche à un autre : l’utilisation peut en être illusoire
                \item S’il existe des contenus communs, la raison doit en être cherchée soit du côté des propriétés objectives de certains êtres naturels ou artificiels, soit du côté de la diffusion et de l’emprunt, c’est-à-dire, dans les deux cas, hors de l’esprit
                \item La raison semble en être que l’opinion turque considère trop restreint le cercle des parents auxquels le Code accorde une réserve
            \end{itemize}
        \end{enumerate}
    \end{enumerate}
\end{enumerate}

\subsubsection{le déterminant possessif}
\begin{enumerate}
    \item son用于complément humain
    \begin{itemize}
        \item Paul, je connais son père. ? J’en connais le père
    \end{itemize}
    \item 只有en用于un infinitif ou une subordonnée
    \begin{itemize}
        \item Paul en a la volonté, de réussir.
    \end{itemize}
    \item 只有en用于le complément d’un nom de quantité ou de proportion
    \begin{itemize}
        \item Ces livres, j’en ai lu une dizaine
        \item Ces livres, j’en ai lu une dizaine. *J’ai lu leur dizaine
        \item Ce livre, j’en ai lu la moitié. * J’ai lu sa moitié
    \end{itemize}
\end{enumerate}



\subsection{lui et y}
\begin{enumerate}
    \item lui和y同时出现很少见,y多与第一二人称同时出现
    \begin{itemize}
        \item Il rentre à l’hôtel. * On doit lui y apporter des pierres remarquables
        \item Vous rentrez à l’hôtel. On doit vous y apporter des pierres remarquables
    \end{itemize}
\end{enumerate}

\subsubsection{lui}
lui varie en personne (me, te, se, lui) et en nombre (nous, vous, leur)
\begin{enumerate}
    \item complément en \emph{à} de certains verbes
    \begin{itemize}
        \item Paul lui parle
    \end{itemize}
    \item complément en \emph{à} de certains adjectifs
    \begin{itemize}
        \item Paul lui est fidèle
        \item Cependant, nous autres, les potes \emph{à} Pollak Henri, les sans-grades, les pékins, nous nous chargions d’arranger la chose
    \end{itemize}
    \item ajoutée comme un bénéficiaire
    \begin{enumerate}
        \item complément en de avec un nom de partie du corps
        \begin{itemize}
            \item Pierre a pris la main de Marie
            \item Pierre lui a pris la main
            \item Je lui serre la main
            \item Les larmes lui montent aux yeux
        \end{itemize}
        \item complément nominal après une préposition; prépositionnel locatif:après, autour, à travers, dedans, dessus, dessous, derrière, devant
        \begin{itemize}
            \item Paul lui a couru après
            \item Pierre est passé devant Marie
            \item Pierre lui est passé devant
            \item Pierre a réparé la porte pour Marie
            \item Pierre lui a réparé la port
        \end{itemize}
    \end{enumerate}
\end{enumerate}

\subsubsection{y}
\begin{enumerate}
    \item une expression de lieu
    \begin{itemize}
        \item Paul y habite, en Belgique
        \item Paul y dort souvent, chez sa sœur.
    \end{itemize}
    \begin{enumerate}
        \item là作为强形式则用于construction clivé和impératif中
        \begin{itemize}
            \item Il est parti en Irlande, parce que c’est là qu’il se sent bien
        \end{itemize}
    \end{enumerate}
    \item complément en \emph{à} d’un verbe non locatif 
    \begin{enumerate}
        \item 只用y不用lui的动词
    \begin{table}[H]
    \small
    \centering
    \begin{tabular}{|>{\RaggedRight\arraybackslash}p{8cm}|>{\RaggedRight\arraybackslash}p{5cm}|}
    \hline
    \rowcolor{cyan!20}
    \textbf{Type de verbes} & \textbf{Exemples} \\ 
    \hline

    % 1. Aspectuels et modaux
    Aspectuels et modaux & \\
    \textit{s'accoutumer, se mettre, s'habituer} & Paul s'y habitue. \\
    \hline

    % 2. Décision et engagement
    Décision et engagement & \\
    \textit{s'attacher, se décider, s'engager, s'opposer, répugner, se refuser, se résoudre, veiller} & Paul s'y oppose. Paul s'y engage. \\
    \hline

    % 3. Désir et orientation
    Désir et orientation & \\
    \textit{s'intéresser, tenir, viser} & Paul y tient. \\
    \hline

    % 4. Essai
    Essai & \\
    \textit{arriver, hésiter, parvenir, recourir, réussir, se risquer} & Paul y réussit. \\
    \hline

    % 5. Incitation et influence
    Incitation et influence & \\
    \textit{aider, autoriser, contraindre, encourager, forcer, inciter, obliger} & Paul y encourage son fils. \\
    \hline

    % 6. Jugement et activité intellectuelle
    Jugement et activité intellectuelle & \\
    \textit{s'attendre, penser, réfléchir, songer} & Paul y songe. \\
    \hline
    \end{tabular}
    \caption{Les classes de verbes admettant y (hors compléments datifs ou locatifs)}
    \end{table}
        \item un infinitif introduit par à
    \begin{itemize}
        \item Paul y a pensé, à aller chez sa sœur
        \item Le directeur y rechigne, à donner certaines informations
    \end{itemize}
    \item un subordonnée introduite par à ce que
    \begin{itemize}
        \item Paul y tient, à ce que tout soit en ordre
        \item Y tenez-vous, à ce que les journaux soient distribués 
    \end{itemize}
    
    \end{enumerate}
\end{enumerate}

\section{Les pronoms personnels forts}
\begin{table}[H]
\centering
\begin{tabular}{|l|l|l|}
\hline
\rowcolor{cyan!20}
\textbf{PERSONNE} & \textbf{SINGULIER} & \textbf{PLURIEL} \\
\hline
1$^{\text{re}}$ pers. & \textit{moi} & \textit{nous} \\
\cline{2-3}
& \textit{moi-même} & \textit{nous-mêmes} \\
\hline
2$^{\text{e}}$ pers. & \textit{toi} & \textit{vous} \\
\cline{2-3}
& \textit{toi-même} & \textit{vous-mêmes} \\
\hline
3$^{\text{e}}$ pers. & \textit{elle, lui} & \textit{elles, eux} \\
\cline{2-3}
& \textit{elle-même, lui-même} & \textit{elles-mêmes, eux-mêmes} \\
\hline
\end{tabular}
\caption{Les pronoms personnels forts}
\end{table}

\subsection{Les fonctions syntaxiques}

\subsubsection{La complémentarité des pronoms forts et des formes faibles}
pronoms forts用于不能用formes faibles的情况
\begin{enumerate}
    \item compléments d’une préposition
    \begin{itemize}
        \item Marie travaille chez eux.
        \item Marie travaille avec moi.
    \end{itemize}
    \item ajout au verbe ou à la phrase
    \begin{itemize}
        \item Paul travaille bien, lui.
    \end{itemize}
    \item périphérique, liés à une autre proforme dans une construction disloquée
    \begin{itemize}
        \item Moi, j’ai froid.
    \end{itemize}
    \item être coordonnés
    \begin{itemize}
        \item Paul et moi sommes amis
        \item Eux et le maire sont amis
    \end{itemize}
\end{enumerate}

\subsubsection{Le pronom personnel fort sujet}
\begin{enumerate}
    \item verbe conjugué的主语不能用第一二人称的强形式,只能用第三人称的强形式
    \begin{itemize}
        \item Eux ne viendront pa
        \item Je viendrai. * Moi viendrai.
    \end{itemize}
    \begin{enumerate}
        \item 当verbe conjugué的主语代词被modifiés ou coordonnés时,不能用弱形式,此时第一二人称强形式可作主语
        \begin{itemize}
            \item Moi seul viendrai.
            \item Paul et moi viendrons
            \item Même vous devriez être capables de comprendre
        \end{itemize}
    \end{enumerate}
    \item infinitif, subordonnées au participe passé, impératif en fonction ajout的主语只能用强形式,不能用弱形式
    \begin{itemize}
        \item Moi, juger
        \item Et lui de répliquer
        \item Eux partis, le concert a pu commencer
        \item Toi, viens ici
    \end{itemize}
\end{enumerate}
\subsubsection{Le pronom personnel fort complément}
\begin{enumerate}
    \item 强形式不能充当动词的直接宾语,要用弱形式
    \begin{itemize}
        \item * Marie verra lui.
        \item * Marie aime moi
    \end{itemize}
    \begin{enumerate}
        \item 但avec le modifieur restrictif \emph{que} ou en cas de coordination时, 直接宾语不能用弱形式只能用强形式
        \begin{itemize}
            \item Marie n’aime que moi
            \item Marie aime son père et moi
        \end{itemize}
    \end{enumerate}
    \item complément d’une forme présentative 
    \begin{itemize}
        \item C’est moi qui viendrai
    \end{itemize}
    \item complément de préposition
    \begin{itemize}
        \item Marie travaille chez eux
        \item Marie travaille avec moi
    \end{itemize}
\end{enumerate}
\subsubsection{Le pronom personnel fort ajout ou périphérique}
\begin{enumerate}
    \item ajout à une phrase ou à un verbe( à un syntagme nominal ou à un nom propre)
    \begin{itemize}
        \item Vincent, lui, travaille bien
        \item Vincent va, lui, à Rome
        \item Vincent est, lui, surement parti.
        \item Vincent va à Rome, lui
    \end{itemize}
    \begin{enumerate}
        \item 作为ajuout时强形式不能置于句首
        \begin{itemize}
            \item * Lui, Vincent a déjà répondu
            \item Lui, Vincent, a déjà répondu
            \item Lui, il a déjà répondu (作为périphérique)
            \item Chez lui, Vincent se sent bien (作为complément de la préposition)
        \end{itemize}
    \end{enumerate}
    \item périphérique dans une construction disloquée
    \begin{enumerate}
        \item 指代un autre pronom fort
        \begin{itemize}
            \item Lui, j’ai déjà discuté avec lui
        \end{itemize}
        \item 指代un pronom faible
        \begin{itemize}
            \item Lui, il a déjà répondu
        \end{itemize}
        \item 指代une proforme faible
        \begin{itemize}
            \item Lui, je l’ai déjà vu.
        \end{itemize}
        \item 指代un déterminant possessif
        \begin{itemize}
            \item Lui, sa voiture est en panne
        \end{itemize}
        \item 指代ce, dans une phrase d’identité 
        \begin{itemize}
            \item Moi, c’est Vincent. ‘je m’appelle Vincent’
        \end{itemize}
        \item 可置于句首、动词后、句末
        \begin{itemize}
            \item Le professeur te verra, toi, mardi prochain
            \item Je viendrai lundi, moi
        \end{itemize}
    \end{enumerate}
\end{enumerate}


\subsection{L’interprétation des pronoms personnels forts}
\begin{enumerate}
    \item contrastif
    \begin{itemize}
        \item Marie a oublié de téléphoner mais heureusement son frère, lui, a appelé
        \item Paul, lui, a arrêté de fumer
    \end{itemize}
    \item réflexive,当强形式与其先行词在同一句子中时
    \begin{itemize}
        \item Jean a pris une photo de lui
    \end{itemize}
    \begin{enumerate}
        \item 当动词指向他者时,不表réflexive
        \begin{itemize}
            \item Jean hérite de lui. Jean ≠ lui
            \item Jean prend exemple sur lui. Jean ≠ lu
        \end{itemize}
        \item 被que修饰时,表réflexive
        \begin{itemize}
            \item Paul ne tient qu’à lui
            \item Paul ne prend exemple que sur lui
        \end{itemize}
        \item 复数形式表réciproque,尤其在entre后
        \begin{itemize}
            \item Les enfants pensent à eux
            \item Les enfants jouent entre eux
        \end{itemize}
    \end{enumerate}
\end{enumerate}
\begin{table}[H]
\centering
\begin{tabular}{|l|>{\RaggedRight\arraybackslash}p{6cm}|>{\RaggedRight\arraybackslash}p{6cm}|}
\hline
\rowcolor{cyan!20}
\textbf{PRÉPOSITION} & \textbf{INTERPRÉTATION RÉFLEXIVE POSSIBLE} & \textbf{INTERPRÉTATION RÉFLEXIVE DIFFICILE} \\
\hline
\textit{à} & \textit{être attentif, faire attention à} & \textit{être accroché, attaché, enchaîné, habitué à} \\
\hline
\textit{avec} & \textit{---} & \textit{être en accord, en contradiction, en harmonie, en rivalité avec} \\
\hline
\textit{contre} & \textit{---} & \textit{être en colère, en rage, furieux, révolté contre} \\
\hline
\textit{de} & \textit{avoir honte de, être content, fier, responsable, satisfait de, faire abstraction de, prendre soin de, tenir compte de} & \textit{avoir envie, besoin, horreur, peur, pitié, raison de, être amoureux, dépendant, différent, fatigué, proche, solidaire de} \\
\hline
\textit{devant} & \textit{---} & \textit{être admiratif, en admiration, en émerveillement, en extase devant} \\
\hline
\textit{en} & \textit{avoir confiance en} & \textit{---} \\
\hline
\textit{envers} & \textit{---} & \textit{avoir une dette, de l'indulgence envers, être bien disposé, indulgent envers, perdre son crédit envers} \\
\hline
\textit{par} & \textit{---} & \textit{être consterné, imité, obsédé, recherché par} \\
\hline
\textit{sur} & \textit{---} & \textit{mener une enquête sur, prendre exemple sur} \\
\hline
\end{tabular}
\caption{ L’interprétation réflexive du pronom fort lui après adjectif attribut ou nom + préposition}
\end{table}

\begin{table}[H]
    \centering
        \begin{tabular}{|l|>{\RaggedRight\arraybackslash}p{6cm}|>{\RaggedRight\arraybackslash}p{6cm}|}
        \hline
        \rowcolor{cyan!20}
        \textbf{PRÉPOSITION} & \textbf{INTERPRÉTATION RÉFLEXIVE POSSIBLE} & \textbf{INTERPRÉTATION RÉFLEXIVE DIFFICILE} \\
        \hline
        \textit{à} & \textit{en arriver à (\textnormal{idiom}.), s'intéresser à, penser à, revenir à (\textnormal{idiom}.), en venir à (\textnormal{idiom}.)} & \textit{s'adresser à, s'attaquer à, se heurter à, se montrer à, se recommander à, se soumettre à, s'en prendre à (\textnormal{idiom}.), s'en remettre à (\textnormal{idiom}.), tenir à (\textnormal{idiom}.)} \\
        \hline
        \textit{avec} & \textit{---} & \textit{collaborer avec, discuter avec, se disputer avec, dormir avec, s'entendre avec, s'entretenir avec, habiter avec, parler avec, rompre avec, sortir avec, travailler avec} \\
        \hline
        \textit{contre} & \textit{---} & \textit{s'acharner contre, conspirer contre, fulminer contre, s'insurger contre, lutter contre, se révolter contre, voter contre} \\
        \hline
        \textit{de} & \textit{douter de, parler de, se ficher de, s'occuper de, rêver de, se soucier de, se souvenir de} & \textit{s'amouracher de, se cacher de, se contenter de, se désintéresser de, hériter de, se méfier de, profiter de, se recommander de, se satisfaire de, se servir, triompher de} \\
        \hline
        \textit{en} & \textit{croire en, rentrer en (\textnormal{idiom}.)} & \textit{---} \\
        \hline
        \textit{hors de} & \textit{être hors de (\textnormal{idiom}.), mettre quelqu'un hors de (\textnormal{idiom}.)} & \textit{---} \\
        \hline
        \textit{malgré} & \textit{agir malgré, faire quelque chose malgré} & \textit{---} \\
        \hline
        \textit{par} & \textit{commencer par, finir par, terminer par} & \textit{---}\\
        \hline
        \textit{pour} & \textit{acheter quelque chose pour, garder ça pour (\textnormal{idiom}.), lutter pour, travailler pour, voter pour} & \textit{en pincer pour (\textnormal{idiom}.)} \\
        \hline
        \textit{sur} & \textit{se replier sur} & \textit{s'appuyer sur, compter sur, copier sur, ironiser sur, se jeter sur, miser sur, se reposer sur, tirer sur, se vautrer sur, veiller sur} \\
        \hline
        \textit{préposition de lieu} & \textit{regarder devant, derrière, sous, à côté de, au-dessus de, autour de, loin de; poser quelque chose sous, rester chez, serrer quelque chose contre} & \textit{lancer vers, regarder vers} \\
        \hline
        \end{tabular}
        \caption{L’interprétation réflexive du pronom fort lui après verbe + préposition}
\end{table}


\subsection{Les pronoms complexes -même}

\subsubsection{Les fonctions syntaxiques}
\begin{enumerate}
    \item 可作主语,但不能作为compléments directs或introduits par \emph{à} datif (sauf modification ou coordination),此时要用les formes faibles
    \begin{itemize}
        \item Moi-même irai là-bas dès que je peux
        \item * Le juge recevra nous-mêmes
        \item Le juge nous recevra.
        \item *Le juge a parlé à nous-mêmes.
        \item Le juge nous a parlé.
        \item * Nous avons rencontré lui-même, par hasard
        \item  Nous avons rencontré lui-même et personne d’autre
    \end{itemize}
    \item compléments de préposition
    \begin{itemize}
        \item L’avocat s’appuiera sur vous-mêmes
    \end{itemize}
    \item compléments de verbe d’identité
    \begin{itemize}
        \item Je suis resté moi-même
    \end{itemize}
    \item compléments d’un verbe présentatif
    \begin{itemize}
        \item C’est moi-même qui l’ai fait
    \end{itemize}
    \item ajout au verbe, à la phrase ou au syntagme nominal;可不加逗号
    \begin{itemize}
        \item Nous irons nous-mêmes parler au juge
        \item Alain Resnais a interviewé Jodie Foster lui-même
        \item Vas-tu l’interviewer lui-même
        \item Alain Resnais lui-même a voulu interviewer Jodie Foster
    \end{itemize}
    \begin{enumerate}
        \item 可用逗号分开,可置于句首,无périphérique形式
        \begin{itemize}
            \item Nous-mêmes, nous aurions préféré ne pas parler au juge
            \item Alain Resnais aurait préféré, lui-même, ne pas interviewer Jodie Foster
        \end{itemize}
    \end{enumerate}
\end{enumerate}
\subsubsection{interprétation contrastive}
\begin{enumerate}
    \item 常出现在construction clivée, modifié par seul ou (ne) que
    \begin{itemize}
        \item C’est [Alain Resnais lui-même] que Jodie Foster aurait voulu rencontrer
        \item Seul [Alain Resnais lui-même] serait capable de répondre à cette question
        \item Jodie Foster ne voulait rencontrer qu’Alain Resnais lui-même
    \end{itemize}
\end{enumerate}
\subsubsection{interprétation réflexive}
\begin{enumerate}
    \item complément de préposition
    \begin{itemize}
        \item Paul a confiance en lui-même
    \end{itemize}
    \item ajout au verbe réfléchi, coréférent à \textit{se}
    \begin{itemize}
        \item Paul se dénigre lui-même
    \end{itemize}
    \item 当动词指向他者时,用lui-même表réflexive(与pronom simple不同)
    \begin{itemize}
        \item En regardant ses premiers films, l’acteur vieillissant est jaloux de lui-même 
        \item En vieillissant, Jean parle surtout avec lui-même
        \item Par une manipulation juridique compliquée, Jean a hérité de lui-même
    \end{itemize}
\end{enumerate}


\subsection{Les pronoms personnels forts modifiés}
\begin{enumerate}
    \item par une subordonnée relative
    \begin{itemize}
        \item Il échappe deux fois à la mort, lui qui devait voyager sur les vols MH17 et MH370
    \end{itemize}
    \item par des adjectifs
    \begin{itemize}
        \item Lui malade, on ne peut rien faire
        \item Lui seul peut nous comprendr
    \end{itemize}
    \item par participes
    \item par noms apposés
    \begin{itemize}
        \item Moi président, il n’y aura plus de chômag
        \item Moi, Paul, promets de vous aider
    \end{itemize}
    \item par les adverbes aussi(在代词后), même(在代词前), et les adjectifs seul, tous
    \begin{itemize}
        \item Lui aussi connait la réponse
        \item Même eux connaissent la réponse
        \item Seul lui connait la réponse
        \item Eux tous connaissent la réponse.
    \end{itemize}
    \item 被aussi, même, seul, tous及subordonnée relative修饰过的强形式可作为主语(包括第一二人称)
    \item 被aussi, même, seul, tous及subordonnée relative修饰过的强形式可作为ajout;在句中的位置自由(但lui不能出现在主语前)
    \begin{itemize}
        \item Jean, lui aussi, connait la réponse
        \item Jean, lui seul, connait la réponse
        \item Paul connait, lui aussi, la réponse.
        \item Paul connait la réponse, lui aussi
        \item * Lui aussi, Paul connait la réponse
    \end{itemize}
\end{enumerate}

\section{Les pronoms indéfinis}
\subsection{On}
\begin{enumerate}
    \item 是弱形式
    \begin{itemize}
        \item * On souvent l’a vu
        \item * Pierre et on viendront te chercher
        \item * On qui a le temps viendra te chercher
    \end{itemize}
    \item 作为suffixé au verbe, 不能与un sujet préverbal连用
    \begin{itemize}
        \item Où va-t-on
        \item Quelqu’un vient-il
        \item * Quelqu’un vient-on
    \end{itemize}
    \item l'on 
    \begin{itemize}
        \item générique
        \item existentielle
        \item équivalente à nous
    \end{itemize}
    \item indéfini existentiel
    \begin{enumerate}
    \item On不能被il或son指代,只可以用se;而quelqu’un可被il或son指代
    \begin{itemize}
        \item On s'est introduit dans la maison pendant notre absence. \textit{Il} n'a rien cassé
        \item On t'a laissé \textit{son} vélo
        \item Quelqu’un s’est introduit dans la maison pendant notre absence. Il n’a rien cassé
    \end{itemize}
    \item indéfini générique
    \begin{enumerate}
        \item 反身代词用soi, soi-même, se
        \begin{itemize}
            \item On se reproche à soi-même ses propres erreurs.
            \item À cette époque-là, en cas de souci, on se confiait à un ami
        \end{itemize}
        \item on是主句主语,subordonnée complétive ou comparative用soi, vous, nous;subordonnée circonstancielle不用soi只用vous, nous
        \begin{itemize}
            \item En cas de grand désarroi, on espère toujours que quelque chose sera fait pour soi
            \item Quand on est parent, on a souvent l’impression que le temps nous dépasse. Les repas, les courses, les activités, les rendez-vous 
            \item On a toujours besoin d’un plus petit que soi
            \item Peut-on encore considérer quelqu’un comme un ami quand il vous a trahi
        \end{itemize}
    \end{enumerate}
    \item équivalent à \textit{nous}
    \begin{enumerate}
        \item 指代nous disloqué 或syntagme nominal disloqué comportant moi ou nous
        \begin{itemize}
            \item Nous, on s’en va
            \item Paul et moi, on s’en va
        \end{itemize}
        \item 作为nous, nous-mêmes, notre的先行词
        \begin{itemize}
            \item On sait depuis longtemps que notre survie à ce niveau passe par aller à Récy et à Charenton
            \item On ne s’occupe plus que de nous-mêmes
            \item On a eu notre heure de gloire
        \end{itemize}
        \item 可被se指代
        \begin{itemize}
            \item On se sent bien chez nous
        \end{itemize}
    \end{enumerate}
    \item équivalent de \textit{tu} ou \textit{vous}
    \begin{itemize}
        \item À un enfant qui se réveille. Alors, on a bien dormi ?
        \item Le kinésithérapeute à la patiente — On s’allonge sur la table
        \item Sale traître à la bourgeoisie, répondit Gabin d’un même rire, on est contente
    \end{itemize}
\end{enumerate}
\end{enumerate}

\subsection{autrui, soi, soi-même}
\begin{itemize}
    \item aller de soi(理所当然), rester sur son quant-à-soi(刻意保持疏离), chacun pour soi(人人为己), en soi, à soi tout seul(独自一人)
    \item Ne fais pas à autrui ce que tu ne veux pas qu’on te fasse à toi-même:己所不欲勿施于人
\end{itemize}
\subsubsection{autrui}
\begin{enumerate}
    \item sujet
    \begin{itemize}
        \item autrui est tour à tour craint, convoité, inaccessible
    \end{itemize}
    \item complément direct
    \begin{itemize}
        \item Rencontrer autrui, cela suppose donc d’une part la vie en communauté
    \end{itemize}
    \item complément de préposition
    \begin{itemize}
        \item Suis-je responsable pour autrui
    \end{itemize}
    \item périphérique disloqué repris par \textit{ce}
    \begin{itemize}
        \item Autrui, c’est d’abord l’autre, le différent
    \end{itemize}
    \item autrui与autre不同,autrui无先行词
\end{enumerate}

\subsubsection{soi}
有先行词
\begin{enumerate}
    \item complément de préposition
    \begin{itemize}
        \item Chacun reste chez soi, bien assis sur les faillites annoncées et confirmées
    \end{itemize}
    \item complément de nom
    \begin{itemize}
        \item la connaissance de soi, la conscience de soi
    \end{itemize}
    \item attribut
    \begin{itemize}
        \item C’est au contact d’autrui que l’on devient soi
    \end{itemize}
    \item 只有当不能用se时soi才作为complément direct出现,如被que修饰
    \begin{itemize}
        \item On a tendance à n’écouter que soi
        \item * On a tendance à écouter soi d’abord
    \end{itemize}
    \item 不能作为主语
    \begin{itemize}
        \item * On pense toujours que soi est le meilleur candidat
    \end{itemize}
\end{enumerate}

\subsubsection{soi-même}
\begin{enumerate}
    \item attribut du sujet
    \begin{itemize}
        \item C’est le genre de choses qu’on n’a pas envie de faire soi-même
    \end{itemize}
    \item complément de préposition
    \begin{itemize}
        \item Comment devenir plus gentil avec soi-même
    \end{itemize}
    \item ajout
    \begin{itemize}
        \item C’est le genre de choses qu’on n’a pas envie de faire soi-même
    \end{itemize}
    \item disloqué, repris par on, en début de phrase
    \begin{itemize}
        \item Soi-même, on fait du meilleur travail quand on a confiance en soi
    \end{itemize}
    \item 不能作为主语
\end{enumerate}

\subsection{Les pronoms indéfinis agglomérés}
\begin{enumerate}
    \item sujet
    \item complément direct
    \item attribut
    \item complément de préposition
    \item autre part et quelque part:compléments obliques ou ajouts
\end{enumerate}

\subsubsection{autre chose, autre part, quelqu’un, quelque chose, quelque part}
\begin{enumerate}
    \item masculins singulier, + de adjectif
    \begin{itemize}
        \item On a vu quelque chose d’étonnant cet après-midi
        \item A-t-il dit autre chose d’important 
    \end{itemize}
    \item quelque -- un(人)chose(物)part(处)
    \begin{enumerate}
        \item 作为attribut使用,表示说话者的敬佩或愤慨
        \begin{itemize}
            \item Ça, c’est quelqu’un
            \item C’est quelque chose, tout de même, de voir ça
        \end{itemize}
    \end{enumerate}
    \item autre --
    \begin{enumerate}
        \item 不同处用que引导比较句
        \begin{itemize}
            \item A-t-il dit autre chose que ce que nous savions
            \item Est-il parti autre part que là où il nous a dit 
        \end{itemize}
    \end{enumerate}
\end{enumerate}

\subsubsection{d’aucuns, grand-chose, grand monde, quelques-uns}
indéfinis de petite quantité
\begin{enumerate}
    \item quelques-uns:不一定指人,可指部分数量
    \begin{itemize}
        \item avant la fin du jour, c’était probable, quelques-uns de ces princes, barons, comtes et généraux seraient morts
        \item Les cavaliers s’aidaient mutuellement à fermer leurs cuirasses, quelques-uns nettoyaient leurs armes avec des rideaux arrachés aux fenêtres
    \end{itemize}
    \item d’aucuns:指有生命物,常用于主语
    \begin{itemize}
        \item D’aucuns l’appellent « patron » quand d’autres se défendent d’une telle appellation
    \end{itemize}
    \item grand-chose et grand monde:只用于否定句,不用于肯定句
    \begin{itemize}
        \item Pierre n’a pas fait grand-chose aujourd’hui
    \end{itemize}
\end{enumerate}

\subsubsection{je/Dieu/on ne sai(s/t) (pas/plus) lequel/qui/quoi}
indéfinis d’ignorance 
\begin{enumerate}
    \item sujet
    \begin{itemize}
        \item je ne sais lequel encore des membres de sa famille était mort au cours d’un assaut
        \begin{enumerate}
            \item je ne sais lequel与qui和quoi不同,有数性变化,可avec un complément partitif en \textit{de}
        \end{enumerate}
    \end{itemize}
    \item complément direct
    \begin{itemize}
        \item Je restais seule à attendre je ne sais quoi, je ne sais qui. Personne ne venait
    \end{itemize}
    \item complément de préposition
    \begin{itemize}
        \item Ce terme [droniste] lancé par je ne sais qui est un solécisme
    \end{itemize}
    \item 加pas/plus修饰
    \begin{itemize}
        \item Merci à une fan de la première heure dont je n’arrive pas à lire le nom et à je ne sais pas qui dont je n’arrive pas à lire la signature
    \end{itemize}
\end{enumerate}

\subsection{l’un et les uns}
\subsubsection{general}
\begin{enumerate}
    \item l’un et les uns是pronom,不能插入修饰词或作为déterminant
    \begin{itemize}
        \item *le premier un
        \item *cet un
        \item *l’une idée
    \end{itemize}
    \item l’autre et les autres不是pronom,可以插入修饰词或作为déterminant
    \begin{itemize}
        \item les deux autres
        \item mes deux autres
        \item l’autre idée
    \end{itemize}
    \item L’un和les uns不是单复数关系
    \begin{enumerate}
        \item l’un可伴随complément partitif en \textit{de}
        \begin{itemize}
            \item L’un de ces tableaux me parait particulièrement intéressant
        \end{itemize}
        \item 表示partitif时不用les uns而用certains
        \begin{itemize}
            \item Certains de ces tableaux me paraissent particulièrement intéressants
        \end{itemize}
    \end{enumerate}
    \item l'un不能与en连用
    \begin{itemize}
        \item Des romans de Balzac, j’en ai lu un.|*J’en ai lu l’un
    \end{itemize}
\end{enumerate}

\begin{table}[H]
\centering
\begin{tabular}{|>{\RaggedRight\arraybackslash}p{6cm}|>{\RaggedRight\arraybackslash}p{8cm}|} % Adjust column widths as needed
\hline
\rowcolor{cyan!20}
\textbf{EMPLOIS} & \textbf{EXEMPLES} \\
\hline
\textit{l'un} + de partitif & \textit{L'un de ces tableaux est particulièrement intéressant.} \\
\hline
\textit{l'un, les uns} corrélatif discursif & \textit{Deux colis : l'un vient d'Aix et le second d'Angers. Les uns ignorent ce que veulent les autres.} \\
\hline
\textit{l'un, les uns} corrélatif ajout & \textit{Ils sont allés les uns à Paris, les autres à Rome.} \\
\hline
\textit{l'un l'autre, les uns les autres} en emploi réciproque & \textit{Ils s'ignorent les uns les autres.} \\
\hline
\textit{l'un} + prép + \textit{l'autre, les uns} + prép + \textit{les autres} en emploi réciproque & \textit{Ils (se) parlent les uns aux autres. Ils sont partis les uns après les autres.} \\
\hline
\end{tabular}
\caption{Les emplois de l’un et les uns}
\end{table}

\subsubsection{Les emplois corrélatifs de l’un... l’autre}
\begin{enumerate}
    \item corrélatif discursif(sujets ou compléments)
    \begin{enumerate}
        \item 可省略成分
        \begin{itemize}
            \item L’un venait d’Essling et l’autre d’Aspern
        \end{itemize}
        \item 可承担不同句法功能
        \begin{itemize}
            \item L’un ne voyage pas sans l’autre.
            \item Les unes ne feraient pas de mal aux autres
        \end{itemize}
        \item l’un et l’autre, l’un ou l’autre, ni l’un ni l’autre
       \begin{itemize}
        \item L’un et l’autre sont aussi indissociables que les pensées et les émotion 
       \end{itemize}
       \item l'un可作为主句主语,l'autre出现在从句
       \begin{itemize}
        \item Les uns pensent que les autres ont tort.
       \end{itemize}
    \end{enumerate}
    \item ajout corrélatif
    \begin{enumerate}
        \item l’un et l’autre出现在从句句首
        \begin{itemize}
            \item J’éprouvai instantanément un immense soulagement comme si l’un et l’autre nous acceptions enfin notre défaite
        \end{itemize}
        \item ajout au syntagme prépositionnel 
        \begin{itemize}
            \item De ces gens qui, en leur temps, ont tranché les uns sur les Florentins du XVe, les autres sur Rembrandt, sur Hals, sur Mantegna ou sur Rubens, qui n’ont guère en commun que leurs lunettes sur le front
        \end{itemize}
        \item ajout au adjectif
        \begin{itemize}
            \item Tout le secteur recèle de nombreux itinéraires de randonnée les uns faciles, les autres plus longs pouvant même comporter des difficultés
        \end{itemize}
    \end{enumerate}
\end{enumerate}


\subsubsection{L’emploi réciproque de l’un l’autre, l’un + préposition + l’autre}
\begin{enumerate}
    \item Le syntagme nominal
    \begin{enumerate}
        \item SN作为动词的ajout,与se连用,se作为complément direct
        \begin{itemize}
            \item Jeanne et Marie s’aident l’une l’autre
            \item Les cultivateurs du coin s’aidaient les uns les autres 
            \item Les prisonniers se suivent l’un l’autre, sans pouvoir se parler
        \end{itemize}
        \item SN réciproque不能作为complément direct, attribut, sujet
        \begin{itemize}
            \item * Les cultivateurs du coin aidaient les uns les autres.
            \item * Ils auraient voulu devenir l’un l’autre
            \item * Paul et Pierre croient que l’un l’autre va arriver immédiatement
        \end{itemize}
    \end{enumerate}
    \item Le syntagme prépositionnel
    \begin{enumerate}
        \item le pronom \textit{l’un} ou \textit{les uns} est ajout au SP, \textit{l’autre} ou \textit{les autres} est le complément de la préposition
        \item le syntagme en \textit{à}可作为ajout à un verbe réfléchi,也可作为complément
        \begin{itemize}
            \item Les cultivateurs du coin se parlaient volontiers les uns aux autres
            \item Les cultivateurs du coin parlaient volontiers les uns aux autres
        \end{itemize}
        \item complément
        \begin{enumerate}
            \item de verbe
            \begin{itemize}
                \item Paul et Marie pensaient l’un à l’autre.
            \end{itemize}
            \item de adjectif
            \begin{itemize}
                \item Très belle situation, appartements trop proches les uns des autres
            \end{itemize}
            \item de nom
            \begin{itemize}
                \item La patronne remarqua qu’ils ne s’abordèrent que longtemps après qu’elle fut rentrée et que leur apparente ignorance l’un de l’autre se prolongea plus que la veille encore.
            \end{itemize}
            \item de participe
            \begin{itemize}
                \item Les poupées russes ou matriochkas [...] sont des séries de poupées de tailles décroissantes placées les unes à l’intérieur des autres
            \end{itemize}
        \end{enumerate}
        \item ajout au verbe
        \begin{itemize}
            \item On les a trouvés l’un contre l’autre
        \end{itemize}
        \item ajout à la phrase,作为句首
        \begin{itemize}
            \item Les uns après les autres, les spectateurs quittaient la salle
        \end{itemize}
    \end{enumerate}
    \item plus + adjectif + l’un + que l’autre
    \begin{itemize}
        \item les autres repas, d’innombrables repas, tous plus splendides et parfaits les uns que les autres
    \end{itemize}
\end{enumerate}

\subsection{tel et untel}
\subsubsection{tel}
\begin{enumerate}
    \item réfère à un être humain comme dans le proverbe
    \begin{itemize}
        \item Tel est pris qui croyait prendre
    \end{itemize}
    \item coordonné à lui-même
    \begin{itemize}
        \item  j’ai appris à ne plus céder à la tentation maladive de les lire, à ne plus me gâcher des journées entières parce que tel ou tel que je ne connais pas dit du mal de moi
    \end{itemize}
    \item avec un complément partitif
    \begin{itemize}
        \item Je sais exactement ce que veut dire tel ou tel de tes sourires, tel ou tel de tes regards
    \end{itemize}
\end{enumerate}

\subsubsection{untel}
someone


\section{Les pronoms démonstratifs}
\subsection{Ce et Ça}
\begin{enumerate}
    \item 都能作为construction clivée的主语, 但cela和celui不能作为该结构的主语
    \begin{itemize}
        \item C’est Jean que tu as rencontré.
        \item Ce/Ça sera Jean qui sera ton interlocuteur.
        \item * Cela/Ceci est Jean que tu as rencontré
    \end{itemize}
    \item les constructions impersonnelles avec une subordonnée complétive:主语用ça, ce, il,很少用cela
    \begin{itemize}
        \item Il est important que nous sachions la vérité
        \item C’est important que nous sachions la vérité
        \item Ça m’ennuie beaucoup que Marie soit partie
        \item ? Cela m’ennuie beaucoup que Marie soit partie
    \end{itemize}
    \begin{enumerate}
        \item 从句可disloquée, finale ou initiale,此时主句主语用ce, ça, cela,不用cela
        \begin{itemize}
            \item C’est important, que nous sachions la vérité
            \item Que nous sachions la vérité, c’est important
            \item Ça m’ennuie beaucoup, que Marie soit partie
            \item Cela m’ennuie beaucoup, que Marie soit partie
            \item * Il est important, que nous sachions la vérité
        \end{itemize}
    \end{enumerate}
\end{enumerate}
\subsubsection{Ce}
遵守弱形式代词的规则

\paragraph{Ce comme sujet}

\begin{enumerate}
    \item 只能与être连用
    \begin{itemize}
        \item C’est beau
        \item * Ce me plait beaucoup
    \end{itemize}
    \item 也可与后跟être的devoir et pouvoir连用
    \begin{itemize}
        \item Ce doit/devait/peut/pourrait être un bon argument
    \end{itemize}
    \item 或作为固定搭配ce me semble (在我看来) 使用
    \begin{itemize}
        \item Marie, ce me semble, est déjà partie
    \end{itemize}
    \item ce作为主语时,il不能作为动词的suffix
    \begin{itemize}
        \item * C’est-il pas vrai ?
        \item Était-ce bien le moment ?
    \end{itemize}
    \item ce只能被ne与动词分开,不能被其他代词分开,否则要用ça
    \begin{itemize}
        \item Ce n’est pas juste
        \item * Ce m’est égal
        \item * Ce nous est accessible
        \item Ça m’est égal
    \end{itemize}
    \item 复合时态中,ce不能与助动词avoir连用,ça不能与助动词être连用
    \begin{itemize}
        \item C’est arrivé d’un seul coup
        \item * Ça est arrivé d’un seul coup
        \item * Ce a été utile
        \item Ça a été utile
    \end{itemize}
    \item verb \textit{être} peut s’accorder en nombre avec le complément nominal
    \begin{itemize}
        \item C’est un moment agréable/une belle année
        \item Ce sont des moments agréables.
        \item Ce furent de belles années
    \end{itemize}
    \begin{enumerate}
        \item 人称代词常用单数
        \begin{itemize}
            \item C’est nous
            \item C’est vous, Madame
        \end{itemize}
    \end{enumerate}
\end{enumerate}

\paragraph{Les autres fonctions syntaxiques de ce}

\begin{enumerate}
    \item ce不能作为普通的complémentde de verbe ou de préposition,而只出现在固定搭配中 : sur ce, pour ce faire, ce faisant
    \begin{itemize}
        \item * J’ai déjà vu ce.
        \item * Marie a déjà parlé de ce.
        \item Marie est arrivée. Sur ce, Paul a frappé à la vitre.
    \end{itemize}
    \item ce 作为 coordination sans verbe : et ce
    \begin{itemize}
        \item Marie a réservé un billet d’avion, et ce immédiatement.
    \end{itemize}
\end{enumerate}

\paragraph{Ce + subordonnée}


\begin{enumerate}
    \item ce + relative
    \begin{enumerate}
        \item sujet de verbes variés
        \begin{itemize}
            \item Ce que tu dis me plait beaucoup.
        \end{itemize}
        \item complément de verbe 
        \begin{itemize}
            \item Tu me diras ce que tu veux
        \end{itemize}
        \item complément de préposition
        \begin{itemize}
            \item Tu me parleras de ce que tu veux
        \end{itemize}
        \item périphérique
        \begin{itemize}
            \item Ce que je veux, c’est une chambre climatisée
        \end{itemize}
        \item ce relative 作为subordonnée interrogative,此时不能用quoi ou que
        \begin{itemize}
            \item Il se demande [ce qu’il faut faire].
            \item * Il se demande qu’il faut faire/quoi il faut faire.
        \end{itemize}
        \item ce与relative不能被其他修饰成分分开
        \begin{itemize}
            \item * Ce donc que tu dis me plait beaucoup.
            \item * Dis-moi ce précisément dont tu as besoin.
        \end{itemize}
    \end{enumerate}
    \item ce + complétive en \textit{que}
    \begin{enumerate}
        \item 用于complétive前有介词时
        \begin{itemize}
            \item dans ce que, en ce que, sur ce que
            \item * dans que, * en que, * sur que
        \end{itemize}
        \item ce也可被cela或le fait替换
        \item 但à ce que和de ce que已经成为固定搭配,不能被cela或le fait替换
    \end{enumerate}
\end{enumerate}


\subsubsection{Ça}
ça不是ce的强形式,是cela的变形
\begin{enumerate}
    \item sujet
    \begin{enumerate}
        \item 作主语时ça部分遵守弱形式规则
        \begin{itemize}
            \item * Ça,parfois,m’intéresse
            \item Qu’est-ce qui ne va pas ? Ça.(可单独用)
            \item Tout ça me dépasse.(可被tout修饰)
            \item Ça, et mille autres choses, m’intéressent(可通过逗号coordonné)
            \item Ça, à soi seul, ne peut constituer un critère(可通过逗号修饰)
        \end{itemize}
        \item être不进行复数变位
        \begin{itemize}
            \item Pour moi, ça sera / * ça seront des poireaux.
        \end{itemize}
    \end{enumerate}
    \item complément direct
    \begin{itemize}
        \item Je veux ça 
    \end{itemize}
    \item complément oblique
    \begin{itemize}
        \item Ils ne pensent qu’à ça
    \end{itemize}
    \item périphérique, repris par \textit{ce} ou personnelle faible
    \begin{itemize}
        \item Ça, c’est du cinéma !
        \item Tout ça, nous pourrons en discuter plus tard
    \end{itemize}
    \item expression figée:ça va
    \begin{enumerate}
        \item ça指对话者
        \begin{itemize}
            \item Comment ça va ?
            \item Ça ira ?
            \item Ça a été ?|Ç’a été?
        \end{itemize}
        \item 表示情绪:j’en ai assez, ça suffit 够了
        \begin{itemize}
            \item De toute façon, ça va j’ai compris que vous aimez les Noirs
        \end{itemize}
    \end{enumerate}
    \item ça很少直接与relative restrictive连用,在construction clivée中relative修饰的也不是ça
    \begin{itemize}
        \item * Je veux ça qui est sur la table
        \item Je veux ça même qui est sur la table
        \item C’est ça qui serait bien
    \end{itemize}
\end{enumerate}


\subsection{Ceci et Cela}
\subsubsection{fonctions syntaxiques}
\begin{enumerate}
    \item sujet:可suffixé -il连用
    \begin{itemize}
        \item Cela t’ennuie-t-il ?
    \end{itemize}
    \item complément direct
    \begin{itemize}
        \item J’aime cela/ceci.
    \end{itemize}
    \item complément oblique
    \begin{itemize}
        \item Marie a parlé de cela/ceci.
    \end{itemize}
    \item périphérique, repris par ce ou par une proforme personnelle faible
    \begin{itemize}
        \item Tout cela, c’est de la magie 
        \item Ceci, ce n’est pas ce que j’attendais.
        \item Mais cela, beaucoup de gens en ont déjà discuté
    \end{itemize}
\end{enumerate}




\subsubsection{Les compléments et ajouts à ceci et cela}
\begin{enumerate}
    \item par un adjectif qui précède
    \begin{itemize}
        \item seul ceci, tout cela
    \end{itemize}
    \item par un adjectif épithète introduit par \textit{de}
    \begin{itemize}
        \item ceci de positif
    \end{itemize}
    \item par une subordonnée relative
    \begin{enumerate}
        \item 引导relative restrictive时,必须被adverbe(précisément) ou adjectif(seul, même)与从句分开
        \begin{itemize}
            \item ceci même qui importe
            \item Cela même que j’appelle signification ne m’apparaît comme pensée sans aucun mélange de langage que par la vertu du langage
        \end{itemize}
        \item 引导relative non restrictive时可直接与其相连
        \begin{itemize}
            \item Je te prie d’insister particulièrement sur ceci qui est capital : ma première visite sera aussi la dernière
            \item Jean-Jacques, de loin, assistait à tout cela qui, au vrai, ne l’intéressait plus
        \end{itemize}
    \end{enumerate}
    \item par une subordonnée complétive en \textit{que}
    
    常用在介词en, par, pour后,或固定搭配à ceci près que et à cela près que中. 可被adjectif, adverb, préposition与从句分开
    \begin{itemize}
        \item en cela qu’il avait tort
        \item Le désaccord des informateurs winnebago offre ceci de remarquable que les deux formes décrites correspondent à des arrangements réels
        \item En cela précisément que n’est réel que ce qui a lieu devant tous, à savoir publiquement
        \item À ceci près que ce serait l’occasion de revenir sur les raisons qui font que Zweig est de nouveau et peut-être de plus en plus
    \end{itemize}
\end{enumerate}

\subsection{celui, celui-ci, celui-là}
\begin{enumerate}
    \item celui不能单独用,只能伴随其他词汇使用
    \begin{itemize}
        \item * Celles ont des chances d’être achalandées
    \end{itemize}
    \begin{enumerate}
        \item d’un complément prépositionnel 
        \begin{itemize}
            \item celui de Paul
        \end{itemize}
        \begin{enumerate}
            \item celui可被même (exactement, précisément) 与限制语分开,但不能被seul分开(与celui-là不同)
            \begin{itemize}
                \item Sous ce ciel d’ardoise, l’harmonie fondamentale de mai est pour moi celle même des toiles de Villon
                \item Me paraît mériter la liberté celui-là seul qui saurait en user pour une autre fin que lui-même
            \end{itemize}
        \end{enumerate}
        \item d’un ajout prépositionnel
        \begin{itemize}
            \item celle en or
        \end{itemize}
        \item d’un ajout adjectival ou au participe passé
        \begin{itemize}
            \item celle relative au téléphone
            \item celle passée
        \end{itemize}
        \item d’un ajout à l’infinitif ou au participe présent
        \begin{itemize}
            \item celle à vendre
            \item celle commençant demain
        \end{itemize}
        \item d’une subordonnée relative
        \begin{itemize}
            \item celle que tu veux
        \end{itemize}
    \end{enumerate}
    \item celui-ci et celui-là无限制
    \item celui可用disloquée en de + nom结构,表示celui的内容
    \begin{itemize}
        \item C’est celui-là, d’argument, qui m’a le plus convaincu
        \item C’est le premier, d’argument, qui m’a le plus convaincu
    \end{itemize}
\end{enumerate}

\subsection{ce dernier}

\begin{enumerate}
    \item 与le dernier不同,ce dernier不能用la dislocation finale en de + nom; Le dernier est bien un SN sans nom, mais pas ce dernier
    \begin{itemize}
        \item  *ce dernier, de soldat, *ce dernier, de lieu
    \end{itemize}
\end{enumerate}

\section{Les proformes verbales}
指代前面出现的infinitif或phrase;le, ça, cela也可指代前面的adjectival或prépositionnel
\begin{enumerate}
    \item les proformes faibles invariables:le, en, y
    \begin{itemize}
        \item Changer de métier, Paul ne le souhaite pas
        \item Changer de métier, Paul n’en a pas envie et il n’y est pas obligé.
        \item Jean veut déménager, j’en/y rêve aussi
        \item Marie était partie. Jean le comprenait maintenant
        \item Marie était partie. Jean s’en doutait mais n’y attachait aucune importance.
        \item Jean est content. Marie l’est aussi
        \item Jean semble en colère. Marie l’est moins
    \end{itemize}
    \item les pronoms démonstratifs:ce, ça, ceci, cela
    \begin{itemize}
        \item Changer de métier, ce n’est pas si facile
        \item Changer de métier, cela lui parait difficile
        \item Marie était partie. Cela ennuyait Jean
        \item Jean est aux anges. Mais il n’est pas que cela
    \end{itemize}
    \item expressions en faire:替代前面出现过的动词
    \begin{enumerate}
        \item seule avec une expression de manière
        \begin{enumerate}
            \item 与anaphorique \textit{ainsi, de même, pareil};adverbe comparatif comme \textit{aussi bien, mieux} 连用
            \begin{itemize}
            \item Ils ont couru pendant une heure. Nous ferons de même demain.
            \item Marie peint de magnifiques tableaux. J’aimerais faire aussi bien
            \end{itemize}
            \item seul en subordonnée comparative:\textit{comme, que}
            \begin{itemize}
                \item Il commença le concert comme il faisait chaque soir, c’est-à-dire qu’il fonçait sur la scène au pas de chasseur, faisait un bref salut
                \item Comment expliquer qu’elle emprisonne ses citoyens plus que ne fait la Chine 
            \end{itemize}
        \end{enumerate}
        \item avec complément nominal
        \begin{enumerate}
            \item \textit{cela, ça, la même chose}
            \begin{itemize}
                \item Il faudrait réparer la douche. Qui peut faire cela
            \end{itemize}
            \item \textit{le} invariable
            \begin{itemize}
                \item On m’a dit de venir mais je ne le ferai pas.
            \end{itemize}
            \item à une subordonnée relative en \textit{que}
            \begin{itemize}
                \item On m’a dit de sortir, ce que j’ai fait.
            \end{itemize}
            \item 与\textit{le, en}组合成固定搭配 \textit{n’en rien faire, en faire autant}
            \begin{itemize}
                \item On lui a dit de travailler davantage. Il persiste à n’en rien faire
            \end{itemize}
        \end{enumerate}
    \end{enumerate}
    \item 第二个宾语用de引导的动词:attendre, demander, dire, espérer
        \begin{enumerate}
            \item en + verbe + autant
            \begin{itemize}
                \item Il fait beau à Londres, nous pouvons en dire autant (du temps) à Paris
                \item Il a fait beau à Londres, nous n’en espérions pas autant à Paris
            \end{itemize}
            \item complément \textit{la même chose}
            \begin{itemize}
                \item Il fait beau à Londres, nous pouvons dire la même chose (du temps) à Paris
            \end{itemize}
        \end{enumerate}
\end{enumerate}


\section{Les interrogatifs, relatifs et exclamatifs}


\begin{table}[H]
\small
\centering
\begin{tabular}{|>{\RaggedRight\arraybackslash}p{2.5cm}|l|>{\RaggedRight\arraybackslash}p{2cm}|>{\RaggedRight\arraybackslash}p{6.5cm}|}
\hline
\rowcolor{cyan!20}
\textbf{FORME} & \textbf{CATÉGORIE} & \textbf{EMPLOI} & \textbf{EXEMPLES} \\
\hline
\textit{auquel, auxquels, auxquelles} & préposition & interrogatif \newline relatif & \textit{Auquel as-tu parlé ?}\newline\textit{l'homme auquel j'ai parlé} \\
\hline
\textit{ce que} & adverbe & exclamatif & \textit{Ce que c'est beau !} \\
\hline
\textit{combien} & adverbe & exclamatif \newline interrogatif & \textit{Combien précieux sont vos conseils !}\newline\textit{Combien voulez-vous ?} \\
\hline
\textit{comme} & adverbe & exclamatif & \textit{Comme c'est beau !} \\
\hline
\textit{comment} & adverbe & interrogatif & \textit{Comment allez-vous ?} \\
\hline
\textit{duquel, desquels, desquelles} & préposition & interrogatif \newline relatif & \textit{Duquel parles-tu ?}\newline\textit{l'homme au frère duquel j'ai parlé} \\
\hline
\textit{laquelle, lequel, lesquels, lesquelles} & pronom & interrogatif \newline relatif & \textit{Lequel préfères-tu ?}\newline\textit{le sac avec lequel je pars} \\
\hline
\textit{lequel, lesquels, lesquelles} & déterminant & relatif & \textit{J'ai vu un notaire, lequel notaire était connu.} \\
\hline
\textit{où} & préposition (locative) & interrogatif \newline relatif & \textit{Où vas-tu ?}\newline\textit{l'endroit où tu vas} \\
\hline
\textit{où} & préposition (temporelle) & relatif & \textit{le jour où tu pars} \\
\hline
\textit{pourquoi} & adverbe & interrogatif \newline relatif & \textit{Pourquoi partez-vous ?}\newline\textit{C'est ce pourquoi il n'est pas venu.} \\
\hline
\textit{quand} & adverbe & interrogatif & \textit{Quand partez-vous ?} \\
\hline
\textit{que} & adverbe & exclamatif \newline interrogatif & \textit{Que c'est beau !}\newline\textit{Que ne l'aviez-vous dit plus tôt ?} \\
\hline
\textit{que} & pronom & interrogatif & \textit{Que voulez-vous ?} \\
\hline
\textit{quel, quelle, quels, quelles} & adjectif & exclamatif \newline interrogatif & \textit{Quel serait mon désespoir !}\newline\textit{Quel est son âge ?} \\
\hline
\textit{quel, quelle, quels, quelles} & déterminant & exclamatif \newline interrogatif & \textit{Quelle chance il a eu !}\newline\textit{Quels élèves a-t-il interrogés ?} \\
\hline
\textit{qu'est-ce que} & adverbe & exclamatif & \textit{Qu'est-ce que c'est beau !} \\
\hline
\textit{qu'est-ce que, qu'est-ce qui} & pronom & interrogatif & \textit{Qu'est-ce que tu veux ?}\newline\textit{Qu'est-ce qui te ferait plaisir ?} \\
\hline
\textit{quid} & pronom & interrogatif & \textit{Quid des retraites anticipées ?} \\
\hline
\textit{qui} & pronom & interrogatif \newline relatif & \textit{Qui est parti ?}\newline\textit{L'homme avec qui je parle} \\
\hline
\textit{qui est-ce que, qui est-ce qui} & pronom & interrogatif & \textit{Qui est-ce qui est parti ?}\newline\textit{Qui est-ce que tu invites ?} \\
\hline
\textit{quoi} & pronom & interrogatif \newline relatif & \textit{À quoi penses-tu ?}\newline\textit{C'est quelque chose à quoi il faut penser.} \\
\hline
\end{tabular}
\caption{Les mots exclamatifs, interrogatifs et relatifs}
\end{table}


\begin{enumerate}
    \item qui est-ce qui/que et qu’est-ce que/qui
    \begin{enumerate}
        \item 可分解为pronom + est-ce que
        \begin{itemize}
            \item À qui était-ce que vous vouliez parler ?
        \end{itemize}
        \item 可插入déjà, donc
        \begin{itemize}
            \item Qui est-ce déjà que tu as vu ?
            \item Qui donc est-ce que tu as vu ?
        \end{itemize}
    \end{enumerate}
    \item subordonnées interrogatives中,不用quoi而用ce que, ce qui
\end{enumerate}

\subsubsection{ Les fonctions syntaxiques des mots interrogatifs, exclamatifs et relatifs}
extrait即本该出现在动词后的成分出现在句首
\begin{enumerate}
    \item mots interrogatifs et exclamatifs可出现在句首
    \begin{enumerate}
        \item sujet
        \begin{itemize}
            \item Lesquels viendront ?
        \end{itemize}
        \item extrait
        \begin{itemize}
            \item Lesquels voulez-vous ?
            \item Comment va-t-il ?
        \end{itemize}
        \item appartenir à un syntagme extrait
        \begin{itemize}
            \item Avec lesquels partez-vous ?
            \item Combien d’épreuves il a rencontrées avant de réussir !
        \end{itemize}
    \end{enumerate}
    \item mots interrogatifs可出现在动词后
    \begin{enumerate}
        \item complément direct/oblique
        \begin{itemize}
            \item Vous voulez lesquels ?
            \item Et il va comment ?
        \end{itemize}
        \item ajout
        \begin{itemize}
            \item Et vous partez quand ?
        \end{itemize}
        \item appartenir à un syntagme
        \begin{itemize}
            \item Vous partez avec qui ?
            \item Tu as fait quelle erreur dans ta dictée ?
        \end{itemize}
    \end{enumerate}
    \item mots exclamatifs较少出现在动词后
    \begin{itemize}
        \item Il a rencontré combien d’épreuves avant de réussir !
    \end{itemize}
    \item le mot relatif出现在subordonnée句首
    \begin{enumerate}
        \item extrait
        \begin{itemize}
            \item Voici l’endroit [où je vais cet été]
        \end{itemize}
        \item appartenir à un syntagme als sujet
        \begin{itemize}
            \item On m’a porté un message, lequel message était codé
        \end{itemize}
        \item appartenir à un syntagme als extrait
        \begin{itemize}
            \item Voici les amis avec qui je pars cet été
            \item Voici la motion pour laquelle je vais voter
        \end{itemize}
    \end{enumerate}
    \item l’adjectif interrogatif \textit{quel},  les exclamatifs \textit{comme} et \textit{ce que}总作为extrait出现在句首
    \begin{itemize}
        \item Quel est le programme ?
        \item * Le programme est quel ?
        \item Comme il a souffert !
        \item * Il a souffert comme !
    \end{itemize}
\end{enumerate}

\subsection{Interrogatifs}
\begin{enumerate}
    \item quid, à quoi bon, pourquoi pas用于句首且无动词
    \begin{enumerate}
        \item quid 后跟complément prépositionnel en \textit{de}
        \begin{itemize}
            \item Quid de l’augmentation du chômage ?
        \end{itemize}
        \item à quoi bon et pourquoi pas后跟作为主语的syntagme nominal或infinitif
        \begin{itemize}
            \item À quoi bon ces protestations ?
            \item Pourquoi pas des vacances à la montagne ?
        \end{itemize}
    \end{enumerate}
\end{enumerate}

\subsubsection{Qui et Lequel}
\paragraph{Qui}
询问人;无先行词
\begin{enumerate}
    \item 阳性单数,即使回答是复数
    \begin{itemize}
        \item * Qui viendront ?
    \end{itemize}
\end{enumerate}
\paragraph{Lequel}
指有生命或无生命
\begin{enumerate}
    \item 可跟complément en \textit{de} dit partitif
    \begin{itemize}
        \item Lequel de ces chapeaux préfères-tu ?
    \end{itemize}
\end{enumerate}


\subsubsection{Que et Quoi : pronoms}
无生命物
\paragraph{Que}
\begin{enumerate}
    \item que是弱形式,quoi是强形式
    \begin{itemize}
        \item J’ai vu quelque chose. * Que ?
        \item J’ai vu quelque chose. Quoi?
        \item * Que d’autre ?
        \item Quoi d’autre ?
        \item * Tu as vu [qui ou que] ?
        \item Tu as vu [qui ou quoi] ?
    \end{itemize}
    \item que不能被sujet或adverbe与动词分开,只能被弱形式分开
    \begin{itemize}
        \item * Que Frédéric veut ?
        \item * Que décidément veut Frédéric ?
        \item Que nous veut Frédéric ?
    \end{itemize}
    \begin{enumerate}
        \item 主语应inversé或suffixé
        \begin{itemize}
            \item Que veut Frédéric ?
            \item Que veut-il ?
        \end{itemize}
    \end{enumerate}
    \item que永远作为extrait置于句首,置于动词后时只能用quoi
    \begin{itemize}
        \item * Vous voulez que ?
        \item * Pierre est devenu que ?
        \item Vous voulez quoi ?
    \end{itemize}
    \begin{enumerate}
        \item 指代complément direct
        \begin{itemize}
            \item Que voulez-vous ?
        \end{itemize}
        \item 指代attribut
        \begin{itemize}
            \item Qu’est devenu Pierre ?
        \end{itemize}
        \item 指代sujet du verbe d’une subordonnée introduit par \textit{qui}
        \begin{itemize}
            \item Que croyez-vous qui arriva ?
        \end{itemize}
        \item 不能作为主语
        \begin{itemize}
            \item * Que peut arriver ?
        \end{itemize}
    \end{enumerate}
\end{enumerate}

\paragraph{Quoi}
\begin{enumerate}
    \item complément d’un verbe
    \begin{itemize}
        \item Frédéric veut quoi ?
    \end{itemize}
    \item complément d’une préposition
    \begin{itemize}
        \item À quoi penses-tu ?
        \item Tu penses à quoi
    \end{itemize}
    \item sujet postverbal
    \begin{itemize}
        \item À Marie reviendrait quoi ?
    \end{itemize}
    \item 只有被修饰的前提下才能作为sujet préverbal
    \begin{itemize}
        \item * Quoi reviendrait à Marie ?
        \item Quoi d’autre pourrait vous satisfaire ?
    \end{itemize}
    \item quoi可appartenir à un syntagme extrait,但除非它与infinitif连用,否则它自己不能单独作为extrait
    \begin{itemize}
        \item À quoi penses-tu ?
        \item * Quoi veut Frédéric ?
        \item Je ne sais pas quoi faire
    \end{itemize}
\end{enumerate}



\subsubsection{Quel : adjectif ou déterminant}
\begin{enumerate}
    \item 询问l’identité d’une entité ou son type
    \item adjectif:作为extrait
    \begin{enumerate}
        \item 询问预设或已知的物体的l’identité d’une entité ou son type
        \item 不能询问对话中引入的新物体的身份或特质,也不能指涉专有名词,否则要用qui或comment
            \begin{itemize}
                \item * Quel êtes-vous ?
                \item Qui êtes-vous ?
                \item * Quel est Jean ?
                \item Qui/Comment est Jean ?
            \end{itemize}
        \item quel不能与动词分开,主语作为proforme suffixée或sujet nominal inversé
        \begin{itemize}
            \item * Quelle donc est la température ?
            \item * Quelle elle est ?
            \item Quelle est-elle ?
            \item Quelle est la température ?
        \end{itemize}
        \item quelle不能置于动词后
        \begin{itemize}
            \item * La température est quelle ?
        \end{itemize}
        \item 不能作为主语
        \begin{itemize}
            \item * Quel prendra le train ?
        \end{itemize}
    \end{enumerate}
    \item déterminant:与名词连用
    \begin{enumerate}
        \item 询问 l’identité d’une entité ou son type
        \begin{itemize}
            \item À quel collègue parlais-tu ?
            \item Quel restaurant va ouvrir ici ?
        \end{itemize}
    \end{enumerate}
\end{enumerate}

\subsubsection{Où et Quand}
\paragraph{Quand}
\begin{enumerate}
    \item extrait
    \begin{itemize}
        \item Quand viendra Paul ?
    \end{itemize}
    \item complément d’un verbe
    \begin{itemize}
        \item Le meilleur moment est quand, à ton avis ?
    \end{itemize}
    \item ajout
    \begin{itemize}
        \item Paul viendra quand ?
    \end{itemize}
    \item complément d’une préposition
    \begin{itemize}
        \item Depuis quand Paul sait-il ce genre de choses ?
    \end{itemize}
\end{enumerate}
\paragraph{Òu}
\begin{enumerate}
    \item extrait
    \begin{itemize}
        \item Où est la meilleure cachette ?
    \end{itemize}
    \item complément oblique
    \begin{itemize}
        \item C’est où, la meilleure cachette ?
    \end{itemize}
    \item ajout
    \begin{itemize}
        \item Tu dors où ?
    \end{itemize}
    \item complément d’une préposition
    \begin{itemize}
        \item Jusqu’où peut-on aller ?
    \end{itemize}
    \item où不是adverbe mais une préposition sans complément
\end{enumerate}


\subsubsection{Comment : adverbe}
询问propriété或manière
\begin{enumerate}
    \item 置于动词后
    \begin{enumerate}
        \item attribut du sujet
        \begin{itemize}
            \item Le spectacle était comment ?
        \end{itemize}
        \item attribut du complément
        \begin{itemize}
            \item Tu trouves mes lunettes comment ?
        \end{itemize}
        \item complément oblique
        \begin{itemize}
            \item Paul s’est conduit comment ?
        \end{itemize}
        \item ajout
        \begin{itemize}
            \item Tu fais ça comment ?
        \end{itemize}
    \end{enumerate}
    \item 作为extrait置于句首
    \begin{enumerate}
        \item attribut du sujet
        \begin{itemize}
            \item Comment était le spectacle ?
        \end{itemize}
        \begin{enumerate}
            \item à l’attribut du sujet de \textit{être} ou \textit{sembler}, mais difficilement à celui de \textit{devenir}, 此时要用que
            \begin{itemize}
                \item ? Comment est-il devenu ?
                \item Qu’est devenu Pierre ?
            \end{itemize}
        \end{enumerate}
        \item attribut du complément
        \begin{itemize}
            \item Comment trouves-tu mes lunettes ?
        \end{itemize}
        \item complément oblique
        \begin{itemize}
            \item Comment s’est conduit Paul ?
        \end{itemize}
        \item ajout
        \begin{itemize}
            \item Comment fais-tu ça ?
        \end{itemize}
        \item 固定搭配comment se fait-il que 表示为什么
        \begin{itemize}
            \item Comment se fait-il que les épisodes de certaines séries télévisées sont diffusés dans le désordre ?
        \end{itemize}
    \end{enumerate}
\end{enumerate}

\subsubsection{Combien : adverbe}
询问数量
\begin{enumerate}
    \item 引导syntagme nominal, avec la fonction spécifieur
    \begin{itemize}
        \item Combien de livres as-tu lus ce mois-ci ?
    \end{itemize}
    \begin{enumerate}
        \item 可与名词分开
        \begin{itemize}
            \item Combien as-tu lu de livres ce mois-ci ?
        \end{itemize}
    \end{enumerate}
    \item 不与名词连用
    \begin{enumerate}
        \item complément de verbe
        \begin{itemize}
            \item Finalement, on était combien ?
        \end{itemize}
        \item complément de préposition
        \begin{itemize}
            \item À combien se montent ces achats ?
        \end{itemize}
        \item extrait
        \begin{itemize}
            \item Combien veut-il pour ce tableau ?
        \end{itemize}
    \end{enumerate}
\end{enumerate}

\subsubsection{Que et Pourquoi : adverbe}


\paragraph{Pourquoi}
\begin{enumerate}
    \item extrait:与其他extrait不同,pourquoi不允许l’inversion du sujet nominal
    \begin{itemize}
        \item Pourquoi Paul vient-il ?
        \item * Pourquoi vient Paul ?
    \end{itemize}
    \item ajout
    \begin{itemize}
        \item Tu as fait ça pourquoi ?
    \end{itemize}
\end{enumerate}
\paragraph{Que}
表示为什么;但只用于extrait et suivi d’un verbe avec ne négatif
\begin{itemize}
    \item Si tu es si peu sûre de moi, que n’as-tu appelé toi-même ? 
\end{itemize}


\subsubsection{La modification des mots interrogatifs}

\paragraph{d’autre}
\textit{d’autre}只修饰pronoms qui et quoi ,不修饰其他疑问词
\begin{itemize}
    \item Qui d’autre vous a parlé ?
    \item Vous voulez quoi d’autre ?
\end{itemize}
\paragraph{diable/diantre}
\textit{diable/diantre}(到底/究竟)只修饰作为extraits的疑问词
\begin{itemize}
    \item * Ce portefeuille est passé où diable ?
    \item Où diable est passé ce portefeuille ?
\end{itemize}
\begin{enumerate}
    \item 不能修饰quel和lequel
    \begin{itemize}
        \item * Quelle diable est la température ?
    \end{itemize}
    \item 可以修饰que,且该组合结构不再是弱形式,故可以与动词分开,不再强制需要sujet inversé
    \begin{itemize}
        \item Que diable voulez-vous ?
        \item Que diable Paul a-t-il en tête ?
    \end{itemize}
\end{enumerate}
\paragraph{donc}
\textit{donc}(究竟)修饰除了que, quel, lequel之外的所有疑问词,且无特殊限制
\begin{itemize}
    \item Qui donc vous a parlé ?
    \item Vous irez où donc ?
    \item * Lequel donc va-t-il acheter ?
\end{itemize}
\paragraph{ça}
\textit{ça}修饰comment, pourquoi, où, quand, qui;常与疑问词单用,demande d’information, indiquant la surprise, ou une demande de clarification
\begin{itemize}
    \item Paul est parti. Où ça ?
    \item Paul est parti. Comment ça ?
    \item Paul est parti. Qui ça ?
\end{itemize}
\begin{enumerate}
    \item 被ça修饰的疑问词用于osition canonique plutôt que comme extrait;不能用sujet suffixé或inversion du sujet
    \begin{itemize}
        \item Il est parti où ça ?
        \item * Où ça est-il parti ?
        \item * Où ça est parti Paul ?
    \end{itemize}
\end{enumerate}

\subsection{Relatifs}
\begin{enumerate}
    \item que和dont是subordonnants而不是pronom
    \begin{enumerate}
        \item 不能作为compléments de préposition ou de nom
        \begin{itemize}
            \item Voici le projet [pour lequel je voterai].
            \item * Voici le projet [pour que je voterai].
            \item Voici Paul, [avec le fils de qui je dois parler].
            \item * Voici Paul, [avec le fils dont je dois parler].
        \end{itemize}
        \item 不能apparaitre dans une relative à l’infinitif.
        \begin{itemize}
            \item Voici un endroit [où aller].
            \item * Voici le projet [que soutenir].
        \end{itemize}
    \end{enumerate}
\end{enumerate}

\subsubsection{Qui et Lequel}
\paragraph{Qui}
\begin{enumerate}
    \item pronom:先行词必须是有生命物
    \begin{enumerate}
        \item 必须与介词连用
        \begin{itemize}
            \item * Voici une personne qui je verrais volontiers
            \item Marie est la fille avec qui Jean parle
        \end{itemize}
        \item 可与不定式连用
        \begin{itemize}
            \item Voici un ami avec qui partir en voyage
        \end{itemize}
    \end{enumerate}
    \item subordonnant:先行词有无生命都可以
    \begin{itemize}
        \item Voici les gens qui sont invités
    \end{itemize}
    \begin{enumerate}
        \item 不能与不定式连用
        \begin{itemize}
            \item * Voici un livre qui avoir du succès
        \end{itemize}
    \end{enumerate}
\end{enumerate}

\paragraph{Lequel}
pronom. 作为complément de préposition
\begin{itemize}
    \item Marie est la fille avec laquelle Jean parle
    \item Voici le livre avec lequel je partirai en vacance
\end{itemize}


\subsubsection{Quoi et Pourquoi}
\paragraph{Quoi}
先行词为无生命物
\begin{enumerate}
    \item 多指代ce, cela, quelque chose, rien
    \begin{itemize}
        \item La pauvreté est ce contre quoi Paul a lutté toute sa vie
    \end{itemize}
    \item 必须与介词连用
    \begin{itemize}
        \item Voici ce pour quoi il se bat
        \item * Voici quelque chose quoi Paul a combattu toute sa vie
    \end{itemize}
\end{enumerate}
\paragraph{pourquoi}
与先行词ce连用
\begin{itemize}
    \item Voici ce pourquoi il est parti.
\end{itemize}


\subsubsection{Où}
作为relatif具有时空双重意味
\begin{enumerate}
    \item complément de verbe
    \begin{itemize}
        \item Je connais l’endroit où vous allez
    \end{itemize}
    \item complément de préposition
    \begin{itemize}
        \item Voilà le point jusqu’où il faut aller
    \end{itemize}
    \item ajout locatif ou temporel
    \begin{itemize}
        \item Je me souviens du jour où je l’ai rencontré
    \end{itemize}
\end{enumerate}

\subsection{Exclamatifs}
\subsubsection{Combien : adverbe}
表达intensité或quantité
\begin{enumerate}
    \item suivi par \textit{de}
    \begin{itemize}
        \item Combien de passion elle y met !(intensité)
        \item Combien de pays elle a visités !(quantité)
    \end{itemize}
    \item extrait, séparé du nom introduit par \textit{de}
    \begin{itemize}
        \item Combien elle y met de passion !(intensité)
        \item Combien elle a visité de pays !(quantité)
    \end{itemize}
    \item modifier un verbe
    \begin{itemize}
        \item Tu sais combien il va au cinéma(fréquence)
        \item Tu sais combien il t’aime(intensité)
    \end{itemize}
    \item modifier un adjectif
    \begin{itemize}
        \item Tu sais combien il nous est précieux(intensité)
    \end{itemize}
    \item modifié par \textit{ô}
    \begin{itemize}
        \item Caen renoue avec une victoire ô combien précieuse
        \item Ayant pourtant, ô combien de fois, personnellement enduré l’épreuve, je ne parvenais tout simplement plus à la ressentir, à l’imaginer
    \end{itemize}
\end{enumerate}


\subsubsection{Comme : adverbe}
\begin{enumerate}
    \item 永远作为extrait置于句首;修饰un verbe, un adjectif ou un adverbe
    \begin{itemize}
        \item Comme j’aime ce tableau ! (degré)
        \item Comme vous avez de grandes oreilles ! (degré)
        \item Comme tu joues bien ! (degré)
        \item Comme tu y vas ! (manière)
        \item Comme tu as voyagé ! (fréquence)
        \item Comme il dort !
    \end{itemize}
\end{enumerate}

\subsubsection{que / ce que / qu’est-ce que : adverbe}
degré,永远置于句首
\begin{enumerate}
    \item que可suivi de \textit{de}或séparé du nom introduit par \textit{de};e que et qu’est-ce que不能与SN连用
    \begin{itemize}
        \item Que de mal j’ai eu à te trouver !
        \item Que/Ce que j’ai eu de mal à te trouver !
        \item * Ce que de mal j’ai eu à te trouver !
        \item * Qu’est-ce que de mal j’ai eu à te trouver !
    \end{itemize}
    \item que近似combien
    \begin{enumerate}
        \item 可表quantité
        \begin{itemize}
            \item Que de pays il a visités !
        \end{itemize}
        \item 可表intensité
        \begin{itemize}
            \item Que de mal j’ai eu à te trouver !
        \end{itemize}
        \item 但不能作为complément d’une préposition
        \begin{itemize}
            \item * Dans que de pays il a voyagé !
            \item Dans combien de pays il a voyagé !
        \end{itemize}
    \end{enumerate}
    \item subordonnées exclamatives不能用que;可用ce que;qu’est-ce que则是非正式用法
    \begin{itemize}
        \item * Regarde [que c’est beau] !
        \item Regarde [ce que c’est beau] !
        \item ! Regarde [qu’est-ce que c’est beau] !
    \end{itemize}
    \item que相比其余两个较少用于表示fréquence
    \begin{itemize}
        \item ? Qu’il a voyagé !
        \item Ce qu’il a voyagé !
        \item Qu’est-ce qu’il a voyagé !
    \end{itemize}
\end{enumerate}



\chapter{Préposition}

\begin{enumerate}
    \item dans/dedans, hors/dehors, sus/dessus et sous/dessous
    \begin{enumerate}
        \item dedans, dessous, dessus et dehors后无complément
        \begin{itemize}
            \item Il va [sous la table]/[dessous].
            \item La balle est [hors du terrain]/[dehors].
        \end{itemize}
        \item 但在de后,以上这些介词可有complément
        \begin{itemize}
            \item Il a surgi [de dessous la table].
            \item * Il a surgi [de dans la niche].
        \end{itemize}
    \end{enumerate}
    \item Prépositions simples archaïques
    \begin{enumerate}
        \item ès = en + les
        \begin{itemize}
            \item docteur ès lettres
        \end{itemize}
        \item lès = à côté de
        \begin{itemize}
            \item Saint-Rémy-lès-Chevreuse
        \end{itemize}
        \item emmi = parmi
        \begin{itemize}
            \item emmi les champs de chardons
        \end{itemize}
        \item fors = hors
        \begin{itemize}
            \item Rien n’a changé, fors le bouc émissaire.
        \end{itemize}
    \end{enumerate}
    \item 介词与副词的根本区别:副词可置于auxiliaire与participe间、infinitif前,但介词都不能
    \begin{itemize}
        \item Il est parti après.|*Il est après parti.
        \item J’ai voté pour.|* J’ai pour voté.
        \item * Paul a décidé de derrière s’arrêter.
        \item * Paul s’est résigné à pour voter.
    \end{itemize}
\end{enumerate}



\section{Les classes syntaxiques de prépositions}

\subsection{Les prépositions à complément nominal ou prépositionnel}

\begin{table}[H]
    \centering
    \small
    \setlength{\extrarowheight}{2pt}
    \begin{tabular}{|>{\raggedright\arraybackslash}p{0.48\textwidth}|>{\raggedright\arraybackslash}p{0.48\textwidth}|}
        \hline
        \rowcolor{cyan!20}
        \textbf{PRÉPOSITIONS} & \textbf{EXEMPLES} \\
        \hline
        \multicolumn{2}{|c|}{\cellcolor[HTML]{F2F2F2}\textbf{SIMPLES}} \\ % Sub-header for SIMPLES
        \hline
        \textbf{sans complément :} & -- \\
        \textit{ailleurs, dedans, dehors, dessous, ici, là, où, partout, tard, tôt} & \\
        \hline
        \textbf{avec un complément nominal :} & \\
        \textit{à, attendu, avec, chez, comme, concernant, côté, dans, de, depuis, derrière, dès, devant, durant, en, hors, entre, envers, malgré, moyennant, outre, par, parmi, passé, pendant, pour, question, rayon, sans, selon, sous, suivant, sur, vers, via, vu} & avec [Marie], côté [cour], comme [mari], depuis [deux mois], hors [les murs], moyennant [finances] \\
        \hline
        \textbf{avec un complément prépositionnel :} & \\
        \textit{auprès (de), autour (de), comme, de, derrière, devant, face (à), faute (de), grâce (à), hors (de), jusque, loin (de), par, près (de), pour, quant (à), suite (à), vers} & auprès [de vous], derrière [chez vous], grâce [à toi], hors [de la ville], jusque [vers midi], vers [chez vous] \\
        \hline
        \textbf{avec deux compléments :} & \\
        \textit{à, avec, contre, dès, sans} & avec [Paul], [aux commandes], une hausse de 10 \% contre [3 \%] [le mois dernier] \\
        \hline
        \multicolumn{2}{|c|}{\cellcolor[HTML]{F2F2F2}\textbf{COMPLEXES}} \\ % Sub-header for COMPLEXE
        \hline
        \textbf{avec un complément nominal :} & \\
        \textit{à même, à part, à travers, de, de par, d'ici, étant donné, il y a, par-dessus, par devers, par-dessous, par-delà, etc.} & à même [le sol], à travers [la forêt], il y a [trois mois], d'ici [le 21] \\
        \hline
        \textbf{avec un complément prépositionnel :} & \\
        \textit{à force (de), au lieu (de), à moins (de), au point (de), à côté (de), à cause (de), au-dessus (de), à l'insu (de), à l'instar (de), à longueur (de), au bout (de), au-delà (de), aux alentours (de), aux environs (de), au-dessous (de), à l'aide (de), au moyen (de), d'ici (à), en cas (de), en dépit (de), en bas (de), en haut (de), en dessous (de), en dehors (de), en face (de), en guise (de), à compter (de), en raison (de), à raison (de), de la part (de), à partir (de), en proie (à), de / par crainte (de), de / par peur (de), par rapport (à), vis-à-vis (de)} & à côté [de l'arbre], d'ici [à 2050], en face [de la gare], vis-à-vis [de nous], \\
        \hline
    \end{tabular}
    \caption{Les principales prépositions à complément nominal ou prépositionnel}
\end{table}

\subsubsection{les prépositions à deux compléments}
avec, dès, sans : un syntagme nominal + un complément prédicatif. 

complément nominal可解释为le sujet du complément prédicatif,可在其前也可在其后
\begin{enumerate}
    \item complément prédicatif est syntagme prépositionnel
    \begin{itemize}
        \item Sans [Marie] [avec nous], on va avoir du mal.
        \item Avec [aux commandes] [un type pareil], on est mal partis.
    \end{itemize}
    \item complément prédicatif est syntagme adjectival ou participe
    \begin{itemize}
        \item Avec [Jean] [malade], on est mal partis.
        \item Avec [Jean] [acquitté par la cour], on va pouvoir repartir du bon pied.
    \end{itemize}
\end{enumerate}


\subsection{Les prépositions à complément infinitif ou avec subordonnée}
\begin{table}[htbp]
    \centering
    \small
    \setlength{\extrarowheight}{2pt} % Adds a little extra height to rows
    \begin{tabular}{|>{\raggedright\arraybackslash}p{0.15\textwidth}|>{\raggedright\arraybackslash}p{0.4\textwidth}|>{\raggedright\arraybackslash}p{0.4\textwidth}|}
        \hline
        \rowcolor{cyan!20}
        \textbf{PRÉPOSITIONS} & \textbf{AVEC COMPLÉMENT INFINITIF} & \textbf{AVEC SUBORDONNÉE EN QUE} \\
        \hline
        simples & \textit{à, afin (de), après, avant (de), contre, de, faute (de), histoire (de), loin (de), pour, quant (à), quitte (à), sans} & \textit{afin (que), après (que), attendu (que), avant (que), depuis (que), dès (que), faute (que), histoire (que), moyennant (que), outre (que), pendant (que), pour (que), pourvu (que), sans (que), selon (que), suivant (que), vu (que)} \\
        \hline
        complexes & \textit{à condition (de), à force (de), au lieu (de), à moins (de), au point (de), de / par crainte (de), de / par peur (de), de manière (à), de façon (à), de là (à), d'ici (à), en sus (de)} & \textit{à condition (que), à force (que), au fur et à mesure (que), au lieu (que), à moins (que), au point (que), compte tenu (de / de ce que), de / par crainte (de), de façon (à ce que), d'ici (à ce que), de là (à ce que), de manière (à ce que), de sorte (que), du moment (que), étant donné (que), de / par peur (que)} \\
        \hline
    \end{tabular}
    \captionsetup{font=small} 
    \caption{Les principales prépositions à complément infinitif ou avec subordonnée}
\end{table}

\subsection{Les prépositions introductrices}
syntagme prépositionnel不能作为sujet或complément direct,只能作为complément oblique de verbe, de nom, d’adjectif, d’adverbe; attribut du sujet, du complément; ajout

但一些prépositions可以作为ajout或marquer引导其他成分 :

\begin{enumerate}
    \item syntagme adjective : 作为épithète ou attribut du complément
    \begin{itemize}
        \item quelque chose de facile
        \item considéré comme fou
    \end{itemize}
    \item syntagme adverbial : 作为épithète ou attribut
    \begin{itemize}
        \item une parole de trop
        \item Cette parole est de trop.
    \end{itemize}
    \item syntagme nominal : 作为sujet ou complément direct
    \begin{itemize}
        \item Jusqu’à sa mère le déteste.
        \item Personne n’a bu de vin.
    \end{itemize}
    \item syntagme verbal : 作为sujet ou complément direct
    \begin{itemize}
        \item De le voir dans cet état nous a fait de la peine.
        \item Paul cherche à vous joindre.
    \end{itemize}
    \begin{enumerate}
        \item 只有de能引导sujet
        \begin{itemize}
            \item À quoi sert [de partir maintenant] ?
        \end{itemize}
    \end{enumerate}
\end{enumerate}

\begin{table}[H]
    \centering
    \small
    \setlength{\extrarowheight}{2pt} % Adds a little extra height to rows
    \begin{tabular}{|>{\raggedright\arraybackslash}p{0.15\textwidth}|>{\raggedright\arraybackslash}p{0.4\textwidth}|>{\raggedright\arraybackslash}p{0.4\textwidth}|}
        \hline
        \rowcolor{cyan!20} % Light blue for header
        \textbf{SYNTAGME} & \textbf{PRÉPOSITIONS} & \textbf{EXEMPLES} \\
        \hline
        adjectival & \textit{comme, de, pour} & On le considère [comme responsable]. \newline Il passe [pour fou]. \\
        \hline
        adverbial & \textit{de, en} & Il a eu une parole [de trop]. \\
        \hline
        nominal & \textit{au-delà de, au-dessous de, au-dessus de, autour de, aux alentours de, aux environs de, au voisinage de, dans les, de, en dessous de, entre, jusqu'à, près de, non loin de, pas loin de, pour, vers les} & Il y avait [aux alentours de trente personnes]. \newline Il est arrivé [jusqu'à 100 patients dans la journée]. \\
        \hline
        verbal & \textit{à, comme, contre, de, en, en sorte (de), en train (de), jusqu'à, par, pour, près (de)} & Il continue [à pleuvoir]. \newline Elle a promis [de travailler davantage]. \newline J'irai [jusqu'à mentir pour toi]. \\
        \hline
    \end{tabular}
    \caption{ Les principales prépositions introductrices de syntagme adjectival, adverbial, nominal ou verbal}
\end{table}

\section{Les classes sémantiques de prépositions}
\subsection{Les prépositions de lieu}

\begin{table}[H]
    \centering
    \small
    \begin{tabularx}{\textwidth}{|>{\raggedright\arraybackslash}X|>{\raggedright\arraybackslash}X|}
    \hline
    \rowcolor{cyan!20}
    \textbf{SENS} & \textbf{FORMES} \\
    \hline
    \rowcolor{cyan!10}
    \multicolumn{2}{|l|}{\textbf{LOCALISATION AVEC OU SANS DÉPLACEMENT}} \\
    \hline
    vers une destination & \`{a}, apr\`{e}s, \`{a} travers, autour (de), avant, dans, de, derri\`{e}re, devant, en, le long (de), loin (de), pr\`{e}s (de), sous, sur, etc. \\
    \hline
    \rowcolor{cyan!10}
    \multicolumn{2}{|l|}{\textbf{LOCALISATION AVEC DÉPLACEMENT}} \\
    \hline
    vers une destination & direction, en destination (de), en direction (de), jusque, jusqu'\`{a}, pour, vers, etc. \\
    \hline
    \`{a} travers un passage & par, via, etc. \\
    \hline
    depuis une origine & \`{a} partir de, de, depuis, etc. \\
    \hline
    \end{tabularx}
    \caption{Le classement sémantique des principales prépositions locatives}
\end{table}

\subsection{Les prépositions temporelles
}


\begin{table}[H]
    \centering
    \small
    \begin{tabularx}{\textwidth}{|>{\raggedright\arraybackslash}p{2.5cm}|>{\raggedright\arraybackslash}p{2.8cm}|>{\raggedright\arraybackslash}X|>{\raggedright\arraybackslash}X|}
    \hline
    \rowcolor{cyan!20}
    \textbf{REPÈRE} & \textbf{RELATION TEMPORELLE} & \textbf{PRÉPOSITIONS} & \textbf{EXEMPLES} \\
    \hline
    \multirow{3}{*}{\textbf{date}} 
    & antériorité & \textit{avant, jusqu’à, d’ici, \% endéans} & avant le dîner, d’ici la fin du mois, \% endéans la fin du mois \\
    & simultanéité, inclusion & \textit{à, vers, en même temps (que), au moment (de), du temps (de)} & vers midi, du temps des Croisades \\
    & postériorité & \textit{après, dès, depuis, à compter (de), à partir (de), suite (à), \% endéans} & depuis hier, à compter d’aujourd’hui \\
    \hline
    \multirow{3}{*}{\textbf{durée}} 
    & antériorité & \textit{il y a, voici, voilà} & il y a un mois, voici trois ans \\
    & simultanéité, inclusion & \textit{avant, dans, de, depuis, d’ici, durant, en, par, pour, au cours (de), pendant, sous, \% endéans} & avant huit jours, de toute la nuit, depuis des années, dans les trois jours, d’ici (à) une semaine, sous huit jours, \% endéans la quinzaine \\
    & postériorité & \textit{après, dans, au bout (de)} & dans trois jours \\
    \hline
    \end{tabularx}
    \caption{L’interprétation des principales prépositions temporelles}
\end{table}

\clearpage
\subsection{Les prépositions abstraites}


\begin{table}[H]
    \centering

\small
\begin{longtable}{|m{4cm}|>{\itshape}m{6cm}|m{5cm}|}
\hline
\rowcolor{cyan!20}
\textbf{RELATION SÉMANTIQUE} & \textbf{\textup{PRÉPOSITIONS}} & \textbf{EXEMPLES} \\
\hline
\endfirsthead
\hline
\textbf{RELATION SÉMANTIQUE} & \textbf{PRÉPOSITIONS} & \textbf{EXEMPLES} \\
\hline
\endhead
\hline
\endfoot
\hline
\endlastfoot

absence, exclusion & à part, à l'exception (de), excepté, exception faite (de), hormis, sans, sauf & \textit{un voyage [sans les enfants]} \\
\hline
accompagnement, inclusion & avec, d'entre, parmi, y compris & \textit{une promenade [avec le chien]} \\
\hline
avantage & au profit (de), au bénéfice (de), en faveur (de), pour. & \textit{un modèle [pour chiens]} \newline \textit{des mesures [en faveur des réfugiés]} \\
\hline
désavantage & aux dépens (de), au détriment (de), contre & \textit{un vote [contre le projet]} \\
\hline
but, direction, motivation, orientation & à, afin (de), après, envers, pour, sur, vers & \textit{courir [après les honneurs]} \newline\textit{de la haine [envers quelqu'un]} \newline \textit{travailler [pour le plaisir]} \\
\hline
cause & de, devant, de par, grâce à, / pour cause (de), à force (de), au gré (de), en cas (de), en raison (de), en vertu (de), par, pour, suite (à) & \textit{agir [au gré de son humeur]} \newline \textit{rouge [de colère]} \newline \textit{une branche cassée [par le vent]} \\
\hline
comparaison, opposition & contre, d'avec, malgré, nonobstant, en dépit (de), au contraire (de), à la différence (de), à côté (de), au fur et à mesure (de), au prorata (de), par rapport (à) & \textit{un faible gain [par rapport aux frais]} \newline \textit{partir [malgré la chaleur]} \\
\hline
instrument, moyen & avec, à l'aide (de), au moyen (de), de, par & \textit{couper le pain [avec un couteau]} \\
\hline
manière & à la façon (de), à la manière (de), comme, façon, selon & \textit{travailler [à la manière d'autrefois]} \newline \textit{travailler [comme peintre]} \\
\hline
possession, transfert de possession & à, chez, en possession (de), pour & \textit{un livre [à moi]} \newline \textit{un cadeau [pour Marie]} \\
& & \textit{trouver du courage [chez Luc]} \\
\hline
prix, quantité, valeur & à, de, fois, moins, moyennant, par, plus, pour, au prix (de), à raison (de), à hauteur (de), sur & \textit{une bague [à 100 euros]} \newline \textit{un modèle [pour 20 euros]} \newline \textit{un sac [de 20 kilos]} \newline \textit{trois [fois trois]} \\
\hline
source du comportement ou de l'information & d'après, à l'instar (de), comme, en, en tant que, en qualité (de), selon & \textit{agir [selon ses principes]} \newline \textit{agir [d'après la loi]} \\
\hline
substitution & pour, au lieu (de), à la place (de), au nom (de), en replacement (de) & \textit{une école [au lieu d'un parking]} \newline \textit{un mort [pour un vivant]} \newline \textit{parler [au nom des enfants]} \\
\hline
topique & concernant, de, à propos (de), au sujet (de), à l'égard (de), envers, quant (à), sur & \textit{une décision [à propos des plantes]} \newline \textit{un livre [de maths]} \newline \textit{un traité [sur les plantes]} \\
\hline
\end{longtable}
\caption{Les principales classes sémantiques de prépositions abstraites}
\end{table}

\begin{enumerate}
    \item cause : à force (de), à/pour cause (de), compte tenu (de), grâce (à), en raison (de), par suite (de), étant donné, sous l’effet (de)
    \item finalité : afin (de), pour, histoire (de), de peur (de), de crainte (de), de façon (à), de manière (à), en sorte (que)
    \item concession : en dépit (de), malgré, nonobstant, sans (que)
    \item condition : à condition (que), à supposer (que), à moins (de), selon, suivant, en fonction (de), moyennant
    \item conséquence : au point (de), de sorte (que)
\end{enumerate}

\section{La structure du syntagme prépositionnel}

\subsection{Les compléments de la préposition}
\begin{enumerate}
    \item complément nominal, avec ou sans déterminant
    \item complément prépositionnel
    \item complément infinitif
    \item complément subordonnée à l’indicatif ou au subjonctif, généralement introduite par que
    \item plus rarement un complément adjectival
    \item avec deux compléments 
    \item sans complément
\end{enumerate}

\subsubsection{Le complément nominal}
à, après, avec, avant, chez, comme, contre, dans, de, depuis, derrière, dès, devant, durant, en, entre, envers, hormis, hors, malgré, outre, par, parmi, pendant, pour, sauf, sans, selon, sous, suivant, sur, vers

\begin{enumerate}
    \item en plein (正) + nom : sans déterminant ; plein可根据nom变为,也可作为agglomérée invariable qui peut accompagner une autre préposition comme ajout
    \begin{itemize}
        \item en pleine rue/* en rue
        \item en plein milieu/* en milieu
        \item en plein sur le crâne
        \item en plein dans le mille
    \end{itemize}
    \item de partitif结构中,de不能直接后跟pronom personnel,而要用d’entre
    \begin{itemize}
        \item ? certains de vous / de nous, * certains d’eux
        \item * plusieurs de nous/de vous/d’eux
        \item certains d’entre nous/d’entre vous/d’entre eux
    \end{itemize}
\end{enumerate}


\subsubsection{Le complément prépositionnel}
\begin{longtable}{|m{7cm}|m{8cm}|}
\hline
\textbf{COMPLÉMENT} & \textbf{EXEMPLES} \\
\hline
\endfirsthead
\hline
\textbf{COMPLÉMENT} & \textbf{EXEMPLES} \\
\hline
\endhead
\hline
\endfoot
\hline
\endlastfoot

introduit par \textbf{à} : & \\
\textit{d'ici, en proie, grâce, jusque, par rapport, quant, suite} & en proie [à l'angoisse], grâce [à vous], jusqu'[à midi], quant [à moi], suite [à votre rapport], d'ici [à Noël] \\
\hline
introduit par \textbf{de} : & \\
\textit{à cause, à compter, à côté, à force, à longueur, à l'aide, à l'insu, à moins, à partir, \% au fur, au fur et à mesure, auprès, autour, au bout, au-dessus, au-dessous, au lieu, au moyen, de la part, en bas, en dehors, en dépit, en face, en guise, en haut, en raison, faute, hors, loin, près, vis-à-vis} & auprès [de vous], à cause [de vous], en face [de la maison], faute [de place], hors [de ma vue], vis-à-vis [de vous] \\
\hline
introduit par une autre préposition : & \\
\textit{comme, de, depuis, derrière, devant, jusque, par, vers} & comme [chez vous], depuis [avant la guerre], devant [chez vous], jusque [dans trois jours], vers [chez Marie] \\
\hline
\end{longtable}


\subsubsection{Le complément adjectival }
en, depuis, entre, côté, à titre de, sous l’emprise de, sous contrôle de, en possession de.
\begin{itemize}
    \item voir les choses en [plus grand], depuis [tout petit]
    \item Dans ce film qui oscille entre très féministe et très misogyne
    \item côté [intellectuel], il n’est pas seulement doué, mais surdoué
    \item à titre d’ami/à titre amical
    \item sous l’emprise du diable / sous l’emprise diabolique, sous l’emprise divine
\end{itemize}


\subsubsection{L’infinitif et la subordonnée compléments}

\paragraph{Les compléments infinitifs de préposition}

\begin{table}[H]
    \centering
    \begin{longtable}{|m{7cm}|m{8cm}|}
    \hline
    \textbf{COMPLÉMENT} & \textbf{EXEMPLES} \\
    \hline
    \endfirsthead
    \hline
    \textbf{COMPLÉMENT} & \textbf{EXEMPLES} \\
    \hline
    \endhead
    \hline
    \endfoot
    \hline
    \endlastfoot
    
    infinitif : & \\
    \textit{à, après, contre, de, entre, pour, sans} & après [être parti], pour [bien réussir], sans [mentir] \\
    \hline
    infinitif introduit par \textbf{à} : & \\
    \textit{de façon, de là, d'ici, de manière, quant, quitte} & de là [à mentir], de manière [à réussir], quant [à partir], quitte [à échouer] \\
    \hline
    infinitif introduit par \textbf{de} : & \\
    \textit{afin, à condition, à force, à moins, au lieu, au point, de peur, en dehors, en train, faute, histoire, loin, par crainte, sur le point} & afin [de réussir], à force [de travailler], loin [de mentir], faute [de réussir] \\
    \hline
    \end{longtable}
    \caption{Les principales prépositions à complément infinitif}
\end{table}

\paragraph{Les subordonnées compléments de préposition}

\begin{table}[H]
    \centering
    \begin{longtable}{|m{7cm}|m{8cm}|}
    \hline
    \textbf{SUBORDONNÉE} & \textbf{EXEMPLES} \\
    \hline
    \endfirsthead
    \hline
    \textbf{SUBORDONNÉE} & \textbf{EXEMPLES} \\
    \hline
    \endhead
    \hline
    \endfoot
    \hline
    \endlastfoot
    
    
    
    en \textbf{que} (indicatif) : & \\
    \textit{\%à cause, à mesure, à moins, au lieu, après, à part, au point, attendu, au fur et à mesure, compte tenu, d'autant, dès, depuis, de même, du fait, du moment, étant donné, excepté, \% enre, hormis, \% moyennant, outre, pendant, sauf, selon, suivant, une fois, vu} & attendu [que tu es parti], depuis [que tu es parti], dès [que tu pourras], pendant [que tu lis], sauf [qu'il était trop tard], selon [que tu viendras ou non] \\
    \hline
    en \textbf{que} (subjonctif) : & \\
    \textit{afin, \%après, avant, à condition, à moins, au point, de, par crainte, de (par peur, de (telle) sorte, en sorte, excepté, \%faute, histoire, hormis, pour, pour peu, pourvu, sans, sauf} & à condition [que tu viennes], afin [que tu viennes], avant [que tu partes], à moins [que tu viennes], histoire [que tu saches], pour [que tu saches], sans [que ce soit prévu] \\
    \hline
    en \textbf{ce que} : & \\
    \textit{en, jusqu'à} & en [ce que rien n'a été fait], jusqu'à [ce que tu viennes] \\
    \hline
    en \textbf{à ce que} : & \\
    \textit{de façon, d'ici, de là, de manière, quitte, sauf} & de façon [à ce que tout soit prêt], d'ici [à ce qu'on l'engage], quitte [à ce qu'on vienne], sauf [à ce qu'il pleuve] \\
    \hline
    en \textbf{de ce que} : & \\
    \textit{à cause} & \%à cause [de ce qu'il pleuvait] \\
    \hline
    autre : & \\
    \textit{à, de, excepté, hormis, sauf, selon} & excepté [quand il pleut], sauf [si tu viens] \\
    \hline
    \end{longtable}
    \caption{Les subordonnées compléments de préposition}
\end{table}

\begin{enumerate}
    \item préposition后可跟subordonnée interrogative, introduite par si ou un mot interrogatif
    \begin{itemize}
        \item \% La santé dépend de [si vous avez eu des enfants]
        \item \% Ton avenir selon [comment tu fumes] ?
    \end{itemize}
    \item  Certains locuteurs acceptent une subordonnée exclamative après à ou de
    \begin{itemize}
        \item \% Quand j’ai une envie de craquer je pense à [comme je serai bien dans mes jeans].
    \end{itemize}
\end{enumerate}

\subsubsection{Les prépositions à deux compléments}
à, avec, sans, dès, contre


\paragraph{à}
\begin{enumerate}
    \item un premier complément nominal et un second complément prépositionnel en emploi locatif ou temporel, la préposition du second complément est souvent \textit{de}, mais pas toujours
    
    \begin{itemize}
        \item Il s’est arrêté [à [un mètre] [avant l’arrivée]].
        \item Il s’est blessé [à [une seconde] [de la fin]].
        \item Il faut les mettre [à [une certaine distance] [de la clôture]].
    \end{itemize}
    \item à + syntagme nominal + près
    \begin{itemize}
        \item Il est [à [un euro] [près]].
        \item Il a été élu [à [une voix près]].
        \item Il a raté son train [à [une seconde] [près]].
        \item Il s’est arrêté [à [un mètre] [près]].
    \end{itemize}
\end{enumerate}


\paragraph{avec, sans, dès}
Le premier complément est nominal (ou infinitif ou même subordonné). Le second est un prédicat de catégorie variée : adjectif, SP, infinitif introduit par à, participe. 二者前后顺序也可调换

\begin{itemize}
    \item avec Paul malade, sans Paul disponible
    \item avec Paul aux commandes, sans Paul ici
    \item avec Paul à remplacer, sans personne à inviter
    \item dès le seuil franchi, avec Paul parti, dès Paul levé
    \item avec [sur la tête] [un chapeau], avec [aux commandes] [un imbécile pareil]
    \item avec [pour seul but] [de s’enrichir]
    \item avec [pour unique condition] [que son salaire soit triplé]
\end{itemize}

\paragraph{Contre}
Lorsque contre signale un contraste, il peut être suivi de deux compléments

\begin{itemize}
    \item L’inflation va augmenter de 1 \% [contre [0,5 ] [l’an dernier]].
    \item Le salaire minimum était ici à 1 000 euros [contre [800 euros] [en Allemagne]].
    \item Le revenu minimum est à 800 euros en travaillant [contre [500 euros] [sans rien faire]].
\end{itemize}

\subsection{nom + préposition + nom}
à, après, contre, par, pour, sur

\subsubsection{La fonction syntaxique}
\begin{enumerate}
    \item un complément prépositionnel
    \begin{itemize}
        \item On met tout ça [bout à bout]. On a rangé les disques [face contre face].
        \item Dans le match entre l’herbe et le maïs, les deux fourrages sont renvoyés [dos à dos]
    \end{itemize}
    \item un ajout prépositionnel à un verbe ou à un nom, parfois entre une préposition et son complément
    \begin{itemize}
        \item danser [joue contre joue], travailler [enfant par enfant]
        \item trois ans [jour pour jour]
        \item Les choses se précipitèrent véritablement à la demi-heure de jeu avec [coup sur coup] deux buts niortais.
    \end{itemize}
    \item un attribut du sujet ou du complément
    \begin{enumerate}
        \item en \textit{à} ou \textit{contre} exprimant la juxtaposition
        \begin{itemize}
            \item Ils restaient [joue contre joue], [face à face].
            \item On les a trouvés [dos à dos].
        \end{itemize}
        \item \textit{pour} et en \textit{contre} exprimant l’échange ou l’opposition avec un sujet désignant une situation (ce)
        \begin{itemize}
            \item C’est [œil pour œil], [dent pour dent], [parole contre parole].
        \end{itemize}
    \end{enumerate}
    \item un sujet nominal ou un complément direct
    \begin{enumerate}
        \item \textit{après} et \textit{sur} exprimant l’accumulation peuvent être compléments directs, et plus difficilement sujets
        \begin{itemize}
            \item Vous n’aurez pas le temps de vous reposer tandis que vous servirez [client après client],
            \item Je fume [cigarette sur cigarette], je bois [verre après verre].
            \item ? Élève après élève se présentaient à l’infirmerie.
        \end{itemize}
        \item \textit{pour} et \textit{contre} exprimant l’échange 只能用于一些verbe的complément direct,不能作为sujet au passif
        \begin{itemize}
            \item * Coup pour coup a été rendu.
            \item * Œil pour œil a été exigé.
        \end{itemize}
    \end{enumerate}
\end{enumerate}

\subsubsection{L’interprétation des constructions}
\paragraph{Exprimant l’accumulation}
\textit{après, sur}
\begin{itemize}
    \item heure après heure = des heures et des heures
    \item milliers sur milliers = des milliers et milliers
\end{itemize}

\paragraph{Exprimant la distribution, la progression spatiale ou temporelle}
\begin{enumerate}
    \item \textit{à, par}exprimant la distribution peuvent se construire avec un cardinal (un à un, deux par deux), aussi une progression spatiale ou temporelle
    \begin{itemize}
        \item Ils avancent [pas à pas], mètre par mètre
        \item Ils progressent minute par minute
    \end{itemize}
    \item \textit{après} exprimant la succession temporelle peut aussi se construire avec l’autre ou l’un et l’autre
    \begin{itemize}
        \item un jour après l’autre, chaque jour l’un après l’autre
    \end{itemize}
\end{enumerate}

\paragraph{Exprimant la juxtaposition}
\textit{contre}, surtout avec les noms de partie du corp,此时可用noms différents ou des déterminants
\begin{itemize}
    \item front contre joue, face contre terre
    \item le front contre la joue, la face contre la terre
\end{itemize}

\subsection{Les prépositions additives et exceptives}
additif (outre), inclusif (y compris) ou exceptif (hormis, sauf)
\subsubsection{Les compléments des prépositions additives et exceptives}
\begin{enumerate}
    \item adjectival
    \begin{itemize}
        \item Il est tout sauf bête.
    \end{itemize}
    \item adverbial
    \begin{itemize}
        \item Il n’est jamais venu, sauf récemment.
    \end{itemize}
    \item nominal
    \begin{itemize}
        \item Tous étaient là, sauf Marie.
    \end{itemize}
    \item prépositionnel
    \begin{itemize}
        \item Elle a cherché partout, sauf dans le jardin.
    \end{itemize}
    \item infinitif
    \begin{itemize}
        \item Elle veut tout sauf partir.
    \end{itemize}
    \item subordonnée
    \begin{itemize}
        \item Il a tout dit, sauf qu’il n’était pas à Paris.
    \end{itemize}
\end{enumerate}


\section{Les fonctions du syntagme prépositionnel}
\begin{enumerate}
    \item attribut du sujet
    \item attribut du complément
    \item complément oblique
    \item ajout
    \item extrait
    \item périphérique
    \item tête de phrase sans verbe
\end{enumerate}

\subsection{Les compléments prépositionnels}

\subsubsection{Les attributs prépositionnels}
\begin{enumerate}
    \item verbes attributifs  (être, rester, sembler)后,介词引导attribut du sujet
    \begin{enumerate}
        \item à表示possessif
        \begin{itemize}
            \item Le livre est [à Marie].
        \end{itemize}
        \item de表示le créateur d’un objet fabriqué
        \begin{itemize}
            \item Ce livre est [de Marie].
        \end{itemize}
        \item pour et contre 表示opinion (赞成/反对)
        \begin{itemize}
            \item Je suis [pour cette proposition].
        \end{itemize}
        \item avec et sans表示accompagnement
        \begin{itemize}
            \item Paul est [avec Marie].
        \end{itemize}
        \item 也可用préposition locative
        \begin{itemize}
            \item Paul est [à Paris].
        \end{itemize}
    \end{enumerate}
    \item 其他动词后,介词引导attribut du complément
    \begin{itemize}
        \item On sait Marie [aux abois].
        \item J’ai trouvé Paul [en forme].
    \end{itemize}
\end{enumerate}


\section{Les prépositions à et de}
\subsection{Les prépositions à et de comme marqueurs}

\subsubsection{À et de suivies d’un verbe infinitif}

\begin{table}[H]
    \centering
    \begin{longtable}{|m{4cm}|m{5.5cm}|m{5.5cm}|}
    \hline
    \rowcolor{cyan!20}
    \textbf{PRÉPOSITION SUIVIE DE} & \textbf{MARQUEUR} & \textbf{TÊTE DE SYNTAGME PRÉPOSITIONNEL} \\
    \hline
    \endfirsthead
    \hline
    \rowcolor{cyan!20}
    \textbf{PRÉPOSITION SUIVIE DE} & \textbf{MARQUEUR} & \textbf{TÊTE DE SYNTAGME PRÉPOSITIONNEL} \\
    \hline
    \endhead
    \hline
    \endfoot
    \hline
    \endlastfoot
    
    nom \newline ou syntagme nominal & \textit{Je n'ai pas [de farine].} \newline \textit{J'ai beaucoup [lu [de livres]].} & \textit{un verre [à vin]} \newline \textit{un verre [de vin]} \\
    \hline
    adjectif \newline ou syntagme adjectival & \textit{quelqu'un [de malade]} & \textit{[De pâle], elle est devenue blanche.} \newline \textit{[À malin], malin et demi.} \\
    \hline
    adverbe \newline ou syntagme adverbial & \textit{quelque chose [de plus]} & \textit{À bientôt !} \newline \textit{Ça date [de longtemps].} \\
    \hline
    syntagme prépositionnel & \textit{quelqu'un [de plus en forme].} & \textit{Il pense [à après le match].} \newline \textit{Il vient [de chez Max].} \\
    \hline
    infinitif \newline ou syntagme verbal & \textit{Paul cherche [à venir].} \newline \textit{Paul promet [de venir].} & \textit{[À le voir ainsi], on l'aurait cru mort.} \newline \textit{[De le voir ainsi], j'ai craint le pire.} \\
    \hline
    \end{longtable}
    \caption{Les fonctions syntaxiques des prépositions à et de}
\end{table}

\paragraph{À ou de + infinitif complément de verbe}
\begin{enumerate}
    \item 作为complément direct时用cela指代
    \begin{itemize}
        \item Jean promet/dit/craint [de venir].|Jean promet/dit/craint cela.
        \item Jean demande/cherche/apprend [à parler].|Jean demande/cherche/apprend cela.
    \end{itemize}
    \begin{enumerate}
        \item 仅部分verbes允许用le指代
        \begin{itemize}
            \item Paul le promet, de venir.
            \item * Paul le cherche, à parler.
        \end{itemize}
    \end{enumerate}
    \item 作为complément oblique时用en, y de cela, à cela指代
    \begin{itemize}
        \item Jean tient/renonce [à venir].|Jean tient/renonce à cela.
    \end{itemize}
\end{enumerate}

\paragraph{À ou de + infinitif complément de préposition}
\begin{enumerate}
    \item 一些介词后直接跟infinitif
    \begin{itemize}
        \item après [être venu]/* après [d’être venu]
    \end{itemize}
    \item avant, afin, faute, près后跟用de引导的infinitif
    \begin{itemize}
        \item  avant [de venir]/* avant [venir]
    \end{itemize}
    \item jusque, quant, quitte, sauf后跟用à引导的infinitif
    \begin{itemize}
        \item J’irai jusqu’[à affirmer que vous avez tort]/* jusqu’[affirmer que vous avez tort].
        \item Pas de commentaire particulier sauf [à dire que la prestation était conforme à mes attentes].
    \end{itemize}
    \item de là + à infinitif / à ce que (subordonnée au subjonctif)
    \begin{enumerate}
        \item comme un ajout à la phrase et introduit un topique (équivalent à ‘quant à’)
        \begin{itemize}
            \item De là [à dire que cela explique la perte totale de l’avion], c’est prématuré.
            \item L’idée de sauter à l’élastique n’était pas nouvelle, mais de là [à faire le grand saut], la question ne s’était jamais posée
        \end{itemize}
        \item comme un énoncé incomplet (sous-entendu : ‘il y a un pas’) 
        \begin{itemize}
            \item À propos de la vigne De là [à en déduire que le réchauffement climatique facilite son retour]...
            \item Certes, Radio-bois-patate était un moyen efficace de pister les gens et même d’éventer leurs petits secrets, mais de là à ce qu’elle fût capable de savoir
        \end{itemize}
    \end{enumerate}
\end{enumerate}



\subsubsection{Les constructions à ce que et de ce que}
\paragraph{subordonnants à ce que et de ce que}
\begin{enumerate}
    \item 引导compléments de nom, de verbe, d’adjectif 
    \item 用于verbe support : avoir peur de ce que, avoir intérêt à ce que, avoir égard à ce que
    \item 用于一些固定表达 : il n’y a que des avantages/pas d’obstacle/rien de mal à ce que
\end{enumerate}


\begin{table}[H]
    \centering
    \begin{longtable}{|m{7cm}|m{8cm}|}
    \hline
    \rowcolor{cyan!20}
    \textbf{SUBORDONNÉE} & \textbf{EXEMPLES} \\
    \hline
    \endfirsthead
    \hline
    \rowcolor{cyan!20}
    \textbf{SUBORDONNÉE} & \textbf{EXEMPLES} \\
    \hline
    \endhead
    \hline
    \endfoot
    \hline
    \endlastfoot
    \rowcolor{cyan!20}
    \multicolumn{2}{|c|}{\textbf{COMPLÉMENT DIRECT}} \\ 
    \hline
    en \textbf{ce que} + subjonctif : & \\
    \textit{aimer, chercher, viser} & \textit{Luc aime [\`{a} ce que tout soit pr\^{e}t].} \newline \textit{Luc cherche [\`{a} ce que tout soit pr\^{e}t].} \\
    \hline
    \rowcolor{cyan!20}
    \multicolumn{2}{|c|}{\textbf{COMPLÉMENT OBLIQUE}} \\ 
    \hline
    en \textbf{de ce que} + indicatif : & \\
    \textit{se plaindre, profiter, se souvenir , r\^{e}ver , rager , venir} & \textit{L\'{e}a se plaint [de ce que rien n'est pr\^{e}t].} \\
    \hline
    en \textbf{de ce que} + subjonctif : & \\
    \textit{douter , se d\^{e}soler, se lamenter, se plaindre, se r\'{e}jouir, rire} & \textit{L\'{e}a se plaint [de ce que rien ne soit pr\^{e}t].} \\
    \hline
    en \textbf{\`{a} ce que} + indicatif : & \\
    \textit{penser , songer, tenir} & \textit{L\'{e}a songe [\`{a} ce que tout sera fini demain].} \\
    \hline
    en \textbf{\`{a} ce que} + subjonctif : & \\
    \textit{aboutir, s'attendre, encourager, s'engager, forcer, renoncer, revenir, tenir, travailler, veiller} & \textit{Luc tient [\`{a} ce que tout soit pr\^{e}t].} \newline \textit{L\'{e}a nous encourage [\`{a} ce que tout soit pr\^{e}t].} \newline \textit{\c{C}a revient [\`{a} ce que tout soit pr\^{e}t demain].} \\
    \hline
    \end{longtable}
\end{table}






\paragraph{Les relatives en ce que}
注意区分subordonnées complétive与ce引导的subordonnée relative :
\begin{enumerate}
    \item subordonnée relative本身不完整
    \begin{itemize}
        \item Elle se plaint [de [ce [que tu lui as dit ] ] ].
        \item Elle pense [à [ce [que vous lui avez raconté ] ] ].
        \item À ce que j’ai compris, ils ont fait la fête
    \end{itemize}
    \item subordonnées complétive是完整的
    \begin{itemize}
        \item Elle se plaint [de ce que tu lui as dit de partir].
        \item Elle pense [à ce que vous devriez partir].
    \end{itemize}
\end{enumerate}

\paragraph{À et de devant d’autres subordonnées}
À et de虽然不能直接引导complétive en que,但其在非正式用法中可直接引导interrogative en si,exclamative en comme,ou une subordonnée de temps 
\begin{enumerate}
    \item \% Temps de travail par semaine + salaire = ça dépend de [si on est agrégé ou certifié]. 
    \item \% Quand j’ai une envie de craquer je pense à [comme je serai bien dans mes jeans]. 
    \item \% Je me souviens de [comme il nous avait fait danser].
    \item \% Ça date de [quand on est partis en vacances, à la montagne ]
\end{enumerate}

\section{Les prépositions locatives}


\begin{table}[H]
    \centering
    \begin{longtable}{|m{7cm}|m{8cm}|}
    \hline
    \rowcolor{cyan!20}
    \textbf{PRÉPOSITIONS} & \textbf{EXEMPLES} \\
    \hline
    \endfirsthead
    \hline
    \rowcolor{cyan!20}
    \textbf{PRÉPOSITIONS} & \textbf{EXEMPLES} \\
    \hline
    \endhead
    \hline
    \endfoot
    \hline
    \endlastfoot
    
    faibles : & \\
    à, de, en & \textit{Il habite [en France], à Paris].} \newline \textit{Il sort [de chez lui].} \\
    \hline
    formées par conversion : & \\
    côté, destination, direction & \textit{Il habite [côté cour].} \newline \textit{Il est parti [direction Marseille].} \\
    \hline
    autres prépositions simples : & \\
    auprès (de), autour (de), chez, contre, dans, dedans, dehors, derrière, dessous, dessus, devant, entre, face (à), hors, ici, jusque, là, loin (de), par, parmi, près (de), sous, sur, vers, via & \textit{Il tourne [autour du rondpoint].} \newline \textit{Il rentre [chez lui].} \newline \textit{Il est [loin de chez lui].} \\
    \hline
    agglomérées : & \\
    à travers, au-dedans (de), au long (de), jusqu'à, par-delà, par devers, à même, en bas (de), en face (de), à côté (de), au coin (de), au-dessous (de), en-dessous (de), au-dessus (de), en deçà (de), au-devant (de), au-delà (de), en dehors (de), vis-à-vis (de), le long (de), à partir (de) & \textit{Il court [à travers les bois].} \newline \textit{Il grimpe [le long du mur].} \newline \textit{Il tourne [au coin de la rue].} \\
    \hline
    locutions prépositionnelles : & \\
    au bout (de), par le bout (de), à (la) gauche (de), à (la) droite (de), sur le coin (de), dans l'enceinte (de), à l'extérieur (de), à l'extrémité (de), à l'intérieur (de) & \textit{Il entre [à l'intérieur de la pièce].} \newline \textit{Il tourne [à gauche de la gare].} \\
    \hline
    \end{longtable}
    \caption{Les classes morphologiques de prépositions locatives}
\end{table}

\subsection{La construction des prépositions locatives}
Le tableau inclut toutes les prépositions simples, mais non toutes les prépositions agglomérées, et encore moins toutes les locutions.
\begin{table}[H]
    \centering
    \begin{longtable}{|m{2.5cm}|m{6cm}|m{6cm}|}
    \hline
    \rowcolor{cyan!20}
    \textbf{COMPLÉMENT} & \textbf{OBLIGATOIRE} & \textbf{FACULTATIF} \\
    \hline
    \endfirsthead
    \hline
    \rowcolor{cyan!20}
    \textbf{COMPLÉMENT} & \textbf{OBLIGATOIRE} & \textbf{FACULTATIF} \\
    \hline
    \endhead
    \hline
    \endfoot
    \hline
    \endlastfoot
    
    nominal & à, chez, côté, dans, de, depuis, dès, direction, durant, en, \% endéans, \% jusque, outre, par, parmi, passé, pendant, pour, ras, sous, sur, vers, via & avant, devant, contre, après, derrière, entre, à travers, par-dessus, par-dessous \\
    \hline
    prépositionnel & auprès (de), au coin (de), dans l'enceinte (de), de, hors (de), jusque, le long (de), par, vers & au long (de), autour (de), à l'avant (de), à l'arrière (de), à droite (de), à gauche (de), au bout (de), au fond (de), à côté (de), au-dessous (de), au-dessus (de), en bas (de), en dessous (de), en face (de), loin (de), près (de) \\
    \hline
    sans complément & ici, là, ailleurs, alentour, dedans, dehors, partout & \\
    \hline
    \end{longtable}
    \caption{Quelques prépositions locatives et leur complément}
    \label{tab:pc}
\end{table}

\chapter{Verbe}
\section{Conjugaison}
\begin{table}[H]
\centering
\small
\setlength{\tabcolsep}{3pt}
\renewcommand{\arraystretch}{1.2}

\begin{tabularx}{\textwidth}{|l|X|X|X|X|X|X|X|X|}
\hline
\rowcolor{cyan!20}
\multicolumn{1}{|c|}{} & \multicolumn{4}{c|}{\textbf{Indicative}} & \multicolumn{2}{c|}{\textbf{Subjunctive}} & \multicolumn{1}{c|}{\textbf{Conditional}} & \multicolumn{1}{c|}{\textbf{Imperative}} \\
\cline{2-9}
\rowcolor{cyan!20}
\multicolumn{1}{|c|}{} & \textbf{Present} & \textbf{Simple past} [\textbf{PAST}] & \textbf{Imperfect} [\textbf{1P}] & \textbf{Future} [\textbf{FUT}] & \textbf{Present} & \textbf{Imperfect} [\textbf{PAST}] & \textbf{Present} [\textbf{FUT}] & \textbf{Present} \\
\hline
\textbf{je} & \textbf{1S}+\textit{e / s / x} & +\textit{ai / ( )s} & +\textit{ais} & +\textit{rai} & \textbf{3P/SUBJ}+\textit{e} &  +\textit{( )sse} & +\textit{rais} & \\
\hline
\textbf{tu} & \textbf{1S}+\textit{es / s / x} & +\textit{(a)s} & +\textit{ais} & +\textit{ras} & \textbf{3P/SUBJ}+\textit{es} & +\textit{( )sses} & +\textit{rais} & \textbf{1S}+\textit{s/t} \textbf{SUBJ}+\textit{e}\\
\hline
\textbf{il/elle} & \textbf{1S}+\textit{e / t /-} & +\textit{a / t} & +\textit{ait} & +\textit{ra} & \textbf{3P/SUBJ}+\textit{e} & +\textit{(\^{})t} & +\textit{rait} & \\
\hline
\textbf{nous} & \textbf{1P}+\textit{ons} & + \textit{(\^{})mes} & +\textit{ions} & +\textit{rons} & \textbf{1P/SUBJ} +\textit{ions}& + \textit{( )ssions} & +\textit{rions} & \textbf{1P/SUBJ} +\textit{ons} \\
\hline
\textbf{vous} & \textbf{1P}+\textit{ez} & + \textit{(\^{})tes} & +\textit{iez} & +\textit{rez} & \textbf{1P/SUBJ} +\textit{iez} & + \textit{( )ssiez} & +\textit{riez} & \textbf{1P/SUBJ} +\textit{ez} \\
\hline
\textbf{ils/elles} & \textbf{3P}+\textit{ent} & + \textit{(\^{})rent} & +\textit{aient} & +\textit{ront} & \textbf{3P/SUBJ} +\textit{ent} & + \textit{( )ssent} & +\textit{raient} & \\
\hline
\end{tabularx}
\end{table}


\begin{itemize}
    \item 1P:Infinitve stem
    \item FUT:Full infintive
\end{itemize}

\begin{enumerate}
    \item Simple Past与Subjuntiv Imperfect中的元音
    \begin{enumerate}
        \item er结尾:a 
        \item ir结尾:i 
        \item oir结尾:u 
        \begin{itemize}
            \item courir与mourir属于该范畴
        \end{itemize}
        \item enir结尾:in
    \end{enumerate}
\end{enumerate}

\section{La valence des verbes}


\begin{table}[H]
    \centering
    \begin{tabular}{>{\RaggedRight}p{3cm} >{\RaggedRight}p{6.5cm} >{\RaggedRight}p{4cm}}
    \toprule
    \textbf{FONCTION} & \textbf{CATÉGORIE} & \textbf{PROFORMES} \\
    \hline
    complément direct & adjectif, adverbe, syntagme nominal, infinitif et subordonnée & le, la, les, en, me, te, se, nous, vous \\
    \hline
    complément oblique & adjectif, adverbe, syntagme nominal, syntagme prépositionnel, infinitif et subordonnée & lui, leur, y, en, me, te, se, nous, vous \\
    \hline
    {attribut du sujet} & adjectif, adverbe, syntagme nominal, syntagme prépositionnel et infinitif & le \\
    \hline
    attribut du complément & adjectif, adverbe, syntagme nominal, syntagme prépositionnel et infinitif & — \\
    \bottomrule
    \end{tabular}
\end{table}

\subsection{Les verbes et leur sujet}


\subsubsection{Les verbes impersonnels}
\begin{enumerate}
    \item Les verbes atmosphériques ou météorologiques
    \begin{enumerate}
        \item verbes simples météorologiques
        \begin{itemize}
            \item pleuvoir 下雨, bruiner 毛毛雨, pleuviner 口语, 极轻微降雨, pleuvioter 口语, 间断性小雨, crachiner 雨夹细雪/冰粒, brouillasser 口语, 薄雾细雨混合, dracher 倾盆大雨, flotter 大雨如注
            \item neiger 下雪, neigeoter 小雪, 飘雪, grêler 下冰雹, givrer 结霜, verglacer 地面结薄冰, geler 结冰/极低温
            \item biser 刮冷风, venter 刮风
            \item tonner 打雷, éclairer 闪电
        \end{itemize}
        \item faire + nom /adjectif météorologiques
        \begin{itemize}
            \item Il fait chaud/froid/beau/sombre.
            \item Il fait soleil/nuit/grand jour/beau temps/un temps affreux.
        \end{itemize}
    \end{enumerate}
\end{enumerate}

\subsubsection{Les verbes à sujet infinitif ou subordonné}


\begin{tabular}{|>{\RaggedRight}p{6cm}|>{\RaggedRight}p{9cm}|} % 两列,带竖线,并自动换行和左对齐
\hline 
\rowcolor{cyan!20}
\textbf{VALENCE} & \textbf{EXEMPLES} \\
\hline
\textbf{sans complément :} \newline \textit{arriver, survenir} & [Qu'il neige en avril] arrive parfois. \\
\hline
\textbf{avec un complément nominal :} \newline \textit{alourdir, améliorer, amuser, causer, changer, constituer, couter, ennuyer, entraîner, étonner, impliquer, nécessiter, réclamer, réjouir, résoudre, signifier, supposer, valoir} & [Que Marie soit partie] complique [les choses]. \newline [Que Marie soit partie] ennuie [Paul]. \newline [Partir demain] réclame [toute une organisation]. \\
\hline
\textbf{avec un complément prépositionnel :} \newline \textit{conduire (à), convenir (à), découler (de), dépendre (de), déplaire (à), importer (à), influer (sur), plaire (à), suffire (à), sortir, venir (de)} & [Persévérer] conduirait [à l'échec]. \newline [Que Marie soit partie] déplaît [à Paul]. \\
\hline
\textbf{avec un complément infinitif ou phrastique :} \newline \textit{commencer (à), conduire (à), être, finir (par), ressembler (à), revenir (à)} & [Souffler] n'est pas [ jouer]. \newline [Partir] reviendrait [à démissionner]. \newline [Que tout va mal] commence [à se savoir]. \\
\hline % 表格底部的横线
\end{tabular}



\subsection{Les verbes sans complément}

\begin{enumerate}
    \item Verbes météorologiques
    \item Verbes d’émission (de son, de lumière ou de substance) 
    \begin{itemize}
        \item briller闪耀, clignoter闪烁, éternuer打喷嚏, respirer呼吸, ronfler打鼾, scintiller闪烁, tousser咳嗽
    \end{itemize}
    \item Verbes d’activité
    \begin{itemize}
        \item danser, dormir, délirer胡言乱语, épiloguer冗长论述, s’éterniser拖延, flâner闲逛, gaffer说错话, jeuner禁食, patienter耐心等待, piqueniquer野餐, rager愤怒, souffrir受苦, somnoler打瞌睡, sommeiller浅睡, travailler, vivre生活居住
    \end{itemize}
    \item Verbes de changement d’état
    \begin{itemize}
        \item arriver发生, se détériorer恶化, dépérir衰弱, s’évanouir昏倒, empirer加重, expirer呼气,到期, péricliter衰败, maigrir变瘦, naitre出生, mourir死亡, se suicider自杀
    \end{itemize}
    \item Verbes de mouvement sur place
    \begin{itemize}
        \item gigoter(小孩)乱蹬腿, sourciller挑眉, sursauter惊跳, trembler颤抖, trépigner跺脚
    \end{itemize}
    \item Verbes de mode de déplacement 
    \begin{itemize}
        \item boiter跛行, claudiquer蹒跚, se dandiner摇摆走路, gambader蹦跳, virevolter旋转
    \end{itemize}
\end{enumerate}

\subsection{Les verbes à complément nominal ou prépositionnel}

\subsubsection{Les verbes à un complément nominal}

\subsubsection{Les verbes à un complément prépositionnel}
\begin{enumerate}
    \item complément oblique ou attribut 
    \item à, après, avant, avec, chez, comme, contre, dans, d’avec, de, d’entre, depuis, devant, derrière, en, entre, envers, jusque, par, parmi, pour, sans, sous, sur, vers
\end{enumerate}


\begin{table}[H]
    \centering
    \small % Optional: Adjust font size
    \renewcommand{\arraystretch}{1.0} % Keep default or adjust slightly if desired

    \begin{tabularx}{\textwidth}{|l|X|X|} % X columns for auto-wrapping and width distribution
    \hline
    \rowcolor{cyan!20}
    \textbf{PRÉPOSITIONS} & \textbf{VERBES} & \textbf{EXEMPLES} \\
    \hline
    % Each example for 'à' is now on a new line using \newline within the same cell
    à & \textit{aller, convenir, correspondre, jouer, manquer, mentir, nuire, obéir, parler, penser, renoncer, ressembler, servir, sourire, suffire, tenir}
    & Cette robe va {à Marie}.\newline Paul ressemble {à son frère}.\newline Cette pièce sert {à mon frère}. \\
    \hline
    après & \textit{s'acharner, \%chercher, courir, \%demander, passer, pleurer, venir}
    & Les parents passent {après les enfants}.\newline Léa court {après la renommée}. \\
    \hline
    avec & \textit{discuter, négocier, parler, permuter, rimer}
    & Paul parle {avec Marie}.\newline Paul négocie {avec Marie}. \\
    \hline
    contre, pour & \textit{échanger, opter, parier, se prononcer, voter}
    & Paul opte {pour votre proposition}.\newline Paul vote {contre votre proposition}. \\
    \hline
    de & \textit{convenir, différer, discuter, douter, jouer, manquer, parler, remplir, rêver, rire, servir, tenir, traiter, se vanter, vider}
    & Paul parle {de son travail}.\newline Paul manque {de place}.\newline Cette pièce sert {de débarras}. \\
    \hline
    en & \textit{consister, dégénérer, finir, muer, partir}
    & Ce logement consiste {en deux pièces}.\newline Le jeu dégénère {en pugilat}. \\
    \hline
    par & \textit{commencer, débuter, finir, terminer}
    & Le spectacle commence {par un concert}. \\
    \hline
    sur & \textit{s'accorder, s'acharner, agir, anticiper, s'appuyer, compter, cracher, crier, discourir, s'entendre, foncer, hurler, interroger, insister, marcher, médire, mentir, parler, passer, pleurer, taper, tirer, régner, se reposer, revenir, sauter, tomber}
    & Paul compte {sur son frère}.\newline Paul insiste {sur sa proposition}.\newline Paul tire {sur un pigeon}.\newline Luc saute {sur l'occasion}. \\
    \hline
    autres prépositions & \textit{choisir (entre), se comporter, se conduire, durer, hésiter (entre), passer (avant), se plaindre (auprès de), se porter, ressusciter (d'entre)}
    & Luc hésite {entre ces meubles}.\newline Max se conduit {avec tact}.\newline La garantie dure {jusqu'en 2020}. \\
    \hline
    \end{tabularx}
\end{table}

\subsubsection{Les verbes à plusieurs compléments nominaux ou prépositionnels}

\begin{table}[H]
    \centering
    \small % Adjust font size if needed
    \renewcommand{\arraystretch}{1.2} % Increase vertical spacing slightly for readability

    
    \begin{adjustbox}{max width=\textwidth}
        \begin{tabularx}{\textwidth}{|l|>{\itshape}Y|Y|} % Use Y for content columns, with vertical lines
        \hline
        \rowcolor{cyan!20}
        \multicolumn{1}{|c|}{\textbf{}} & \multicolumn{1}{c|}{\textbf{VERBES}} & \multicolumn{1}{c|}{\textbf{EXEMPLES}} \\
        \hline
        \rowcolor{cyan!20}
        \multicolumn{3}{|c|}{\textbf{DEUX COMPLÉMENTS NOMINAUX}} \\
        \hline
        \multicolumn{1}{|l|}{} & acheter, envoyer, expédier, payer, vendre & Luc vend \{sa voiture\} \{5 000 euros\}. \newline Paul envoie \{Max\} \{place Maubert\}. \\
        \hline
        \rowcolor{cyan!20}
        \multicolumn{3}{|c|}{\textbf{UN COMPLÉMENT NOMINAL ET UN COMPLÉMENT PRÉPOSITIONNEL}} \\
        \hline
        à & apprendre, avouer, comparer, dire, donner, écrire, emprunter, envoyer, imputer, montrer, offrir, opposer, préférer, prendre, prêter, proposer, rapprocher, voler & Paul préfère \{ceci\} \{à cela\}. \newline Paul donne \{un livre\} \{à Marie\}. \newline Paul écrit \{une lettre\} \{à Marie\}. \\
        \cline{1-3} % Full line across all columns
        avec & associer, comparer, échanger, marier, négocier, permuter, unir & Paul compare \{ceci\} \{avec cela\}. \newline Léa négocie \{un accord\} \{avec Max\}. \\
        \cline{1-3}
        contre, pour & donner, échanger, parler, taper & Paul échange \{ceci\} \{contre cela\}. \newline Paul donne \{30 euros\} \{pour le projet\}. \newline Paul tape \{la règle\} \{contre la table\}. \\
        \cline{1-3}
        de & apprendre, approcher, combler, convaincre, couvrir, différer, informer, obtenir, orner, ôter, prévenir, prolonger, remplir, séparer, sortir, tenir, tirer, traduire, traiter, vider & Paul approche \{l'échelle\} \{du mur\}. \newline Luc remplit \{le salon\} \{de meubles\}. \newline Léa comble \{Max\} \{de cadeaux\}. \\
        \cline{1-3}
        en & aménager, changer, couper, découper, déguiser, diviser, partager, répartir, traduire, transformer & Luc transforme \{la chambre\} \{en bureau\}. \newline Luc partage \{le gâteau\} \{en six parts\}. \newline Paul traduit \{le texte\} \{en anglais\}. \\
        \cline{1-3}
        autres prépositions & choisir (parmi), compter (parmi), descendre, diluer (dans), dissoudre (dans), héberger (inclure (dans)), interroger (sur), loger, mettre, monter, noyer, placer, poser, pousser, prendre (pour), \newline traiter & Paul interroge \{Max\} \{sur sa leçon\}. \newline Paul met \{la vaisselle\} \{dans le placard\}. \newline Luc compte \{Jean\} \{parmi ses amis\}. \newline Paul prend \{Max\} \{pour témoin\}. \newline Paul traite \{Marie\} \{sans égards\}. \\
        \hline
        \rowcolor{cyan!20}
        \multicolumn{3}{|c|}{\textbf{DEUX COMPLÉMENTS PRÉPOSITIONNELS}} \\
        \hline
        à et de & conclure, parler, servir & Cette clef sert \{à Luc\} \{de tournevis\}. \newline Luc parle \{à Max\} \{de son travail\}. \\
        \hline
        à et à & servir & L'anglais sert \{à Paul\} \{à son travail\}. \\
        \hline
        à et une autre préposition & jouer (à) (avec), renoncer (à) (pour) & Paul joue \{au tennis\} \{avec Marie\}. \\
        \hline
        de et de & hériter & Léa hérite \{d'un buffet\} \{de sa mère\}. \\
        \hline
        de et une autre préposition & arguer (de) (auprès de), convenir (de) (avec), débattre (de) (avec),décider (de) (avec), discuter (de) (avec), hériter (de) (de), se plaindre (de) (auprès de), profiter (de) (pour), répondre (de) (devant), témoigner (de) (auprès de), traiter (de) (avec) & Léa discute \{de tout\} \{avec Max\}. \newline Paul répond \{de ses actes\} \{devant nous\}. \newline Paul témoigne \{de sa bonne foi\} \{auprès du juge\}. \\
        \hline
        autres prépositions & compter (sur) (pour), s'entendre (avec) (sur), parler (avec) (sur) & Luc parle \{avec Max\} \{sur le bleu\}. \\
        \hline
        \end{tabularx}
    \end{adjustbox}
\end{table}

\newpage

\subsection{Les verbes à complément infinitif}


\begin{table}[H]
    \centering
    \small
    \begin{adjustbox}{max width=\textwidth}
        \begin{tabular}{| L{6.8cm} | L{2.5cm} | L{5.5cm} |} % Adjusted column widths
        \hline
        \rowcolor{cyan!20}
        \textbf{VERBES} & \textbf{INFINITIF} & \textbf{EXEMPLES} \\
        \hline
        \rowcolor{cyan!20}
        \multicolumn{3}{|c|}{\textbf{À UN COMPLÉMENT}} \\ % Merged 3 columns now
        \hline
        \textbf{aspectuel :} \newline aller, commencer (\textit{à / par}), continuer (\textit{à / de}), être en train (\textit{de}), finir (\textit{de / par}), se mettre (\textit{à}), venir (\textit{de}) & complément direct ou oblique & \textit{Luc finit [de travailler].} \newline \textit{Luc commence [à comprendre].} \newline \textit{Cela finira [par se savoir].} \\
        \hline
        % ... (continue with the rest of your table, applying the same change)
        \textbf{de décision :} \newline accepter (\textit{de}), décider (\textit{de}), opter (\textit{pour}), refuser (\textit{de}), renoncer (\textit{à}), voter (\textit{pour}) & complément direct ou oblique & \textit{Luc choisit [de partir].} \newline \textit{Luc opte [pour partir].} \\
        \hline
        \textbf{de déplacement :} \newline aller, courir, descendre, monter, partir, venir & complément oblique & \textit{Luc court [acheter du pain].} \\
        \hline
        \textbf{de désir et de sentiment :} \newline aimer, s’attendre (\textit{à}), craindre (\textit{de}), désirer, détester, redouter (\textit{de}), regretter (\textit{de}), souhaiter (\textit{à}), vouloir & complément direct ou oblique & \textit{Luc aime [dormir].} \newline \textit{Luc tient [à venir].} \\
        \hline
        \textbf{d’essai :} \newline chercher (\textit{à}), essayer (\textit{de}), réussir (\textit{à}), tenter (\textit{de}) & complément direct & \textit{Luc essaie [de comprendre].} \newline \textit{Luc cherche [à comprendre].} \\
        \hline
        \textbf{d’identité :} \newline être, paraitre, passer pour, rester (\textit{à}), sembler & attribut ou complément oblique & \textit{Le problème est [de trouver].} \newline \textit{Luc semble [ne rien faire].} \newline \textit{Luc reste [à ne rien faire].} \\
        \hline
        \textbf{modal :} \newline avoir (\textit{à}), devoir, falloir nécessiter (\textit{de}), pouvoir, réclamer (\textit{de}), risquer (\textit{de}) & complément direct & \textit{Il faut [partir].} \newline \textit{Luc a [à travailler].} \newline \textit{Luc risque [d’échouer].} \\
        \hline
        \textbf{d’opinion et d’activité intellectuelle :} \newline comprendre, croire, imaginer, oublier, penser, savoir, se souvenir (\textit{de}) & complément direct ou oblique & \textit{Luc pense [avoir raison].} \newline \textit{Luc se souvient [d’être venu].} \\
        \hline
        \rowcolor{cyan!20}
        \multicolumn{3}{|c|}{\textbf{À DEUX COMPLÉMENTS}} \\ % Merged 3 columns now
        \hline
        \textbf{causatif :} \newline empêcher (\textit{de}), faire, laisser & complément oblique & \textit{Luc laisse Max [partir].} \newline \textit{Luc empêche Max [de travailler].} \\
        \hline
        \textbf{causatif de déplacement :} \newline envoyer, expédier & complément oblique & \textit{Luc envoie Max [acheter du pain].} \\
        \hline
        \textbf{de communication :} \newline affirmer (\textit{à}), avouer (\textit{à}), déclarer (\textit{à}), dire (\textit{à}), écrire (\textit{à}), expliquer (\textit{à}), prétendre (\textit{à}) & complément direct & \textit{Luc nous affirme [tout comprendre].} \\
        \hline
        \textbf{d’engagement :} \newline s’engager (\textit{envers}) (\textit{à}), jurer (\textit{à}) (\textit{de}), promettre (\textit{à}) (\textit{de}) & complément direct & \textit{Luc jure à Max [de rester].} \\
        \hline
        \textbf{d’ordre ou d’influence :} \newline conseiller (\textit{à}) (\textit{de}), convaincre (\textit{de}), dire (\textit{à}) (\textit{de}), interdire (\textit{à}) (\textit{de}), obliger (\textit{à}), ordonner (\textit{à}) (\textit{de}), permettre (\textit{à}) (\textit{de}) & complément direct ou oblique & \textit{Luc permet à Max de partir.} \newline \textit{Luc oblige Max à partir.} \\
        \hline
        \textbf{de perception :} \newline écouter, entendre, observer, regarder, sentir, voir & complément oblique & \textit{Luc entend Max chanter un air.} \\
        \hline
        \end{tabular}
    \end{adjustbox}
\end{table}


\newpage
\subsection{Les verbes à complétive}

\subsubsection{General}
\paragraph{La distinction entre subordonnées complétive et circonstancielle}
\begin{enumerate}
    \item complétive必须存在
    \begin{itemize}
        \item Paul souhaite [que tu viennes].|*Paul souhaite.
    \end{itemize}
    \item 只有subordonnées complétive可以被le, cela, en, y替代,或转化为ce..., c’est que...形式
    \item si
    \begin{enumerate}
        \item 只有用作complétive interrogative时,si从句动词才可用futur ou conditionnel
        \begin{itemize}
            \item J’ignore si elle viendra.
            \item * Paul viendra si elle viendra.
        \end{itemize}
        \item ou non可加在interrogative句尾,而不能加于conditionnelle
        \begin{itemize}
            \item Paul demande si tu viendras ou non.
        \end{itemize}
        \item si在verbe de perception, d’opinion ou d’activité intellectuelle后也可引导exclamante
        \begin{itemize}
            \item Tu as vu [si c’est beau] !
            \item *Si c’est beau, tu as vu !
            \item Tu as vu ça !
        \end{itemize}
    \end{enumerate}
    \item comme
    \begin{enumerate}
        \item complément de manière, comme se conduire
        \begin{itemize}
            \item Paul se conduit [comme il faut].|* Paul se conduit.
        \end{itemize}
        \item complétive exclamative, comme admirer
        \begin{itemize}
            \item Paul admire [comme elle court].|Paul admire cela.
        \end{itemize}
    \end{enumerate}
\end{enumerate}

\subsubsection{La fonction des complétives}
\paragraph{compléments directs}
\begin{enumerate}
    \item 用le, ça ou cela 指代complétives
\end{enumerate}

\paragraph{compléments obliques}
\begin{enumerate}
    \item à ce que, de ce que, en ce que, autre préposition + que 引导从句
    \begin{itemize}
        \item Paul tient [à ce que Marie vienne].|Paul y tient.
        \item Paul se plaint [de ce qu’on ne l’appelle pas].|Paul s’en plaint.
        \item  Le sens de la politique consiste [en ce que les hommes libres ont entre eux des relations d’égaux] .
        \item Paul opte [pour qu’on parte le 28].
    \end{itemize}
    \item compléments obliques有时也可只用que引导
    \begin{enumerate}
        \item se plaindre ou convaincre可用que,并用en指代
        \begin{itemize}
            \item Paul se plaint [qu’on ne l’appelle pas].|Paul s’en plaint.
            \item Paul convainc Marie [qu’elle a tort].|Paul en convainc Marie.
        \end{itemize}
        \item s’attendre可用à ce que或只用que,并用y指代
        \begin{itemize}
            \item Paul s’attend [à ce que Marie vienne].|Paul s’y attend.
            \item Paul s’attend [que Marie vienne].|Paul s’y attend.
        \end{itemize}
    \end{enumerate}
\end{enumerate}


\section{Les constructions passives, neutres et impersonnelles}
\subsection{La construction passive}
\begin{enumerate}
    \item verbes impersonnels et réfléchis不能用于passif
\end{enumerate}
\subsubsection{Le complément d’agent au passif}
\begin{enumerate}
    \item 用de引导complément d’agent
    \begin{enumerate}
        \item 可被en指代
        \begin{itemize}
            \item Le film a été apprécié des élèves.
            \item Le film en a été apprécié.
        \end{itemize}
        \item le sujet est un infinitif
        \begin{itemize}
            \item Avoir été puni désole/surprend Max.
            \item Max est désolé/surpris d’avoir été puni.
        \end{itemize}
        \item le sujet est un subordonnée,介词消失(de ce que转化成que),但仍然被en指代
        \begin{itemize}
            \item Que l’éruption soit imminente étonne/effraie/soulage tout le monde.
            \item Tout le monde est étonné/effrayé/soulagé que l’éruption soit imminente
            \item Tout le monde en est étonné/effrayé/soulagé.
        \end{itemize}
        \begin{table}[H]
    \centering
    % Uncomment one of the following lines if the table still overflows or if you want it smaller:
    % \small        % Slightly smaller font
    % \footnotesize % Even smaller font
    % \scriptsize   % Very small font

    \begin{tabular}{| L{7cm} | L{8cm} |} % Two columns with specified widths
    \hline
    \textbf{VERBES} & \textbf{EXEMPLES} \\
    \hline
    \textbf{d'accompagnement :} \newline \textit{accompagner, escorter} & \textit{Le colis est accompagné d'un mode d'emploi.} \\
    \hline
    \textbf{d'assistance :} \newline \textit{aider, appuyer, assister, seconder} & \textit{Le directeur est aidé d'un assistant.} \\
    \hline
    \textbf{donnant la composition d'un groupe :} \newline \textit{composer, constituer, former} & \textit{Le gouvernement est constitué de quinze ministres.} \\
    \hline
    \textbf{instrumentaux :} \newline \textit{entourer, recouvrir} & \textit{Le jardin est recouvert de feuilles.} \\
    \hline
    \textbf{de localisation :} \newline \textit{précéder, suivre} & \textit{Le diner sera suivi d'un concert.} \\
    \hline
    \textbf{d'opinion ou d'activité intellectuelle :} \newline \textit{admettre, approuver, comprendre, connaître, rejeter, tolérer} & \textit{La situation n'a pas été comprise de la population.} \\
    \hline
    \textbf{de perception :} \newline \textit{écouter, entendre, voir} & \textit{Son cri a été entendu de tous.} \\
    \hline
    \textbf{de sentiment :} \newline \textit{aimer, apprécier, craindre, détester, envier, inquiéter, mépriser, surprendre} & \textit{Ce livre est apprécié de tous.} \newline \textit{Je suis étonné du départ de Max.} \\
    \hline
    \end{tabular}
\end{table}
    \end{enumerate}
    \item 用dans引导complément d’agent
    \begin{enumerate}
        \item 只限于描述situation d’inclusion (contenir, (r)enfermer, englober, inclure) 
        \begin{itemize}
            \item Cette boite contient tous mes bijoux.
            \item  Tous mes bijoux sont contenus dans/*par cette boite.
        \end{itemize}
        \item actif主语是locatif,如果是人则不能用dans要用par
        \begin{itemize}
            \item La police a contenu les manifestants.
            \item Les manifestants ont été contenus par/*dans la police.
        \end{itemize}
    \end{enumerate}
\end{enumerate}

\subsubsection{Le passif impersonnel}
complément direct de l’actif在这个结构中仍保持为complément
\begin{enumerate}
    \item verbes transitifs
    \begin{itemize}
        \item Il a été perdu une montre.
        \item Une montre a été perdue.
    \end{itemize}
    \begin{enumerate}
        \item syntagme nominal défini在这个结构中很不自然
        \begin{itemize}
            \item La clef a été volée.
            \item *Il a été volé la clef
        \end{itemize}
    \end{enumerate}
    \item verbes intransitif: sursoir, procéder, parler, travailler
    \begin{itemize}
        \item il a été procédé à la désignation d’un secrétaire de séance.
        \item il a été travaillé à un plan de reprise des activités pour l’ensemble des services
    \end{itemize}
    \begin{enumerate}
        \item changement de lieu, apparition ou de disparition类的动词,不能用于passif impersonnel
        \begin{itemize}
            \item *Il a été arrivé/sorti/né ce jour-là (par des milliers de réfugiés).
            \item * Il a été beaucoup né ce jour-là.
        \end{itemize}
    \end{enumerate}
    \item verbes à complément infinitif ou à complétive:用passif impersonnel比passif personnel更自然
    \begin{itemize}
        \item On m’a demandé de réagir à ces tranches de vie.
        \item Il m’a été demandé de réagir à ces tranches de vie
        \item En 1804, on a décidé que la réunion suivante aurait lieu à La Haye
        \item En 1804, il a été décidé que la réunion suivante aurait lieu à La Haye.
        \item * En 1804, que la décision suivante aurait lieu à La Haye a été décidé.
    \end{itemize}
    \item Le passif impersonnel des locutions verbales
    
    \begin{enumerate}
        \item verbales incluant un nom sans déterminant用passif impersonnel比passif personnel更自然
        \begin{itemize}
            \item Il sera mis fin à des pratiques qui ont reçu récemment une fâcheuse publicité
            \item * Fin sera mise à des pratiques qui ont reçu récemment une fâcheuse publicité.
        \end{itemize}
        \item 一些短语 (faire fi de, mettre fin à, tenir compte de)只能用impersonnel表示passif
        \begin{itemize}
            \item Il a été fait fi de nos remarques.
            \item * Fi a été fait de vos remarques.
        \end{itemize}
    \end{enumerate}
    \item Le complément d’agent du passif impersonnel大部分是有生命的,用par引导,很少用de
\end{enumerate}

\subsection{La construction médiopassive}

\begin{enumerate}
    \item le complément direct à l’actif devient sujet
    \item le verbe est réfléchi
    \item le sujet de l’actif不能出现
    \begin{enumerate}
        \item actif必须是有生命物,否则不能用médiopassive
        \begin{itemize}
            \item * Les grands chênes se foudroient sans difficulté.
        \end{itemize}
    \end{enumerate}
\end{enumerate}
\begin{itemize}
    \item Nager s’apprend à tout âge
    \item Il se confirme que la guerre va bientôt finir. (médiopassif impersonnel)
    \item  Ce film ne se montre pas à des enfants
\end{itemize}


\subsection{La construction neutre}

\begin{enumerate}
    \item verbes transitif构成construction intransitive,complément de devenir sujet
    \begin{itemize}
        \item Le vent a cassé la branche.
        \item La branche a cassé.
    \end{itemize}
    \item 与médiopassif的区别是neutre不包含agent implicite.
\end{enumerate}


\subsection{La construction impersonnelle}
主语永远是第三人称单数
\begin{enumerate}
    \item verbes à sujet nominal, généralement indéfini
    \begin{itemize}
        \item Un accident est arrivé. Il est arrivé un accident
    \end{itemize}
    \item verbes à sujet infinitif
    \begin{itemize}
        \item Crier ne sert à rien. Il ne sert à rien de crier
    \end{itemize}
    \item verbes à sujet subordonné
    \begin{itemize}
        \item Qu’il neige importe peu. Il importe peu qu’il neige
    \end{itemize}
    \item passif, médiopassif, neutre intransitive
    \begin{itemize}
        \item Plusieurs spectateurs ont été blessés. Il a été blessé plusieurs spectateurs.
        \item De nombreux livres se vendent sur Internet. Il se vend de nombreux livres sur Internet.
        \item Une erreur s’est glissée dans le raisonnement. Il s’est glissé une erreur dans le raisonnement.
    \end{itemize}
\end{enumerate}

\subsubsection{La construction impersonnelle avec un syntagme nominal}
\begin{enumerate}
    \item SN作为complément direct
    \begin{enumerate}
        \item SN可被en指代
        \begin{itemize}
            \item Des malheurs, il lui en est arrivé plusieurs
        \end{itemize}
        \item SN可被que和quoi指代
        \begin{itemize}
            \item Que se passe-t-il ?
            \item Il se passe quoi ?
        \end{itemize}
        \begin{enumerate}
            \item 可用于quoi引导concessive
            \begin{itemize}
                \item Quoi qu’il t’arrive, je serai à tes côtés.
            \end{itemize}
            \item 区别于其他complément direct,SN不能用于relative
            \begin{itemize}
                \item * Le malheur qu’il lui est arrivé est indescriptible.
            \end{itemize}
        \end{enumerate}
    \end{enumerate}
    \item 可用动词
    \begin{enumerate}
        \item verbe intransitif
        \item verbe transitif sans complément direct
        \begin{itemize}
            \item Il mange plein de gens du quartier dans ce restaurant populaire
        \end{itemize}
        \item 不能用verbe de mesure comme couter ou peser
        \begin{itemize}
            \item Certains sacs pèsent 30 kilos. *Il pèse 30 kilos certains sacs.
        \end{itemize}
        \item attribut不能用impersonnel
        \begin{itemize}
            \item Un voisin est médecin. *Il est médecin un voisin.
            \item Plusieurs participants restaient silencieux. *Il restait silencieux plusieurs participants.
        \end{itemize}
    \end{enumerate}
\end{enumerate}

\subsubsection{La construction impersonnelle avec un infinitif ou une complétive}

\begin{enumerate}
    \item infinitif用de引导,很少用à
    \begin{itemize}
        \item Il convient de partir tout de suite
        \item Il est prévu de faire de cette ancienne usine un centre culturel
        \item Il a été appris à voyager léger
    \end{itemize}
    \begin{enumerate}
        \item 在valoir mieux后,infinitif无引导词
        \begin{itemize}
            \item Il vaut mieux écouter l’avis de votre patron
        \end{itemize}
        \item infinitf可以被cela替换
        \begin{itemize}
            \item Il est prévu cela
            \item Il a été appris cela
        \end{itemize}
    \end{enumerate}
    \item complétive
    \begin{itemize}
        \item Il est de notoriété publique que la véritable passion des Hobbits est la boustifaille
    \end{itemize}
    \item 可用动词
    \begin{enumerate}
        \item verbe sans complément
        \begin{itemize}
            \item Il convient de réserver à l’avance
        \end{itemize}
        \item verbe à complément prépositionnel 
        \begin{itemize}
            \item Il est revenu à nos oreilles que tu voulais démissionner
        \end{itemize}
        \item verbe transitif当complément被proforme (me)替代,或属于une expression figée
        \begin{itemize}
            \item Il me démange de supprimer la partie déroulement du jeu qui n’apporte rien à l’article
            \item Il ne fait aucun doute que Paul sera élu
        \end{itemize}
        \item verbe à complément direct nominal ou adjectival不能用impersonnel
        \begin{itemize}
            \item *Il embête Marie que Paul parte.
            \item *Il coute cher de voyager.
        \end{itemize}
        \item 不用于syntagme nominal,ajout不阻止infinitif ou complétive使用impersonnel
        \begin{itemize}
            \item Il semble vain de vouloir tout comprendre
            \item Il est vrai que Marie boit
        \end{itemize}
    \end{enumerate}
\end{enumerate}

\subsubsection{Le sujet impersonnel: : il et ça}
\begin{enumerate}
    \item ce ou ça
    \begin{enumerate}
        \item le verbe peut être transitif avec un complément nominal ou adjectival
        \begin{itemize}
            \item Ça embête Marie que Paul parte.
            \item Ça coute cher de voyager.
        \end{itemize}
        \item la complétive (ou l’infinitif) peut apparaitre disloquée en début de phrase
        \begin{itemize}
            \item Que tu ne sois pas venu, c’est dommage.
        \end{itemize}
        \item la complétive (ou l’infinitif) peut être omise
        \begin{itemize}
            \item C’est dommage.
        \end{itemize}
        \item d’autres subordonnées sont possibles, introduites par si ou par quand 
        \begin{itemize}
            \item Ça vous dérange si je mets mon vélo là ?
        \end{itemize}
    \end{enumerate}
\end{enumerate}

\section{Les auxiliaires avoir et être}

\subsection{L’accord du participe passé aux temps composés}

\subsubsection{les verbes non réfléchis}
\paragraph{L’accord du participe avec le sujet}
être
\begin{itemize}
    \item Je suis partie hier.
    \item Quand êtes-vous parties, Mesdames ?
    \item On est entré par effraction.
    \item On est entrés par effraction.
\end{itemize}

\paragraph{L’accord du participe avec le complément}
avoir : 当complément direct (proforme ou d’un syn- tagme nominal) 出现在avoir前时, le participe s’accorde avec le complément

\begin{enumerate}
    \item pronominalisé
    \begin{itemize}
        \item Paul nous a embrassés. (embrasser quelqu’un)
        \item Paul nous a parlé. (parler à quelqu’un)
    \end{itemize}
    \begin{enumerate}
        \item 用en指代时,participe保持masculin singulier
        \begin{itemize}
            \item Je l’ai écrite, cette lettre.
            \item J’en ai écrit, des lettres, dans ma vie !
        \end{itemize}
    \end{enumerate}
    \item en interrogative
    \begin{itemize}
        \item Quelles lettres a-t-il promises ?
        \item Quelle lettre as-tu écrite ?
    \end{itemize}
    \begin{enumerate}
        \item 用lequel指代,participe varie en genre et en nombre
        \begin{itemize}
            \item Lesquelles as-tu prises ?
        \end{itemize}
        \item 用que, qui指代,participe保持masculin singulier
        \begin{itemize}
            \item Qu’as-tu fait comme bêtise ?
            \item Qui a-t-il pris pour femme ?
        \end{itemize}
        \item combien用于句首,participe保持masculin singulier
        \begin{itemize}
            \item Combien as-tu payé ?
        \end{itemize}
        \item combine + de + nom用于句首,participe varie en genre et en nombre;de + nom出现在动词后,participe则保持masculin singulier
        \begin{itemize}
            \item Combien as-tu écrit de lettres ?
            \item Combien de lettres as-tu écrites ?
        \end{itemize}
    \end{enumerate}
    \item en exclamative
    \begin{itemize}
        \item Quelle chance tu as eue !
    \end{itemize}
    \item antécédent d’une relative introduite par \textit{que}
    \begin{itemize}
        \item C’est une vie horrible qu’il a vécue.
        \item Voici les lettres qu’il a écrites.
    \end{itemize}
\end{enumerate}




\subsubsection{les verbes réfléchis}
\begin{enumerate}
    \item intrinsèquement réfléchis / médiopassifs / neutres,réfléchi不作为complément direct,此时le participe s’accorde avec le sujet。除了se rire, se (com)plaire (à)
    \begin{itemize}
        \item Elle s’est évanouie.
        \item Elle s’est aperçue de son erreur.
        \item Les livres se sont bien vendus.
        \item La foule s’est dispersée.
    \end{itemize}
    \item 当réfléchis作为complément direct 时,le participe s’accorde avec se
    \begin{itemize}
        \item Elle s’est inscrite à un cours.
        \item Marie s’est lavée.
    \end{itemize}
    \item 当réfléchis作为complément prépositionnel时,le participe s’accorde avec le complément direct qui précède
    \begin{itemize}
        \item Marie se les est achetés.
        \item Voici la voiture que Paul s’est offerte.
    \end{itemize}
\end{enumerate}

\section{Les constructions verbales avec un attribut}
\subsection{Les constructions à attribut du sujet}
\small % 或 \scriptsize, 根据实际大小需求
\begin{longtable}{|
  >{\itshape\raggedright\arraybackslash}p{2.4cm}|
  >{\raggedright\arraybackslash}p{2.4cm}|
  >{\raggedright\arraybackslash}p{2.4cm}|
  >{\RaggedRight\arraybackslash}p{2.4cm}|
  >{\raggedright\arraybackslash}p{2.4cm}|
  >{\raggedright\arraybackslash}p{2.4cm}|}
\hline
\rowcolor{cyan!20}
\textbf{Verbe} & \textbf{Adjectif} & \textbf{Nom} & \textbf{Syntagme Nominal} & \textbf{Syntagme Prépositionnel} & \textbf{Adverbe} \\
\hline
\endfirsthead

\hline
\rowcolor{cyan!20}
\textbf{Verbe} & \textbf{Adjectif} & \textbf{Nom} & \textbf{Syntagme Nominal} & \textbf{Syntagme Prépositionnel} & \textbf{Adverbe} \\
\hline
\endhead

\hline
\endfoot

\hline
\endlastfoot
\rowcolor{cyan!20}
\multicolumn{6}{|c|}{\textbf{VERBES D'ÉTAT}} \\
\hline
se trouver & fort & musicien & le meilleur & à l’aise & mieux \\
se vouloir & honnête & adulte & son ami & avec nous & ainsi \\
\hline
\rowcolor{cyan!20}
\multicolumn{6}{|c|}{\textbf{VERBES DE CHANGEMENT D'ÉTAT}} \\
\hline
commencer & premier & apprenti, comme apprenti & — & en bonne position & bien \\
débuter & petit & apprenti, comme apprenti & — & en bonne position & bien \\
devenir & fou & capitaine & mon ami & de meilleure humeur & mieux \\
entrer & premier & apprenti, comme apprenti & — & en transe & — \\
se faire & bête & acteur & son ami & — & — \\
finir & dernier & directeur, comme directeur & — & en bonne position & mal \\
passer & inaperçu & directeur & — & de mode & — \\
rester & pauvre & professeur, comme professeur & mon ami & en colère & ainsi \\
sortir & premier & major & — & en bonne position & mieux \\
terminer & dernier & patron, comme patron & — & en bonne place & mal \\
tomber & malade & — & — & en arrêt & mal \\
tourner & court & chèvre & — & à son avantage & mal \\
virer & malhonnête & cambrioleur & — & à l’aigre & — \\
\hline
\rowcolor{cyan!20}
\multicolumn{6}{|c|}{\textbf{AUTRES VERBES D'ÉTAT}} \\
\hline
s’affirmer & aisé, comme aisé & médecin & un bon maire, comme un bon maire & en hausse & — \\
s’annoncer & houleux, comme houleux & — & un succès, comme un succès & en difficulté & mal \\
apparaître & facile, comme facile & ami & un handicap, comme un handicap & en forme & ainsi \\
s’avérer & difficile & — & un atout & contre nous & — \\
avoir l’air & facile & médecin & — & d’un médecin & mieux \\
demeurer & tranquille & professeur & son ami & en forme & ainsi \\
être & sage & médecin & un musicien & en avance & mieux \\
faire & idiot & médecin & un bon professeur & — & bien \\
se montrer & curieux & (bon) joueur & un bon maire & en forme & ainsi \\
paraître & calme & ami & un bon professeur, comme le meilleur candidat & en forme & mieux \\
passer pour & fou & \% médecin & un bon médecin & — & — \\
se révéler & compétent, comme incompétent & observateur & un bon acteur, comme un bon acteur & à sec & mieux \\
sembler & intelligent & étudiant & un bon père & en forme & mieux \\
se sentir & faible & adulte & un homme, comme un homme & à l’aise & bien \\
\end{longtable}

\subsection{Les constructions à attribut du complément}
\begin{tabular}{|p{4cm}|p{3.5cm}|p{3.5cm}|>{\RaggedRight\arraybackslash}p{3.5cm}|}
\hline
\textbf{VERBES} & \textbf{ADJECTIF} & \textbf{NOM ou SYNTAGME NOMINAL} & \textbf{ADVERBE ou SYNTAGME PRÉPOSITIONNEL} \\
\hline
\textbf{de communication} : \newline \textit{affirmer (comme), déclarer, \newline dire, inscrire, marquer, \newline mettre, porter, promettre, \newline qualifier (de), traiter (de)} & \textit{On le dit [riche].} \newline \textit{On le traite [de fou].} \newline \textit{On déclare [important] que vous veniez.} & \textit{On le dit [un bon maire].} \newline \textit{On le déclare [candidat].} & \textit{On le dit [à l'aise].} \\
\hline
\textbf{de classification} : \newline \textit{classer, coder, \newline décrire (comme), \newline donner, étiqueter, jouer, \newline prendre (pour), présenter, \newline représenter (comme)} & \textit{On le donne [partant].} \newline \textit{On les classe [dangereux].} & \textit{On le décrit [comme le chef].} & \textit{On les a classés [à risque].} \\
\hline
\textbf{de nomination} : \newline \textit{consacrer, désigner (comme), \newline élire, faire, instituer, introniser, \newline mettre, nommer, proclamer, \newline sacrer} & \quad --- & \textit{On l'a élu [président].} & \textit{On l'a élu [à la mairie].} \\
\hline
\textbf{modaux} : \newline \textit{falloir} & \textit{Il le faut [vivant].} & & \textit{Il en faut [dans chaque classe].} \\
\hline
\textbf{existentiels et présentatifs} : \newline \textit{c'est, il y a, il reste, voici, \newline voilà} & \textit{Il y en a deux [malades].} \newline \textit{Le voilà [content].} & \textit{Il y en a deux [médecins].} \newline \textit{Le voilà [notre allié].} & \textit{Il y en a deux [avec nous].} \newline \textit{Voilà Paul [en forme].} \newline \textit{Me voilà [mieux].} \\
\hline
\textbf{de perception} : \newline \textit{apercevoir, écouter, entendre, \newline regarder, sentir, voir} & \textit{Je le sens [triste].} \newline \textit{Je le vois [malheureux].} & \textit{Je le vois bien [médecin].} & \textit{Je le vois [en forme].} \newline \textit{Je le sens [à l'aise].} \\
\hline
\textbf{de possession} : \newline \textit{avoir, conserver, garder, \newline maintenir, prendre (pour), tenir} & \textit{Il a un fils [malade].} \newline \textit{Il la maintient [debout].} & \textit{Il a un fils [médecin].} \newline \textit{Il la prend [pour guide].} & \textit{Il a un père [en forme].} \newline \textit{Il la maintient [en ordre].} \\
\hline
\textbf{d'activité intellectuelle} : \newline \textit{caractériser (comme), \newline certifier, concevoir, \newline considérer (comme), \newline croire, découvrir, deviner, \newline envisager, estimer, \newline se figurer, imaginer, juger, \newline penser, prédire, pressentir, \newline présumer, prétendre, prévoir, \newline reconnaître (comme), \newline regarder (comme), savoir, \newline supposer, trouver, tenir (pour)} & \textit{Je le croyais [fidèle].} \newline \textit{Je le considère [comme malade].} \newline \textit{Je trouve [nécessaire] de partir.} \newline \textit{Je juge [important] que vous veniez.} & \textit{Je le sais [étudiant].} \newline \textit{Je crois [mon ami].} \newline \textit{Je le considère [comme un ami].} & \textit{Je sais Luc [avec nous].} \newline \textit{Je crois Paul [mieux].} \newline \textit{Je le considère [comme en avance].} \newline \textit{On le juge [en difficulté].} \\
\hline
\textbf{de désir et de volonté} : \newline \textit{aimer, désirer, espérer, \newline exiger, préférer, proposer, \newline souhaiter, vouloir} & \textit{J'aimerais Luc [plus attentif].} & \textit{J'espérais Luc [notre allié].} & \textit{Je le préfère [en noir].} \\
\hline
\textbf{causatifs} : \newline \textit{faire, laisser, mettre, rendre} & \textit{Laissez-le [tranquille]!} \newline \textit{Cette musique me rend [joyeux].} & \textit{Cette aventure les a rendus [amis].} & \textit{Laissez-le [en paix]!} \newline \textit{Ça les a mis [en colère].} \\
\hline
\end{tabular}

\section{Les constructions causatives}
\subsection{Les constructions de faire avec un infinitif}
\begin{enumerate}
    \item infinitif不能被causataire与faire分开,必须置于其后
    \begin{itemize}
        \item Jacques fait entrer [les invités] au salon
    \end{itemize}
    \item compléments de l’infinitif se pronominalisent sur faire
    \begin{itemize}
        \item Paul le fait lire aux enfants.
        \item Paul empêche les enfants de le lire.
    \end{itemize}
    \item participe passé de faire ne s'accrod pas avec le causataire
    \begin{itemize}
        \item On a fait rire [les filles].|On les a fait rire.
        \item Quelles filles a-t-on fait rire ?
        \item Voici les filles qu’on a fait rire.
    \end{itemize}
    \item faire suivi d’un infinitif ne se passive pas. 只存在于古典法语中
    \begin{itemize}
        \item * Les enfants ont été faits travailler.
        \item Il a été fait entrer,et a été chargé de la part de l’Assemblée, d’aller chez M.Le Tellier
    \end{itemize}
    \begin{enumerate}
        \item 用on来替代passif
        \begin{itemize}
            \item On a fait travailler les enfants.
        \end{itemize}
        \item 可使用médiopassif
        \begin{itemize}
            \item Les voitures se font réparer chez un concessionnaire.
        \end{itemize}
    \end{enumerate}
\end{enumerate}

\subsubsection{Le causataire}
le causataire est complément direct ou prépositionnel
\begin{enumerate}
    \item infinitif a un complément direct时causataire为complément prépositionnel (à, par, de)
    \begin{itemize}
        \item Jean fait apprendre leurs leçons aux enfants
        \item Jean leur fait apprendre leurs leçons.
        \item Jean les fait apprendre aux enfants
        \item Je ferai couper le bois [par Jean].
        \item Son humeur joviale l’a fait aimer [de ses élèves].
    \end{itemize}
    \begin{enumerate}
        \item complément de l’infinitif est une subordonnée complétive时,causataire用à引导
        \begin{itemize}
            \item Paul fait comprendre [à Marie] qu’elle a tort.
            \item Ça fait penser [à Paul] qu’il a rêvé.
        \end{itemize}
        \item faire est réfléchi时,只能用par,不能用à
        \begin{itemize}
            \item * Max s’est fait coiffer [à son frère].
            \item Max s’est fait coiffer [par son frère].
        \end{itemize}
        \item à引导causataire animé et influencé voire contraint par le sujet;pour引导causataire inanimé ou ayant une relation plus indirecte avec le sujet
        \begin{itemize}
            \item On fera ranger leur chambre [aux enfants].
            \item On fera inonder la campagne [par la Marne] pour protéger Paris.
            \item On fera repeindre la chambre [par un professionnel].
        \end{itemize}
        \item infinitif a un complément animé时,只能用par,不能用à
        \begin{itemize}
            \item Jacques a fait soigner l’enfant [par sa mère] / ? [à sa mère]
            \item Nous avons fait prévenir Jacques [par Pierre] / * [à Pierre].
            \item Nous lui avons fait prévenir Jacques.
        \end{itemize}
    \end{enumerate}
    \item infitif无complément direct或complément de l’infinitif est figé时,causataire为complément direct
    \begin{itemize}
        \item Paul fait rire les enfants. Paul les fait rire.
        \item La révolution a fait prendre corps [une grande espérance].
    \end{itemize}
\end{enumerate}

\subsubsection{Les proformes}
\paragraph{Le réfléchi avec faire}
remplacer le complément direct ou prépositionnel de l’infinitif.
\begin{enumerate}
    \item le réfléchi reprenant la cause s’attache à faire
    \begin{itemize}
        \item Max se fait coiffer par son frère.
        \item Marie s’est fait acheter un manteau par son assistant.
    \end{itemize}
    \item le réfléchi reprenant le causataire s’attache à l’infinitif
    \begin{itemize}
        \item Paul fait se laver les enfants.
        \item Le froid a fait s’acheter un manteau à Paul.
    \end{itemize}
\end{enumerate}

\paragraph{Les proformes figées avec faire}


\subsection{Les constructions de laisser avec infinitif}
可construction infinitive ordinaire或construction fusionnée
\begin{itemize}
    \item Paul laisse son fils voir ces films.
    \item Paul laisse voir ces films à son fils.
\end{itemize}


\section{Les verbes de perception}


\begin{table}[H]
    \centering
    \begin{adjustbox}{max width = \textwidth}
        \begin{tabular}{|l|l|l|}
    \hline
    \rowcolor{cyan!20}
    \textbf{PERCEPTION} & \textbf{NON INTENTIONNELLE} & \textbf{INTENTIONNELLE} \\
    \hline
    auditive & \textit{ENTENDRE}, \textit{ouïr} & \textit{ÉCOUTER} \\
    \hline
    gustative & \textit{SENTIR}, \% \textit{gouter} &\textit{GOÛTER}, \textit{déguster, savourer}  \\
    \hline
    olfactive & \textit{SENTIR} & \textit{SENTIR}   \textit{flairer, humer, renifler} \\
    \hline
    tactile & \textit{SENTIR}, \textit{ressentir}  & \textit{SENTIR}, \textit{palper, tâter, toucher} \\
    \hline
    visuelle & \textit{VOIR}, \textit{apercevoir, entrevoir, observer} & \textit{REGARDER},   \textit{contempler, fixer, guetter, lorgner, mirer, scruter, visionner} \\
    \hline
    non restreinte & \textit{PERCEVOIR}, \textit{déceler, discerner, noter, remarquer} & \quad --- \\
    \hline
    \end{tabular}
    
    \end{adjustbox}
\end{table}

\section{Les verbes support}

\subsection{Le passif des constructions à verbe support}

\subsubsection{Le passif personnel}
\begin{enumerate}
    \item le nom prédicatif est sujet. 只有当le verbe support a un complément direct才能使用passif,并用de引导complément d’agent
    \begin{itemize}
        \item La panique a pris/saisi les marchés.
        \item Les marchés ont été pris/saisis de panique.
    \end{itemize}
    \item le nom prédicatif est complément direct
    \begin{enumerate}
        \item pour verbales figées
        \begin{enumerate}
            \item nom prédicatif sans déterminant不能用passif,le déterminant est facultatif时可以用passif
            \begin{itemize}
                \item Paul a pris peur. | * Peur a été prise (par Paul).
                \item On rendra (un) hommage aux victimes.
                \item Un hommage/Hommage sera rendu aux victimes.
                \item On a pris (un) rendez-vous pour demain.
                \item Un rendez-vous/Rendez-vous a été pris pour demain.
            \end{itemize}
            \item le verbe support a un sujet non agentif时不能用passif. apprécier ou craindre是例外
            \begin{itemize}
                \item * La fuite a été prise.
                \item Les invités font honneur à ce diner.
                \item Honneur a été fait à ce diner par les invités.
                \item Leurs scrupules font honneur à vos amis.
                \item * Honneur est fait à vos amis (par leurs scrupules).
            \end{itemize}
        \end{enumerate}

        \item pour les verbes simples,描述état或qualité不能用passif;prendre与douche ou décision连用时可用passif
        \begin{itemize}
            \item Une décision a été prise.
            \item Combien de douches ont été prises depuis ce matin ?
            \item * Où le courage a-t-il été pris de parler ainsi ?
        \end{itemize}
    \end{enumerate}
\end{enumerate}

\subsubsection{Le passif impersonnel}
\begin{itemize}
    \item Il a été procédé à une enquête.
    \item * À une enquête a été procédé.
    \item Il a été fait appel à la population.
    \item ? Appel a été fait à la population.
\end{itemize}

\subsubsection{Les verbes supports converses}
un complément prépositionnel peut devenir sujet, le sujet du verbe support peut être supprimé, ou devenir complément prépositionnel introduit par \textit{de} ou \textit{de la part de}.


\begin{table}[H]
    \centering
    \begin{tabular}{|l|l|}
    \hline
    \rowcolor{cyan!20}
    \textbf{CONSTRUCTION DE BASE} & \textbf{CONSTRUCTION CONVERSE} \\
    \hline
    \textit{commettre un vol (contre qqn)} & \textit{subir un vol (de la part de qqn)} \\
    \hline
    \textit{donner l'autorisation à qqn de} & \textit{avoir / recevoir l'autorisation (de qqn) de} \\
    \hline
    \textit{faire des compliments à qqn} & \textit{recevoir des compliments de (la part de) qqn} \\
    \hline
    \textit{faire / pratiquer une opération sur qqn} & \textit{subir une opération} \\
    \hline
    \textit{infliger des brimades à qqn} & \textit{subir des brimades de (la part de) qqn} \\
    \hline
    \textit{lancer des injures à qqn} & \textit{essuyer des injures de (la part de) qqn} \\
    \hline
    \end{tabular}
\end{table}
\begin{itemize}
    \item Jean donne une gifle à Paul.
    \item Paul reçoit une gifle (de Jean).
    \item Jean commet une agression contre Paul.
    \item Paul subit une agression (de la part de Jean).
\end{itemize}


\begin{table}[H]
    \centering
    \small
    \setlength{\extrarowheight}{2pt} % Adds a little extra height to rows
    
    \begin{adjustbox}{max width = \textwidth}
        \begin{tabular}{|>{\raggedright\arraybackslash}p{0.48\textwidth}|>{\raggedright\arraybackslash}p{0.48\textwidth}|}
            \hline
            \rowcolor{cyan!20} % Light blue for header
            \textbf{VERBE SANS COMPLÉMENT PROPRE} & \textbf{VERBE AVEC COMPLÉMENT PROPRE} \\
            \hline
            \rowcolor{cyan!20}
            \multicolumn{2}{|c|}{\textbf{SUJET PRÉDICATIF}} \\ % Merged header row
            \hline
            (un événement) arriver, (un accident) avoir lieu, (un bruit) courir, (le match) se dérouler, (un spectacle) se donner, (un orage) éclater, (une histoire) se passer, (un événement) se produire, (le silence) régner, (un événement) survenir, (une réunion) se tenir & (une tempête) frapper (quelque part), (un événement) prendre place (quelque part), (une fête) tomber (tel jour), (un ouragan) toucher (un endroit) \\
            \hline
            \rowcolor{cyan!20}
            \multicolumn{2}{|c|}{\textbf{COMPLÉMENT DIRECT PRÉDICATIF}} \\ % Merged header row
            \hline
            accumuler (les erreurs), adopter (une position), arborer (une mine fière), arrêter (une décision), attraper (un coup de soleil), avoir (du courage), bâtir (un projet), bercer (un projet), brandir (une menace), caresser (un projet), commettre (un crime), concevoir (une idée), conduire (une enquête), connaître (la sérénité), contracter (un engagement), dégager (une odeur), déposer (une plainte), détenir (un secret), diriger (une enquête), dresser (un constat), édicter (un règlement), édifier (un projet), effectuer (une opération), élever (une objection), émettre (un avis), endurer (des souffrances), engager (une discussion), entreprendre (une action), éprouver (de la joie), essuyer (un orage), exercer (une influence), faire (un éloge), fixer (une règle), forger (un plan), former (des vœux), goupiller (un plan), instaurer (des règles), instruire (une affaire), jouer (un rôle), livrer (un combat), machiner (un complot), magouiller (une escroquerie), manifester (du mécontentement), mener (un combat), mitonner (un plan), monter (une combine), montrer (du courage), nourrir (de l'espoir), opérer (une retraite), ourdir (un complot), passer (un accord), perpétrer (un crime), piquer (une colère), pondre (un article), porter (une attaque), posséder (un savoir-faire), pousser (un cri), pratiquer (une opération), prendre (la fuite), produire (un effet), promulguer (un décret), prononcer (un discours), propager (une rumeur), ratifier (un accord), relever (un défi), rendre (un avis), ressentir (de la tristesse), soulever (une objection), souscrire (une assurance), soutenir (une idée), subir (un orage), tisser (des intrigues), tramer (un complot) & accorder (son pardon) (à), administrer (une correction) (à), adresser (des reproches) (à), allonger (une gifle) (à), allouer (une indemnité) (à), asséner (une gifle) (à), balancer (une plaisanterie) (à), concéder (un prêt) (à), conférer (une distinction) (à), cracher (des injures) (à), dicter (ses ordres) (à), dispenser (des consolations) (à), donner (son accord) (à), ficher (des coups) (à), filer (des coups) (à), flanquer (des coups) (à), formuler (des remarques) (à), impartir (un délai) (à), infliger (des reproches) (à), intenter (un procès) (à), intimer (un ordre) (à), lâcher (des injures) (à), lancer (un défi) (à), libeller (un message) (à), marteler (un ordre) (à), octroyer (une aide) (à), offrir (son aide) (à), poser (une question) (à), prêter (un appui) (à), procurer (une aide) (à), prodiguer (des conseils) (à), recevoir (un conseil) (de), vouer (de l'admiration) (à) \\
            \hline
            \rowcolor{cyan!20}
            \multicolumn{2}{|c|}{\textbf{COMPLÉMENT OBLIQUE PRÉDICATIF}} \\ % Merged header row
            \hline
            s'adonner (à l'escrime), faire étalage (de courage), faire montre (de courage), faire preuve (d'intelligence), jouir (d'une bonne santé), se livrer (à des confidences), procéder (à une opération), procéder (d'une grande méchanceté) & accabler (qqn) (de reproches), gratifier (qqn) (d'une récompense), nantir (qqn) (de provisions), soumettre (qqn) (à la torture) \\
            \hline
        \end{tabular}
    \end{adjustbox}
\end{table}








\chapter{Phrase}







\section{Les phrases verbales}
\subsection{La phrase à l’indicatif et au subjonctif}
\subsubsection{Le sujet}
\begin{enumerate}
    \item form de sujet
    \begin{enumerate}
        \item nom/pronom
        \item une phrase subordonnée
        \begin{itemize}
            \item Qu’il faille changer le papier peint ennuie le locataire
        \end{itemize}
        \item un syntagme à l’infinitif
        \begin{itemize}
            \item Réserver votre billet par Internet est impossible actuellement.
        \end{itemize}
    \end{enumerate}
    \item sujet inversé
    \begin{enumerate}
        \item 主语过长
        \begin{itemize}
            \item Ont été désignés pour cette mission Luc, Jean et Paul
        \end{itemize}
        \item interrogatives
        \begin{itemize}
            \item Où va Paul
        \end{itemize}
        \item exclamatives
        \begin{itemize}
            \item Quelle chance a Paul
        \end{itemize}
        \item certaines phrases au subjonctif
        \begin{itemize}
            \item Puisse Paul nous aider
        \end{itemize}
    \end{enumerate}
    \item sujet omis:大部分是verbes impersonnels;\textit{mieux vaut, peu importe, voici}
    \begin{itemize}
        \item Peu importe les conséquences.
        \item Faudrait prendre un peu plus de temps.
        \item Suffit qu’elle veuille, dit Joseph
    \end{itemize}
    \begin{enumerate}
        \item sujet omis的情况下,ne也应该省略,除了n’empêche和n’importe这个固定搭配
        \begin{itemize}
            \item N’empêche que tu aurais pu faire attention.
            \item Faut pas exagérer !
        \end{itemize}
    \end{enumerate}
\end{enumerate}

\subsubsection{Les compléments}

\begin{table}[H]
    \centering
    
    \begin{adjustbox}{max width=\textwidth}
        \begin{tabular}{|l|p{0.2\textwidth}|>{\RaggedRight}p{0.25\textwidth}|>{\RaggedRight}p{0.25\textwidth}|}
        \hline
        \rowcolor{cyan!20}
        \textbf{CATÉGORIE} & \textbf{ATTRIBUT} & \RaggedRight \textbf{COMPLÉMENT DIRECT} & \textbf{COMPLÉMENT OBLIQUE} \\
        \hline
        adjectif ou syntagme adjectival & \textit{Paul est [content].} & \textit{Ce tableau coute [cher].} & \textit{Luc vend [cher] ce tableau.} \\
        \hline
        adverbe ou syntagme adverbial & \textit{Ce tableau est [mieux].} & \textit{Celui-ci coute [davantage].} & \textit{Luc va [très bien].} \\
        \hline
        syntagme nominal & \textit{Paul est [mon ami].} & \textit{Paul mange [la pomme].} & \textit{Luc va [rue Madame].} \\
        \hline
        syntagme prépositionnel & \textit{Paul est [en forme].} & $-$ & \textit{Luc pense [à Marie].} \\
        \hline
        syntagme verbal & \textit{Ce tableau est [à vendre].} & \textit{Paul veut [venir demain].} & \textit{Luc va [faire les courses].} \\
        \hline
        phrase subordonnée & $-$ & \textit{Paul veut [que tu viennes].} & \textit{Je me souviens [qu'il neigeait].} \\
        \hline
        \end{tabular}
    \end{adjustbox}
\end{table}



\subsubsection{Les ajouts}
\begin{enumerate}
    \item adverbe ou syntagme adverbial
    \begin{itemize}
        \item Paul travaille [bien]
    \end{itemize}
    \item syntagme prépositionnel
    \begin{itemize}
        \item Paul garde espoir [malgré la crise]
    \end{itemize}
    \item adjectif ou syntagme adjectival invariable ou lié à un nom
    \begin{itemize}
        \item Lou a refusé [net]
        \item Lou est partie [furieuse]
    \end{itemize}
    \item syntagme nominal
    \begin{itemize}
        \item Paul travaille [le samedi]
    \end{itemize}
    \item pronom contrastif ou quantifieur
    \begin{itemize}
        \item Paul viendra, [lui]
        \item Les élèves viendront [tous]
    \end{itemize}
    \item syntagme verbal
    \begin{itemize}
        \item Haussant le ton, il est intervenu.
    \end{itemize}
    \item subordonnée circonstancielle, comparative, ou relative extraposée
    \begin{itemize}
        \item Alex viendra [quand il pourra]
        \item Alex ment [comme il respire].
        \item Des gens sont arrivés, [qui étaient énervés].
    \end{itemize}
    \item des incises ou commentaires 
    \begin{itemize}
        \item Lou a, [je crois], terminé son travail.
    \end{itemize}
    \item des termes d’adresse nominaux, des interjections ou des particules de discours
    \begin{itemize}
        \item Venez, [les enfants] !
        \item Tu peux me passer le sel, [s’il te plait] ?
    \end{itemize}
\end{enumerate}





\subsubsection{Les éléments en début de phrase verbale}
\begin{enumerate}
    \item élément marqueur:conjonction de coordination, subordonnant
    \item élément extrait:le mots, syntagmes interrogatifs ou exclamatifs
    \begin{enumerate}
        \item un pronom
        \begin{itemize}
            \item Qui Paul a-t-il rencontré hier ?
        \end{itemize}
        \item un syntagme nominal
        \begin{itemize}
            \item Quelle chance tu as !
        \end{itemize}
        \item un syntagme prépositionnel 
        \begin{itemize}
            \item À qui est-ce que vous pensez ?
        \end{itemize}
        \item un syntagme verbal
        \begin{itemize}
            \item Le laver, il faut.
        \end{itemize}
        \item un adjectif
        \begin{itemize}
            \item Quelle est la température ?
        \end{itemize}
        \item un adverbe
        \begin{itemize}
            \item Quand partiras-tu ?
        \end{itemize}
    \end{enumerate}
    \item un mot ou un syntagme ajout
    \item un mot ou un syntagme périphérique
    \begin{enumerate}
        \item nom, pronom ou syntagme nominal 
        \begin{itemize}
            \item Paul, on ne lui parle plus.
        \end{itemize}
        \item adjectif ou syntagme adjectival
        \begin{itemize}
            \item Plus grand que toi, personne ne peut l’être.
        \end{itemize}
        \item infinitif ou syntagme verbal
        \begin{itemize}
            \item Avoir vingt ans, ce n’est pas forcément le plus bel âge de la vie.
        \end{itemize}
        \item phrase subordonnée
        \begin{itemize}
            \item Que tu viennes demain, ça rassure tout le monde.
        \end{itemize}
    \end{enumerate}
\end{enumerate}

\begin{table}[H]
    \centering
    \begin{adjustbox}{max width =\textwidth}
        \begin{tabular}{|l|p{0.65\textwidth}|} % Adjust 0.65\textwidth as needed for your document layout
        \hline
        \rowcolor{cyan!20}
        \textbf{FONCTIONS} & \textbf{EXEMPLES} \\
        \hline
        MARQUEUR + EXTRAIT + AJOUT + PÉRIPHÉRIQUE & \textit{Mais comment, d'ailleurs, Paul, j'ai l'ai raté ?} \\
        MARQUEUR + EXTRAIT + PÉRIPHÉRIQUE + AJOUT & \textit{Mais comment, Paul, d'ailleurs, je l'ai raté ?} \\
        MARQUEUR + PÉRIPHÉRIQUE + EXTRAIT + AJOUT & \textit{Mais Paul, comment, d'ailleurs, je l'ai raté ?} \\
        MARQUEUR + PÉRIPHÉRIQUE + AJOUT + EXTRAIT & \textit{Mais Paul, d'ailleurs, comment je l'ai raté ?} \\
        MARQUEUR + AJOUT + PÉRIPHÉRIQUE + EXTRAIT & \textit{Mais d'ailleurs, Paul, comment je l'ai raté ?} \\
        MARQUEUR + AJOUT + EXTRAIT + PÉRIPHÉRIQUE & \textit{Mais d'ailleurs, comment, Paul, j'e l'ai raté ?} \\
        \hline
        EXTRAIT + PÉRIPHÉRIQUE + AJOUT + MARQUEUR & \textit{Comment Paul d'ailleurs est-ce que je l'ai raté?} \\
        EXTRAIT + AJOUT + MARQUEUR + PÉRIPHÉRIQUE & \textit{Comment d'ailleurs est-ce que Paul, je l'ai raté?} \\
        EXTRAIT + MARQUEUR + AJOUT + PÉRIPHÉRIQUE & \textit{Comment est-ce que d'ailleurs, Paul, je l'ai raté?} \\
        EXTRAIT + MARQUEUR + PÉRIPHÉRIQUE + AJOUT & \textit{Comment est-ce que Paul, d'ailleurs, je l'ai raté?} \\
        EXTRAIT + AJOUT + PÉRIPHÉRIQUE + MARQUEUR & \textit{Comment, d'ailleurs, Paul, est-ce que je l'ai raté?} \\
        EXTRAIT + PÉRIPHÉRIQUE + MARQUEUR + AJOUT & \textit{Comment, Paul, est-ce que d'ailleurs j'ai raté?} \\
        \hline
        PÉRIPHÉRIQUE + MARQUEUR + AJOUT + EXTRAIT & \textit{Moi, est-ce que, d'ailleurs, à Paul, j'ai parlé?} \\
        PÉRIPHÉRIQUE + AJOUT + EXTRAIT + MARQUEUR & \textit{Moi d'ailleurs, à Paul, est-ce que j'ai parlé?} \\
        PÉRIPHÉRIQUE + AJOUT + MARQUEUR + EXTRAIT & \textit{Moi d'ailleurs est-ce que à Paul, j'ai parlé?} \\
        PÉRIPHÉRIQUE + MARQUEUR + EXTRAIT + AJOUT & \textit{Moi, est-ce que, à Paul, d'ailleurs j'ai parlé?} \\
        PÉRIPHÉRIQUE + EXTRAIT + MARQUEUR + AJOUT & \textit{Moi, à Paul, est-ce que d'ailleurs j'ai parlé?} \\
        PÉRIPHÉRIQUE + EXTRAIT + AJOUT + MARQUEUR & \textit{Moi, à Paul, d'ailleurs, est-ce que j'ai parlé?} \\
        \hline
        \end{tabular}
    \end{adjustbox}
    \caption{\textbf{L'ordre des éléments en début de phrase}}
\end{table}

\subsection{La phrase à l’impératif}
\begin{enumerate}
    \item 无sujet syntaxique
    \item le, la, les, en, y替代complément时,它们要通过连字符置于动词后
    \begin{itemize}
        \item Apporte-le sur la table !
        \item Allons-y les amis !
    \end{itemize}
    \begin{enumerate}
        \item 否定句中proforme在动词前
        \begin{itemize}
            \item Ne l’apporte pas sur la table !
        \end{itemize}
    \end{enumerate}
    \item impératif只能是phrase indépendante,不能被subordonnant引导作为从句,但可被conjonction de coordination引导
\end{enumerate}

\subsection{Les phrases à l’infinitif et au participe présent}
\subsubsection{La phrase à l’infinitif}
不是句子而是syntagmes verbaux.
\begin{enumerate}
    \item sujet
    \begin{enumerate}
        \item syntagme nominal
        \begin{itemize}
            \item Et le silence de retomber 
        \end{itemize}
        \item nom propre
        \begin{itemize}
            \item Paul, se marier !
        \end{itemize}
        \item pronom fort
        \begin{itemize}
            \item Et lui de répliquer
        \end{itemize}
        \item 不能为弱形式
        \begin{itemize}
            \item *Et il de répliquer.
        \end{itemize}
    \end{enumerate}
    \item complement在动词后,但tout和rien在动词前
    \begin{itemize}
        \item Je lui ai dit d’arrêter [et lui de tout nier].
    \end{itemize}
    \item pas, 与其他副词一样置于动词前
    \begin{itemize}
        \item Paul, ne pas venir ?
        \item Et Paul de bien insister sur ce point.
    \end{itemize}
\end{enumerate}


\subsubsection{La phrase au participe présent}
\begin{enumerate}
    \item subordonnées circonstancielles de temps ou de cause
    \begin{itemize}
        \item L’hiver approchant, il faut rentrer les cactus.
        \item La discussion n’a pas pu aboutir, Paul ayant refusé de prendre part au vote
    \end{itemize}
    \item 被en引导时,不是subordonnées而是syntagme verbal
    \begin{itemize}
        \item En vieillissant, on comprend bien des choses.
    \end{itemize}
\end{enumerate}

\section{Les subordonnées}
\subsection{Les subordonnées sujet ou complément}

\begin{table}[H]
    \centering
    \small
    \begin{tabular}{|p{2.5cm}|p{3.5cm}|>{\RaggedRight}p{2.5cm}|p{4.5cm}|}
    \hline
    \rowcolor{cyan!20}
    \textbf{TYPE} & \textbf{INTRODUCTEUR} & \textbf{MODE} & \textbf{FONCTION} \\
    \hline
    déclarative & \textit{que, de ce que} & indicatif ou subjonctif & sujet, complément direct ou oblique \\
    \hline
    désidérative & \textit{que, à de que} & subjonctif & sujet, complément direct ou oblique \\
    \hline
    interrogative & \textit{si}, mot, syntagme interrogatif & indicatif &  sujet, complément direct ou oblique \\
    \hline
    exclamative & mot, syntagme exclamatif, \textit{que} & indicatif ou subjonctif &  sujet, complément direct ou oblique \\
    \hline
    \end{tabular}
\end{table}

\subsubsection{Les subordonnées sujets}
\begin{enumerate}
    \item Le mode des déclaratives sujets大部分是subjonctif,但也可以是indicatif
    \begin{itemize}
        \item Qu’on n’arrête pas de grandir désespérait les mères, obligées de rallonger les robes d’une bande de tissu
        \item Qu’il faille chauffer en mai arrive rarement
    \end{itemize}
    \item la subordonnée sujet也可出现在动词后
    \begin{itemize}
        \item Plût au ciel qu’il pleuve
        \item À cela s’ajoute qu’aucune décision n’a été prise
        \item Si vous n’avez pas permis que je devienne bon, d’où vient que vous m’ayez ôté l’envie d’être méchant
    \end{itemize}
\end{enumerate}

\subsubsection{Les subordonnées compléments}
\begin{enumerate}
    \item 用le ou ça 指代complément direct
    \begin{itemize}
        \item Paul pense [que tout va bien].|Paul le pense
    \end{itemize}
    \item 用en ou y 指代complément oblique
    \begin{itemize}
        \item Il se souvient [que le président était encore vivant à ce moment-là].|Il s’en souvient
        \item On fait l’hypothèse [que l’atome est sécable].|On en fait l’hypothèse
        \item Paul est certain [qu’il a bien répondu].|Paul en est certain
        \item Elle songe [à ce que tout sera prêt].|Elle y songe
    \end{itemize}
    \item 可作为adverbes或prépositions的compléments,承担circonstancielle的功能
    \begin{itemize}
        \item Nous pouvons parler, encore que vous ayez l’air pressé
        \item Nous commencerons l’inventaire, avant que tu arrives
    \end{itemize}
\end{enumerate}

\subsection{Les subordonnées périphériques}
\begin{enumerate}
    \item 置于句首或句末,主句用ce, en, y等proforme指代
    \begin{itemize}
        \item Qu’il faille recourir au référendum, c’est probable
        \item Que vous soyez en avance, qui s’en plaindrait ?
        \item Si et quand on sortira de la crise, personne ne le sait
        \item Je trouve ça incroyable, comme il a changé
    \end{itemize}
    \item 在subordonnée interrogative ou exclamative中,更倾向于用 subordonnée périphérique而不是subordonnée sujet 
    \begin{itemize}
        \item C’est mieux [que tu partes].|?[Que tu partes] est mieux
        \item Ça m’est égal [avec qui il négocie].|?[Avec qui il négocie] m’est égal.
        \item Ça m’épate [comme il est malin].|*[Comme il est malin] m’épate.
    \end{itemize}
\end{enumerate}

\subsection{Les subordonnées ajouts}

\begin{table}[H]
    \centering

\begin{adjustbox}{max width =\textwidth}
    \begin{tabular}{|>{\RaggedRight}p{3cm}|>{\RaggedRight}p{5.5cm}|>{\RaggedRight}p{3.5cm}|>{\RaggedRight}p{5cm}|}
    \hline
    \rowcolor{cyan!20}
    \textbf{SUBORDONNÉE} & \textbf{INTRODUCTEUR} & \textbf{MODE} & \textbf{EXEMPLES} \\
    \hline
    circonstancielle & \textit{lorsque, que, si}, etc., mot ou syntagme antéposé, adverbe ou préposition + \textit{que} ou sans introducteur & indicatif, subjonctif, ou participe présent & Je viendrai [si je peux]. Il est fatigué [tant il travaille]. [Depuis qu'il est rentré], il ne peut rien faire. [Le temps pressant], on va rentrer. \\
    \hline
    comparative & \textit{que, comme} & indicatif & Il a plus travaillé [qu'on lui avait dit]. Il a travaillé [comme on lui avait dit]. \\
    \hline
    relative & \textit{dont, que, qui, où}, ou syntagme avec un mot relatif & indicatif ou subjonctif & la fille [que je vois]   un endroit [où j'irai]   une femme [à qui je puisse parler] \\
    \hline
    incise & sans introducteur & indicatif & Paul, [fit-il], est idiot. \\
    \hline
    \end{tabular}
\end{adjustbox}

\end{table}

\begin{enumerate}
    \item circonstancielles commencent
    \begin{enumerate}
        \item par adverbe
        \begin{itemize}
            \item l est fatigué, tant il travaille
        \end{itemize}
        \item par syntagme antéposé suivi de \textit{que}
        \begin{itemize}
            \item Aussi habile qu’il soit, il ne peut pas réussir.
        \end{itemize}
        \item sans introducteur, quand elles sont au participe présent ou au subjonctif avec un sujet suffixé
        \begin{itemize}
            \item Le temps pressant, on va rentrer.
            \item Je ne l’écouterai pas, fût-il ministre
        \end{itemize}
    \end{enumerate}
\end{enumerate}

\section{Les coordonnées}

\subsection{Les phrases coordonnées}

\subsection{Les phrases juxtaposées}

\subsubsection{Les phrases juxtaposées coordonnées}

\subsubsection{Les phrases juxtaposées subordonnées}
temporelle, causale, concessive, conditionnel
\begin{enumerate}
    \item participe présent
    \begin{itemize}
        \item Il faudra, [l’hiver approchant], rentrer les cactus
    \end{itemize}
    \item subjonctif avec sujet suffixé
    \begin{itemize}
        \item Le directeur, [eût-il été plus attentif ], ne pouvait tout contrôler.
    \end{itemize}
    \item imparfait
    \begin{itemize}
        \item N’était la difficulté à trouver à manger, les réfugiés se sentaient à l’abri.
    \end{itemize}
    \item conditionnel
    \begin{itemize}
        \item Me supplierait-il à genoux, je ne recevrai pas cet homme
    \end{itemize}
    \item Était-il heureux, il chantait.
    \item 第二个句子可以被que引导
    \begin{itemize}
        \item Était-il heureux, qu’il chantait
    \end{itemize}
\end{enumerate}


\section{Les phrases à extraction}
\subsection{Les constructions à extraction}

\subsubsection{Les interrogatives partielles avec extraction}
主句或从句
\begin{itemize}
    \item À qui veux-tu parler ?
    \item Je me demande comment Paul va se rendre à Paris
\end{itemize}
\subsubsection{Les exclamatives avec extraction}
主句或从句
\begin{itemize}
    \item On sait combien il s’est sacrifié pour ses enfants
    \item Que de mensonges il se croit obligé d’inventer
\end{itemize}
\subsubsection{Les déclaratives avec extraction}
\begin{enumerate}
    \item Les antépositions avec inversion du sujet
    
    complément prépositionnel ou un adjectif attribut出现在句首,sujet在动词后 
    \begin{itemize}
        \item À cette potion amère, s’ajoute peu à peu un arrière-goût insidieux de mensonge.
        \item Rares sont ceux qui approuvent cette décision
        \item De ces contradictions si apparentes provient probablement le sentiment de malaise dont on ne peut se défaire tout au long de ce livre
    \end{itemize}
    \item Les antépositions sans inversion du sujet
    \begin{itemize}
        \item De cette affaire, on ne parle presque plus
        \item À cela, j’ai modestement pensé
        \item Légalement, ce dossier n’est pas défendable
    \end{itemize}
\end{enumerate}
\subsubsection{Les subordonnées relatives et l’extraction}
\begin{itemize}
    \item Voici le livre auquel je pense
\end{itemize}
\begin{enumerate}
    \item Les relatives sans antécédent et l’extraction
    \begin{itemize}
        \item Tu peux inviter qui tu préfères
        \item Paul s’adressera à qui on lui conseillera de s’adresser
    \end{itemize}
\end{enumerate}
\subsubsection{Les constructions clivées}
c’est + le foyer de la clivée + une subordonnée relative qui est une phrase à extraction
\subsubsection{Les subordonnées circonstancielles avec extraction}
\begin{enumerate}
    \item Les circonstancielles de cause avec extraction
    
    tant, tellement引导的从句
    \begin{itemize}
        \item Nous connaissions les traits de son visage mieux que ceux de notre mère ou de notre femme, tant nous avions passé de temps à étudier ses photos, à les comparer
        \item Il ne pouvait pas bouger, tellement il était saisi
    \end{itemize}
    \item Les circonstancielles de concession avec extraction
    \begin{enumerate}
        \item 含pronom(quoi, qui que ce soit)的syntagme nominal ou prépositionnel后跟que
        \begin{itemize}
            \item Quoi que tu me dises, je ne changerai pas d’avis
            \item À qui que ce soit que je m’adresse, on me repousse
        \end{itemize}
        \item aussi/si/quelque/tout + syntagme adjectival ou adverbial + que;没有que时用sujet suffixé
        \begin{itemize}
            \item Joseph, disait-elle, tout intelligent qu’il était, avait aussi sa bêtise
            \item Si généreusement qu’il se comporte, il ne se rachètera pas auprès du public
            \item Il nous reste un espoir, si mince soit-il
        \end{itemize}
    \end{enumerate}
\end{enumerate}
\subsubsection{Les comparatives avec extraction}

comme, que作为adverbes extraits




\subsection{Les propriétés des phrases à extraction}

\subsubsection{La fonction de l’élément manquant}

\paragraph{L’élément manquant correspond à un complément}
\begin{enumerate}
    \item complément de préposition作为extrait时不能没有préposition
    \begin{itemize}
        \item Pour quel projet as-tu voté ?
        \item * Quel projet as-tu voté pour ?
    \end{itemize}
\end{enumerate}
\paragraph{L’élément manquant correspond à un sujet}

当句首成分被看作是从句的主语时,它此时是extrait,从句被qui引导
\begin{itemize}
    \item Quel genre de personne croyez-vous qui viendra ?
\end{itemize}
\paragraph{L’élément manquant correspond à un spécifieur}
\begin{itemize}
    \item Combien avez-vous d’enfants ? (combien d’enfants)
    \item Que vous avez de chance ! (que de chance)
    \item Nous connaissions les traits de son visage mieux que ceux de notre mère ou de notre femme, tant nous avions passé de temps à étudier ses photos (tant de temp)
\end{itemize}
\paragraph{L’élément manquant correspond à un ajout}
\begin{itemize}
    \item Comment Paul travaille-t-il ?
    \item C’est l’endroit où Paul travaille.
    \item Pourquoi penses-tu qu’il vienne ?
\end{itemize}


\subsubsection{L’inversion du sujet dans les phrases à extraction}


\begin{enumerate}
    \item interrogatives indépendantes
    \begin{itemize}
        \item Quel livre lit Paul en ce moment ?
    \end{itemize}
    \item interrogatives subordonnées
    \begin{itemize}
        \item Je me demande quels livres lit Paul en ce moment
    \end{itemize}
    \item subordonnées relatives
    \begin{itemize}
        \item J’ai lu le livre que lit Paul en ce moment
    \end{itemize}
    \item subordonnées comparatives
    \begin{itemize}
        \item Il se conduit comme se conduisait son père
    \end{itemize}
    \item subordonnées circonstancielles
    \begin{itemize}
        \item Il faut y croire, si mince que cet espoir paraiss
        \item Il faut y croire, si mince que paraisse cet espoir
    \end{itemize}
    \item constructions clivées
    \begin{itemize}
        \item C’est ce livre que lisait Paul cet été
        \item C’est ce livre que lisait Paul cet été
    \end{itemize}
    \item constructions locatives ou attributives
    \begin{itemize}
        \item Sur la place se dressait une cathédrale.
        \item Rares sont les bacheliers de moins de 16 ans.
    \end{itemize}
    \item 疑问句中疑问词不在句首作为extrait时,不能用sujet inversé
    \begin{itemize}
        \item Paul viendra avec qui ?
        \item * Viendra Paul avec qui ?
    \end{itemize}
    \item extrait指代从句缺失的成分时,从句的主语可倒置
    \begin{itemize}
        \item Avec qui penses-tu [que Paul viendra ◊] ?
        \item Avec qui penses-tu [que viendra Paul ◊] ?
        \item Quand Marie a-t-elle dit [qu’était mort Paul ◊] ?
        \item Où l’expert a-t-il écrit [que baisserait la production ◊] ?
    \end{itemize}
\end{enumerate}


\subsubsection{La relation à distance entre l’élément extrait et l’élément manquant}
\begin{itemize}
    \item 不能作为extraction的结构
    \begin{enumerate}
        \item les sujets infinitifs ou subordonnés
        \begin{itemize}
            \item Saluer ce voisin m’arrive rarement
            \item  * Quel voisin est-ce que [saluer ◊] t’arrive rarement ?
        \end{itemize}
        \item les compléments prépositionnels
        \item les subordonnées interrogatives
        \item les subordonnées relatives et les constructions clivées
        \item les subordonnées circonstancielles
    \end{enumerate}
\end{itemize}
\paragraph{L’élément manquant appartient à un sujet}
\begin{enumerate}
    \item 名词是主语时,该名词的complément de nom可作为extrait
    \begin{itemize}
        \item  De quel livre [l’auteur ◊] est-il célèbre ?
    \end{itemize}
    \item 作为sujet infinitif ou à une subordonnée sujet,该动词的complément de verbe很难作为extrait
    \begin{itemize}
        \item Qu’on réduise l’espace réservé aux voitures est une de nos priorités
        \item * Quel espace [qu’on réduise ◊] est-il une de nos priorités ?
        \item Quel voisin t’arrive-t-il [de saluer ◊] ?
        \item * Quel voisin est-ce que [saluer ◊] t’arrive rarement ?
    \end{itemize}
\end{enumerate}
\paragraph{L’élément manquant appartient à un complément prépositionnel}
\begin{enumerate}
    \item préposition引导complément infinitif时可以extraire complément de verb
    \begin{itemize}
        \item Je vais finir par résoudre ce problème 
        \item Quel problème vas-tu finir [par résoudre ◊ SV] ?
    \end{itemize}
    \item préposition引导complément nominal很难extraire complément de nom
    \begin{itemize}
        \item Je vais finir par un livre de cet auteur
        \item * De quel auteur vas-tu finir par un livre
    \end{itemize}
\end{enumerate}
\paragraph{L’élément manquant appartient à une subordonnée interrogative}
\begin{enumerate}
    \item complément d’une subordonnée interrogative à l’infinitif时可以extraction
    \begin{itemize}
        \item Quel problème savez-vous [comment expliquer ◊ ◊] ?
        \item Vous savez [comment expliquer ce problème ◊].
    \end{itemize}
    \item verb conjugué则难extraire
    \begin{itemize}
        \item * Quel problème savez-vous [comment le professeur a expliqué ◊ ◊] ?
    \end{itemize}
    \begin{enumerate}
        \item complément prépositionnel比complément nominal更容易extraire
        \begin{itemize}
            \item \% À quel problème savez-vous [si le professeur a pensé ◊] ?
        \end{itemize}
    \end{enumerate}
\end{enumerate}
\paragraph{L’élément manquant appartient à une circonstancielle}
\begin{enumerate}
    \item complement de verb conjugué很难extraire
    \begin{itemize}
        \item Tu seras soulagé [quand tu verras Pierre].
        \item * Qui seras-tu soulagé [quand tu verras ◊] ?
    \end{itemize}
    \item pour + infinitif结构的complement可以extraire
    \begin{itemize}
        \item Il a fallu trente ans [pour construire ce bâtiment].
        \item C’est un bâtiment qu’il a fallu trente ans [pour construire ◊].
    \end{itemize}
\end{enumerate}
\paragraph{L’élément manquant appartient à une subordonnée relative}
\begin{enumerate}
    \item L’extraction d’un second élément hors de la relative n’est pas possible
    \begin{itemize}
        \item Je connais un endroit [où acheter des cigarettes ◊].
        \item * Quel genre de cigarettes connais-tu un endroit [où acheter ◊ ◊] ?
        \item Je connais celui [qui a écrit ce livre].
        \item * Quel livre connais-tu celui [qui a écrit ◊] ?
    \end{itemize}
\end{enumerate}
\paragraph{L’élément extrait appartient à une coordination}
\begin{itemize}
    \item * Quels amis as-tu [appelés ◊ et invité leurs enfants] ?
    \item Quels amis as-tu [appelés ◊ et invités ◊] ?
\end{itemize}


\section{Type de phrase}
\begin{table}[H]
\centering
\begin{tabular}{|l|l|l|}
\hline
\rowcolor{cyan!20}
\textbf{TYPE DE PHRASE} & \textbf{SOUS-TYPE} & \textbf{EXEMPLES} \\
\hline
déclarative & — & \textit{Marie a lu ces livres.} \\
& & \textit{Marie a lu ces livres ?} \\
& & \textit{Marie a lu ces livres !} \\
\hline
désidérative & impératif & \textit{Lis davantage de livres !} \\
\cline{2-3}
& subjonctif & \textit{Que Marie lise ces livres !} \\
& & \textit{Puisse Marie réussir !} \\
\hline
exclamative & à mot exclamatif & \textit{Comme Marie semble heureuse !} \\
& & \textit{Quelle chance elle a !} \\
\cline{2-3}
& à mot intensif-exclamatif & \textit{Marie a lu tant de livres !} \\
& & \textit{Paul a un tel courage !} \\
\hline
interrogative & partielle & \textit{Quels livres Marie a lus ?} \\
& & \textit{Qui est venu ?} \\
\cline{2-3}
& totale & \textit{Est-ce que Marie a lu ces livres ?} \\
& & \textit{Marie a-t-elle lu ces livres ?} \\
& & \textit{A-t-elle lu ces livres ?} \\
\hline
\end{tabular}
\end{table}

\begin{enumerate}
    \item désidératives
    \begin{enumerate}
        \item 第一三人称不用impératif,用subjonctif introduit par \textit{que}
        \begin{itemize}
            \item Qu’il vienne !
            \item Que je sois pendu si je comprends ce qui se passe !
        \end{itemize}
        \item subjonctif不加que也属于正式用法
        \begin{enumerate}
            \item souhait
            \begin{itemize}
                \item Puisse-t-elle lui donner la même éducation que Gunilla à Victor
            \end{itemize}
            \item regret
            \begin{itemize}
                \item Plût au ciel qu’il soit venu !
            \end{itemize}
        \end{enumerate}
        \item impératif用法强调une réalisation ou une action future
        \begin{enumerate}
            \item injonction
            \begin{itemize}
                \item Sortons !
            \end{itemize}
            \item souhait
            \begin{itemize}
                \item Dormez bien !
            \end{itemize}
        \end{enumerate}
        \item subjonctif用法强调愿望表达不一定以现实可行性为前提
        \begin{enumerate}
            \item injonction
            \begin{itemize}
                \item Qu’il vienne !
            \end{itemize}
            \item souhait
            \begin{itemize}
                \item Que la force soit avec toi !
            \end{itemize}
            \item imprécation
            \begin{itemize}
                \item Que je sois pendu si je comprends ce qui se passe !
            \end{itemize}
        \end{enumerate}
    \end{enumerate}
\end{enumerate}

\subsection{Les types de phrases indépendantes}
\begin{table}[H]
\centering
\begin{tabular}{|l|p{0.35\textwidth}|p{0.45\textwidth}|}
\hline
\rowcolor{cyan!20}
\textbf{TYPE DE PHRASE} & \textbf{FORME DU VERBE} & \textbf{AUTRE ÉLÉMENT LEXICAL} \\
\hline
déclarative & $-$ indicatif (\textit{Paul viendra.}) & $-$ \\
& $-$ infinitif (\textit{Et Paul de sursauter.}) & \\
\hline
désidérative & $-$ impératif (\textit{Venez ici !}) & $-$ \textit{que} + subjonctif (\textit{Qu'il vienne !}) \\
& $-$ subjonctif à sujet inversé ou suffixé (\textit{Puisse Marie vous entendre !}) & $-$ \textit{pourvu que} + subjonctif (\textit{Pourvu qu'il vienne !}) \\
\hline
exclamative & $-$ indicatif (\textit{Comme il pleut !}) & $-$ mot exclamatif : \textit{ce que, combien, comme, comment, que, quel, qu'est-ce que} \\
& $-$ infinitif (\textit{Paul, faire tant de bruit !})
& $-$ adverbe ou adjectif intensif-exclamatif : \textit{si, tant, tel, tellement} \\
\hline
interrogative & $-$ indicatif (\textit{Où allez-vous ?}) & $-$ mot interrogatif : \textit{combien, comment, lequel, où, pourquoi, quand, que, quel, qu'est-ce que, qu'est-ce qui, quoi, quoi, qui est-ce que, qui est-ce qui} \\
& $-$ indicatif à sujet suffixé (\textit{Viendrez-vous ?})
& $-$ introducteur : \textit{est-ce que} \\
\hline
\end{tabular}
\end{table}

\subsection{Les types de subordonnées complétives}


\begin{table}[H]
\centering
\begin{adjustbox}{max width=\textwidth}
    \begin{tabular}{|l|l|l|p{0.45\textwidth}|}
    \hline
    \rowcolor{cyan!20}
    \textbf{TYPE DE COMPLÉTIVE} & \textbf{INTRODUCTEUR} & \textbf{FORME DU VERBE} & \textbf{EXEMPLES} \\
    \hline
    déclarative & \textit{que} & indicatif ou subjonctif & \textit{Pierre pense [qu'elle a lu ces livres].} \\
    & & & \textit{Pierre regrette [qu'elle ait lu ces livres].} \\
    \hline
    désidérative & \textit{que} & subjonctif & \textit{Pierre ordonne [qu'elle lise ces livres].} \\
    & & & \textit{Pierre souhaite [qu'elle réussisse].} \\
    \hline
    interrogative & \textit{si} ou mot interrogatif & indicatif & \textit{Pierre se demande [quels livres elle a lus].} \\
    & & & \textit{Pierre se demande [qui est venu].} \\
    & & & \textit{Pierre se demande [si elle a lu ces livres].} \\
    \hline
    exclamative & \textit{que} ou mot exclamatif & indicatif ou subjonctif & \textit{Pierre sait [comme elle est heureuse].} \\
    & & & \textit{Pierre regrette [qu'elle soit si triste].} \\
    \hline
    \end{tabular}
\end{adjustbox}
\end{table} 



\section{Les phrases déclaratives}
\subsection{Subordonnées}
\subsubsection{La subordonnée déclarative sujet}
\begin{enumerate}
    \item 从句动词大多用subjonctif,个别用indicatif
    \begin{itemize}
        \item Que Paul soit absent m'étonne
        \item Ça m'étonne, que Paul soit absent
        \item Que la vie n'est pas rose en France et exige beaucoup d'opiniâtreté commence à se savoir
    \end{itemize}
    \item 可用Périphérique结构,主语用ce, cela, ça替代
    \begin{itemize}
        \item Ça m’étonne, que Paul soit absent
    \end{itemize}
    \item subordonnée sujet可倒置于动词后
    \begin{itemize}
        \item À cela s’ajoute qu’aucune décision n’a été prise
        \item D’où vient que Paul est en retard
        \item À quoi sert que vous ayez pris tant de précautions
    \end{itemize}
\end{enumerate}


\subsubsection{La subordonnée déclarative complément de nom}
\begin{enumerate}
    \item à l’indicatif après un nom de message ou d’objet abstrait
    \begin{itemize}
        \item La nouvelle que la fusée avait réussi le largage du satellite arriva enfin
        \item Personnen’aformulél’hypothèse que l’ancêtre commun pouvait être un hominidé
    \end{itemize}
    \item au subjonctif après un nom de qualité ou un nom de sentiment
    \begin{itemize}
        \item Ils allaient envisager la possibilité que le bâtiment soit construit
        \item L’obligation que les élèves soient prêts pour l’examen était évidente
        \item Personne n’éprouva le regret que le spectacle soit raté
    \end{itemize}
\end{enumerate}

\subsubsection{La subordonnée déclarative complément d’adjectif}

\begin{enumerate}
    \item à l’indicatif avec les adjectifs d’opinion
    \begin{itemize}
        \item Paul est certain que tout se passera comme prévu
    \end{itemize}
    \item au subjonctif après les adjectifs de sentiment 
    \begin{itemize}
        \item Paul est stupéfait que vous soyez arrivé à temps
        \item Paul est heureux que vous soyez arrivé à temps
    \end{itemize}
    \item 可以出现在construction disloquée en fonction périphérique 
    \begin{itemize}
        \item Paul en est certain, que tout se passera comme prévu
    \end{itemize}
    \item 伴随表达des propriétés de situations ou de propositions的形容词 (faux, légal, normal, possible, vrai)
    \begin{enumerate}
        \item subordonnée sujet 
        \begin{itemize}
            \item Que les enfants soient accueillis dès 8 heures est tout à fait normal
        \end{itemize}
        \item complément dans une construction impersonnelle
        \begin{itemize}
            \item Il est tout à fait normal que les enfants soient accueillis dès 8 heures
            \item Il est clair que Paul a fait une gaffe
        \end{itemize}
        \item périphérique dans une construction disloquée
        \begin{itemize}
            \item C’est tout à fait normal, que les enfants soient accueillis dès 8 heures
            \item C’est clair, que Paul a fait une gaffe
        \end{itemize}
        \item 表达propriétés de situations (imminent, inattendu, interdit) ou jugement de valeur sur une situation (bizarre, étonnant, important, intéressant, magnifique, regrettable)的形容词,la subordonnée est au subjonctif 
        \begin{itemize}
            \item Il est tout à fait normal que les enfants soient accueillis dès 8 heures
        \end{itemize}
        \item 表达propriétés de propositions  (avéré, clair, faux, vrai)的形容词
        \begin{enumerate}
            \item la subordonnée \textit{sujet} est au subjonctif
            \begin{itemize}
                \item Que Paul ait fait une gaffe est clair
            \end{itemize}
            \item 但subordonnée est à l’indicatif dans les constructions impersonnelles ou disloquées
            \begin{itemize}
                \item Il est clair que Paul a fait une gaffe
                \item C’est clair, que Paul a fait une gaffe
            \end{itemize}
        \end{enumerate}
    \end{enumerate}
\end{enumerate}

\subsubsection{La subordonnée déclarative après une préposition ou un adverbe}
\begin{enumerate}
    \item On devrait prévoir l’assemblée générale avant que le conseil de surveillance s’en mêle
    \item Le bureau se réunit pendant que les délégués préparent leurs interventions
    \item Le bureau se réunit alors que les délégués ne sont pas encore arrivés
\end{enumerate}




\subsubsection{La subordonnée déclarative complément de verbe}



\begin{table}[H]
\centering
\small   
\begin{tabular}{|p{0.5\textwidth}|p{0.35\textwidth}|}
\hline
\rowcolor{cyan!20}
\textbf{VERBES} & \textbf{EXEMPLES} \\
\hline

\textbf{D’activité intellectuelle et d’opinion :} \newline
\textit{admettre, s’apercevoir, apprendre, calculer, comprendre, constater, contester, croire, découvrir, douter, estimer, évaluer, exclure, ignorer, imaginer, inventer, juger, oublier, penser, savoir, se rappeler, se souvenir (de), supposer, trouver, vérifier}
&
\textit{Je sais [que Paul est rentré].} \newline
\textit{Je crois [qu’il va pleuvoir].} \\
\hline

\textbf{De communication :} \newline
\textit{affirmer, annoncer, avertir, dire, s’écrier, écrire, s’exclamer, expliquer, informer, murmurer, se plaindre (de), prétendre, prévenir, protester, raconter}
&
\textit{Marie dit [qu’elle a faim].} \newline
\textit{Paul lui murmura [qu’il devait partir].} \newline
\textit{Luc se plaint [de ce qu’on l’a prévenu trop tard].} \\
\hline

\textbf{De choix et de décision :} \newline
\textit{choisir, contrôler, décider, décréter, garantir, renoncer (à)}
&
\textit{J’ai décidé [que tout devait être fini pour demain].} \newline
\textit{On renonce [à ce que la réunion ait lieu ce soir].} \\
\hline

\textbf{D’engagement :} \newline
\textit{assurer, s’engager (à), jurer, promettre}
&
\textit{Je vous promets [que je serai à la hauteur].} \newline
\textit{Je m’engage [à ce que tout soit fini demain].} \\
\hline

\textbf{D’identité :} \newline
\textit{être, rester, sembler}
&
\textit{Le problème est [qu’on n’a plus le temps].} \newline
\textit{L’intérêt de l’infiltration reste [qu’on apaise la douleur].} \\
\hline

\textbf{Modaux :} \newline
\textit{il faut, il importe, nécessiter, il se peut}
&
\textit{Il se peut [que Paul vienne à Paris].} \newline
\textit{Il faut [que vous preniez une assurance].} \\
\hline

\textbf{De perception et présentatifs :} \newline
entendre, observer, sentir, voici, voilà, voir
&
\textit{Luc entend [que Max joue du piano].} \newline
\textit{Paul voit [que Marie est partie].} \\
\hline

\textbf{De sentiment et de réaction émotive :} \newline
admirer, apprécier, détester, s’étonner (de), regretter, se réjouir (de)
&
\textit{Paul regrette [que Marie soit partie].} \\
\hline
\end{tabular}
\end{table}


\subsubsection{La subordonnée complétive en quand}
verbes de sentiment et de réaction émotive (adorer, aimer, détester)
\begin{enumerate}
    \item circonstancielle de temps
    \begin{itemize}
        \item Jeanne est sortie quand Marie est arrivée
    \end{itemize}
    \item subordonnée interrogative
    \begin{itemize}
        \item Je me demande quand Marie arrive
    \end{itemize}
    \item complément de verbe
    \begin{itemize}
        \item J’aime quand on mange sur la terrasse
    \end{itemize}
    \begin{enumerate}
        \item complément direct,可被le ou ça指代
        \begin{itemize}
            \item J’aime ça, quand on mange sur la terrasse
        \end{itemize}
        \item complément oblique,必须被en指代
        \begin{itemize}
            \item on s’aperçoit quand elle manque
            \item On s’en aperçoit.
            \item *On s’aperçoit.
        \end{itemize}
    \end{enumerate}
    \item 与complétive en que的区别
    \begin{enumerate}
        \item quand不能作为主语
        \begin{itemize}
            \item * [Quand ils mettent tout le monde dans le même panier] m’énerve
        \end{itemize}
        \item 永远à l’indicatif,que在verb de sentiment后用subjonctif
        \begin{itemize}
            \item J’aime quand on part tôt
            \item J’aime qu’on parte tôt
        \end{itemize}
        \item quand不能extraire un complément hors d’une complétive
        \begin{itemize}
            \item Où est-ce que tu aimes [qu’on mange ◊] ?
            \item * Où est-ce que tu aimes [quand on mange ◊] ?
        \end{itemize}
    \end{enumerate}
\end{enumerate}


\section{Les phrases désidératives}
\subsection{Indépendantes}
\subsubsection{Les phrases désidératives à l’impératif}
主语用第二人称单数,第一二人称复数
\begin{enumerate}
    \item 无形式主语;pronom, nom propre ou un syntagme nominal vocatif只作为ajout
    \begin{itemize}
        \item Toi, viens par ici !
        \item Marie, attendez un instant.
        \item Les enfants, écoutez bien les recommandations.
    \end{itemize}
    \item proformes suffixées au verbe impératif
    \begin{itemize}
        \item Donne-le-moi !
        \item Vas-y !
    \end{itemize}
    \begin{enumerate}
        \item 否定句中proformes préfixées au verbe 
        \begin{itemize}
            \item M’en parlez pas !
            \item N’en prends pas !
            \item T’inquiète. = ne t’inquiète pas
            \item T’occupe. = ne t’en occupe pas
        \end{itemize}
    \end{enumerate}
\end{enumerate}

\subsubsection{Les phrases désidératives indépendantes au subjonctif}
\begin{enumerate}
    \item \textit{que} 引导,主语用不能出现在imperatif的人称:第一人称单数,第三人称单复数
    \begin{itemize}
        \item Qu’il aille au diable !
        \item Qu’ils viennent me voir !
        \item Que je sois pendu si je me trompe !
    \end{itemize}
    \item \textit{pourvu que} 引导 (souhait),主语能用所有人称
    \begin{itemize}
        \item Pourvu qu’il vienne !
        \item Pourvu que tu sois reçu à l’examen !
    \end{itemize}
    \item 无que引导
    \begin{enumerate}
        \item sujet avant le verbe 
        \begin{itemize}
            \item Dieu vous garde !
            \item Dieu soit loué !
            \item Grand bien lui fasse !
        \end{itemize}
        \item sujet inversé
        \begin{itemize}
            \item Soit x un nombre réel positif.
            \item Puisse le ciel vous aider !
        \end{itemize}
        \item sujet suffixé
        \begin{itemize}
            \item Puisse-t-il vous aider !
        \end{itemize}
        \item le sujet est une relative sans antécédent ou une subordonnée
        \begin{itemize}
            \item Comprenne qui pourra !
            \item Plût au ciel que vous l’ayez mieux mérité
        \end{itemize}
        \item 无主语
        \begin{itemize}
            \item Ne vous en déplaise !
            \item Plaise à Dieu !
        \end{itemize}
    \end{enumerate}
\end{enumerate}


\subsubsection{Les phrases désidératives sans verbe}
置于主语前后主语后
\begin{enumerate}
    \item un adjectif
    \begin{itemize}
        \item Haut les mains !
        \item Bas les pattes !
    \end{itemize}
    \item un syntagme prépositionnel 
    \begin{itemize}
        \item Les mains sur la table !
        \item Au diable l’avarice !
    \end{itemize}
    \item adverb
    \begin{itemize}
        \item Vivement les vacances !
        \item Doucement les basses !
        \item Plus vite la cadence !
    \end{itemize}
    \item \textit{vive} (万岁) 作为介词使用
    \begin{itemize}
        \item Vive les vacances !
        \item Vive nous !
    \end{itemize}
\end{enumerate}


\subsection{Les subordonnées désidératives}
用subjonctif,que / à ce que / de ce que引导,无人称限制
\begin{enumerate}
    \item sujets
    \begin{itemize}
        \item Qu’il vienne est souhaitable.
    \end{itemize}
    \item compléments de verbe
    \begin{itemize}
        \item Paul a ordonné que tout soit fini demain.
        \item Paul tient à ce que vous veniez
    \end{itemize}
    \item compléments d’adjectif
    \begin{itemize}
        \item Paul est désireux que Marie vienne.
        \item Il est requis que vous restiez
    \end{itemize}
    \item compléments de nom
    \begin{itemize}
        \item On a transmis l’ordre que tout soit fini demain.
    \end{itemize}
    \item compléments de préposition
    \begin{itemize}
        \item On travaille pour que tout soit fini demain.
    \end{itemize}
    \item périphériques
    \begin{itemize}
        \item Je le souhaite, [qu’il vienne].
    \end{itemize}
\end{enumerate}


\subsubsection{La subordonnée désidérative complément de verbe}
\begin{table}[H]
    \centering
    % Uncomment one of the following lines if the table still overflows or if you want it smaller:
    % \small        % Slightly smaller font
    % \footnotesize % Even smaller font
    % \scriptsize   % Very small font

    \begin{tabular}{| L{6.5cm} | L{8.5cm} |} % Two columns with specified widths
    \hline % Top horizontal line
    \rowcolor{cyan!20}
    \textbf{VERBES} & \textbf{EXEMPLES} \\
    \hline % Middle horizontal line (after header)
    \textbf{causatifs :} \newline \textit{empêcher, éviter, faire, tâcher} & \textit{Paul évite [que Marie vienne].} \newline \textit{La tempête n'a pas empêché [que ce rendez-vous soit un succès].} \\
    \hline
    \textbf{de désir et de volonté :} \newline \textit{attendre, chercher (à ce que), craindre, désirer, prier, souhaiter, tenir (à ce que), veiller (à ce que), viser (à ce que), vouloir} & \textit{Paul veut [que Marie vienne].} \newline \textit{Paul tient [à ce que Marie vienne].} \\
    \hline
    \textbf{d'identité :} \newline \textit{être, rester, sembler} & \textit{La consigne est [qu'on ait fini pour demain].} \newline \textit{L'objectif reste [qu'on soit plus nombreux].} \\
    \hline
    \textbf{d'influence et d'ordre :} \newline \textit{attendre (de qqn), autoriser (à ce que), conseiller (à qqn), convaincre, défendre (à qqn), demander (à qqn), dire (à qqn), encourager (à ce que), enjoindre à, exiger (de qqn), exhorter (à), forcer (à ce que), interdire (à qqn), inviter (à ce que), ordonner (à qqn), permettre (à qqn), persuader, prier, proposer (à qqn), réclamer (à qqn), recommander (à qqn), sommer} & \textit{Luc nous ordonne [que tout soit prêt].} \newline \textit{Luc nous encourage [à ce que tout soit prêt].} \newline \textit{Luc exige de nous [que tout soit prêt].} \\
    \bottomrule % Bottom horizontal line
    \end{tabular}
\end{table}


\subsubsection{L’usage des phrases désidératives}
\begin{table}[H]
\centering
\begin{adjustbox}{max width=\textwidth}
\begin{tabular}{|p{0.2\textwidth}|p{0.25\textwidth}|p{0.25\textwidth}|p{0.25\textwidth}|}
\hline
\rowcolor{cyan!20}
\textbf{DÉSIRÉRATIVES / PERFORMANCES} & \textbf{EXEMPLES} & \textbf{PARTICULES DE DISCOURS} & \textbf{DISCOURS RAPPORTÉ} \\
\hline
\rowcolor{cyan!10}
\multicolumn{4}{|l|}{\textbf{OBLIGATION}} \\
\hline
commande, ordre &
\textit{Taisez-vous !} \newline \textit{Viens ici !} \newline \textit{Que vos hommes soient prêts à l’aube !} &
\textit{un point c’est tout, point barre, et que ça saute, etc.} &
\textit{Il nous a ordonné [de nous taire].} \newline
\textit{Il a commandé [que nos hommes soient prêts à l’aube].} \\
\hline

conseil, instruction &
\textit{Couvre-toi bien !} \newline \textit{Battez les œufs en neige ferme.} &
--- &
\textit{Il nous a recommandé [de battre les œufs en neige ferme].} \\
\hline

exigence &
\textit{Remboursez !} &
\textit{donc} &
\textit{Il a exigé [qu’on le rembourse].} \\
\hline

prière, requête &
\textit{Veuillez patienter un instant !} &
\textit{s’il vous plaît, s’il te plaît, je vous en prie, voulez-vous, etc.} &
\textit{Il nous a priés [de patienter un instant].} \\
\hline

proposition, suggestion &
\textit{Essayez la crème brûlée !} &
\textit{je vous en prie, donc, plutôt} &
\textit{On nous suggère d’essayer la crème brûlée.} \\
\hline

invitation, offre &
\textit{Passez nous voir !} \newline \textit{Reprends une part de tarte !} &
\textit{donc} &
\textit{Il nous a invités à passer le voir.} \\
\hline

\rowcolor{cyan!10}
\multicolumn{4}{|l|}{\textbf{PERMISSION}} \\
\hline

autorisation, consentement, permission, etc. &
\textit{Fume, si tu en as envie.} \newline \textit{Servez-vous.} &
\textit{si vous voulez, si tu y tiens, d’accord, O.K., etc.} &
\textit{On m’a permis de fumer.} \newline
\textit{On nous a autorisés à nous servir.} \\
\hline

\rowcolor{cyan!10}
\multicolumn{4}{|l|}{\textbf{SOUHAIT}} \\
\hline

malédiction, souhait, vœu, etc. &
\textit{Dors bien !} \newline \textit{Que le meilleur gagne !} \newline \textit{Le diable l’emporte !} &
--- &
\textit{On souhaite [que le diable l’emporte].} \\
\hline

\end{tabular}
\end{adjustbox}

\end{table}

\section{Les phrases interrogatives}
\subsection{Les phrases interrogatives indépendantes}
\begin{table}[H]
\centering
\begin{adjustbox}{max width=\textwidth}
\begin{tabular}{|p{0.25\textwidth}|p{0.25\textwidth}|p{0.25\textwidth}|p{0.25\textwidth}|}
\hline
\rowcolor{cyan!20}
\textbf{INTERROGATIVE} & \textbf{TOTALE} & \textbf{PARTIELLE} & \textbf{ALTERNATIVE} \\
\hline

\textit{est-ce que} &
\textit{Est-ce qu’il part ?} &
\textit{Où est-ce que tu vas ?} &
\textit{Est-ce qu’il part ou non ?} \newline \textit{Est-ce que tu vas à Aix ou à Nice ?} \\
\hline

sujet suffixé &
\textit{Paul part-il ?} &
\textit{Où vas-tu ?} &
\textit{Paul part-il ou non ?} \newline \textit{Vas-tu à Aix ou à Nice ?} \\
\hline

sujet inversé &
--- &
\textit{Où va Paul ?} &
--- \\
\hline

avec seulement mot interrogatif &
--- &
\textit{Où tu vas ?} \newline \textit{Tu vas où ?} &
--- \\
\hline
\end{tabular}
\end{adjustbox}
\end{table}


\subsubsection{Les interrogatives totales}
询问la vérité de la proposition, 用oui, non来回答
\begin{enumerate}
    \item 不能用sujet inversé
    \begin{itemize}
        \item * Part Paul ?
        \item Paul part-il?
    \end{itemize}
\end{enumerate}
\paragraph{avec est-ce que}
à tous les temps , à l’indicatif
\begin{enumerate}
    \item construction présentative的疑问形式中,être可根据不同时态而改变形式;但est-ce que所有时态都保持不变
    \begin{itemize}
        \item Si Pierre n’est pas là, c’est qu’il est en retard.
        \item  Pourquoi Pierre n’est-il pas venu ? Était-ce qu’il était malade ?
        \item Si Pierre n’est pas là, serait-ce qu’il est malade ?
        \item Pierre n’est pas là. Est-ce qu’il est malade ?
    \end{itemize}
    \item constructions clivées的疑问形式中,c’est est séparé de que,区别于与est-ce que
    \begin{itemize}
        \item C’est Paul que tu as invité.
        \item Est-ce Paul que tu as invité ?
        \item Est-ce que Paul est arrivé ?
    \end{itemize}
    \item est-ce que置于句首或跟随adverbe, extrait ou périphérique
    \begin{itemize}
        \item Est-ce qu’on peut faire confiance à Paul ?
        \item Habituellement, est-ce que tu fais confiance à Paul ?
        \item À Paul, est-ce qu’on peut faire confiance ?
        \item Marie, est-ce que tu l’as vue récemment ?
    \end{itemize}
    \item est-ce que不能与所有形式的l’inversion du sujet连用
    \begin{itemize}
        \item * Est-ce que part-il ?
        \item * Est-ce que Paul part-il ?
        \item * Est-ce que part Paul ?
    \end{itemize}
\end{enumerate}
\paragraph{avec un verbe à sujet pronominal suffixé}
\begin{enumerate}
    \item une phrase sans sujet nominal
    \begin{itemize}
        \item Aura-t-elle le temps de nous parler ?
    \end{itemize}
    \item sujet nominal
    \begin{itemize}
        \item Marie aura-t-elle le temps de nous parler ?
    \end{itemize}
    \item sujet syntagme verbal infinitif
    \begin{itemize}
        \item Aller au théâtre vous plairait-il ?
    \end{itemize}
    \item sujet subordonnée
    \begin{itemize}
        \item Que Paul vienne au théâtre est-il envisageable ?
    \end{itemize}
\end{enumerate}


\subsubsection{Les interrogatives partielles}
询问un participant ou un élément de la situation,le mot interrogatif correspondant à l’information manquante

\begin{enumerate}
    \item un mot ou syntagme interrogatif 在句首:
    \begin{enumerate}
        \item + est-ce que
        \begin{itemize}
            \item Quand est-ce que vous partez ?
        \end{itemize}
        \item + un sujet inversé
        \begin{itemize}
            \item Quand part Paul ?
        \end{itemize}
        \item + un sujet suffixé
        \begin{itemize}
            \item Quand partez-vous ?
            \item Quand Paul part-il ?
        \end{itemize}
    \end{enumerate}
    \item le mot interrogatif 也可出现在动词后
    \begin{itemize}
        \item Vous partez quand ?
    \end{itemize}
\end{enumerate}


\paragraph{Les différents mots interrogatifs}

\begin{table}[H]
\centering
\renewcommand{\arraystretch}{1.3}
\begin{tabular}{|>{\bfseries}l|p{6cm}|p{6cm}|}
\hline
\rowcolor{cyan!20}
CATÉGORIE & \textbf{MOT SIMPLE} & \textbf{MOT AGGLOMÉRÉ} \\
\hline
adjectif & \textit{quel} & --- \\
\hline
adverbe & \textit{combien, comment, pourquoi, quand, que} & \textit{pourquoi pas} \\
\hline
déterminant & \textit{quel} & --- \\
\hline
préposition & \textit{auquel, duquel, où} & \textit{à quoi bon} \\
\hline
pronom & \textit{lequel, que, qui, quid, quoi} & \textit{que diable, qui est-ce qui, qui est-ce que, qu’est-ce qui, qu’est-ce que} \\
\hline
\end{tabular}
\end{table}
\begin{enumerate}
    \item Qui crois-tu [être le meilleur] ?
    \item * Tu crois qui être le meilleur ?
    \item Qui crois-tu [qui soit le meilleur] ?
    \item * Qui crois-tu que soit le meilleur ?.
\end{enumerate}

\paragraph{Les interrogatives partielles avec mot interrogatif en début de phrase}
\begin{enumerate}
    \item pronom que只能指代:
    \begin{enumerate}
        \item un complément direct
        \begin{itemize}
            \item Que fais-tu cette semaine ?
        \end{itemize}
        \item un attribut
        \begin{itemize}
            \item Que devient Paul ?
        \end{itemize}
        \item sujet d’une subordonnée
        \begin{itemize}
            \item Que m’as-tu dit qui était arrivé ?
        \end{itemize}
    \end{enumerate}
    \item adjectif quel只能指代un attribut
\end{enumerate}


\paragraph{Les interrogatives partielles avec est-ce que}
\begin{enumerate}
    \item est-ce que可与sujet nominal inversé ou non搭配,但不能与 sujet suffixé搭配
    \begin{itemize}
        \item Où donc est-ce que Paul part ?
        \item Où donc est-ce que part Paul ?
        \item * Où donc est-ce que Paul part-il ?
    \end{itemize}
    \item est-ce que不能与un syntagme interrogatif en fonction sujet搭配
    \begin{itemize}
        \item Quel livre vient de paraitre ?
        \item * Quel livre est-ce que vient de paraitre ?
        \item Combien de personnes sont venues ?
        \item * Combien de personnes est-ce que sont venues ?
    \end{itemize}
    \item Avec est-ce que, le mot interrogatif ne peut pas être en position postverbale, sauf question de reprise
    \begin{itemize}
        \item Je n’ai pas bien entendu : Est-ce que Paul part où ?
    \end{itemize}
\end{enumerate}

\paragraph{Les interrogatives partielles avec un verbe à sujet suffixé}
\begin{enumerate}
    \item syntagme interrogatif在句首时,用sujet suffixé;sujet nominal会构成 inversion complexe
    \begin{itemize}
        \item Où vas-tu ?
        \item Où Paul va-t-il ?
    \end{itemize}
    \item interrogatif作为sujet时,不能用sujet suffixé
    \begin{itemize}
        \item * Qui est-il parti ?
        \item * Lesquels sont-ils arrivés ?
    \end{itemize}
\end{enumerate}


\paragraph{Les interrogatives partielles avec mot interrogatif après le verbe}
非正式表达
\begin{enumerate}
    \item quoi只能用于postverb
    \begin{itemize}
        \item Finalement, on fait quoi ?
        \item * Quoi fait-on ?
    \end{itemize}
    \item Est-ce que与sujet suffixé 不能与interrogatif postverbal连用
    \begin{itemize}
        \item * Est-ce que tu pars où ?
        \item * Pars-tu où ?
    \end{itemize}
    \item le syntagme interrogatif不能被extrait用interrogatif postverbal
    \begin{enumerate}
        \item un ajout circonstanciel
        \begin{itemize}
            \item Il s’est défendu en accusant qui ?
            \item * Qui s’est-il défendu en accusant ?
        \end{itemize}
        \item un ajout à un nom
        \begin{itemize}
            \item Il a acheté un appartement de quelle surface ?
            \item * De quelle surface a-t-il acheté un appartement ?
        \end{itemize}
        \item un ajout à un complément de nom introduit par une préposition autre que \textit{de}
        \begin{itemize}
            \item Tu as raté la sortie vers quelle ville ?
            \item \# Vers quelle ville as-tu raté la sortie ?
        \end{itemize}
    \end{enumerate}
    \item interrogatif dans une subordonnée
    \begin{enumerate}
        \item subordonnée circonstancielle ou relative的疑问词只可出现在从句中,不能出现在该句句首
        \begin{itemize}
            \item Il était là [quand son fils passait quel concours] ?
            \item * Quel concours était-il là [quand son fils passait ◊] ?
            \item Il a construit une machine [qui sert à quoi] ?
            \item * À quoi a-t-il construit une machine [qui sert ◊] ?
        \end{itemize}
        \item subordonnée complétive:从句的疑问词可出现在句首
        \begin{enumerate}
            \item que引导从句 + interrogatif出现在从句中 = 疑问句
            \begin{itemize}
                \item Tu crois [qu’on devrait partir quand] ? = Quand crois-tu [qu’on devrait partir ◊] ?
                \item Il faut qu’on aille où, à ton avis ? = Où faut-il [qu’on aille ◊], à ton avis ?
            \end{itemize}
            \item interrogatif在从句首引导从句 = 陈述句
            \begin{itemize}
                \item Il a dit [que Marie devait s’adresser à qui] ? = À qui a-t-il dit que Marie devait s’adresser ?
                \item il a dit [à qui Marie devait [s’adresser ◊]].
            \end{itemize}
        \end{enumerate}
    \end{enumerate}
\end{enumerate}


\paragraph{Les énoncés interrogatifs à l’infinitif}

\begin{enumerate}
    \item sujet est omis
    \begin{itemize}
        \item Qui contacter ?
        \item Que dire?
        \item Quel scénario envisager ?
    \end{itemize}
    \item 不能用quel, qui est-ce que, qu’est-ce que, est-ce que
    \begin{itemize}
        \item * Quel être ?
        \item * Qui est-ce que contacter ?
        \item * Pourquoi est-ce que se mettre en colère ?
    \end{itemize}
\end{enumerate}


\subsubsection{Les interrogatives alternatives}
与interrogatives totales的形式相同
\begin{enumerate}
    \item avec est-ce que 
    \begin{itemize}
        \item Est-ce qu’on part ou est-ce qu’on ne part pas ?
    \end{itemize}
    \begin{enumerate}
        \item ou si可替换第二个问句的疑问词
        \begin{itemize}
            \item Est-ce que tu préfères que je te fasse manger avant de partir, ou si tu aimes mieux que je te prépare tout pour que tu manges à midi ?
        \end{itemize}
    \end{enumerate}
    \item avec un verbe à sujet suffixé
    \begin{itemize}
        \item Viendrez-vous à la remise du prix ou préférez-vous rester chez vous ?
    \end{itemize}
\end{enumerate}

\subsection{Les subordonnées interrogatives}

\subsubsection{Les formes des subordonnées interrogatives}
\begin{table}[H]
    \centering
    % 如果表格溢出或您希望字体更小,可以取消注释以下行之一:
    % \small        % 略小字体
    % \footnotesize % 更小字体
    % \scriptsize   % 非常小字体

    % 表格定义:4列,所有列都有竖线
    % 列宽根据内容估算
    \begin{tabular}{| L{3.5cm} | L{4cm} | L{4.5cm} | L{3cm} |}
    \hline
    \textbf{INTERROGATIVE} & \textbf{SI} & \textbf{MOT OU SYNTAGME \newline INTERROGATIF} & \textbf{SANS VERBE} \\
    \hline
    totale & \textit{Je demande [si Luc vient].} & — & — \\
    \hline
    partielle & — & \textit{Je demande [qui viendra].} \newline \textit{Je demande [qui mange quoi].} & \textit{Je demande [comment].} \\
    \hline
    alternative & \textit{Je demande [si Luc vient ou non].} \newline \textit{Je demande [si Léa va à Aix ou à Nice].} & — & — \\
    \hline
    \end{tabular}
\end{table}
\paragraph{Les subordonnées interrogatives partielles}
\begin{enumerate}
    \item que和quoi替换为ce que, ce qui
    \begin{itemize}
        \item * On a cherché quoi tu avais fait aujourd’hui
        \item * On a cherché qu’avait fait Paul aujourd’hui
        \item On a cherché ce que tu avais fait aujourd’hui
        \item On a cherché ce qui te plairait
    \end{itemize}
    \begin{enumerate}
        \item quoi只有在被modifié时才能使用
        \begin{itemize}
            \item On a cherché quoi d’autre te plairait
        \end{itemize}
    \end{enumerate}
    \item 从句首用si ou un mot interrogatif 引导,只有 interrogatives multiples时interrogatif才能出现在动词后
    \begin{itemize}
        \item elle tenait quand même à tout passer en revue, comme un propriétaire un peu maniaque explique à son futur locataire [où on range quoi] 
    \end{itemize}
    \begin{enumerate}
        \item La présence de si est incompatible avec celle d’un mot interrogatif, sauf s’il s’agit d’une question de reprise
        \begin{itemize}
            \item Je n’ai pas bien entendu : tu me demandes [si Paul va où] ?
        \end{itemize}
    \end{enumerate}
\end{enumerate}

\paragraph{Les compléments interrogatifs à l’infinitif}
不能构造interrogatives alternatives ou totales
\begin{itemize}
    \item On se demande [où aller].
    \item Il faudrait décider [quoi faire].
    \item Je vais vous dire [quand partir].
    \item * On se demande [si partir ou non].
    \item *Je vais vous dire [si partir].
\end{itemize}

\subsubsection{La subordonnée interrogative sujet}
\begin{itemize}
    \item Si ces différences entrainent ou non des comportements violents n’est pas clair du tout
    \item À qui tu devras t’adresser dépendra des circonstances
    \item Qui elle rencontre importe peu.
\end{itemize}
\begin{enumerate}
    \item 可作为sujet inversé
    \begin{itemize}
        \item Peu leur importait qui régnait sur les Flandres, ou si c’était l’an 1529 de l’Incarnation du Christ
    \end{itemize}
    \item 可作为périphérique
    \begin{itemize}
        \item Qui est invité, qui ne l’est pas, ce n’est pas clair
    \end{itemize}
\end{enumerate}

\subsubsection{La subordonnée interrogative complément d’adjectif, de nom ou de préposition}

\begin{itemize}
    \item On peut se poser la question si la Suisse doit continuer à honorer d’anciens contrats
    \item Alors, je ne suis pas certain si je dois répondre aux questions posées
    \item Je n’ai aucune idée avec qui il sort
\end{itemize}
从句前可用介词
\begin{itemize}
    \item On n’a aucune idée [de quand il rentre].
    \item Je suis déterminée et consciente [de pour quoi et pour qui je suis là] 
    \item Il était intéressé [par qui faisait quoi]
\end{itemize}

\subsubsection{La subordonnée interrogative complément de verbe}

\begin{enumerate}
    \item compléments de verbe (cela, le)
    \begin{itemize}
        \item Je sais [si Marie viendra].|Je le sais.
        \item  Je me demande [qui viendra].|Je me le demande.
    \end{itemize}
    \begin{enumerate}
        \item introduites par une préposition ou un adverbe interrogatif,依旧是compléments de verbe
        \begin{itemize}
            \item Je sais [comment je vais faire].|Je sais cela.
            \item Je me demande [à qui elle parle].|Je me le demande.
        \end{itemize}
    \end{enumerate}
    \item complément oblique (en, de cela)
    \begin{itemize}
        \item Je ne me souviens pas [quand on part].|Je ne m’en souviens pas.
        \item On s’étonne [avec qui elle sort].|On s’en étonne.
        \item Tout dépend [si on a le temps de faire le voyage].|Tout dépend de cela.
    \end{itemize}
    \item de + subordonnée interrogative
    \begin{itemize}
        \item Le vote L. P. ne dépend pas de la façon dont on parle de l’immigration ou de l’Islam, mais [de si on en parle ou pas]
        \item Elle voulait s’assurer [de qui il était et de comment il agissait avant de s’engager]
        \item Ça dépend [de combien il gagne par mois].
        \item Qu’est-ce que tu sais [de qui a fait quoi] ? 
    \end{itemize}
\end{enumerate}

\subsubsection{Les subordonnées interrogatives et les relatives sans antécédent}
qui, où, prép. + qui与relatives sans antécédent不同


\subsection{Les phrases interrogatives sans verbe}

\subsubsection{Les phrases interrogatives sans verbe indépendantes}
\begin{enumerate}
    \item quid, pourquoi pas, à quoi bon, à quand只能用于该类形式
    \begin{itemize}
        \item Quid de votre projet ?
        \item Et Papillon, euh, pourquoi ce nom ? 
    \end{itemize}
    \item 可用ou ça, 但在position initiale不能使用
    \begin{itemize}
        \item Le concert a lieu demain. Où ça ?
        \item Le concert a lieu où ça ?
        \item * Où ça a lieu le concert ?
    \end{itemize}
\end{enumerate}



\subsubsection{Les subordonnées interrogatives sans verbe}
à quand, pourquoi
\begin{itemize}
    \item Je me demande [à quand la blouse d’infirmière]
    \item On se demande [pourquoi tant de haine].
    \item Il a dit qu’il viendrait. Je ne sais pas [avec qui].
\end{itemize}


\subsection{La variation dans les phrases interrogatives}

\begin{table}[H]
    \centering
    \begin{tabular}{| L{3.5cm} | L{5.5cm} | L{5.5cm} |}
    \hline
    \rowcolor{cyan!20}
    \textbf{INTERROGATIVE} & \textbf{TOTALE} & \textbf{PARTIELLE} \\
    \hline
    mot interrogatif & — & \% \textit{Où c'est que tu vas ?} \newline \% \textit{C'est où que tu vas ?} \\
    \hline
    est-ce que ou que & \! \textit{Est-ce que l'homme peut-il tout connaitre ?} & \% \textit{Où est-ce que c'est que tu vas ?} \newline \% \textit{Où'ce que tu vas ?} \newline \% \textit{Où est-ce tu vas ?} \newline \! \textit{Où que tu vas ?} \\
    \hline
    particule -\textit{ti} ou -\textit{tu} & \% \textit{T'es-ti content ?} \newline \% \textit{Paul est-ti content ?} \newline \% \textit{T'es-tu content ?} \newline \% \textit{Paul est-tu content ?} & — \\
    \hline
    \end{tabular}
\end{table}

\subsubsection{La variation dans les phrases interrogatives totales indépendantes}

\begin{enumerate}
    \item L’extension de la forme \textit{-t-il}
    \begin{itemize}
        \item 可与主语的人称不一致
        \begin{itemize}
            \item ! Et l’autre personne ne serait-il pas J.-B. H. ?
            \item ! Son exclusion de la direction du P. le renvoie-t-il à la case martyr ?
            \item ! La courte victoire de G. l’autorisera-t-il à promouvoir le changement ou à le retenir ?
        \end{itemize}
        \item 有时可与est-ce que连用
        \begin{itemize}
            \item ! Est-ce que l’homme peut-il tout connaitre ?
        \end{itemize}
    \end{itemize}
    \item verbe + ti, verbe + tu (Québec)
    \begin{itemize}
        \item \% C’est-ti prêt ?
        \item \% Le gouvernement te donne-tu beaucoup d’argent ?
        \item \% Tu t’achètes-tu du linge des fois ? 
    \end{itemize}
    \begin{enumerate}
        \item 不能与est-ce que, mot interrogatif (除了固定搭配comment ça va)连用, 或用于subordonnée中
        \begin{itemize}
            \item \% Si c’est ti pas mignon ça !
            \item * On sait pas où on va-tu.
        \end{itemize}
    \end{enumerate}
\end{enumerate}

\subsubsection{La variation dans les interrogatives partielles indépendantes}
\begin{enumerate}
    \item mot interrogatif + est-ce que, + c’est que/qui ou ce que(Québec)
    \begin{itemize}
        \item ! Comment que ça s’appelle déjà donc ce coin-là ?
        \item ! Comment ce qu’on dit ça ?
        \item Quand est-ce que tu es allée en Tunisie ?
        \item Où c’est que t’as été te promener, dit cette punaise de Chantal 
        \item \% Où c’est que t’as été te promener, dit cette punaise de Chantal
        \item \% Qui c’est qui commande ici ?
        \item \% Comment c’qu’y va, Bobby ?
    \end{itemize}
    \item mot interrogatif + que (Québec)
    \begin{itemize}
        \item ! Pourquoi que ça a été encouragé de de donner le biberon ?
        \item ! Depuis quand que t’as un souffle au cœur hein ?
    \end{itemize}
    \begin{enumerate}
        \item où c’que, ousque
        \begin{itemize}
            \item ! Ousque j’irais, un coup divorcée ?
        \end{itemize}
        \item qui que, qui qui
        \begin{itemize}
            \item ! Qui qui a installé ça ?
            \item ! Qui que tu as vu ?
        \end{itemize}
    \end{enumerate}
    \item mot interrogatif en position canonique
    \begin{itemize}
        \item On va où ?
    \end{itemize}
    \item mot interrogatif initial sans est-ce que ni sujet inversé
    \begin{itemize}
        \item Où on va comme ça ?
    \end{itemize}
\end{enumerate}

\subsubsection{La variation dans les subordonnées interrogatives}
\begin{enumerate}
    \item est-ce que用于les subordonnées interrogatives totales
    \begin{itemize}
        \item ! elle est belle elle se demande [est ce que elle va se marier avec Aladin ou...]
        \item ! Je me suis demandé [est-ce qu’ils auront le temps de finir].
    \end{itemize}
    \item La variation dans les subordonnées interrogatives partielles
    
\end{enumerate}
\begin{table}[H]
        \begin{tabular}{|l|l|}
        \hline
        \rowcolor{cyan!20}
        \textbf{MOT INTERROGATIF} & \textbf{EXEMPLES} \\
        \hline
        + \textit{est-ce que}, \textit{qui est-ce que}, \textit{qu'est-ce que} & {\% J'ai entendu [\textit{où est-ce qu'il est allé}]}. \\
        & ! Je me demande [\textit{qu'est-ce qu'il cherche}]. \\
        \hline
        + \textit{c'est que} / \textit{qui} & {\% Je me demande [\textit{où c'est qu'il est allé}]}. \\
        & ! Je me demande [\textit{qu'est-ce que c'est qu'il a bu}]. \\
        \hline
        + \textit{que} & ! Je me demande [\textit{où qu'il est allé}]. \\
        \hline
        + verbe à sujet suffixé & ! On se demande [\textit{quand arriveront-ils}]. \\
        \hline
        en position \textit{canonique} & ! Je me demande [\textit{c'est quoi, son problème}]. \\
        & ! Je sais pas [\textit{il va où}]. \\
        \hline
        \end{tabular}
    \caption{La variation dans les subordonnées interrogatives partielles}
\end{table}

\section{Les phrases exclamatives}

\subsection{Les phrases exclamatives à mot exclamatif}


\begin{table}[H]
    \centering 
    \begin{tabular}{|l|l|l|}
    \hline
    \rowcolor{cyan!20}
    \textbf{CATÉGORIE} & \textbf{FORME} & \textbf{EXEMPLES} \\
    \hline
    adjectif & \textit{quel} & \textit{Quelle ne fut pas sa surprise !} \\
    \hline
    adverbe & \textit{ce que}, \textit{qu'est-ce que} & \textit{Ce qu'on a pu rire !} \\
    & & \textit{Qu'est-ce qu'on a ri !} \\
    \hline
    adverbe & \textit{combien} & \textit{Combien il a souffert !} \\
    & & \textit{Combien de bêtises il a dites !} \\
    \hline
    adverbe & \textit{comme} & \textit{Comme il a souffert !} \\
    & & \textit{Comme c'est beau !} \\
    \hline
    adverbe & \textit{que} & \textit{Que de fois il s'est trompé !} \\
    & & \textit{Qu'il est beau !} \\
    \hline
    déterminant & \textit{quel} & \textit{Sur quel ton il nous parle !} \\
    & & \textit{Quelle chance a Marie !} \\
    \hline
    \end{tabular}
\end{table}

\subsubsection{quel}

\begin{enumerate}
    \item 作为adjectif,引导包含être的exclamative,总用在句首作为attribut du sujet
    \begin{itemize}
        \item Et s’il allait à la chasse sur son éléphant et qu’il rencontrât un de ces tigres mangeurs d’homme [...], quel serait mon désespoir !
    \end{itemize}
    \item 作为déterminant
    \begin{enumerate}
        \item 在句首引导sujet或extrait,前可用介词
        \begin{itemize}
            \item Quel calme se répandait !
            \item avec quelle mesquinerie d’esprit et de moyens, et dans quel désordre, ne cherchons-nous pas encore aujourd’hui !
        \end{itemize}
        \item syntagme prépositionnel 可偶尔出现在动词后
        \begin{itemize}
            \item Vous m’avez trahi, abusé de quelle façon ! 
        \end{itemize}
    \end{enumerate}
\end{enumerate}

\subsubsection{combien}
普遍作为extrait出现在句首
\begin{enumerate}
    \item 最常作为ajout
    \begin{itemize}
        \item Combien il a souffert, le pauvre !
        \item Oh, combien il faut se méfier des majuscules !
    \end{itemize}
    \item 较少作为peser ou couter等动词的complement
    \begin{itemize}
        \item \% Combien toutes ces responsabilités pèsent sur ses épaules !
        \item \% Combien ça coute, toutes ces visites !
    \end{itemize}
    \item combien +  adjectival/adverbial
    \begin{itemize}
        \item Mais [combien long] est le circuit !
        \item Combien facilement elle donne au spectacle une apparence d’art ! 
    \end{itemize}
    \begin{enumerate}
        \item combien可与其修饰词的形容词/副词分开
        \begin{itemize}
            \item Combien le circuit est long !
        \end{itemize}
    \end{enumerate}
    \item combien + de + nom : sujet, complément, ajout
    \begin{itemize}
        \item Et combien de livres sont devenus littéralement introuvables !
        \item Il a été trahi par combien d’amis
        \item Combien de voyages insensés il a faits !
        \item Combien de fois s’est-il trompé !
    \end{itemize}
    \begin{enumerate}
        \item combien可与de nom分开,但当该SN为complément de préposition时二者不能分开
        \begin{itemize}
            \item Combien il a fait [de voyages insensés] !
            \item Combien il s’est trompé [de fois] !
            \item * Combien il a été trahi par d’amis !
        \end{itemize}
        \item combien可单独作为sujet使用,作为syntagme nominal sans nom;指人或“一般物”
        \begin{itemize}
            \item Combien sont morts !
            \item Combien reste à faire !
        \end{itemize}
    \end{enumerate}
\end{enumerate}



\subsubsection{que}
\begin{enumerate}
    \item 作为extrait ajout总用于句首,修饰verbe或attribut
    \begin{itemize}
        \item Que je t’aime !
        \item Que Paul est futé !
        \item Qu’il risque gros dans cette affaire !
    \end{itemize}
    \item que + de + nom : sujet ou extrait,que也可与de nom分开 
    \begin{itemize}
        \item Que de personnes se sont trompées !
        \item Que de progrès il a faits !
        \item Qu’il a fait de progrès !
    \end{itemize}
    \begin{enumerate}
        \item 与SN en combien不同,que引导的SN不能作为complément出现在verbe或préposition后
        \begin{itemize}
            \item Égalité, [que de crimes] on commet en ton nom !
            \item * Égalité, on commet [que de crimes] en ton nom !
            \item À combien de guichets fermés il s’est heurté !
            \item * À que de guichets fermés il s’est heurté !
        \end{itemize}
    \end{enumerate}
\end{enumerate}

\subsubsection{comme et ce que, qu’est-ce que}
\begin{enumerate}
    \item comme指代
    \begin{enumerate}
        \item complément de manière
        \begin{itemize}
            \item Comme il se comporte, celui-là !
        \end{itemize}
        \item attribut de sujet
        \begin{itemize}
            \item Comme il est, celui-là !
        \end{itemize}
        \item ajout
        \begin{itemize}
            \item Comme tu as eu raison de partir !
            \item Comme il est futé !
            \item Comme elle faisait de fines reprises, ta mère !
        \end{itemize}
    \end{enumerate}
    \item ce que et qu’est-ce que指代
    \begin{enumerate}
        \item complément direct
        \begin{itemize}
            \item Ce qu’une mère peut faire pour ses enfants !
            \item Qu’est-ce que ça peut couter, tous ces déplacements !
        \end{itemize}
        \item ajout 
        \begin{itemize}
            \item Qu’est-ce qu’on a ri !
            \item Ce que ça coute cher, tous ces déplacements !
        \end{itemize}

    \end{enumerate}
\end{enumerate}

\subsubsection{La structure des phrases exclamatives à mot exclamatif}

\begin{enumerate}
    \item extrait允许sujet inversé
    \begin{itemize}
        \item Comme me pèsent ces réunions incessantes !
        \item Que de problèmes fait éclater chacun de ces mots ! 
    \end{itemize}
    \item combien, que, quel还允许sujet suffixé
    \begin{itemize}
        \item Dans quel état te mets-tu !
    \end{itemize}
    \item extrait自身可属于subordonnée或complément infinitif
    \begin{itemize}
        \item Combien d’incapables ne m’a-t-il pas dit [qu’il connaissait ◊] !
        \item Combien de sottises faut-il [que je commette ◊] !
        \item Quels beaux voyages je sais [que tu vas faire ◊] !
        \item À combien de sollicitations je regrette aujourd’hui [de n’avoir pas cédé ◊] !
    \end{itemize}
\end{enumerate}

\subsection{Les autres phrases exclamatives}
\begin{enumerate}
    \item mot intensif-exclamatif
    \begin{enumerate}
        \item adverb si, tant, tellement
        \item adjectif tel
    \end{enumerate}
    \item sans verbe 
    \begin{enumerate}
        \item avec un mot exclamatif
        \item avec mot intensif-exclamatif
    \end{enumerate}
\end{enumerate}


\begin{table}[H]
    \centering 
    \begin{tabular}{|l|l|l|}
    \hline
    \rowcolor{cyan!20}
    \textbf{CATÉGORIE} & \textbf{FORME} & \textbf{EXEMPLES} \\
    \hline
    adjectif & \textit{tel} & \textit{Il m'a fait un tel caprice !} \\
    \hline
    adverbe & \textit{si} & \textit{Ce tableau est si beau !} \\
    \hline
    adverbe & \textit{tant} & \textit{Elle a tant de chance !} \\
    & & \textit{Je l'aime tant !} \\
    \hline
    adverbe & \textit{tellement} & \textit{Elle a tellement de chance !} \\
    & & \textit{Ce tableau est tellement beau !} \\
    & & \textit{Je l'aime tellement !} \\
    \hline
    \end{tabular}
\caption{Les phrases exclamatives à mot intensif-exclamatif}
\end{table}

\subsubsection{Les phrases exclamatives avec adverbe intensif-exclamatif}
与déclarative构造一样
\begin{enumerate}
    \item tant与只verbe或de + nom连用;si与除verbe之外的词连用;tellement与所有的词连用
    \item complément de verbe
    \begin{itemize}
        \item Ce sac pèse tant !
    \end{itemize}
    \item ajout 
    \begin{enumerate}
        \item 修饰adjectif
        \begin{itemize}
            \item Il est si beau !
        \end{itemize}
        \item 修饰adverbe
        \begin{itemize}
            \item Tout est allé si vite !
        \end{itemize}
        \item 修饰verbe 
        \begin{itemize}
            \item Ils travaillent tellement !
        \end{itemize}
        \item 修饰nom
        \begin{itemize}
            \item J’ai eu si peur !
        \end{itemize}
    \end{enumerate}
\end{enumerate}


\paragraph{Emplois non exclamatifs de tant, tellement et si}

\begin{enumerate}
    \item 前指某种相同程度 “这样,那样”
    \begin{itemize}
        \item Est-il si riche ?
        \item Pourquoi tant de haine ?
        \item Arrête de tant travailler !
    \end{itemize}
    \item si, tant et tellement …… que + subordonnée consécutif
    \begin{itemize}
        \item Il marchait tellement vite que personne ne pouvait le suivre.
        \item Il n’a pas tant de tort qu’on pourrait croire.
    \end{itemize}
\end{enumerate}


\paragraph{Tant et tellement + nom}
\begin{enumerate}
    \item sujet
    \begin{itemize}
        \item Tant de soldats sont morts !
    \end{itemize}
    \item complément de verbe
    \begin{itemize}
        \item Elle a eu tellement de chance !
    \end{itemize}
    \item complément de préprosition
    \begin{itemize}
        \item Elle marche avec tant d’élégance !
    \end{itemize}
    \item ajout
    \begin{itemize}
        \item Elle s’est trompée tant de fois !
    \end{itemize}
    \item tant和tellement可与de nom分开,除了作为complément de préprosition这种情况
    \begin{itemize}
        \item Elle a tellement eu [de chance] !
        \item Elle s’est tant trompée [de fois] !
        \item * Elle a tant marché avec d’élégance !
    \end{itemize}
    \item 可单独用,作为syntagme nominal sans nom;指人或“一般物”
    \begin{itemize}
        \item Tellement/Tant sont morts !
        \item Tant/Tellement reste à faire !
        \item J’ai tant/tellement à faire !
    \end{itemize}
\end{enumerate}

\subsubsection{Les phrases exclamatives avec adjectif intensif-exclamatif}

\begin{enumerate}
    \item 用于nom前或article indéfini后,不能作为attribut après le verbe
    \begin{itemize}
        \item Pour entrer dans mon garage, c’est une telle acrobatie ! 
        \item * Il est tel !
    \end{itemize}
\end{enumerate}

\paragraph{Les emplois non exclamatifs de tel}

\begin{enumerate}
    \item subordonnée consécutive
    \begin{itemize}
        \item Il éprouva un tel plaisir [qu’il se mit à sourire].
    \end{itemize}
    \item comparative
    \begin{itemize}
        \item Il portait un foulard tel [qu’en portent les femmes].
    \end{itemize}
    \item anaphorique
    \begin{itemize}
        \item Comment a-t-il pu commettre une telle erreur ?
    \end{itemize}
\end{enumerate}

\subsubsection{Les phrases exclamatives sans verbe}
\begin{enumerate}
    \item un mot exclamatif ou intensif-exclamatif 引导 syntagme nominal / syntagme adjectival作为prédicat ; syntagme nominal (可用que引导), infinitif, subordonnée作为sujet,可前置或后置
    \begin{itemize}
        \item Ce Pierre, quel génie !
        \item Quel génie que ce Pierre !
        \item Quel dommage de partir si tôt !
        \item Quelle chance que vous soyez là !
        \item Tellement heureux, le cousin !
    \end{itemize}
    \item 感叹词不能用 comme, ce que et qu’est-ce qu,其余可用
\end{enumerate}

\subsection{L’interprétation des phrases exclamatives}
\subsubsection{La phrase exclamative négative}
\begin{enumerate}
    \item exclamatif用于否定结构前,不能在其中
    \item 使用combien de, que de et quel;很难用comme, ce que, qu’est-ce que
    \item 意思等同于肯定感叹句
    \begin{itemize}
        \item Combien de fois n’ai-je pas entendu ce genre d’anecdote !
        \item Que d’impairs n’a-t-il pas commis pendant l’entrevue !
        \item À quels drames du passé n’ont-ils pas échappé !
    \end{itemize}
\end{enumerate}

\subsection{L’exclamative de degré}

\paragraph{L’exclamative de grande quantité}
\begin{enumerate}
    \item 用combien, que, ce que, qu’est-ce que引导,较少用quel,还可用tant, tellement, 较少用tel 
    \item 与复数或不可数单数名词相连
\end{enumerate}
\begin{itemize}
        \item Combien de fois ai-je entendu ce genre d’anecdote !
        \item Que d’eau a été répandue !
        \item Quelle poussière ils ont transportée !
        \item Il dépense un tel argent !
        \item Comme il pleut en Normandie ! (longtemps, souvent)
        \item Ce qu’on a marché ce matin ! (la durée ou la distance)
        \item Il a tellement toussé cette nuit ! (la fréquence)
        \item Qu’est-ce qu’on meurt dans cette ville ! (le nombre de personnes concernées)
    \end{itemize}


\paragraph{L’exclamative de grande intensité}
可用所有感叹词

\begin{itemize}
    \item Comme il a souffert !
    \item Il a tellement insisté pour voir ce film !
    \item Il a eu tant de peine !
    \item Comme elle est belle !
    \item Il lui infligea une telle honte !
    \item Quelle honte il lui infligea !
\end{itemize}


\paragraph{L’exclamative et les expressions de degré extrême}
\begin{enumerate}
    \item 自身已经包含degré extrême的形容词 (catastrophique, extraordinaire, immense, sublime) 难与très或combien, comme, que, ce que, qu’est-ce que连用,但可与quel连用
    \begin{itemize}
        \item \% Que cette réforme est catastrophique !
        \item \% Comme cet arbre est immense !
        \item Quel arbre immense il a !
        \item Quel film sublime j’ai vu hier !
    \end{itemize}
    \item 感叹词也很难与adverbe de haut degré或superlatif连用
    \begin{itemize}
        \item \# Comme elle est très intelligente !
        \item \# Comme elle est la meilleure !
    \end{itemize}
\end{enumerate}


\subsubsection{L’exclamative basée sur un modèle}
quel引导; ce modèle virtuel exemplifie toutes les qualités, ou au contraire les défauts, de la classe d’objets ou d’individus décrits par le nom, et l’exclamative dit à quel point un objet ou un individu s’en rapproche ou s’en éloigne
\begin{itemize}
    \item Quel chapeau tu as !
    \item Quel spectacle j’ai vu hier ! Un enchantement !
    \item Quel avocat était votre père ! 
\end{itemize}



\subsubsection{L’exclamative de manière}
只用comme引导;L’interprétation d’une exclamative de manière oppose en quelque sorte ‘une très bonne manière’ à ‘une très mauvaise manière’. 
\begin{itemize}
    \item Comme la convenance a explosé pendant ce diner !
    \item Comme il a éteint le feu de cheminée, le jeune pompier, hier soir !
    \item Comme il a dormi !
\end{itemize}


\subsection{Les subordonnées exclamatives}
可作为从句的感叹句:
\begin{enumerate}
    \item introduites par un mot exclamatif,除了que
    \item incluent un mot intensif-exclamatif,并用que引导从句,根据动词决定用indicatif还是subjonctif
    \begin{itemize}
        \item Je sais qu’il a tant souffert. 
        \item Je regrette qu’il ait tant souffert.
    \end{itemize}
\end{enumerate}


\subsubsection{La subordonnée exclamative sujet}
\begin{enumerate}
    \item 只有使用mot intensif-exclamatif的subordonnée exclamative才可作为sujet,从句动词普遍用subjonctif
    \begin{itemize}
        \item Qu’il ait tant souffert nous étonne.
        \item * Comme il est malin est incroyable.
    \end{itemize}
    \item 可在construction disloquée置于句尾
    \begin{itemize}
        \item Ça nous étonne, qu’il ait tant souffert.
    \end{itemize}
    \item variantes à l’infinitif
    \begin{itemize}
        \item C’est incroyable de souffrir tellement !
    \end{itemize}
\end{enumerate}


\subsubsection{La subordonnée exclamative complément d’adjectif}
\begin{enumerate}
    \item 两种类型的exclamative都可使用,subordonnée à mot intensif-exclamatif用que引导,较少用de ce que
    \begin{itemize}
        \item Paul est heureux qu’on ait tant ri
        \item Je suis d’ailleurs toujours très étonné [à quel point les pêcheurs sont au courant de ce qui s’est pris dans les étangs].
        \item aul est heureux de ce qu’on ait tant ri
    \end{itemize}
    \item 有时可作为complément d’une préposition
    \begin{itemize}
        \item \% Paul était indifférent à [combien elle avait souffert].
    \end{itemize}
\end{enumerate}

\subsubsection{ La subordonnée exclamative complément de nom}
subordonnée à mot intensif-exclamatif 用que引导,较少置于介词后
\begin{itemize}
    \item le fait [qu’elle ait tant souffert]
    \item l’impression [qu’elle a tant souffert]
    \item \% le récit de [comme elle était heureuse]
\end{itemize}

\subsubsection{La subordonnée exclamative complément de verbe}
que引导,较少用de ce que
\begin{itemize}
    \item Il nous raconte [comme il a été heureux].|Il nous le raconte.
    \item Il se souvient [comme il a été heureux].|Il s’en souvient.
    \item Il sait [qu’il a été tellement heureux].
    \item Il se souvient [de ce qu’il a été tellement heureux].
\end{itemize}

\subsection{La variation dans les phrases exclamatives}

\subsubsection{Comment}
\begin{enumerate}
    \item comme的变形,与comme不同的是,comment可以用在动词后
    \begin{itemize}
        \item ! Comment elle parle, celle-là !
        \item ! Comment c’est vieux, ce truc !
    \end{itemize}
    \item comment + que / est-ce que / ce que
    \begin{itemize}
        \item ! Comment que je suis trop fort !
        \item ! T’as vu comment que c’était beau !
    \end{itemize}
\end{enumerate}

\subsubsection{verbe + -ti / -tu}
\begin{enumerate}
    \item ti可加可不加连字符,可写为-ty ou -t’y
    \begin{itemize}
        \item \% Il sonne-ti fort, celui-là, hein ! 
    \end{itemize}
    \item -tu Québec
    \begin{enumerate}
        \item 不能与 mot exclamatif 连用或用于subordonnée
        \begin{itemize}
            \item * Comme c’est-tu beau !
            \item * Je pense que c’est-tu beau !
            \item \% C’est-tu assez fort !
        \end{itemize}
        \item -tu + pas不表示否定,而表示intensité
        \begin{itemize}
            \item \% C’est-tu pas damnant !
            \item \% C’est-tu pas innocent !
        \end{itemize}
    \end{enumerate}
    \item donc/don Québec
    \begin{itemize}
        \item \% C’est donc ben beau la réussite !
        \item \% C’est don ben plate ! 
    \end{itemize}
    \item -tu与assez但不能与ben连用,donc可与ben但不能与assez连用
    \begin{itemize}
        \item * C’est-tu ben beau !
        \item * C’est don assez plate !
    \end{itemize}
\end{enumerate}

\chapter{La subordonnée relative}
\subsubsection{Introducteur}

\begin{enumerate}
    \item une proforme relative
    \item un syntagme contenant une proforme relative
    \item un subordonnant invariable : qui, que, dont
    \begin{enumerate}
        \item 有两种不同功能的qui
        \begin{enumerate}
            \item un pronom relatif, qui suit une préposition
            \begin{itemize}
                \item J’avais rendez-vous avec Paul, [[au fils de qui] tu m’avais dit de m’adresser].
            \end{itemize}
            \item un subordonnant qui correspond toujours à un sujet
            \begin{itemize}
                \item l’homme [qui est parti]
            \end{itemize}
        \end{enumerate}
    \end{enumerate}
    \item 只有introduites par un mot relatif ou un syntagme contenant un mot relatif的从句动词才可用infinitif
    \begin{itemize}
        \item un homme [avec qui parler]
    \end{itemize}
\end{enumerate}


\begin{table}[H]
    \centering
    \small
    \begin{longtable}{|>{\RaggedRight}m{4cm}|m{3cm}|m{7cm}|}
    \hline
    \rowcolor{cyan!20}
    \textbf{INTRODUCTEUR} & \textbf{ÉLÉMENT RELATIVISÉ} & \textbf{EXEMPLES} \\
    \hline
    \endfirsthead
    \hline
    \rowcolor{cyan!20}
    \textbf{INTRODUCTEUR} & \textbf{ÉLÉMENT RELATIVISÉ} & \textbf{EXEMPLES} \\
    \hline
    \endhead
    \hline
    \endfoot
    \hline
    \endlastfoot
    
    \textit{auquel}, \textit{duquel}, \textit{où}, prép. + \textit{lequel}, prép. + \textit{qui}, prép. + \textit{quoi}, \textit{pourquoi} & complément \newline ou ajout & \textit{la fille [de qui je parle]} \newline \textit{l'homme [auquel je pense]} \newline \textit{un endroit [où aller]} \\
    \hline
    \textit{lequel}, \textit{lequel} + nom & sujet & \textit{Paul, [lequel était parti].} \newline \textit{un notaire, [lequel notaire était parti]} \\
    \hline
    \textit{qui} & sujet & \textit{l'homme [qui est venu]} \\
    \hline
    \textit{que} & complément \newline ou ajout & \textit{l'homme [que j'ai vu]} \newline \textit{le jour [qu'il est parti]} \\
    \hline
    \textit{dont} & complément en \textit{de} \newline ou proforme & \textit{la fille [dont je parle]} \newline \textit{une difficulté [dont il est clair qu'il faut la surmonter]} \\
    \hline
    \end{longtable}
    \caption{Les principaux types de subordonnées relatives}
\end{table}

\subsubsection{Antécédent}
\begin{enumerate}
    \item un nom, un pronom ou un syntagme nominal
    \item un adjectif
    \begin{itemize}
        \item satisfait [que j’étais]
    \end{itemize}
    \item une préposition 
    \begin{itemize}
        \item là [où je vais]
    \end{itemize}
    \item un syntagme verbal ou une phrase
    \begin{itemize}
        \item Il a crié, à quoi je n’ai pas réagi.
    \end{itemize}
    \item extraposée, c’est-à-dire ajout au verbe
    \begin{itemize}
        \item Un homme est venu [qui était énervé].
    \end{itemize}
    \item second complément d’un perception verbe transitif ou dans une construction clivée
    \begin{itemize}
        \item Je le vois qui arrive.
        \item C’est Paul [que je vois].
    \end{itemize}
\end{enumerate}

\section{La structure des subordonnées relatives}
\subsection{Les subordonnées relatives introduites par une proforme relative}


\begin{table}[H]
    \centering
    \begin{tabular}{|l|l|l|}
    \hline
    \rowcolor{cyan!20}
    \textbf{CATÉGORIE} & \textbf{MOT} & \textbf{EXEMPLES} \\
    \hline
    adverbe & {\textit{pourquoi}} & {\textit{ce [pourquoi on s'est battu]}} \\
    \hline
    déterminant & {\textit{lequel}} & {\textit{un notaire, [lequel notaire était sourd]}} \\
    \hline
    préposition & {\textit{auquel, duquel, où}} & {\textit{l'homme [auquel je pense]}} \\
    & & {\textit{l'endroit [où je vais]}} \\
    \hline
    pronom & {\textit{lequel, qui, quoi}} & {\textit{un ami [sur qui compter]}} \\
    & & {\textit{un ami, [lequel était parti]}} \\
    \hline
    \end{tabular}
    \caption{Les mots relatifs}
\end{table}

\begin{enumerate}
    \item qui用于介词后,指代人
    \item quoi只能指代无生命物
    \begin{itemize}
        \item * Voilà la personne [à quoi je pense].
    \end{itemize}
    \item lequel不能用于référence vague
    \begin{itemize}
        \item * Il s’est produit ce [auquel on pouvait s’attendre].
        \item * Il n’y a rien [sur lequel on ne se soit pas interrogé].
    \end{itemize}
    \item où指代时间或空间
    \item pourquoi指代ce
\end{enumerate}

\subsubsection{La fonction syntaxique des proformes relatives}



\begin{table}[H]
    \centering
    \small
    \begin{tabular}{|l|l|l|}
    \hline
    \rowcolor{cyan!20}
    \textbf{FONCTION DE L'INTRODUCTEUR} & \textbf{FORME} & \textbf{EXEMPLES} \\
    \hline
    sujet & \textit{lequel, lequel} + nom & \textit{un notaire, [lequel était mort depuis longtemps]} \\
    & & \textit{cette hypothèse, [laquelle hypothèse s'est révélée exacte]} \\
    \hline
    extrait & \textit{auquel, duquel, où}, \textit{prép + lequel} & \textit{l'homme [auquel je parle]} \\
    correspondant à un complément & \textit{prép} + \textit{où} & \textit{l'homme [à qui je parle]} \\
    & \textit{prép + qui}, \textit{prép + quoi} & \textit{l'endroit [où je vais]} \\
    & & \textit{ce [à quoi je pense]} \\
    \hline
    extrait & \textit{auquel cas, où, pourquoi,} & \textit{la maison [où Paul a grandi]} \\
    correspondant à un ajout & \textit{prép} + \textit{auquel} / \textit{duquel} / \textit{lequel} & \textit{l'outil [avec lequel Paul travaille]} \\
    & \textit{prép} + \textit{où,} & \textit{Viens, [sans quoi on sera en retard].} \\
    & \textit{prép} + \textit{qui, prép + quoi}& \textit{Luc est parti, ce [pourquoi tout va mal].} \\
    & \textit{adverbe} + \textit{prép} + \textit{qui} / \textit{lequel} & \textit{le rail [parallèlement auquel la route s'étire]} \\
    \hline
    \end{tabular}
    \caption{Les subordonnées relatives avec mot relatif}
\end{table}

\begin{enumerate}
    \item lequel用于relative non restrictive,总是加逗号
    \begin{enumerate}
        \item 不能作为complément oblique
        \begin{itemize}
            \item * Ils avaient emprunté 5 000 euros, lequel prix ils avaient payé la voiture.
        \end{itemize}
        \item 不能作为ajout
        \begin{itemize}
            \item * Ils se souviennent de ce jour-là, lequel on était allé à la mer.
        \end{itemize}
    \end{enumerate}
\end{enumerate}


\subsection{Les subordonnées relatives introduites par que et qui}
\subsubsection{Les introducteurs que et qui}
\begin{table}[H]
    \centering
    \small
    \begin{tabular}{|l|c|l|}
    \hline
    \rowcolor{cyan!20}
    \textbf{FONCTION DE L'INTRODUCTEUR} & \textbf{FORME} & \textbf{EXEMPLES} \\
    \hline
    marqueur correspondant au sujet & \textit{qui} & \textit{l'homme [qui est venu]} \\
    \hline
    marqueur correspondant au complément direct & \textit{que} & \textit{l'homme [que j'ai vu]} \\
    \hline
    marqueur correspondant à l'attribut du sujet & \textit{que} & \textit{le médecin [qu'il est devenu]} \\
    \hline
    marqueur correspondant au sujet d'un verbe subordonné & \textit{que} & \textit{l'homme [que je crois être un génie]} \\
    \hline
    \end{tabular}
    \caption{Les subordonnées relatives en que et qui}
\end{table}

\paragraph{Que comme subordonnant}
\begin{enumerate}
    \item invariable 
    \item 不能作为complément d’une préposition
    \begin{itemize}
        \item * Elle vit celui [avec que Jean avait vécu si longtemps].
        \item * C’est la raison [pour que Marie est partie].
    \end{itemize}
    \item 其后从句可用indicatif ou subjonctif,但不能用infinitif
    \begin{itemize}
        \item * Il pensait seul, sans un ami que rencontrer régulièrement.
        \item Il pensait seul, sans un ami avec qui discuter.
    \end{itemize}
\end{enumerate}

\paragraph{Les deux qui}
\begin{enumerate}
    \item un pronom relatif complément d’une préposition 
    \begin{enumerate}
        \item 先行词只能为有生命物
        \begin{itemize}
            \item Elle vit l’homme [avec qui elle avait vécu si longtemps].
            \item * Elle vit le tournevis [avec qui elle pourrait ouvrir la porte].
        \end{itemize}
        \item 从句可用infinitif
        \begin{itemize}
            \item Elle avait quelqu’un [avec qui parler].
        \end{itemize}
    \end{enumerate}
    \item un subordonnant qui correspond à un sujet en début de relative
    \begin{enumerate}
        \item 对先行词无要求
        \begin{itemize}
            \item Elle vit un homme [qui entrait au cinéma].
            \item Elle vit un tournevis [qui était sur la table]
        \end{itemize}
        \item 从句可用indicatif ou subjonctif,但不能用infinitif
        \begin{itemize}
            \item Je connais un secrétaire [qui sait le chinois].
            \item Je connais un secrétaire [qui sait le chinois].
            \item * Elle cherchait quelqu’un [qui parler chinois].
        \end{itemize}
    \end{enumerate}
\end{enumerate}

\paragraph{Les relatives en que avec complétive en qui}
只有部分动词允许qui引导completive : affirmer, croire, dire, penser, savoir, voir, comprendre, estimer, espérer, ignorer, imaginer, juger, prétendre, supposer, trouver, vouloir
\begin{itemize}
    \item Voici ce [qu’on a dit [qui arriverait]].
    \item Voici ce [que j’espérais [qui arriverait]].
    \item Théophile, [que l’on sait [que vous aimez ◊]], viendra demain. (第一个que引导relative第二个引导complétive)
    \item Théophraste Renaudot [que l’on sait [qui n’est pas un sot]] insère dedans ses gazettes quelquefois de pires sornettes.
    \item le livre [que je crois [qui est sur la table]]
    \item Peut-être que je me trompe, que je confonds ce [que je croyais [qui arriverait à Hélène Lagonelle]] avec ce départ obligé réclamé par sa mère.
\end{itemize}

\subsubsection{Les relatives introduites par que et qui}
Que et qui introduisent des relatives dans lesquelles il manque un syntagme nominal
\begin{enumerate}
    \item qui引导缺失主语的从句
    \item que引导缺失其他成分的从句
    \begin{enumerate}
        \item 缺失le complément direct
        \begin{itemize}
            \item Voici la maison [que Pierre a bâtie]
        \end{itemize}
        \item 缺失l’attribut du sujet
        \begin{itemize}
            \item voilà l’homme [qu’il était devenu] 
        \end{itemize}
        \item 缺失un ajout temporel,取代où的作用
        \begin{itemize}
            \item le peu [que j’ai lu de livres sur le sujet]
        \end{itemize}
        \begin{itemize}
            \item un jour [que Francis Ford Coppola rendait visite à sa fille Sophie Coppola] 
            \item Un jour que grand-mère me lisait un conte, elle s’était approchée de nous sans faire de bruit
            \item l’année que je suis arrivée là, ils avaient réservé à La Soupe 
        \end{itemize}
        \item 缺失un spécifieur de quan- tité
    \end{enumerate}
    \item que引导的从句允许l’inversion du sujet nominal
    \begin{itemize}
        \item Je voudrais un livre [qu’aiment lire ◊ tous les enfants].
    \end{itemize}
    \item relative缺失的成分不能属于
    \begin{enumerate}
        \item syntagme sujet infinitif 
        \begin{itemize}
            \item * Il avait gardé un accent [que [de supprimer ◊] est une tendance chez les immigrés].
        \end{itemize}
        \item subordonnée circonstancielle
        \begin{itemize}
            \item * Nous redoutons cette décision [que nous partirions [s’ils prenaient ◊]].
        \end{itemize}
    \end{enumerate}
    \item relative可包含plusieurs éléments manquants renvoyant à l’antécédent
    \begin{itemize}
        \item Cet enfant [qu’on a menacé ◊ [d’envoyer ◊ en pension]] continue à faire des bêtises.
    \end{itemize}
\end{enumerate}

\subsection{Les subordonnées relatives en dont}

\subsubsection{Dont comme subordonnant}
\begin{enumerate}
    \item  il ne peut pas faire partie d’un syntagme relatif en tant que complément de nom ou de préposition,区别于duquel et de qui
    \begin{itemize}
        \item * Un peu plus tard survint Jojo, [[dans la vie dont] l’amour ne tenait aucune place].
        \item Un peu plus tard survint Jojo, [[dans la vie duquel] l’amour ne tenait aucune place]
        \item * Même Patrick, [[contre le corps dont] son corps était collé], ne l’entendrait pas.
        \item Même Patrick, [[contre le corps de qui] son corps était collé], ne l’entendrait pas.
        \item * Il est aux aguets, à l’instar des animaux [[auprès dont] il vit].
        \item il est aux aguets, à l’instar des animaux [[auprès desquels] il vit]. 
    \end{itemize}
    \item dont后从句可用indicatif ou subjonctif,plus difficilement à l’infinitif
    \begin{itemize}
        \item \% Comment choisir ce [dont parler] ?
    \end{itemize}
    \item 与que类似,dont不能构成无先行词的relative
    \begin{itemize}
        \item Il parlait volontiers [de qui/* dont il aimait se souvenir].
    \end{itemize}
\end{enumerate}

\subsubsection{La relative en dont avec syntagme manquant}
dont指代syntagme prépositionnel en de.

\begin{enumerate}
    \item complément d’un verbe
    \begin{itemize}
        \item Récitez chacun un poème [dont vous vous souvenez bien].
    \end{itemize}
    \item complément d’un nom
    \begin{itemize}
        \item La maison, [dont les volets sont soigneusement clos], demeure toujours aussi silencieuse.
    \end{itemize}
    \begin{enumerate}
        \item syntagme nominal可以是sujet
        \begin{itemize}
            \item toutes ces vertus prolétariennes [dont il imaginait que [l’étalage] pourrait infléchir les sentiments de Ginette]
        \end{itemize}
        \item syntagme nominal可以是attribut ou complément direct
        \begin{itemize}
            \item toutes ces vertus [dont il évitait [l’étalage]]
        \end{itemize}
        \item 但syntagme nominal不能出现在syntagme prépositionnel中
        \begin{itemize}
            \item * toutes ces vertus [dont il pensait [à l’étalage]]
            \item * Je remercie mon réalisateur, dont je n’aurais pas réussi [sans la confiance].
        \end{itemize}
        \item complément des noms d’approximation (espèce, genre, type) et des noms affectifs (amour, crétin, drôle, merveille)不能用dont
        \begin{itemize}
            \item * Il y a des choses [dont ce genre est perdu]
            \item * Cela s’est passé la semaine [dont le journal s’est perdu].
            \item * C’est cette surface [dont un appartement est disponible].
        \end{itemize}
    \end{enumerate}
    \item complément d’un adjectif
    \begin{itemize}
        \item Disciple de Socrate, il cherche à imiter son maître, [dont il est fanatique à l’instar d’Apollodore]
    \end{itemize}
    \item plus rarement complément de préposition
    \begin{itemize}
        \item dont on est loin
    \end{itemize}
\end{enumerate}

\paragraph{Les relatives en dont comme constructions à extraction}
\begin{enumerate}
    \item dont允许l’inversion du sujet nominal
    \begin{itemize}
        \item C’est un épisode [dont se souvient Marie].
    \end{itemize}
    \item dongt允许缺失的成分作为subordonnée complétive ou complément infinitif的一部分
    \begin{itemize}
        \item Apprenez chacun un poème [dont vous êtes sûr [que vous vous souviendrez bien]].
        \item C’est un épisode [dont Marie refuse [de se souvenir]].
    \end{itemize}
\end{enumerate}


\subsubsection{La relative en dont avec proforme}
Dont peut aussi introduire une subordonnée complète, sans complément manquant,此时必须用pronom (elle, ils)或proforme (la) reprend l’antécédent de la relative
\begin{enumerate}
    \item 只能用phrase complexe,不能用relative simple
    \begin{itemize}
        \item Il a pointé une difficulté [dont il déplorait qu’elle ne soit pas encore résolue].
        \item nous vîmes monter à vive allure une colonne de véhicules de police [dont il ne faisait aucun doute qu’ils allaient droit chez nous].
        \item Il a pointé une difficulté [dont il ne savait qui la résoudrait].
        \item * Il a pointé une difficulté [dont elle n’est pas encore résolue].
    \end{itemize}
\end{enumerate}

\paragraph{La variété des proformes dans la relative en dont}
\begin{enumerate}
    \item pronoms personnels
    \begin{itemize}
        \item son problème avait été de se débarrasser d’une philosophie de Hegel [dont il est évident que le critique est de plain-pied avec elle] 
        \item Il a pointé une difficulté [dont il déplorait qu’elle ne soit pas encore résolue].
    \end{itemize}
    \item proformes faibles
    \begin{itemize}
        \item L’autorité ne renvoie pas à ses propres capacités comme la puissance qu’on jauge et [dont on se demande si on pourra lui résister ou si on devra essayer de se la concilier].
    \end{itemize}
    \item pronom démonstratif (ça, ce)
    \begin{itemize}
        \item À moins que l’U. ait de l’argent à gaspiller dans des trucs [dont il est évident que ça ne peut pas marcher].
    \end{itemize}
    \item déterminant possessif
    \begin{itemize}
        \item Leur préférence dans ce coin là va bien plutôt à Jung, [dont il est clair que sa position est exactement opposée]
    \end{itemize}
    \item le pronom \textit{tous}
    \begin{itemize}
        \item Voici la liste des rubriques, [dont il est évident que toutes ne sont pas remplies pour chacun des émigrés].
    \end{itemize}
    \item prépositions anaphoriques comme \textit{là-bas, dessus, dessous, dedans}
    \begin{itemize}
        \item Des gens qui ont su consacrer leur temps aux autres, [dont on sait que l’on peut compter dessus].
    \end{itemize}
\end{enumerate}

\paragraph{Les verbes intermédiaires}
\begin{enumerate}
    \item verbes de communication : affirmer, dire, prétendre
    \item verbes d’opinion et d’attitude intellectuelle : croire, douter, penser, savoir, avoir le sentiment, la certitude, être clair, évident, voir
    \item les modaux épistémiques : possible, il faut bien
\end{enumerate}

\subsection{Les subordonnées relatives sans verbe}
Ce sont des relatives non restrictives, ou appositives, souvent marquées par une virgule
\subsubsection{Introduite par un syntagme prépositionnel comportant un mot relatif}
\begin{enumerate}
    \item 表示cause ou origine : d’où
    \begin{itemize}
        \item Ses parents étaient sourds, [d’où sa connaissance de la langue des signes].
    \end{itemize}
    \item 表示relation partitive : parmi lesquels, au nombre desquels (其中包括), au premier/dernier rang desquels, et plus rarement, au sein desquels, au sommet desquels
    \begin{itemize}
        \item Et l’amitié de quelques jeunes talents au premier rang desquels Roger Nimier. 
        \item Celle-ci arrivait à Constantinople, chargée des présents de son dangereux fils pour Sélim, [au nombre desquels la Perle des Perles, Vassiliki, une jeune vierge grecque]
    \end{itemize}
    \item Les relatives avec adjectif épithète
    \begin{itemize}
        \item plusieurs pulls, dont deux rouges
    \end{itemize}
    \item Les relatives avec participe présent, passé ou passif
    \begin{itemize}
        \item quelques spécimens de farces et attrapes, parmi lesquels un couteau à ressort cédant à la moindre pression, et une grosse araignée noire assez effroyablement imitée.
        \item J’ai reçu des paquets, parmi lesquels deux partis depuis plusieurs mois.
        \item J’ai reçu des paquets, parmi lesquels deux envoyés par ma tante.
    \end{itemize} 
\end{enumerate}

\subsubsection{Introduite par dont}

\paragraph{Dont + syntagme nominal}
类似parmi lesquels,syntagme nominal表示une partie de l’antécédent.

\begin{itemize}
    \item Les migrants bloqués en Méditerranée accueillis dans huit pays, dont la France.
    \item Quatre-vingts figurants, dont des vignerons de la région, ont participé au tournage.
    \item Nous allons étudier plusieurs romans du XIXe siècle, dont deux de Balzac.
\end{itemize}

\paragraph{Dont + syntagme nominal + syntagme adverbial/nominal/prépositionnel}
\begin{itemize}
    \item Plusieurs étudiants doivent rendre leur travail, dont [Émilie] [très bientôt].
    \item Plusieurs étudiants doivent rendre leur travail, dont [Émilie] [cet après-midi même].
    \item \% Plusieurs enfants ont reçu des cadeaux, dont [Marie] [un livre ancien].
\end{itemize}

\section{Les fonctions des subordonnées relatives et leur interprétation}
\subsection{La subordonnée relative ajout à une catégorie nominale}

\subsubsection{La subordonnée relative dans le syntagme nominal}
\paragraph{La relative ajout à un syntagme nominal sans nom}
\begin{enumerate}
    \item quantifieurs aucun et nul后不能直接跟relative en fonction ajout,但可跟relative extraposée
    \begin{itemize}
        \item * Aucun [qui n’en aura pas fait la demande] ne recevra son diplôme.
        \item Aucun ne recevra son diplôme, [qui n’en aura pas fait la demande].
    \end{itemize}
    \item indéfinis (beaucoup, certains) 可被relative non restrictive修饰
    \begin{itemize}
        \item Le directeur a prévenu certains, [qui étaient absents].
        \item Beaucoup, qui étaient absents, n’avaient pas leur convocation.
    \end{itemize}
\end{enumerate}


\subsubsection{La subordonnée relative ajout à un pronom}
\begin{enumerate}
    \item pronom personnel
    \begin{itemize}
        \item Lui [qui avait tant de mal à se maintenir sur une selle], comment allait-il grimper là-haut ?
    \end{itemize}
    \item pronom démonstratif
    \begin{itemize}
        \item Celui [qui m’en convaincra] n’est pas encore venu.
    \end{itemize}
    \begin{enumerate}
        \item ceci et cela允许直接被relative non restrictive修饰
        \begin{itemize}
            \item Cela, [que j’appelle d’ailleurs signification], ne m’apparait comme pensée sans aucun mélange de langage que par la vertu du langage.
        \end{itemize}
        \item 但ceci et cela不允许直接被relative restrictive修饰,其与relative间必须有même, surtout, précisément等修饰词
        \begin{itemize}
            \item * Cela [que j’appelle signification] fait intervenir le langage.
            \item Cela même [que j’appelle signification] ne m’apparaît comme pensée sans aucun mélange de langage que par la vertu du langage
        \end{itemize}
    \end{enumerate}
    \item pronom indéfini
    \begin{itemize}
        \item La solitude de Marie est quelque chose [qui me tracasse].
    \end{itemize}
    \item pronom quantifieur
    \begin{enumerate}
        \item quantifieurs négatifs rien et personne
        \begin{itemize}
            \item Il ne se passera rien [que nous n’ayons prévu].
            \item La vente ne comporte rien d’exceptionnel, qui aurait pu attirer de grands collectionneurs.
        \end{itemize}
        \item quantifieurs universels tout et tous后不能跟relative, 但可跟ce引导的relative
        \begin{itemize}
            \item * Il s’est passé tout [que nous prévoyions].
            \item * Tous [que nous attendions] sont arrivés.
            \item * Tous sont arrivés, que nous attendions depuis longtemps.
            \item Tout ce [que nous prévoyions] est arrivé.
            \item Tous ceux [que nous attendions] sont arrivés.
        \end{itemize}
    \end{enumerate}
    \item 不能作为ajout aux pronoms faibles, sauf \textit{ce}
    \begin{itemize}
        \item * Il [qui demandait à entrer] était là.
        \item Ce [que j’appelle signification] fait intervenir le langage.
    \end{itemize}
    \begin{enumerate}
        \item pronoms faibles虽然不能被作为ajout的relative修饰,但可其可以作为verbes d’existence ou de perception后relative的先行词
        \begin{itemize}
            \item Il était là [qui demandait à rentrer].
            \item On les a vus [qui entraient dans le magasin].
            \item J’en ai acheté [qui sont de meilleure qualité que ceux-là].
        \end{itemize}
    \end{enumerate}
\end{enumerate}


\subsection{La subordonnée relative ajout à une catégorie non nominale}

\subsubsection{La subordonnée relative ajout à une préposition locative}
ici, là, là-bas + où
\begin{enumerate}
    \item là où可表示比喻义 : le point, l’aspect sur lequel
    \begin{itemize}
        \item Là [où il était lui aussi optimiste, Samuel], c’est lorsqu’il se figurait qu’il allait réintégrer bien vite son palace, retrouver ses meubles Louis XV, Louis XVI, Napoléon
    \end{itemize}
    \item là où可引导subordonnée circonstancielle d’opposition
    \begin{itemize}
        \item Il y avait dix personnes, là où il en aurait fallu cent. = alors qu’il en aurait fallu cent
    \end{itemize}
\end{enumerate}

\subsubsection{La subordonnée relative ajout à un adjectif}
\begin{enumerate}
    \item adjectif可作为que的先行词,从句所缺失的成分是attribut du sujet
    \begin{itemize}
        \item puritain [que je suis]
        \item Bête [que tu es], tu crois tout dès que ton frère parle.
        \item Fier de ses travaux [qu’il était], Paul dédaignait les autres invités.
        \item Fier [qu’il était de ses travaux], Paul dédaignait les autres invités.
    \end{itemize}
    \item tel que, quel que 中的tel和quel不是引导relative的先行词
    \begin{itemize}
        \item Il était tel [que sont les Français]. (comparative) 
        \item Je ne l’engagerai pas, [quel que soit son mérite]. (subordonnée concessive )
    \end{itemize}
    \item L’adjectif au superlatif avec une relative, relative包含pouvoir并可出现省略
    \begin{itemize}
        \item Je serrai le plus fort [que je pouvais].
        \item J’y allais le plus vite [que je pouvais]
        \item Le plus tôt [que le docteur puisse vous prendre]
    \end{itemize}
\end{enumerate}

\subsubsection{La relative ajout à une catégorie verbale}
relative peut être ajout à un syntagme verbal ou à une phrase
\begin{itemize}
    \item sinon je souffre de migraines et [ne puis m’empêcher d’écrire], à quoi je reconnais l’artiste.
    \item Paul les avait vus [se radicaliser], contre quoi il n’avait rien pu faire.
    \item [Il a prétendu que j’avais été absent], à quoi je n’ai rien répondu.
    \item [L’un clamait que Paul avait raison et l’autre qu’il avait tort], à quoi Paul lui-même ne répondait rien, ni à l’un, ni à l’autre.
\end{itemize}


\paragraph{Les introducteurs des relatives ajouts à une catégorie verbale}
\begin{enumerate}
    \item préposition + quoi
    \begin{itemize}
        \item David a battu Goliath, sur quoi personne n’aurait beaucoup parié.
        \item * David a battu Goliath, quoi a été une surprise.
    \end{itemize}
    \item lequel demande un antécédent nominal ; la reprise par lequel + nom est très douteuse
    \begin{itemize}
        \item ? David a battu Goliath, lequel évènement a été une surprise.
    \end{itemize}
    \item 不能单独用dont, que et qui, 要么其前加ce, 要么其前加nom qui reprend l’antécédent (évènement, issue, problème, question, situation) 
    \begin{itemize}
        \item * David a battu Goliath, qui était une surprise pour beaucoup / que peu de gens avaient prévu / dont peu de gens avaient l’espoir.
        \item David a battu Goliath, ce qui était une surprise.
        \item Je ne puis m’empêcher d’écrire, ce à quoi on reconnait l’artiste.
        \item David a battu Goliath, une issue que peu de gens avaient prévue.
        \item Il a demandé pourquoi un tel personnage faisait encore partie du groupe, une question dont on se serait bien passé.
        \item David a battu Goliath, ce que peu de gens avait prévu.
    \end{itemize}
\end{enumerate}

\subsection{Les subordonnées relatives extraposées}
\begin{enumerate}
    \item Une subordonnée relative peut être séparée de son antécédent nominal (nom, pronom, syntagme nominal)
        \begin{itemize}
            \item \underline{Le temps} n’est plus [où les avocats tiraient leur renommée des cours d’assises]. 
            \item \underline{Trois étapes} sont prévues [au terme desquelles l’économie russe va retrouver un rythme de croissance positif]
            \item \underline{Les auteurs} l’ont bien compris [qui jouent résolument la carte des plats de jadis présentés à la manière d’antan].
        \end{itemize}
    \item quel ou combien后可跟relative ajout,也可跟relative extraposée
    \begin{itemize}
        \item \underline{Quelle solution} qui te parait intéressante a-t-on proposée ?
        \item \underline{Quelle solution} a-t-on proposée [qui te parait intéressante] ?
        \item \underline{Combien d’expériences} n’ont pas donné de résultats [qui paraissaient prometteuses] ?
    \end{itemize}
    \item qui和que后不能跟relative ajout,但其可作relative extraposée的先行词
    \begin{itemize}
        \item * Qui que tu ne connaissais pas déjà as-tu rencontré ?
        \item \underline{Qui} as-tu rencontré [que tu ne connaissais pas déjà] ?
        \item Avec \underline{qui} avez-vous parlé [que vous ne connaissiez pas déjà] ?
        \item \underline{Qui} n’a pas pu venir à la réunion, [qui n’était pas prêt] ?
    \end{itemize}
\end{enumerate}

\subsubsection{La fonction syntaxique de la relative extraposée}
La relative extraposée est ajout au syntagme verbal ou à la phrase
\begin{enumerate}
    \item relative extraposée可置于subordonnée circonstancielle前或后
    \begin{itemize}
        \item \underline{Des dispositions complémentaires} se révèleront donc nécessaires, comme on peut le voir dès aujourd’hui, qui empêcheront les attitudes individualistes des états membres.
        \item \underline{Une femme} est arrivée alors que l’on n’attendait plus personne, qui portait elle aussi un grand chapeau noir.
    \end{itemize}
    \item relative extraposée可有多个先行词,因此从句动词用复数
    \begin{itemize}
        \item \underline{Une femme} est arrivée, et \underline{un homme} s’est installé au bar, qui avaient l’air de se connaitre.
        \item Paul aime \underline{les croissants}, mais Marie préfère \underline{les pâtisseries danoises}, qu’ils dégustent avec leur thé de l’après-midi.
    \end{itemize}
    \item en指代complément direct indéfini时,可作为relative的先行词
    \begin{itemize}
        \item (Des nouvelles) On en a déjà publié [qui ont eu du succès].
    \end{itemize}
    \begin{enumerate}
        \item relative可出现在complément之前
        \begin{itemize}
            \item (Des textes) On en a expliqué [qui étaient plutôt difficiles] aux étudiants de première année.
        \end{itemize}
        \item relative可coordonnée avec un syntagme adjectival introduit par \textit{de}
        \begin{itemize}
            \item (Des hypothèses) On en discute chaque semaine [[d’indispensables] [mais qui sont plutôt difficiles]].
        \end{itemize}
        \item 此结构不适用于其他proformes faibles
        \begin{itemize}
            \item (Les textes) * On les a expliqués [qui étaient plutôt difficiles].
        \end{itemize}
    \end{enumerate}
\end{enumerate}

\subsection{Les subordonnées relatives compléments d’un verbe}
Certaines subordonnées relatives peuvent être le second complément d’un verbe :
\subsubsection{La relative complément dans les constructions présentatives}
\begin{enumerate}
    \item impersonnelle: il y a, il est, il existe, avoir(informel), voilà, voici. 
    \begin{itemize}
        \item Il y a Paul [qui voudrait te parler].
        \item Il y en a [qui veulent de la glace].
        \item Marie a son fils [qui est tombé malade]. (le verbe avoir avec un sujet personnel, sans interprétation de possession)
        \item Le voilà [qui nous attend].
    \end{itemize}
    \item localisation existentielle : être, rester là, se tenir, se trouver
    \begin{itemize}
        \item Marie est là [qui attend].
    \end{itemize}
    \item tout虽然不能被relative ajout修饰,但在construction présentative中tout后可使用relative complément
    \begin{itemize}
        \item Il y a tout [qui va mal] !
        \item Voilà tout [qui arrive en même temps].
        \item * Tout [qui va mal] nous déprime.
    \end{itemize}
\end{enumerate}


\subsubsection{La relative complément de verbes de perception}
La relative est introduite par qui. 先行词可以是proforme faible
\begin{itemize}
    \item On l’a aperçu [qui partait en courant].
    \item On l’a trouvé qui se lamentait sur son sort.
    \item Victor a été aperçu [qui partait en courant].
    \item On a aperçu Victor hier [qui partait en courant].
\end{itemize}

\subsubsection{La relative complément dans les constructions clivées}

\begin{enumerate}
    \item introduite par qui ou que
    \begin{itemize}
        \item C’est Marie [qui a raison].
        \item C’est Marie [à qui je suis prêt à faire confiance].
    \end{itemize}
    \item que作为引导词时,先行词可以是不同成分
    \begin{enumerate}
        \item syntagme prépositionnel
        \begin{itemize}
            \item C’est avec Paul [que nous discuterons].
        \end{itemize}
        \item adjectival
        \begin{itemize}
            \item C’est tout jeune [qu’il a décidé de devenir cinéaste].
        \end{itemize}
        \item adverbe 
        \begin{itemize}
            \item C’est trop rapidement [qu’il a franchi ce cap].
        \end{itemize}
        \item syntagme verbal infinitif
        \begin{itemize}
            \item C’est de ne pas l’avoir su [que je regrette].
        \end{itemize}
        \item subordonnée complétive
        \begin{itemize}
            \item C’est plutôt que tu ne puisses pas venir [que je regrette].
        \end{itemize}
        \item circonstancielle
        \begin{itemize}
            \item C’est seulement si tu m’accompagnes [que je partirai].
        \end{itemize}
    \end{enumerate}
\end{enumerate}


\subsection{Les subordonnées relatives non restrictives}
lequel pronom sujet ou lequel déterminant引导的relative是non restrictives. lequel修饰的名词可与先行词不同
\begin{itemize}
    \item Certains livres, lesquels naturellement sont au programme, ont disparu de la bibliothèque.
    \item Il y a une autre voiture ; je crains que ma mère n’ait quelques-uns de ses invités habituels. Auquel cas nous repartons, répliqua Luc.
    \item Il a été admis parmi les Chevaliers du Tastevin, laquelle société fête ses nouveaux membres chaque année en grande pompe.
\end{itemize}

\section{Les subordonnées relatives au subjonctif ou à l’infinitif}
\subsection{Les subordonnées relatives au subjonctif}
\begin{enumerate}
    \item relative au subjonctif不能用于 antécédent défini后
    \item relative au subjonctif不能用于 relative non restrictive
    \item relative au subjonctif中不能加incises comme parait-il, dit-on, je crois
    \begin{itemize}
        \item Jean veut pêcher un poisson [qui a, parait-il, des nageoires violettes].
        \item \# Jean veut pêcher un poisson [qui ait, je crois, des nageoires violettes].
    \end{itemize}
\end{enumerate}
\subsubsection{La subordonnée relative au subjonctif par attraction}
主句动词用subjonctif时,relative可用indictive ou subjonctif
\begin{itemize}
    \item bien que pour celles-ci elle n’eût plus tout à fait l’âge [qui convînt]
    \item lorsqu’elle n’est pas au seul service de la justice quelle que soit d’ailleurs la conception [que l’on s’en puisse faire].
    \item bien que pour celles-ci elle n’eût plus tout à fait l’âge [qui convenait].
    \item il semblerait que ce soit le motard [qui n’ait pas respecté la priorité à droite].
    \item Pas étonnant donc que ce soit un responsable de la communication [qui ait été mandaté pour recevoir]
    \item Il semblerait que ce soit le motard [qui n’a pas respecté la priorité à droite].
\end{itemize}


\subsubsection{La subordonnée relative au subjonctif avec un antécédent indéfini}
\begin{enumerate}
    \item 当先行词为indéfini,即对说话人non identifiable时,relative用subjonctif. 若为défini则不能用subjonctif
    \begin{itemize}
        \item Je cherche une secrétaire [qui sache le chinois].
        \item Même sans être un spécialiste en matière d’allaitement, il a cherché autre chose [qui aille mieux], mais n’a pas trouvé.
        \item Je cherche la secrétaire [qui sait le chinois].
        \item \# Je cherche la secrétaire [qui sache le chinois].
    \end{itemize}
    \begin{enumerate}
        \item 该先行词也可被ajout non identifiant修饰, choix libre : n’importe lequel, quelconque, quel qu’il soit
        \begin{itemize}
            \item J’ai besoin d’un collaborateur [qui sache le chinois], n’importe lequel.
            \item Qu’elle épouse un garçon [qui ait de la terre], peu importe lequel !
            \item Avez-vous rencontré une quelconque voisine qui ait accepté de signer la pétition ?
        \end{itemize}
    \end{enumerate}
    \item 当先行词为négatif或interrogatif时,relative也可用subjonctif
    \begin{itemize}
        \item Elle ne connait aucune voisine [qui ait accepté de signer la pétition].
        \item Je ne vois personne [qui puisse te sortir de cette situation].
        \item Qu’a-t-il fait [qui soit si remarquable] ?
    \end{itemize}
\end{enumerate}

\subsubsection{La subordonnée relative au subjonctif avec un superlatif}
表示le caractère exceptionnel de l’antécédent, comparé à d’autres
\begin{enumerate}
    \item 先行词包含un adjectif au superlatif,relative可用subjonctif
    \begin{itemize}
        \item C’est le meilleur film [qu’il ait réalisé jusqu’à maintenant].
        \item Le prestige exercé, bien au-delà de nos frontières, par le titre d’ingénieur civil des mines est pour les écoles des mines le plus bel encouragement [qui puisse leur être donné].
    \end{itemize}
    \item 先行词包含un adjectif d’ordre : dernier, premier, principal, seul, unique
    \begin{itemize}
        \item Elle possède l’unique maison du village [qui soit ancienne].
        \item Fleur de lotus est le premier film [que l’on ait réalisé en Technicolor].
    \end{itemize}
    \item 先行词包含un syntagme nominal partitif de la forme : un des rares / premiers / derniers/seuls + nom
    \begin{itemize}
        \item On m’a volé un des rares livres [que j’aie lus avec plaisir].
    \end{itemize}
    \item ne... que允许verbe présentatif ou de perception后的relative使用subjonctif
    \begin{itemize}
        \item Il n’y a que mon chien [qui me comprenne].
        \item Je ne vois que Paul [qui puisse m’aider].
        \item 
    \end{itemize}
\end{enumerate}

\subsubsection{La subordonnée relative au subjonctif qui définit un type d’entité}
antécédent interprété comme un \textbf{type}, 此时défini ou indéfini都允许 relative au subjonctif
\begin{itemize}
    \item Cet entrepreneur a construit une maison [qui puisse résister aux plus fortes tempêtes].
    \item Sur toute la surface de la terre, il a semé des forêts profondes [où le pêcheur puisse retrouver la bonne chance]
    \item Le partenariat que nous menons avec la Roumanie permet aussi de montrer qu’il n’y a pas de système monolithique, mais seulement des bases conjointes qui doivent leur permettre de construire le système [qui leur soit le plus approprié]
\end{itemize}


\subsection{Les subordonnées relatives à l’infinitif}
\subsubsection{La syntaxe de la subordonnée relative à l’infinitif}
À l’écrit, relatives à l’infinitif ne prend normalement pas de virgule
\begin{itemize}
    \item \# J’ai vu Paul, à qui parler.
\end{itemize}
\paragraph{Les introducteurs de la relative à l’infinitif}
\begin{enumerate}
    \item relatif prépositionnel : où
    \begin{itemize}
        \item J’ai trouvé un endroit [où me baigner].
        \item Est-ce que tu as en tête un endroit [où proposer [d’organiser la réception ◊ ]] ?
    \end{itemize}
    \item SP contenant un pronom relatif : prép. + lequel, prép. + qui, prép. + quoi
    \begin{itemize}
        \item Elle aura quelqu’un [aux côtés de qui dormir].
        \item J’en ramenais des vieilleries : un paon en faïence [dans les plumes duquel mettre le courrier].
        \item Il souhaite rencontrer quelqu’un [avec qui espérer [discuter ◊ ]].
        \item en même temps, il imaginait des préoccupations communes, trouvait quelqu’un [à qui pouvoir [se confier ◊ sans qu’il se moquât]].
    \end{itemize}
    \item \textit{lequel} comme sujet, et \textit{que, qui, dont} comme subordonnants不能引导relative au infinitif
\end{enumerate}

\subsubsection{L’antécédent de la subordonnée relative à l’infinitif}
\paragraph{La relative à l’infinitif avec antécédent indéfini ou négatif}
\begin{itemize}
    \item le chat, le chien ont besoin d’au moins chacun une femme, un homme à soi [avec qui échanger ne serait-ce que des regards].
    \item j’entreprends un nouveau livre pour avoir un compagnon, un interlocuteur, quelqu’un [avec qui manger et dormir], [auprès duquel rêver et cauchemarder], le seul ami présentement tenable
    \item Il n’avait personne [à qui confier ces sentiments-là], pas même Foy.
    \item il n’y avait nulle part [où fuir], aucun salut à espérer où que ce soit ni de qui que ce soit
\end{itemize}

\paragraph{Les autres antécédents possibles pour la relative à l’infinitif}
先行词作为complément de être,可作为défini引导relative à l’infinitif

\begin{itemize}
    \item il est le socle [sur lequel édifier l’œuvre]
    \item Antoine,c’estautrechose,l’hommeàquiparler,avecmalgrétout,çasedevine,lesangchaud
\end{itemize}

\section{Les relatives sans antécédent}

\subsection{Les introducteurs des relatives sans antécédent}

\begin{table}[H]
    \centering 
    \begin{tabular}{|l|l|l|}
    \hline
    \rowcolor{cyan!20}
    \textbf{CATÉGORIE} & \textbf{FORME} & \textbf{EXEMPLES} \\
    \hline
    adverbe & \textit{comme} & \textit{Je ferai [comme tu veux].} \\
    \hline
    adverbe & \textit{quand} & \textit{Je partirai [quand tu partiras].} \\
    \hline
    préposition & \textit{où} & \textit{J'irai [où tu iras].} \\
    \hline
    pronom & \textit{qui, prép. + quoi, quiconque} & \textit{Je verrai [qui tu verras].} \\
    & & \textit{[Quiconque a compris] lève la main !} \\
    \hline
    \end{tabular}
    \caption{Les introducteurs des relatives sans antécédent}
\end{table}

\begin{enumerate}
    \item dont, que, auquel, duquel, lequel不能引导relatives sans antécédent
    \begin{itemize}
        \item * Nous achèterons [que l’on nous dira].
        \item * Il a parlé justement [dont nous ne devions pas parler].
        \item * Il a parlé justement [duquel nous ne devions pas parler].
        \item * Lequel donnera des renseignements à la police recevra une récompense.
    \end{itemize}
    \item que只用于固定搭配 : n’avoir que faire de, advienne que pourra, coute que coute, vaille que vaille
\end{enumerate}

\subsubsection{Les pronoms introducteurs de relatives sans antécédent}

\paragraph{Qui et quoi}
\begin{enumerate}
    \item qui指代有生命物
    \begin{enumerate}
        \item complément de préposition
        \begin{itemize}
            \item Adresse-toi donc [à qui Paul s’est lui-même adressé].
        \end{itemize}
        \item sujet
        \begin{itemize}
            \item Qui donnera des renseignements à la police recevra une récompense.
        \end{itemize}
        \item extrait,  correspondant à un complément direct ou à un attribut
        \begin{itemize}
            \item Je recevrai [qui tu me diras].
            \item Il est enfin devenu [qui il voulait].
        \end{itemize}
    \end{enumerate}
    \item quoi指代无生命物
    \begin{enumerate}
        \item 只能作为complément de préposition
        \begin{itemize}
            \item Il va travailler [sur quoi j’ai travaillé moi-même il y a des années].
            \item *Il lit [quoi l’intéresse].
        \end{itemize}
        \item 其他用法要用ce qui, ce que替代quoi
        \begin{itemize}
            \item Il lit ce [qui l’intéresse].
        \end{itemize}
    \end{enumerate}
\end{enumerate}

\paragraph{Quiconque}
\begin{enumerate}
    \item 只能引导relative sans antécédent, 指代有生命物
    \item 只能作为relative中的主语,不能作为complément
    \begin{itemize}
        \item Je recevrai [quiconque me le demandera].
        \item Quiconque donnera des renseignements à la police recevra une récompense.
        \item Je parlerai à [quiconque en fera la demande].
        \item * Je recevrai [quiconque tu me désigneras].
    \end{itemize}
\end{enumerate}


\subsubsection{Les autres proformes relatives sans antécédent}
\paragraph{Où et quand}
\begin{enumerate}
    \item Où在该结构只能表示locative, 不表示temporel
    \begin{itemize}
        \item Nous partirons [où tu veux].
    \end{itemize}
    \item quand
    \begin{itemize}
        \item Il partira [quand nous avons décidé qu’il partirait].
        \item Garde cette lettre pour [quand je serai mort].
    \end{itemize}
    \begin{enumerate}
        \item subordonnant quand peut être repris par que dans une subordonnée coordonnée, 但引导relatif sans antécédent的quand不行
        \begin{itemize}
            \item Il viendra [quand il en aura envie et qu’il en aura la permission].
            \item * Il viendra [quand il a dit pouvoir venir et qu’il en aura la permission].
        \end{itemize}
    \end{enumerate}
\end{enumerate}

\paragraph{Comme}
adverbe de similarité, introduire une comparative ajout ou complément
\begin{itemize}
    \item Il ment [comme il respire].
    \item Il s’est conduit [comme nous avions prévu qu’il se conduirait].
\end{itemize}

\subsection{La syntaxe des relatives sans antécédent}

\subsubsection{La catégorie et la fonction syntaxique des relatives sans antécédent}
\paragraph{La relative sans antécédent forme un syntagme nominal}
qui ou quiconque forme un syntagme nominal
\begin{enumerate}
    \item sujet, éventuellement sujet postverbal
    \begin{itemize}
        \item Quiconque omettait de se prosterner au passage de la yourte] était aussitôt décapité. 
        \item y a le panneau de la cellule, à la sortie du métro école Militaire, en profite [qui veut], un maximum, le boulot militant c’est ça
    \end{itemize}
    \item complément direct
    \begin{itemize}
        \item Mais je considère comme un ennemi [quiconque l’attaque de l’extérieur]
    \end{itemize}
    \item attribut du sujet
    \begin{itemize}
        \item Il n’est pas devenu [qui il est] en un jour !
    \end{itemize}
    \item complément de préposition
    \begin{itemize}
        \item l’entrée était gratuite pour [quiconque apportait un chien]
    \end{itemize}
    \item disloquée en fonction périphé- rique
    \begin{itemize}
        \item Quiconque a reçu en cadeau, pour son malheur, la flûte du preneur de rats, on l’empêchera difficilement de mener les enfants à la rivière.
    \end{itemize}
\end{enumerate}


\paragraph{La relative sans antécédent forme un syntagme prépositionnel ou adverbial}
\begin{enumerate}
    \item introducteur dans syntagme prépositionnel ou \textit{où} forme un syntagme prépositionnel
    \begin{enumerate}
        \item complément oblique
        \begin{itemize}
            \item Il vaut mieux s’adresser [[à qui] Paul s’est déjà adressé].
            \item Il a fini [[par quoi] on commence d’habitude ].
            \item Nous irons [[jusqu’où] nos pas nous porteront ].
        \end{itemize}
        \item ajout 
        \begin{itemize}
            \item Il travaille [où travaillait son père ].
        \end{itemize}
        \item extrait en début de phrase
        \begin{itemize}
            \item Où tu iras j’irai.
            \item Nous nous sommes rendus [où avait habité ◊ Chateaubriand].
            \item Nous nous rendons [où on nous a demandé [d’aller ◊]].
        \end{itemize}
    \end{enumerate}
    \item comme ou quand forme un syntagme adverbial
    \begin{enumerate}
        \item ajout
        \begin{itemize}
            \item Nous partirons [quand vous nous avez dit de partir ].
        \end{itemize}
        \item complément oblique
        \begin{itemize}
            \item Il se comporte [comme on doit se comporter ].
        \end{itemize}
        \item attribut
        \begin{itemize}
            \item Il est [comme on est à son âge ].
        \end{itemize}
    \end{enumerate}
\end{enumerate}

\subsubsection{La structure interne des relatives sans antécédent}
\paragraph{Les relatives sans antécédent à l’infinitif}
\begin{enumerate}
    \item 固定搭配avoir [de qui tenir], trouver [à qui parler]
    \begin{itemize}
        \item En Pandora O’Shaughnessy, il trouvait [à qui parler].
    \end{itemize}
    \item 固定搭配avoir où aller, se procurer où habiter, trouver où se réfugier
    \begin{itemize}
        \item Il est avantageux d’avoir [où aller] 
    \end{itemize}
    \item de quoi : dénote une quantité et indique le moyen de l’action. 
    \begin{itemize}
            \item J’étais du petit nombre qui avait [de quoi manger, de quoi boire, de quoi lire, de quoi me promener].
            \item De quoi vous remplir quelques volumes, la mythomanie étant souvent la matière brute du romancier. 
        \end{itemize}
    \begin{enumerate}
        \item 可用于non restrictives,en apposition
        \begin{itemize}
            \item un budget fictif mais nonobstant colossal : plus de cinq milliards d’anciens francs, de quoi acheter les trois tableaux les plus chers du monde ou de quoi acquérir une cinquantaine de Klee
        \end{itemize}
    \end{enumerate}
    \item 与其他relatives sans antécédent introduites par une préposition不同,该解雇可以构成SN并作为complément nominal direct
        \begin{itemize}
            \item En Pandora O’Shaughnessy, il trouvait [à qui parler]
            \item Il est avantageux d’avoir [où aller]
            \item J’étais du petit nombre qui avait [de quoi manger, de quoi boire, de quoi lire, de quoi me promener].
        \end{itemize}
\end{enumerate}

\chapter{Les subordonnées circonstancielles}
\section{Les constructions circonstancielles}
\subsection{La syntaxe des subordonnées circonstancielles}
\subsubsection{Les subordonnées circonstancielles à verbe conjugué}
\begin{enumerate}
    \item introducteur est un subordonnant comme puisque, si
    \begin{enumerate}
        \item construction corrélative : 主句有un adjectif (tel) ou un adverbe (tant)
        \begin{itemize}
            \item Son enthousiasme était d’autant plus surprenant [qu’elle ne nous avait jamais parlé de ce projet].
            \item Elle met une telle énergie dans son travail [qu’elle force l’admiration].
        \end{itemize}
    \end{enumerate}
    \item introducteur est un adverbe ou une préposition suivis de \textit{que} 
    \begin{itemize}
        \item Alors qu’il devait partir tôt, Paul est toujours là.
        \item Maintenant que Paul est là, le travail avance plus vite.
        \item Sa tenue était étonnante, [surtout que c’était l’hiver].
        \item Depuis que Paul est là, le travail avance plus vite.
    \end{itemize}
    \item introducteur est une expression (en fonction extrait) antéposée : causales ou concessives
    \begin{enumerate}
        \item adverbe (tant)
        \begin{itemize}
            \item le chevalier de Beltram forçait l’admiration de tous ces Mongols [tant il avait les reins et le cœur solides]. 
        \end{itemize}
        \item adjectif (quelle)
        \begin{itemize}
            \item Quelle que soit la difficulté, tu dois continuer.
        \end{itemize}
        \item syntagme adjectival (si mince)
        \begin{itemize}
            \item Il y a toujours un espoir, [si mince soit-il].
        \end{itemize}
        \item syntagme adverbial (aussi vite)
        \begin{itemize}
            \item Aussi vite qu’il coure, il sera en retard.
        \end{itemize}
    \end{enumerate}
\end{enumerate}

\subsubsection{Les subordonnées circonstancielles sans introducteur}
\begin{enumerate}
    \item La subordonnée circonstancielle au participe présent
    \begin{itemize}
        \item Paul étant arrivé, la réunion a pu commencer.
        \item Étant absent à la réunion, Paul doit être à l’étranger.
        \item Achète du pain, [en rentrant de l’école].
    \end{itemize}
    \item La subordonnée circonstancielle au participe passé ou passif
    \begin{enumerate}
        \item sujet nominal + participe passé ou passif
        \begin{itemize}
            \item Le délai de quinze jours passé, le patient peut considérer sa demande comme acceptée.
            \item Une fois Paul renvoyé, le travail a repris comme auparavant.
        \end{itemize}
        \item sujet nominal + syntagme prépositionnel ou adjectif
        \begin{itemize}
            \item La tête dans les nuages, Paul ne participait pas à la discussion.
            \item Paul malade, le travail a été bouleversé.
        \end{itemize}
    \end{enumerate}
\end{enumerate}

\subsection{Les autres constructions circonstancielles}

\subsubsection{Les ajouts circonstanciels à l’infinitif}
syntagmes prépositionnels circonstanciels de temps, de condition, de concession, de finalité, de cause, de conséquence.
\begin{itemize}
    \item Ferme la fenêtre [avant de partir] !
    \item Ils ne pourront pas passer [à moins de casser le cadenas].
    \item Il a bon cœur [sans être vraiment généreux].
\end{itemize}

\subsubsection{Les ajouts circonstanciels prédicatifs}
\begin{enumerate}
    \item participe passé ou passif : causale, temporelle
    \begin{itemize}
        \item Bouleversé par le film, Marc est sorti de la salle.
    \end{itemize}
    \item syntagme prépositionnel
    \begin{itemize}
        \item En colère, Paul ne participait pas à la discussion.
    \end{itemize}
    \item adjectif : causale, temporelle, plus rares condition
    \begin{itemize}
        \item Fourbues, les bêtes n’avançaient guère.
    \end{itemize}
    \item nom : causale, temporelle, plus rares condition
    \begin{itemize}
        \item Père depuis peu, Paul semble bien fatigué.
        \item Enfant, il avait été effrayé par ces récits. = quand il était enfant
    \end{itemize}
\end{enumerate}

\subsubsection{Les autres syntagmes prépositionnels circonstanciels}
une relation de temps, de cause, de concession, de condition ou de but 


\section{Les subordonnées conditionnelles}
\subsection{General}
\subsubsection{Les différentes subordonnées conditionnelles}
\begin{table}[H]
    \centering
    \small
    \begin{adjustbox}{max width = \textwidth}
        \begin{tabular}{|l|l|l|}
        \hline
        \rowcolor{cyan!20}
        \textbf{INTRODUCTEUR} & \textbf{FORMES} & \textbf{EXEMPLES} \\
        \hline
        adverbe + \textit{que} & \textit{pour autant, pour peu} & \textit{Il réussira [pour peu qu'il ait confiance en lui].} \\
        \hline
        préposition + \textit{que} & \textit{à moins, selon, suivant, \% moyennant, pourvu,} à \textit{supposer} , & \textit{Je ne viendrai pas, [à moins que tu viennes].} \\
        & \textit{en supposant, en admettant, \% supposé} & \\
        \hline
        subordonnant & \textit{si, si jamais, si tant est que, que} & \textit{[Si tu es prêt], n'hésite pas.} \\
        & & \textit{[Qu'on nous invite ou non], nous irons à la réunion.} \\
        \hline
        syntagme & \textit{au cas, dans le cas, dans l'hypothèse, dans l'éventualité,} & \textit{[Au cas où votre lettre arrive], je vous appelle.} \\
        prépositionnel + \textit{où} & \textit{dans la mesure, pour le cas} & \\
        \hline
        syntagme & \textit{à (la) condition, sous réserve} & \textit{Vous pouvez concourir [sous réserve que votre inscription soit} \\
        prépositionnel + \textit{que} & & \textit{validée].} \\
        \hline
        --- & \textit{n'était, n'eût été} + \textit{que} & \textit{Tout allait bien, [n'était qu'il pleuvait].} \\
        \hline
        --- & \textit{verbe à sujet suffixé} & \textit{[Était-il heureux], il chantait.} \\
        & & \textit{[Paul était-il heureux], il chantait.} \\
        \hline
        --- & \textit{sujet + prédicat non verbal} & \textit{[L'esprit ailleurs], il aurait eu du mal à réussir.} \\
        \hline
        \end{tabular}
    \end{adjustbox}
    \caption{Les principales subordonnées conditionnelles}
\end{table}
\paragraph{La position des subordonnées conditionnelles}
\begin{enumerate}
    \item 条件从句可置于主句前后,或主句sujet et verbe, verbe et son complément之间
    \begin{itemize}
        \item Le candidat serrait, [s’il le pouvait], toutes les mains qui se tendaient.
        \item Henri Pollak notre pote à nous, [si toutefois il n’était ni de garde, ni de piquet d’incendie, ni consigné, ni puni], serrait les mains molles de Karabinowicz 
    \end{itemize}
    \item ajouts à phrase 
    \item ajouts au verbe
    \item ajouts au adjectif
    \begin{itemize}
        \item Il y a un élève [capable de meilleures notes [s’il travaillait un peu plus]].
    \end{itemize}
    \item ajouts au nom
    \begin{itemize}
        \item Maire pour cinq ans [à moins que sa majorité le lâche], il s’attaquait à de grands travaux.
        \item À l’heure [s’il avait pris le tram], Paul était maintenant obligé de courir à toutes jambes.
    \end{itemize}
    \item ajouts au syntagme prépositionnel
\end{enumerate}


\subsubsection{Les autres ajouts conditionnels}

\begin{table}[H]
    \centering 
    \small
    \begin{adjustbox}{max width =\textwidth}
        \begin{tabular}{|l|l|l|}
        \hline
        \rowcolor{cyan!20}
        \textbf{INTRODUCTEUR} & \textbf{FORMES} & \textbf{EXEMPLES} \\
        \hline
        préposition + \textit{à} + infinitif & \textit{sauf} & \textit{Tu échoueras [sauf à travailler davantage].} \\
        \hline
        préposition ou syntagme & \textit{à (la) condition, à moins,} & \textit{Tu seras en retard [à moins de partir tout de suite].} \\
        prépositionnel + \textit{de} + infinitif & \textit{\% moyennant, sous réserve} & \textit{Vous entrez, [sous réserve de verser quelque chose].} \\
        \hline
        préposition + syntagme nominal & \textit{moyennant, selon, etc.} & \textit{Il va réussir, [moyennant quelque effort].} \\
        \hline
        préposition ou syntagme & \textit{à (la) condition, à moins, sous réserve,} & \textit{[En cas de doute], on vous appellera.} \\
        prépositionnel + \textit{de} + syntagme & \textit{dans l'hypothèse, dans le cas, en cas,} & \\
        nominal & \textit{dans la mesure, dans l'éventualité, etc.} & \\
        \hline
        --- & \textit{syntagme prédicatif initial} & \textit{[Moins fatigué], il aurait mieux réussi.} \\
        \hline
        \end{tabular}
    \end{adjustbox}
    \caption{Les autres ajouts conditionnels}
\end{table}

\begin{enumerate}
    \item en + participe present, 置于句首 : participe passé ou passif, adjectifs, nom, syntagmes prépositionnels
    
    主句动词必须用conditionnel形式,否则只能解释为causale ou temporelle
    \begin{itemize}
        \item Parti plus tôt, Paul serait arrivé à l’heure.
        \item Président, il aurait modernisé le pays.
        \item Réparée à temps, l’église aurait pu être sauvée.
        \item Plus en forme, Paul courrait plus vite.
        \item En travaillant davantage, il aurait mieux réussi.
    \end{itemize}
\end{enumerate}

\subsubsection{Les autres constructions conditionnelles}
\paragraph{Les coordinations à sens conditionnel}
et可省略,ou必须存在
\begin{enumerate}
    \item 第一句用impératif
    \begin{itemize}
        \item Avance et je te casse la figure. = si tu avances
        \item Avance, je te casse la figure !
    \end{itemize}
    \item 第一句用que + subjonctif
    \begin{itemize}
        \item Qu’il vienne et il aura affaire à moi ! = s’il vient
        \item Que le dictateur s’écroulât et ils s’effondraient avec lui, le parti Mouquawat se scindait en deux ou trois factions, et Salah Rouzi revenait immédiatement de Libye pour épurer le pays et mettre en place ses amis.
        \item Qu’il vienne, il aura affaire à moi !
    \end{itemize}
    \item ou 
    \begin{itemize}
        \item Ou tu vas trouver les gendarmes, ou bien on monte à Malataverne tous les deux. = si tu ne vas pas trouver les gendarmes, on monte à Malataverne et vice versa
        \item Arrête ou tu vas avoir affaire à moi ! = si tu n’arrêtes pas, tu vas avoir affaire à moi
    \end{itemize}
\end{enumerate}

\subsection{La syntaxe des subordonnées conditionnelles}
\subsubsection{Les introducteurs de la subordonnée conditionnelle}
\paragraph{un subordonnant}
\begin{enumerate}
    \item si, si jamais, si tant est que : subjonctif ou indicative 
    \item comme si, sauf si, excepté si : 与si jamais, si tant est que不同,它们间可被副词分开
    \begin{itemize}
        \item sauf, bien sûr, si tu ne peux pas te déplacer
    \end{itemize}
    \item que + subjonctif 其中必须含有disjonction, parfois réduite à pas ou à non
    \begin{itemize}
        \item Que Marie ait raison ou qu’elle ait tort, il faut écouter ses arguments.
        \item qu’il parût avoir été frappé par un brigand ou par un fanatique, le peuple penserait que Voltaire avait eu ce qu’il méritait.
        \item Que ça vous plaise ou pas, les Indiens y vous le diront comme moi.
        \item Il faut écouter les arguments de Marie, qu’elle ait raison ou qu’elle ait tort
    \end{itemize}
\end{enumerate}

\paragraph{un adverbe ou une préposition suivi de que}
\begin{enumerate}
    \item adverbe : pour autant, pour peu + que subjonctif
    \begin{itemize}
        \item Nous irons à la plage, [pour peu qu’il fasse beau].
    \end{itemize}
    \item préposition : pourvu, selon, suivant, à supposer, en admettant, en supposant + que
    \begin{itemize}
        \item de manière à pouvoir bénéficier d’un accès permanent à ceux-ci, où que l’on se trouve, [pourvu évidemment que l’on soit connecté au web].
        \item Les systèmes peuvent, en effet, s’opposer, [pourvu que les conceptualisations choisies soient assez étroites et que les idées qui ordonnent ces conceptualisations soient arrêtées]
        \item En supposant que toute cette magie réussisse, il n’y aurait jamais aucun tri 
    \end{itemize}
\end{enumerate}







\paragraph{un syntagme prépositionnel suivi de où ou que}
\begin{enumerate}
    \item à condition, au cas + complétive en \textit{que} ou relative en \textit{où}
    \begin{itemize}
        \item Je veux que Danglard puisse me repérer [au cas où mon portable me lâche].
        \item Je viendrai [à condition que tu me préviennes à temps].
    \end{itemize}
\end{enumerate}

\subsubsection{La subordonnée conditionnelle avec verbe à sujet suffixé}
\begin{enumerate}
    \item n’était, n’eût été, si ce n’est/n’ét ait + sujet inversé en \textit{que} ou sujet nominal inversé
    \begin{itemize}
        \item Pour le reste, une bonne moitié des poèmes d’Alcools me laissent indifférent, n’était que leur érudition biscornue, [...], appâte l’imagination presque à chaque page, n’était aussi qu’Apollinaire est un admirable inventeur, ou dénicheur, de noms propres
        \item Une bonne moitié des poèmes d’Alcools me laissent indifférents, [si ce n’est que leur érudition appâte l’imagination presque à chaque page].
        \item Je pourrais continuer longtemps à recopier la liste dressée par Paulhan, [n’eût été la possibilité de fatiguer outre mesure le lecteur]
        \item N’eût été la possibilité de fatiguer outre mesure le lecteur, je pourrais continuer longtemps à recopier la liste dressée par Paulhan.
    \end{itemize}
    \item sujet suffixé, verbe au conditionnel, à l’imparfait ou au subjonctif plus-que-parfait
    \begin{itemize}
        \item Paul était-il heureux, il chantait.
        \item Marie eût-elle agi différemment, le don n’aurait pas été si merveilleux.
        \item Arriverait-il maintenant, je partirais aussitôt.
        \item Entrouvrait-il un placard, c’était un fatras nauséabond. Risquait-il un œil dans le tiroir du bureau, il tombait sur des coquillages et des peaux de banane semi-fossilisées
        \item elle se louait d’avoir appris et de s’être gardée, puisque eût-elle agi différemment le don n’aurait pas été si merveilleux
    \end{itemize}
    \begin{enumerate}
        \item 区别于concessives à sujet suffixé,conditionnelle不能置于除了句首的其他位置
        \begin{itemize}
            \item Je ne l’écouterai pas, [fût-il ministre].
            \item Fût-il ministre, je ne l’écouterai pas.
            \item * Il chantait, [était-il heureux].
        \end{itemize}
        \item 第二句前可加que
        \begin{itemize}
            \item Était-il heureux qu’il chantait.
        \end{itemize}
    \end{enumerate}
\end{enumerate}


\subsubsection{Le mode et le temps du verbe dans la subordonnée conditionnelle}


\begin{table}[H]
    \centering 
    \small
    \begin{adjustbox}{max width = \textwidth}
        \begin{tabular}{|l|l|l|}
        \hline
        \rowcolor{cyan!20}
        \textbf{INTRODUCTEUR} & \textbf{MODE DU VERBE} & \textbf{EXEMPLE} \\
        \hline
        \textit{au cas (où), dans le cas (où), dans l'éventualité (où),} & indicatif & \textit{Je viendrai [dans l'hypothèse où il fait / ferait beau].} \\
        \textit{dans l'hypothèse (où), dans la mesure (où),} & & \textit{Paul sortira [suivant qu'il fait beau ou non].} \\
        \textit{pour le cas (où), selon (que), suivant (que)} & & \textit{Paul viendra [au cas où il ferait fait beau].} \\
        \hline
        \textit{n'eût été (que), si, si jamais} & indicatif (sauf conditionnel) & \textit{[S'il fait beau], je viendrai.} \\
        \hline
        \textit{en admettant (que), en supposant (que),} & indicatif, subjonctif & \textit{Je viendrai [pour autant qu'il fait / ferait / fasse beau].} \\
        \textit{\% moyennant (que), pour autant (que), sous réserve (que)} & & \textit{Je viendrai, [sous réserve qu'il fera / ferait / fasse beau].} \\
        \hline
        \textit{à supposer (que)} & indicatif (sauf conditionnel), & \textit{[À supposer qu'il fait / fasse beau], je viendrai.} \\
        & subjonctif & \\
        \hline
        \textit{à condition (que), à moins (que), pour peu (que),} & subjonctif & \textit{Je ne viendrai pas, [à moins qu'il fasse beau].} \\
        \textit{pourvu (que), que... ou que, que... ou pas, que... ou non,} & & \textit{Je viendrai, [qu'il fasse beau ou non].} \\
        \textit{si tant est que} & & \\
        \hline
        \end{tabular}
    \end{adjustbox}
    \caption{L’indicatif et le subjonctif dans la subordonnée conditionnelle}
\end{table}

\begin{enumerate}
    \item si et si jamais后不能用le conditionnel et le futur.
    \begin{itemize}
        \item * S’il pleuvra, je ne viendrai pas.
    \end{itemize}
    \begin{enumerate}
        \item conditionnel contrastive后可用le futur
        \begin{itemize}
            \item Si le réchauffement ouvrira de nouvelles possibilités de culture, notamment dans les pays de hautes latitudes, la raréfaction de l’eau et l’appauvrissement des sols poseront d’immenses problèmes dans les basses latitudes. 
        \end{itemize}
        \item même si concessif 后可用le conditionnel et le futur
        \item subordonnée interrogative en si可用le futur
    \end{enumerate}
\end{enumerate}




\subsection{L’interprétation des subordonnées conditionnelles}


\subsubsection{Les conditionnelles douteuses ou irréelles}
indicatif imparfait, conditionnel ou subjonctif
\begin{enumerate}
    \item si et au cas où可表达 possible, probable, improbable全部三种可能,但pour autant, si jamais et si tant est que只表达doute
    \begin{itemize}
        \item Je viendrai si l’équipe passe le premier tour, ce qui est possible.
        \item Je viendrai au cas – fort improbable – où l’équipe passerait le premier tour.
        \item Je viendrai si jamais l’équipe passe le premier tour, ce qui serait une surprise.
        \item \# Je viendrai si jamais l’équipe passe le premier tour, ce qui est probable.
    \end{itemize}
\end{enumerate}
\begin{table}[H]
    \centering 
    \small
    \begin{tabular}{|l|l|l|}
    \hline
    \rowcolor{cyan!20}
    \textbf{INTRODUCTEUR} & \textbf{TEMPS DE LA CONDITIONNELLE} & \textbf{EXEMPLES} \\
    \hline
    \textit{si} & situation présente : & \textit{S'il était là maintenant...} \\
    & imparfait, plus-que-parfait & \textit{S'il avait été là maintenant...} \\
    \cline{2-3}
    & situation passée ou future : & \textit{S'il avait été là hier...} \\
    & plus-que-parfait & \textit{S'il avait été là demain...} \\
    \hline
    autres introducteurs & conditionnel ou subjonctif & \textit{Pour autant qu'il aurait été là hier...} \\
    & à un temps composé & \textit{À condition qu'il ait été là hier...} \\
    & & \textit{Dans la mesure où il aurait été là maintenant...} \\
    & & \textit{En admettant qu'il ait été là maintenant...} \\
    & & \textit{Au cas où il aurait été là demain...} \\
    & & \textit{Pour peu qu'il ait été là demain...} \\
    \hline
    \end{tabular}
    \caption{Les temps verbaux dans la conditionnelle irréelle}
\end{table}

\begin{enumerate}
    \item l’imparfait et le plus-que-parfait在语义上无区别,两者可coordonnés
    \begin{itemize}
        \item Si Paul était là et s’il avait eu un peu de temps, il aurait réglé le problème.
        \item Si Paul avait été là et s’il avait un peu de temps, il aurait réglé le problème.
    \end{itemize}
    \item 主句动词常用conditionnel. 主句用imparfait,当从句用plus-que-parfait或imparfait(非正式)时
    \begin{itemize}
        \item Si les horaires n’avaient pas changé hier, il était possible de faire le trajet en train.
        \item \% Si les horaires ne changeaient pas aujourd’hui, il était possible de faire le trajet en train.
    \end{itemize}
\end{enumerate}

\subsubsection{Les conditionnelles de condition suffisante ou nécessaire}

\subsubsection{Les différentes interprétations des subordonnées en si}
\paragraph{L’emploi contrastif de la subordonnée en si}
si不表示hypothétique,而是建立在对立之上(grand/petit). 从句可用futur ou conditionnel,主句的对立可用pour autant, en revanche; un pronom contrastif (eux); aussi加强语气
\begin{itemize}
    \item Si Paul est grand, Marie est petite.
    \item Siladirectionfinancièredel’A.sefrottelesmains],lessupporters,eux,fontgrisemine.
    \item Si le réchauffement ouvrira de nouvelles possibilités de culture, notamment dans les pays de hautes latitudes, la raréfaction de l’eau et l’appauvrissement des sols poseront d’immenses problèmes dans les basses latitudes.
    \begin{itemize}
        \item 表示s’il est vrai que... il est encore plus vrai que...
    \end{itemize}
    \item Si Paul est grand, il n’est pour autant pas doué en sport.
\end{itemize}


\section{Les subordonnées circonstancielles de cause}

\subsection{Les subordonnées circonstancielles de cause}

\begin{table}[H]
    \centering
    \small
    
    \begin{adjustbox}{max width =\textwidth}
        \begin{tabular}{|l|l|l|}
        \hline
        \rowcolor{cyan!20}
        \textbf{INTRODUCTEUR} & \textbf{FORMES} & \textbf{EXEMPLES} \\
        \hline
        adverbe & \textit{tant, tellement} & \textit{Il se fatigue [tellement il court].} \\
        \hline
        adverbe + \textit{que} & \textit{d'autant, dès lors, maintenant, non, surtout} & \textit{[Maintenant que l'année a commencé], on ne se voit plus.} \\
        \hline
        locution & \textit{du moment, à partir du moment} & \textit{[À partir du moment où c'est obligatoire], tu dois le faire.} \\
        prépositionnelle + \textit{où} & & \\
        \hline
        locution & \textit{au prétexte, sous (le) prétexte, du fait, du moment} & \textit{Je n'y ai pas droit, [sous prétexte que je suis trop jeune].} \\
        prépositionnelle + \textit{que} & & \\
        \hline
        préposition + \textit{que} & \textit{attendu, compte tenu, étant donné, vu} & \textit{[Étant donné qu'il pleut], Paul ne sort pas.} \\
        \hline
        subordonnant & \textit{comme, parce que, puisque, (soit) que} & \textit{Je pars [parce que la réunion est finie].} \\
        \hline
        --- & \textit{sujet + participe présent} & \textit{[Paul étant en retard], la réunion n'a pu commencer.} \\
        \hline
        --- & \textit{sujet + syntagme adjectival, nominal} & \textit{[L'esprit ailleurs], Paul a chuté sur un gros caillou.} \\
        & \textit{ou prépositionnel} & \\
        \hline
        \multicolumn{3}{|c|}{\textbf{CONSTRUCTION CORRÉLATIVE}} \\
        \hline
        \textit{d'autant} & \textit{d'autant mieux, d'autant moins, d'autant plus} & \textit{Je suis d'autant plus d'accord [que j'y ai déjà réfléchi].} \\
        \textit{+ comparatif... que} & & \\
        \hline
        \end{tabular}
    \end{adjustbox}
    \caption{Les principales subordonnées de cause}
\end{table}

\paragraph{La fonction syntaxique des subordonnées de cause}
\begin{enumerate}
    \item ajout à la phrase
    \begin{itemize}
        \item Une autre fois, elle lui tricotait d’autorité un châle, [sous prétexte que Marie nous faisait honte avec ses guenilles sur les épaules].
    \end{itemize}
    \item ajout au verbe
    \begin{itemize}
        \item Elle lui tricotait, [sous prétexte qu’elle avait honte], un nouveau châle.
    \end{itemize}
    \item ajout au nom 
    \begin{itemize}
        \item Orphelin [puisqu’il a perdu son père], Paul doit travailler.
    \end{itemize}
    \item ajout au adjectif 
    \begin{itemize}
        \item Malade [parce qu’il avait trop bu], Aymeric est resté au lit.
    \end{itemize}
\end{enumerate}

\paragraph{Les subordonnées complétives de cause}
complétive que subjonctif可表示casual
\begin{itemize}
    \item Paul est triste [que tout soit fini].
    \item Paul se réjouit [que tout soit fini].
\end{itemize}

\paragraph{La position des subordonnées de cause}
\begin{enumerate}
    \item mobile
    \item tant et tellement 引导的从句只能出现在主句后
    \begin{itemize}
        \item Paul avait mérité de réussir, tant il s’était donné de mal.
        \item * Tant il s’était donné de mal, Paul avait mérité de réussir.
    \end{itemize}
    \item d’autant + comparatif用于主句,que引导的从句只能出现在主句后
    \begin{itemize}
        \item Paul a d’autant plus de mérite [qu’il s’est donné du mal].
        \item * Qu’il s’est donné du mal, Paul a d’autant plus de mérite.
    \end{itemize}
\end{enumerate}

\subsection{Les autres constructions de cause}

\begin{table}[H]
    \centering 

    \begin{adjustbox}{max width =\textwidth}
        \begin{tabular}{|l|l|l|}
        \hline
        \rowcolor{cyan!20}
        \textbf{INTRODUCTEUR} & \textbf{FORMES} & \textbf{EXEMPLES} \\
        \hline
        préposition & \textit{avec, de par, étant donné, par, pour, vu} & \textit{Nous sommes coincés [avec cette tempête].} \\
        \hline
        préposition + \textit{à} & \textit{grâce, suite} & \textit{Nous avons réussi [grâce à Paul].} \\
        \hline
        locution prépositionnelle & \textit{à cause, à force, au prétexte, du fait, en raison, faute,} & \textit{Le combat s'arrêta [faute de munitions].} \\
        ou préposition + \textit{de} & \textit{sous (le) prétexte} & \textit{[En raison de leur âge], ils ne pouvaient pas entrer.} \\
        \hline
        préposition + infinitif & \textit{à, à force (de), de, faute (de), pour, sous prétexte de, etc.} & \textit{Mes yeux fatiguent [à force de fixer l'écran].} \\
        & & \textit{Il a été puni [pour avoir chanté faux].} \\
        & & \textit{Il a été puni [faute d'avoir prévenu à temps].} \\
        \hline
        --- & \textit{participe passé ou présent} & \textit{[Parti tôt], Paul est arrivé à l'heure.} \\
        \hline
        --- & \textit{syntagme prédicatif} & \textit{[Maligne comme elle est], elle réussira.} \\
        & & \textit{[Professeur depuis peu], il manque d'expérience.} \\
        \hline
        \end{tabular}
    \end{adjustbox}
    \caption{Les autres ajouts de cause}
\end{table}

\subsubsection{Les syntagmes prépositionnels de cause}
\paragraph{Les syntagmes prépositionnels de cause à l’infinitif ou au participe présent}
de + infinitif, en + 
\begin{enumerate}
    \item prépositions : à, de, faute, pour
    \begin{itemize}
        \item Paul se fatigue [à force de courir].
        \item À les voir ainsi désolés, si le métro était ouvert, on s’y jetterait, pour fuir ce spectacle mortel.
        \item C’est en 1950 qu’il entre à l’usine Renault : de 1950 à 1955, il part pour Paris, laissant seuls son épouse et ses enfants à Céaucé, [faute de trouver un logement sur son lieu de travail]
        \item B. L. R. [...] a semé une jolie pagaille [en indiquant, mercredi 12 février, que l’idée de bloquer l’avancement des agents publics était à l’étude].
        \item Paul était fatigué [d’avoir couru].
    \end{itemize}
    \item locutions prépositionnelles : à force, sous prétexte
\end{enumerate}

\paragraph{L’ajout de cause et l’ajout de but}
\begin{enumerate}
    \item pour + infinitif passé 表示cause 
    \begin{itemize}
        \item Socrate lui-même est condamné à mort [pour avoir perverti la jeunesse]. 
    \end{itemize}
    \item pour + que 表示but
    \begin{itemize}
        \item Paul a déménagé [pour qu’on ne puisse pas le retrouver].
    \end{itemize}
\end{enumerate}

\paragraph{Autres syntagmes prépositionnels de cause}
\begin{enumerate}
    \item prépositions : à force (de), à cause (de), attendu, compte tenu (de), en raison (de), étant donné, faute (de), grâce (à), vu
    \begin{itemize}
        \item Vu l’épidémie de grippe, le match est reporté.
        \item Étant donné leur différence d’âge, les deux enfants ne s’intéressaient pas aux mêmes jeux.
    \end{itemize}
    \item locutions prépositionnelles : sous (le) prétexte (de)
    \item par ou de + un nom de sentiment ou d’état intérieur
    \begin{itemize}
        \item les mauvaises langues dirent que c’était le père, [par cupidité], qui avait fait exploser la grenade
        \item Paul est rouge [de honte].
    \end{itemize}
\end{enumerate}

\paragraph{Les syntagmes prépositionnels en avec}
avec也可表示cause 
\begin{enumerate}
    \item avec + un complément nominal
    \begin{itemize}
        \item Avec cette grève des transtrucs en commachin, on peut plus rien faire de ce qu’on veut. 
        \item Avec un tel emploi du temps, le bébé ne peut pas découvrir son articulation au monde de la société
    \end{itemize}
    \item avec + un syntagme nominal et un prédicat
    \begin{itemize}
        \item Avec Alba d’un côté et Hippolyte de l’autre, J. S. n’était pas – et de loin – l’élément le plus brillant de la khâgne de Henri-IV 
    \end{itemize}
\end{enumerate}

\subsubsection{Les ajouts prédicatifs de cause}
\paragraph{Les ajouts de cause au participe passé ou passif}
\begin{itemize}
    \item Parti de bonne heure, Paul est arrivé à l’heure.
    \item Convaincu que l’illustre physicien devait avoir raison, Planck s’orienta dans cette nouvelle voie
\end{itemize}
一旦participe前有主语,则其不再表示cause而表示temporelle
\begin{itemize}
    \item à peine mon article paru, je me suis précipité rue Chaptal pour en obtenir un autre.
    \item Montecullo et le Fort Umberto une fois enlevés par les nôtres, le 7 avril, la légion entra en trombe dans Massaouah, pêle-mêle avec une foule d’Italiens en déroute
\end{itemize}

\paragraph{Les syntagmes adjectivaux de cause}
\begin{enumerate}
    \item un syntagme adjectival initial 
    \begin{itemize}
        \item Malade depuis trois mois, Paul nous manque beaucoup.
    \end{itemize}
    \item adjectif + une subordonnée relative en \textit{que} avec le verbe être à l’indicatif
    \begin{itemize}
        \item Le Christ vous parle, sur TF1, à une heure de grande écoute, et vous ne comprenez pas son message, [mécréants que vous êtes] ?
        \item c’est toujours la même monotonie pesante, qui nous épuise, [sûrs que nous sommes de n’y pouvoir échapper]. 
    \end{itemize}
    \item adjectif + une subordonnée comparative en \textit{comme} avec être ou un autre verbe attributif (se montrer, sembler, voir)
    \begin{itemize}
        \item Méchante comme elle est, elle ne leur laissera pas un sac de ducats.
        \item Malchanceux comme tu es, César, c’est comme si c’était fait.
        \item Attentif comme on le croit aux besoins de ses collaborateurs, il parait le candidat idéal.
        \item Elle ne risque pas de laisser passer l’heure, sa vieille, insomniaque et nerveuse comme elle est.
        \item Taper des enveloppes et tenir des fichiers à longueur de journée : intelligente comme elle est, c’est un crime.
    \end{itemize}
\end{enumerate}

\paragraph{Les autres ajouts prédicatifs de cause}
\begin{enumerate}
    \item nominaux
    \begin{itemize}
        \item Professeur depuis peu, Paul manque d’expérience. = parce qu’il est professeur depuis peu
        \item Nous voilà condamnés, [orphelins que nous sommes], à poursuivre un fantôme en reconnaissance de paternité. 
    \end{itemize}
    \item prépositionnels
    \begin{itemize}
        \item Au courant de sa curieuse situation d’engagé volontaire contraint, ces braves types considéraient Simon comme un des leurs et lui simplifiaient de leur mieux sa tâche.
        \item En retard comme tu es, tu vas rater ton train.
    \end{itemize}
\end{enumerate}


\subsubsection{Les coordinations et juxtapositions à interprétation causale}
\paragraph{Les coordinations à sens causal}
coordination \textit{car},它只能引导coordonnées而不能引导subordonnée causale,因为它引导的从句不能置于主句前
\begin{itemize}
    \item Paul a réussi car il a beaucoup travaillé.
    \item * Car il a beaucoup travaillé, Paul a réussi.
\end{itemize}

\paragraph{Les phrases juxtaposées à sens causal}
connecteurs aussi ou en effet 
\begin{itemize}
    \item Paul a réussi. En effet, il avait beaucoup travaillé.
\end{itemize}

\subsection{La syntaxe des subordonnées de cause}
\subsubsection{La subordonnée de cause introduite par un subordonnant}
\textit{parce que, puisque et comme}
\begin{itemize}
    \item Elle a réussi [parce qu’elle a travaillé].
    \item Vous pouvez partir [puisque la réunion est finie].
    \item Comme les guides refusaient de les accompagner et que le temps ne se prêtait pas à lever la moindre caravane, ils se firent indiquer les points d’eau et les points de non-retour 
\end{itemize}

\begin{enumerate}
    \item comme引导comparative时是adverbe,因此不能reprise en que en cas de coordination.
    \begin{itemize}
        \item Il vit comme il veut et comme il a décidé/*et qu’il a décidé.
    \end{itemize}
    \item parce que et puisque可引导une subordonnée sans verbe, ou elliptique
    \begin{itemize}
        \item Celle-ci était tantôt fade [parce que trop fraiche], tantôt avariée pour ne pas dire pourrie
        \item grisé par leur odeur encore plus enivrante [puisque nocturne], il s’achemina vers la grille
    \end{itemize}
\end{enumerate}

\paragraph{Les subordonnées de cause dans une construction corrélative}
主句用d’autant mieux, d’autant moins, d’autant plus修饰adjectif / verbe,从句用que引导. 主句有时也可无verbe
\begin{itemize}
    \item Il conçut pour ce livre un enthousiasme d’autant plus surprenant [que c’était certainement un des tout premiers qu’il lisait].
    \item Il a d’autant plus travaillé qu’il avait un espoir de promotion.
    \item Je la comprends d’autant mieux [que depuis quelque temps mes mobiles personnels font place à des motifs plus généraux]
    \item Préoccupation légitime, d’autant plus [que la politique d’externalisation de certaines activités a déjà com- mencé depuis fort longtemps]. 
\end{itemize}

\subsubsection{La subordonnée de cause introduite par un adverbe ou une préposition}
\paragraph{un adverbe suivi de que}
\begin{enumerate}
    \item \textit{d’autant, maintenant, non, surtout} + que ; d’autant + comparatif (d’autant moins, d’autant plus) + que
    \begin{itemize}
        \item Zingaro traînait la patte comme moi, [maintenant que le chaud de l’alerte était passé]. 
        \item Les boutiques n’y sont pas très reluisantes et les cafés eux-mêmes sont peu nombreux, [non que ces gens-là soient particulièrement sobres] mais plutôt parce qu’ils préfèrent aller boire ailleurs.
        \item La manière dont elle s’éventait avec cette feuille m’intriguait, [surtout que c’était l’hiver] 
        \item Je devais encore avoir mon accent du Loiret, [d’autant que je venais d’y passer deux mois de vacances l’été précédent].
        \item Il a d’autant moins protesté [qu’il n’avait pas la conscience tranquille].
    \end{itemize}
    \begin{enumerate}
        \item non que后只能用subjonctif,不能用indicatif
        \begin{itemize}
            \item ils se déclarèrent volontaires pour les aider, [non que la prospérité du tailleur les intéressât], mais parce qu’ils trouvaient une occasion de flânerie.
            \item \% toutefois je restais le plus hésitant sur cette démarche, [non que je n’avais pas une confiance absolue en Bill]
        \end{itemize}
    \end{enumerate}
    \item \textit{dès lors} + que ; 其与que间可被其他修饰成分分开
    \begin{itemize}
        \item Mais pourquoi le jeter dans de périlleuses convulsions, [dès lors que, de toute manière, il va recouvrer la santé] ?
        \item Dès lors en effet que l’éducation ne vaut que parce qu’elle permet de produire, elle ne vaut plus rien au-delà de ce qu’exige la productivité.
    \end{itemize}
    \item \textit{soit que / autre syntagme... soit que / autre syntagme} + subjonctif. soit与que可被分开
    \begin{itemize}
        \item Soit qu’il ait été malade, soit qu’il ait eu un empêchement, Paul n’est pas venu.
        \item Mais il s’agit bien encore ici de pointes et non de couteaux à dos, [soit que la partie retouchée reste encore trop tranchante pour être ainsi utilisée, soit précisément que le bord coupant ait été détruit] 
        \item Mais Jean-Jacques n’y alla pas, [soit par son indolence habituelle, soit qu’il ne tînt pas à connaître trop précisément comment et où s’imprimait très illégalement son livre]
    \end{itemize}
\end{enumerate}

\paragraph{un adverbe antéposé}
\begin{enumerate}
    \item \textit{tant, tellement} en fonction extrait et correspondent à un ajout de haut degré dans la subordonnée
    \begin{itemize}
        \item Tout le monde apprécie Paul, [tellement il est gentil].
        \item L’arbre généalogique lui paraissait aussi immense que foisonnant [tant il y avait eu de croisements entre les différentes maisons de noblesse]. 
    \end{itemize}
    \item si + adjectif avec sujet inversé
    \begin{itemize}
        \item On se sent tout de suite à son aise, [si chaleureux est leur accueil].
    \end{itemize}
    \begin{enumerate}
        \item 该结构的subordonnée concessive形式 : 须用subjonctif且可加que
        \begin{itemize}
            \item Et je tirai parti de cette complicité de classe, si scandaleuse fût-elle au regard de nos principes, sans le moindre scrupule.
            \item Et je tirai parti de cette complicité de classe, si scandaleuse qu’elle fût au regard de nos principes.
        \end{itemize}
    \end{enumerate}
\end{enumerate}

\paragraph{une préposition ou une locution prépositionnelle, suivie de que }
\begin{enumerate}
    \item prépositions \textit{attendu, compte tenu, étant donné, vu}
    \begin{itemize}
        \item Étant donné que le revenu du pays dépend du pétrole, la situation s’est détériorée.
    \end{itemize}
    \item locutions prépositionnelles \textit{au prétexte, du fait, du moment, sous prétexte}
    \begin{itemize}
        \item Je n’ai pas droit aux allocations, [sous prétexte que je suis encore gérante].
    \end{itemize}
\end{enumerate}

\subsubsection{La subordonnée de cause sans introducteur}
\paragraph{au participe présent avec sujet}
无主语则不能算subordonnée,只能看作是 un syntagme verbal
\begin{itemize}
    \item Le Proche-Orient relevant d’une logique très particulière, les récriminations des uns et des autres sont de bon augure.
    \item Il s’arrêta à deux reprises pour visiter des maisons isolées, au bord de la route. Comprenant tout de suite qu’on ne lui achèterait rien, il en ressortit presque aussitôt.
\end{itemize}

\paragraph{au participe passé avec un nom de partie du corps}
\begin{itemize}
    \item L’esprit ensommeillé par le ronflement de la voiture, il ne pensait plus rien
    \item les bras nus, elle paraissait beaucoup plus jeune qu’à Paris
    \item Farfouillant un peu plus, [l’esprit occupé], il n’entendit pas des pas revenir. 
\end{itemize}


\subsubsection{Les subordonnées de cause modifiées par des adverbes}
\begin{enumerate}
    \item parce que, sous (le) prétexte que
    \begin{itemize}
        \item par adverbe modal : sans doute, peut-être, probablement
        \begin{itemize}
            \item Je pense maintenant qu’elle avait dû se mettre à faire semblant d’avaler ses calmants, peut-être pour nous embêter, [peut-être parce que j’avais parlé et qu’elle essayait de vouloir m’aider]
        \end{itemize}
        \item par un adverbe qui la situe dans une série de causes : aussi
        \begin{itemize}
            \item C’est-à-dire qu’à aucun moment je n’ai essayé de lui parler ou le raisonner, [parce que je sentais bien que c’était inutile mais aussi parce que je m’y refusais]. 
        \end{itemize}
        \item par un un comparatif : moins, plus
        \begin{itemize}
            \item Mon poids « idéal » se trouve être plus lourd que le poids de départ de mon régime [...] [sans doute sous prétexte que j’ai plus de 50 ans].
        \end{itemize}
    \end{itemize}
    \item comme, étant donné, maintenant que, puisque 不能被任何adverbes修饰
    \begin{itemize}
        \item \# Je vais essayer d’être heureux [surtout puisque tu es heureuse].
        \item \# Je ne veux pas m’en servir [certainement maintenant que les diables y ont touché].
    \end{itemize}
\end{enumerate}



\subsection{L’interprétation des subordonnées de cause}
\subsubsection{Les subordonnées de cause et le contenu principal}
Quand elles faire partie du contenu principal, les subordonnées peuvent être : clivée, interrogée, niée
\paragraph{Les subordonnées peuvent faire partie du contenu principal}
\begin{enumerate}
    \item subordonnées en \textit{parce que, surtout que, du fait que, sous prétexte que}
    \begin{enumerate}
        \item 以上词汇引导的原因从句对主句起到补充作用,为主句提供了新的信息,所以它们可以回答以pourquoi提问的问句,且可以clivée, interrogée, directement niée
    \end{enumerate}
    \item ajouts en \textit{à cause de, à force de, en raison de, faute de, par, pour}
\end{enumerate}
\begin{itemize}
    \item C’est [parce qu’il est malade] que Paul reste à la maison.
    \item Est-ce [parce qu’il est malade] que Paul reste à la maison ?
    \item Ce n’est pas [parce que le soleil devient pâlichon] qu’il faut se résigner à le devenir aussi.
    \item Il éprouve une autre peur aussi, non parce que je suis blanche mais parce que je suis si jeune, si jeune qu’il pourrait aller en prison si on découvrait notre histoire.
    \item Est-ce [sous prétexte que c’est l’Afrique] que l’on doit faire ce que l’on veut ?
    \item C’est [faute de trouver un logement sur son lieu de travail] qu’il laisse seuls son épouse et ses enfants.
    \item C’est [pour avoir perverti la jeunesse] que Socrate est condamné à mort.
\end{itemize}

\paragraph{Les subordonnées ne peuvent pas faire partie du contenu principal}
这些从句提供的只是已预设的共识,说话者认为对方已知的信息,所以它们不能用于clivées, directement niées ou interrogées.
\begin{enumerate}
    \item subordonnées en \textit{puisque}
    \begin{itemize}
        \item \# C’est [puisqu’il est malade] que Paul n’est pas venu.
        \item \# Est-ce [puisqu’il est malade] que Paul n’est pas venu ?
        \item \# Paul n’est pas venu, non [puisqu’il est malade], mais par oubli.
    \end{itemize}
    \item subordonnées en \textit{attendu que, comme, dès (lors) que, du moment que, étant donné que, maintenant que, vu que}, 可以clivées,但此时只能表示temporel et non causal.
    \begin{itemize}
        \item C’est [maintenant qu’il est à la retraite] que Paul reste plus souvent à la maison. [Temps]
        \item C’est [comme il partait] que Paul a compris son erreur. [Temps]
        \item \# C’est [étant donné qu’il est à la retraite] que Paul reste plus souvent à la maison.
        \item Je ne veux pas m’en servir [maintenant que les diables y ont touché]. 
    \end{itemize}
\end{enumerate}


\section{Les subordonnées circonstancielles de finalité}
\subsection{subordonnée de finalité}
\subsubsection{La relation de finalité}
\begin{enumerate}
    \item subordonnée de but
    \begin{itemize}
        \item Paul a déménagé en Patagonie [pour que personne ne le retrouve].
    \end{itemize}
    \item subordonnée de crainte
    \begin{itemize}
        \item je m’étais tassé au fond du siège le plus bas [de crainte que l’homme que j’avais reconnu puisse me reconnaître à son tour]
    \end{itemize}
\end{enumerate}

\subsubsection{La diversité des subordonnées de finalité}


\begin{table}[H]
    \centering
    \small
    \begin{adjustbox}{max width =\textwidth}
        \begin{tabular}{|p{3cm}|p{6cm}|p{6cm}|}
        \hline
        \rowcolor{cyan!20}
        \textbf{INTRODUCTEUR} & \textbf{FORMES} & \textbf{EXEMPLES} \\
        \hline
        avec, dans + nom de but ou de crainte + que & dans le but, dans la crainte, dans le dessein, avec l'intention & Je suis venu [avec l'intention que tout soit prêt pour le soir]. \\
        & & Je suis venu [dans le but que tout se passe bien]. \\
        & & Elle vivait [dans la crainte qu'on l'abandonne]. \\
        \hline
        de, en + nom de manière + à ce que, que & de façon, de telle sorte, de manière, de manière telle & Il faut écrire [de telle manière qu'on puisse vous répondre]. \\
        \hline
        préposition + que & afin, de crainte, de peur, de sorte, en sorte, \%histoire, par crainte, par peur, pour & Elle secoua son sac [afin que pas une miette ne se perde]. \\
        & & La petite hurlait [de crainte qu'on ne l'abandonne]. \\
        & & Je te raconte cela [pour que tu comprennes]. \\
        \hline
        subordonnant & que & Arrête cette musique [qu'on ait un peu la paix] ! \\
        \hline
        \end{tabular}
    \end{adjustbox}
    \caption{Les principales subordonnées circonstancielles de finalité}
\end{table}




\end{document}